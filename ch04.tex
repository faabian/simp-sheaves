% include latex header (\usepackage, \newcommand etc.) 
\documentclass[a4paper]{article}
% \usepackage[left=3cm,right=3cm,top=3cm,bottom=2cm]{geometry} % page settings
\usepackage{amsmath}
\usepackage{amssymb}
\usepackage{amsthm}
\usepackage{etoolbox}
\usepackage[ngerman]{babel}
\usepackage[utf8]{inputenc}
\usepackage{mathtools}
\usepackage{tikz-cd}
\usepackage{enumitem}
\usepackage{hyperref}

\setlength{\parskip}{\medskipamount}
\setlength{\parindent}{0pt}

\theoremstyle{plain}
\newtheorem{theorem}{Theorem}
\newtheorem{lemma}[theorem]{Lemma}
\newtheorem{prop}[theorem]{Proposition}
\newtheorem{kor}[theorem]{Korollar}
\newtheorem{satz}[theorem]{Satz}
%% \providecommand*{\lemmaautorefname}{Lemma}
%% \providecommand*{\propautorefname}{Prop.}
%% \providecommand*{\korautorefname}{Korollar}
%% \providecommand*{\satzautorefname}{Satz}

\theoremstyle{definition}
\newtheorem{defn}[theorem]{Definition}

\theoremstyle{remark}
\newtheorem{bem}[theorem]{Bemerkung}

\DeclareMathOperator{\Cat}{Cat}
\DeclareMathOperator{\poset}{poset}
\DeclareMathOperator{\EnsX}{Ens_{/X}}
\DeclareMathOperator{\pEnsX}{pEns_{/X}}
\DeclareMathOperator{\AbX}{Ab_{/X}}
\DeclareMathOperator{\pAbX}{pAb_{/X}}
\DeclareMathOperator{\OffX}{Off_X}
\DeclareMathOperator{\Ens}{Ens}
\DeclareMathOperator{\Ob}{Ob}
\DeclareMathOperator{\Der}{Der}
\DeclareMathOperator{\Ab}{Ab}
\DeclareMathOperator{\sKons}{s-Kons}
\DeclareMathOperator{\Ket}{Ket}
\DeclareMathOperator{\EnsB}{Ens_{/\B}}
\DeclareMathOperator{\im}{im}
\DeclareMathOperator{\Id}{Id}
\DeclareMathOperator{\id}{id}
\DeclareMathOperator{\colf}{colf}
\DeclareMathOperator{\limf}{limf}
\DeclareMathOperator{\Top}{Top}

\newcommand{\etalespace}[1]{\overline{#1}}
\newcommand{\B}{\mathcal{B}}
\newcommand{\op}{^\mathrm{op}}
\newcommand{\iso}{\xrightarrow{\sim}}
\newcommand{\qiso}{\xrightarrow{\approx}}
\newcommand{\fromqiso}{\xleftarrow{\approx}}
\newcommand{\open}{\subset\kern-0.58em\circ}  % only possible in math mode
\newcommand{\K}{\mathcal{K}}
\newcommand{\Z}{\mathbb{Z}}
\newcommand{\R}{\mathbb{R}}
\newcommand{\DerAbK}{\Der(\Ab_{/|\K|})}
\newcommand{\DerskK}{\Der_{\mathrm{sk}}(|\K|)}
\newcommand{\DerpskK}{\Der^+_{\mathrm{sk}}(|\K|)}
\newcommand{\AbKr}{\Ab_{/|\K|}}
\newcommand{\sKonsK}{\sKons(\K)}
\newcommand{\inj}{\hookrightarrow}
\newcommand{\surj}{\twoheadrightarrow}
\newcommand{\Iff}{\Leftrightarrow}
\newcommand{\Implies}{\Rightarrow}
\newcommand{\cc}{^{\bullet}}  % chain complex
\newcommand{\from}{\leftarrow}


\begin{document}

\title{Simpliziale Garben}
\author{Fabian Glöckle}
\date{\today}
% \maketitle

\chapter{Simpliziale Garben}

In diesem Kapitel beschäftigen wir uns mit simplizialen Objekten in
der Kategorie der Garben auf einem topologischen Raum $X$. Während
Simplizialkomplexe ungerichtete Graphen verallgemeinern,
verallgemeinern simpliziale Mengen gerichtete Graphen, und werden uns
so bei der geometrischen Realisierung von Objekten in
Diagrammkategorien von Garben zur Verfügung stehen.

\section{Simpliziale Mengen}

Wir betrachten die Menge $\Delta$ der nichtleeren endlichen
Ordinalzahlen. Ihre Elemente sind von der Form $\{0, 1, \dots, n\}$
für ein $n \in \N$, welche wir kurz mit $[n]$ bezeichnen werden. Wir
verstehen diese Mengen $[n]$ als angeordnete Mengen.

\begin{defn}
  Die Simplex-Kategorie $\Delta$ ist die Kategorie der endlichen
  nichtleeren Ordinalzahlen versehen mit monotonen Abbildungen als
  Morphismen.

  Ist $C$ eine Kategorie, so bezeichnen wir eine Prägarbe auf $\Delta$
  mit Werten in $C$ (d. h. einen Funktor $\Delta\op \to C$) als
  simpliziales Objekt in $C$.
\end{defn}

Unseren gewohnten Sprechweisen folgend nennen wir simpliziale Objekte
in $C$ auch kurz ``simpliziale $C$'' und sprechen etwa von
simplizialen Mengen und simplizialen Garben. Wir notieren die
Funktorkategorien simplizialer Objekte in $C$ auch kurz mit $\s C =
[\Delta\op, C]$. Opponiert bezeichnen wir einen Funktor $\Delta \to C$
als kosimpliziales Objekt in $C$.

Für $X \in \s C$ ein simpliziales Objekt notieren wir kurz $X_n =
X([n])$ und für eine monotone Abbildung $f: [m] \to [n]$ auch $f^* =
X(f)$.

Die monotonen Abbildungen $[m] \to [n]$ werden von zwei besonders
einfachen Klassen monotoner Abbildungen erzeugt: von den
\emph{Randabbildungen}, den eindeutigen Injektionen $d_i^n: [n - 1] \to [n]$,
die $i \in \{0, \dots, n\}$ nicht treffen, und den
\emph{Degenerationsabbildungen}, den eindeutigen Surjektionen $s_i^n: [n +
1] \to [n]$, für die $i \in \{0, \dots n\}$ zweielementiges Urbild
hat.

\begin{lemma}[\cite{GM}, I.2, ex. 1] \label{face-gen}

  \begin{enumerate}[label=(\roman*)] \item \label{itm:face-gen-rel}
  Die Rand- und Degenerationsabbildungen er\-fül\-len die Relationen
  \begin{align*}
    d_j^{n+1} d_i^n &= d_i^{n+1} d_{j-1}^n \quad \text{für} \quad i < j, \\
    s_j^n s_i^{n+1} &= s_i^n s_{j+1}^{n+1} \quad \text{für} \quad i \leq j, \\
    s_j^{n-1} d_i^n &=
    \begin{cases}
    d_i^{n-1} s_{j-1}^{n-2} \quad \text{für} \quad i < j, \\
    \id_{[n-1]} \quad \text{für} \quad i = j \text{ oder } i = j+1, \\
    d_{i-1}^{n-1} s_j^{n-2} \quad \text{für} \quad i > j + 1.
    \end{cases}
  \end{align*}
 
  \item \label{itm:face-gen-form} Sei $f: [m] \to [n]$ monoton. Dann
  hat $f$ eine eindeutige Darstellung
  \[ f = d_{i_1}^n \dots d_{i_s}^{n-s+1} s_{j_t}^{m-t} \dots s_{j_1}^{m-1} \]
  mit $n \geq i_1 > \dots > i_s \geq 0$, $m > j_1 > \dots > j_t \geq
  0$ und $n = m-t+s$.
    
  \item \label{itm:face-gen-cat} Die vom Köcher der endlichen
  nichtleeren Ordinalzahlen mit den Rand- und Korandabbildungen und
  den angegebenen Relationen auf den Morphismenmengen erzeugte
  Pfadkategorie ist isomorph zu $\Delta$.
\end{enumerate}
\end{lemma}
\begin{proof}
  \begin{enumerate}[label=(\roman*)]
  \item Durch Bildchen Zeichnen oder explizites Nachrechnen: Wir
  werten beide Abbildungen simultan auf allen Elementen aus und
  erhalten zum Beispiel für die erste Aussage:
  \begin{align*}
  &(0, 1, \dots, i-1, i, i+1, \cdots, j-2, j-1, j, \cdots, n-1) \\
  \xmapsto{d_i^n} \quad
  &(0, 1, \cdots, i-1, i+1, i+2, \cdots, j-1, j, j+1, \cdots, n) \\
  \xmapsto{d_j^{n+1}} \quad
  &(0, 1, \cdots, i-1, i+1, i+2, \cdots, j-1, j+1, j+2, \cdots, n+1)
  \end{align*}
  sowie
  \begin{align*}
  &(0, 1, \cdots, i-1, i, i+1, \cdots, j-2, j-1, j, \cdots, n-1) \\
  \xmapsto{d_{j-1}^n} \quad
  &(0, 1, \cdots, i-1, i, i+1, \cdots, j-2, j, j+1, \cdots, n) \\
  \xmapsto{d_i^{n+1}} \quad
  &(0, 1, \cdots, i-1, i+1, i+2, \cdots, j-1, j+1, j+2, \cdots, n+1).
  \end{align*}

  \item In einer solchen Darstellung gibt $m - t$ die Anzahl der
  Funktionswerte der Abbildung an, die $j_k$ beschreiben die Partition
  von $(0, \cdots, m)$ in zusammenhängende Abschnitte mit demselben
  Funktionswert und die $i_k$ bestimmen die Funktionswerte selbst.

  \item Jeden Morphismus in der Pfadkategorie, d. h. jedes Tupel
  komponierbarer Rand- und Degenerationsabbildungen kann mit den
  Relationen aus dem ersten Teil auf eine Form wie
  in \ref{itm:face-gen-form} gebracht werden. Umgekehrt ist
  nach \ref{itm:face-gen-form} jede monotone Abbildung aber auch als
  ein solcher Pfad darstellbar. Den Isomorphismus definiert also der
  Funktor, der auf Objekten durch die Identität und auf Morphismen
  durch die offensichtliche Zuordnung der Erzeuger der Pfadkategorie
  auf die jeweiligen Rand- und Degenerationsabbildungen gegeben ist.
\end{enumerate}
\end{proof}

Die einfachsten Beispiele nichttrivialer simplizialer Mengen sind die
Standard-$n$-Simplizes, die sich als die darstellbaren Funktoren
$\Delta^n = \Delta(\cdot, [n])$ beschreiben lassen. Wir erhalten den
Funktor
\begin{align*}
  R: \Delta\op &\to \s\Ens, \\
  [n] &\mapsto \Delta^n = \Delta(\cdot, [n]),
\end{align*}
der auf Morphismen durch Vorschalten gegeben ist, welches wir mit
$f \mapsto f^*$ notieren.

Ist $X \in \s\Ens$ eine simpliziale Menge, so bezeichnen wir $X_n :=
X([n])$ als die Menge der $n$-Simplizes von $X$. Nun sind die
$n$-Simplizes von $X$ gerade die ``Bilder des $n$-ten
Standardsimplizes in $X$'' sind, präziser
\begin{equation} \label{eq:simp-as-hom}
X_n \iso \s\Ens(\Delta^n, X).
\end{equation}
In der Tat besagt das Yoneda-Lemma, dass die Transformationen des
freien Funktors $\Delta^n = \Delta(\cdot, [n])$ zum Funktor $X$ in
natürlicher Bijektion stehen zu $X([n])$.

Das anschließende Lemma zeigt, dass sich eine simpliziale Menge $X$
vollständig durch ihre Simplizes $\Delta^n \to X$ verstehen
lässt. Dazu erklären wir zunächst Slice-Kategorien, Kategorien von
Objekten über einem gegebenen Objekt bzw. Spezialfälle von
Komma-Kategorien.
\begin{defn}
  Sei $r: C \to D$ ein Funktor und $X \in D$ ein Objekt. Dann
  bezeichnet $\slicecat{C}{r}{X}$ die Slice-Kategorie der Objekte von
  $C$ über $X$ mittels $r$, deren Objekte Objekte $Y \in C$ samt einem
  Morphismus $\pi_Y: rY \to X$ sind und deren Morphismen Morphismen
  $f: Y \to Z$ in $C$ sind, für die $rf$ ein Morphismus über $X$ ist,
  d. h. $\pi_Y = \pi_Z \circ rf$ gilt.
\end{defn}
Wir bezeichnen die Slice-Kategorie $\slicecat{\Delta}{r}{X}$ als die
Simplexkategorie von $X$. Konkret ist darin ein Objekt ein Morphismus
$\Delta^n \to X$ und ein Morphismus ein von $[n] \to [m]$ induzierter
Morphismus simplizialer Mengen $\Delta^n \to \Delta^m$ über $X$.

\begin{lemma}
  Sei $X \in \s\Ens$ eine simpliziale Menge. Dann gilt
  \[ X \iso \col_{\Delta^n \to X} \Delta^n, \]
  mit dem Kolimes über die Simplexkategorie von $X$.
\end{lemma}
\begin{proof}
  Bezeichne $K$ obigen Kolimes. Das System, über das der Kolimes
  gebildet wird, sichert uns nach der universellen Eigenschaft des
  Kolimes einen Morphismus $K \to X$. Wir müssen also zeigen, dass
  dieser Bijektionen auf den $n$-Simplizes induziert. Tatsächlich
  definiert ein $n$-Simplex $\Delta^n \to K$ durch Nachschalten von
  $K \to X$ einen $n$-Simplex in $X$. Diese Zuordnung ist bijektiv,
  denn wir erhalten eine Umkehrabbildung, wenn wir $\Delta^n \to X$
  den zugehörigen Morphismus in den Kolimes $\Delta^n \to K$ zuordnen.
\end{proof}
Der Beweis benötigte keine konkreten Eigenschaften der simplizialen
Mengen. Wir halten mit wörtlich übertragenen Beweis allgemein fest:
\begin{prop}
  Sei $F: C \to \Ens$ ein Funktor. Dann ist $F$ ein Kolimes über
  darstellbare Funktoren $C(X, \cdot)$.
\end{prop}
\begin{proof}
  Wir betrachten mittels des Funktors $r: C \to \Ens^C, X \mapsto
  C(X, \cdot)$ die Slice-Kategorie $\slicecat{C}{r}{F}$ der
  darstellbaren Funktoren über $F$ und dann das System in $\Ens^C$,
  das daraus durch Vergessen der Transformationen nach $F$
  hervorgeht. Wir bezeichnen wieder den Kolimes darüber mit $K$ und
  erhalten mit der universellen Eigenschaft eine Tranformation
  $K \Trafo F$. Diese ist eine Isotransformation, wenn sie auf allen
  Objekten $X \in C$ Bijektionen $KX \iso FX$ induziert. Nach dem
  Yoneda-Lemma entsprechen diese den Transformationen
  $C(X, \cdot) \Trafo K$ bzw. $C(X, \cdot) \Trafo F$. Diese stehen
  aber nach der universellen Eigenschaft des Kolimes und der
  Definition der Kategorie $\slicecat{C}{r}{F}$ in Bijektion durch
  Nachschalten von $K \Trafo F$.
\end{proof}

\section{Der kosimpliziale Raum der Standardsimplizes}

Die geometrsiche Realsierung einer simplizialen Menge soll sich mit
geeigneten Identifikationen aus den geometrischen Realisierungen von
Standard-$n$-Simplizes zusammensetzen. Diese definieren wir als die
abgeschlossenen geometrischen Standard-$n$-Simplizes
\[ |\Delta^n| = \set{x \in \R^{n+1}}{0 \leq x_i \leq 1, \sum_{i=0}^n x_i = 1}. \]

Ein Morphismus $f: [m] \to [n]$ induziert eine stetige Abbildung $|f|:
|\Delta^m| \to |\Delta^n|$ auf den zugehörigen Simplizes. Nach
\ref{face-gen} reicht es, diese für Rand- und Degenerationsabbildungen
anzugeben. Für die Randabbildung $|d_i|$ handelt es sich dabei um die
Inklusion der $i$-ten Kante, d. h. in Koordinaten das Einfügen einer
Null an der $i$-ten Stelle, für die Degenerationen $|s_i|$ um den
Kollaps der $i$-ten Kante, d. h. in Koordinaten die Ersetzung der
$i$-ten und ihrer darauffolgenden Koordinate durch ihre Summe. In
Formeln:
\begin{align*}
  |d_i^n|(x_0, \cdots, x_{n-1})
  &= (x_0, \cdots, x_{i-1}, 0, x_i, \cdots, x_{n-1}), \\
  |s_i^n|(x_0, \cdots, x_{n+1})
  &= (x_0, \cdots, x_{i-1}, x_i + x_{i+1}, x_{i+2}, \cdots, x_{n+1}).
\end{align*}
Wir erhalten einen Funktor $R: \Delta \to \Top, [n] \mapsto \Delta^n$,
genannt der \emph{kosimpliziale Raum der Standardsimplizes}.

\section{Geometrische Realisierung simplizialer Mengen}

Wir erklären nun die geometrische Realisierung simplizialer
Mengen. Der Unterschied zur geometrischen Realsierung von
Simplizialkomplexen ist im Wesentlichen die Möglichkeit, Simplizes
wiederzuverwenden und zu degenerieren, was zu Identifikationen in der
geometrischen Realisierung führt. Ein Fall von ``Wiederverwendung''
ist etwa die Realisierung der $S^1$ als 1-Simplex, dessen Endpunkte
übereinstimmen. Degeneration bedeutet, dass niedererdimensionale
Simplizes auch die Rolle höherdimensionaler Simplizes übernehmen
können. Wir können etwa unser Beispiel modifizieren und die $S^n$ als
$n$-Simplex realisieren, bei dem alle niederdimensionalen Kanten in
einem Punkt zusammenfallen.

Die geometrische Realisierung von Standard-$n$-Simplizes haben wir
gerade gesehen. Nun fordern wir, dass sich die Realisierung mit
Kolimites vertrage:
\begin{equation} \label{eq:real-cocont}
|\col_i X_i| \iso \col_i |X_i|.
\end{equation}
Wenn dies der Fall sein soll, so müssen wir mit der Darstellung
\[ X := \col_{\Delta^n \to X} \Delta^n \]
aus \ref{sset-colim} auf jeden Fall setzen
\begin{equation} \label{eq:real-colim}
|X| := \col_{\Delta^n \to X} |\Delta^n|.
\end{equation}
Debei wird der Kolimes wieder über die Simplexkategorie von $X$
gebildet, nun allerdings mit den induzierten stetigen Abbildungen aus
\ref{} als Systemmorphismen.

Ein Morphismus simplizialer Mengen $X \to Y$ induziert nun durch
Nachschalten von $X \to Y$ einen Funktor auf den Simplexkategorien
$\Delta \downarrow X \to \Delta \downarrow Y$ und damit auch auf den
Kolimites eine stetige Abbildung $|X| \to |Y|$. Wir erhalten also den
Funktor der geometrischen Realisierung $| \cdot |: \s \Ens \to \Top$.

Unsere Konstruktion erfüllt tatsächlich:
\begin{prop}
  Der Funktor der geometrischen Realisierung simplizialer Mengen $|
  \cdot|$ vertauscht mit beliebigen Kolimites über kleine
  Indexkategorien.
\end{prop}
\begin{proof}
  Zu zeigen ist, dass die Bildung der Simplexkategorie mit Kolimites
  vertauscht, d. h. für einen Funktor $X: I \to \s\Ens$ gilt:
  \[ \slicecat{\Delta}{r}{\col_i X_i} \iso \col_i \slicecat{\Delta}{r}{X_i}. \]
  Beide Kategorien bestehen aber aus Objekten der Form $\Delta^n \to
  X_i$ für ein $i \in I$, die identifiziert werden, falls sie durch
  Nachschalten von $X_i \to X_j$ auseinander hervorgehen.
\end{proof}

Wie ganz allgemein können wir den Kolimes topologischer Räume auch
explizit mittels Koprodukt und Koegalisator ausschreiben. Wir
verstehen die Mengen $X_n$ als diskrete topologische Räume und
erhalten:
\[ |X| \iso (\coprod_n X_n \times |\Delta^n|) / \sim \]
mit der Quotiententopologie, die durch die von
\[ (x, |f|(p)) \sim (f^* x, p) \]
für alle monotonen $f: [m] \to [n]$ erzeugte Äquivalenzrelation
gegeben ist.

\section{Sparsame Realisierung durch nichtdegenerierte Simplizes}

Während die Definition simplizialer Mengen und ihrer Realisierung mit
Degenerationsabbildungen wie gesehen von einem formalen Standpunkt aus
sehr elegant ist, ist für die konkrete Arbeit häufig eine explizitere
Form der Realisierung praktischer, die ``unnötige Simplizes von
vornherein weglässt''.
\begin{defn}[\cite{GM}, I.2.9]
  Sei $X \in \s\Ens$ eine simpliziale Menge. Ein Simplex $x \in X_n$
  heißt degeneriert, falls es einen Simplex $y \in X_m$ und eine
  surjektive monotone Abbildung $s: [n] \to [m], n > m$ gibt mit $x =
  s^* y$.
\end{defn}
Andernfalls heißt ein Simplex nichtdegeneriert. Für $X \in \s\Ens$
bezeichne $NX_n$ die Menge der nichtdegenerierten $n$-Simplizes von
$X$.
\begin{lemma}[\cite{MIT}, Prop. 9] \label{degen-repr}
  Sei $X \in \s\Ens$ eine simpliziale Menge und $x \in X_n$ ein
  $n$-Simplex. Dann gibt es eine eindeutige Darstellung $x = s^* y$
  für $y \in NX_m$ einen nichtdegenerierten Simplex von $X$ und $s:
  [n] \to [m]$ eine surjektive monotone Abbildung.
\end{lemma}
\begin{proof}
  Ist $x$ nichtdegeneriert, wähle $y = x$ und $s =
  \id_{[n]}$. Andernfalls gibt es nach Definition ein $y$ eine
  Surjektion mit der gewünschten Eigenschaft. Diese sind eindeutig
  nach den Relationen aus \ref{face-gen}, vergleiche \cite{MIT},
  Prop. 9.
\end{proof}
Wir notieren für eine solche Darstellung $x = s^* y$ auch $y =
N_s(x)$.

Wir definieren nun die \emph{sparsame geometrsiche Realisierung} einer
simplizialen Menge $X$ wie folgt:
\[ \| X \| := \coprod_n NX_n \times |\Delta^n| / \sim_N ,\]
wobei wie gehabt $NX_n$ die diskrete Topologie trägt und die
Äquivalenzrelation erzeugt ist von
\[ (x, |d_i|(p)) \sim_N (N_s(d_i^* x), |s|(p)), \]
mit der eindeutigen Darstellung Degenerierter aus
\ref{degen-repr}. Diese Äquivalenzrelation lässt sich interpretieren
als das Umgehen der mittleren Schritte in der Rechnung
\[ (x, d_i p) \sim (d_i^* x, p) \sim (s^* y, p)
   \sim (y, |s|(p)) \]
mit $y = N_s(d_i^* x)$.
   
\begin{satz}
  Die von der Inklusion
  \[ \coprod_n NX_n \times |\Delta^n| \inj \coprod_n X_n \times |\Delta^n| \]
  induzierte Abbildung $\|X\| \iso |X|$ ist ein Homöomorphismus.
\end{satz}
\begin{proof}
  Die Abbildung existiert und ist stetig nach der universellen
  Eigenschaft topologischer Quotienten, denn die Äquivalenzrelation
  $\sim$ umfasst $\sim_N$. Sie ist bijektiv, denn für die
  dazukommenden Punkte in degenerierten Simplizes $s^* x$ gilt ohnehin
  $(s^* x, p) \sim (x, |s|(p))$. Weiter ist sie offen: Ist $U \open
  \|X\|$ eine offene Teilmenge, so berechnen wir ihr Bild in $|X|$
  durch das Bild ihres Urbilds $V$ in $\coprod_n NX_n \times
  |\Delta^n|$ unter
  \[ \coprod_n NX_n \times |\Delta^n|
  \inj \coprod_n X_n \times |\Delta^n| \xrightarrow{q} |X|. \]

  Bezeichne $\overline{V}$ den Abschluss von $V \open \coprod_n X_n
  \times |\Delta^n|$ unter $\sim$-Äquivalenz. Es gilt $\overline{V} =
  q^{-1}(q(V))$ und wir müssen nach Definition der Quotiententopologie
  zeigen, dass $\overline{V}$ offen ist. Bezeichne für $x$ einen
  nichtdegenerierten Simplex von $X$ den Schnitt von $V$ mit dem zu
  $x$ gehörigen geometrischen Simplex $|\Delta^n|$ mit $V_x$. Bei
  Übergang von $V$ zu $\overline{V}$ kommen dann alle Punkte $(s^* x,
  p)$ mit $(x, |s|(p)) \in U_x$ hinzu. Das Urbild der offenen Menge $U_x$
  unter dem stetigen $s$-Kollaps $|s|$ ist dann natürlich wieder
  offen.
\end{proof}

\section{Iterative Konstruktion der geometrischen Realisierung}

Wir geben eine weitere Interpretation der geometrischen Realisierung
als iteratives Ankleben geometrischer Simplizes an ihre Ränder an und
werden so insbesondere sehen, dass die geometrische Realisierung einen
Hausdorffraum liefert (\cite{Moer}, III.1).

Bezeichne dazu $\|X^{\leq k}\| = \coprod_{n = 0}^k NX_n \times
|\Delta^n| / \sim_N $ die geometrische Realisierung durch
nichtdegenerierte Simplizes der Dimension $\leq k$. Wir erhalten
Einbettungen $\|X^{\leq k}\| \inj \|X\|$ sowie $\|X\| = \bigcup_k
\|X^{\leq k}\|$, denn die Äquivalenzrelation von ganz $\|X\|$ fügt
keine neuen Identifikationen auf den Teilmengen $\|X^{\leq k}\|
\subset \|X\|$ hinzu.

Wir können die $\|X^{\leq k}\|$ iterativ konstruieren. Betrachte dazu
die stetigen Abbildungen $\pi_k$, die uns den ``$k$-dimensionalen
Teil'' von $\|X\|$ liefern:
\[ \pi_k: NX_k \times |\Delta^k|
  \inj \coprod_{n=0}^k NX_n \times |\Delta^n|
  \surj \|X^{\leq k}\|. \]

\begin{prop}[\cite{Moer}, III.1, \cite{GJ}, I.2.3]
  Sei $X \in \s\Ens$ eine simpliziale Menge. Dann ist
  \shorthandoff{"}
  \[ \begin{tikzcd}
    NX_k \times \del |\Delta^k| \arrow[r, hook] \arrow[d]
    \arrow[dr, phantom, "\ulcorner"]
    & NX_k \times |\Delta^n| \arrow[d, "\pi_k"] \\
    \|X^{\leq k-1}\| \arrow[r, hook]
    & \|X^{\leq k}\|
  \end{tikzcd} \]
  ein Pushout topologischer Räume. Dabei ist die Abbildung links auf
  der $i$-ten Kante $|d_i|: |\Delta^{k-1}| \inj \del |\Delta^k|$
  gegeben durch
  \begin{align*}
    NX_k \times |\Delta^{k-1}| &\to \|X^{\leq k-1}\| \\
    (x, p) &\mapsto [(N_s(d_i^* x), |s|(p))].
  \end{align*}
\end{prop}
\begin{proof}
  Die Definition der Abbildung links ist sinnvoll, da sie auf den
  niederdimensionalen Überschneidungskanten von $|d_i| |\Delta^k|$ und
  $|d_j| |\Delta^k|$ übereinstimmt.
  
  Sowohl der Pushout des Diagramms als auch $\|X^{\leq k}\|$ sind
  Quotienten von $\coprod_{n=0}^k NX_n \times |\Delta^n|$ nach
  gewissen Äquivalenzrelationen. Bezeichne die Einschränkung von
  $\sim_N$ auf $\coprod_{n=0}^k NX_n \times |\Delta^n|$ mit
  $\sim_N^k$. Die erzeugenden Relationen für $\sim_N^k$ sind:
  \[ (x, |d_i|(p)) \sim (N_s(d_i^* x), |s|(p))
  \quad \text{für} x \in NX_n, n \leq k. \]
  Das sind für $x \in NX_n$ und $n \leq k-1$ genau die erzeugenden
  Relationen von $\sim_N^{k-1}$ und für $n = k$ genau die Relationen
  aus der Definition des Pushouts.
\end{proof}

\begin{kor} \label{real-hd}
  Die geometrische Realisierung $|X|$ einer simplizialen Menge $X \in
  \s\Ens$ ist ein CW-Komplex und insbesondere ein Hausdorffraum.
\end{kor}
\begin{proof}
  Die vorangegangene Proposition zeigt, dass es sich um einen
  CW-Komplex handelt. Die Pushouts sind dabei das Ankleben von
  $k$-Zellen $D^k \cong |\Delta^k|$. Konkret zur
  Hausdorff-Eigenschaft: Seien $p, q \in \|X\|$, $p \neq q$. Dann gibt
  es ein $k$ mit $p, q \in \|X^{\leq k}\| \setminus \|X^{\leq
    k-1}\|$. Somit liegt ohne Einschränkung $p$ im Inneren eines
  $k$-Simplizes und kann ohne Probleme durch disjunkte offene
  Umgebungen $U_k, V_k \open \|X^{\leq k}\|$ von Randpunkten dieses
  Simplexes und Punkten in anderen Simplizes getrennt werden. Wir
  erhalten daraus disjunkte offene Umgebungen $U$ und $V$ in ganz
  $|X|$, indem wir sie ``baryzentrisch ergänzen'': Sei dazu zu jedem
  geometrischen Standardsimplex $|\Delta^n|$ ein ausgezeichneter Punkt
  $z_n$ in seinem offenen Inneren fest gewählt (etwa das
  Baryzentrum). Wir setzen für $n \geq k$ induktiv
  \begin{align*}
    U_{n+1} = \big\{ \: [(y, r)] \in \|X^{\leq n+1}\| \quad \big| \quad
    & y \in NX_{n+1}, \exists t \in (0, 1], w \in |\Delta^n|: \\
    & (d_i^* y, w) \in U_n, r = t w + (1-t) z_y \: \big\},
  \end{align*}
  den konvexen Abschluss zu $z_y$ von $U_n$ in den an $U_n$
  angrenzenden Simplizes $y$ und dann $U = \bigcup_{n \geq k} U_n$
  sowie $V$ entsprechend. Diese Teilmengen sind nach Induktion
  disjunkt, offen und enthalten $p$ bzw. $q$.
\end{proof}

\section{Kompakt erzeugte Hausdorffräume}

Die geometrischen Realisierungen simplizialer Mengen liegen
tatsächlich sogar in einer in ihren kategoriellen Eigenschaften
bequemen (engl. \emph{convenient}) Kategorie topologischer Räume, den
kompakt erzeugten Hausdorffräumen.

\begin{defn}[\cite{Steenrod}] \label{def:cg}
  Ein topologischer Raum $X$ heißt kompakt erzeugt, falls gilt: eine
  Teilmenge $A \subset X$ ist abgeschlossen, falls $A \cap K$ in $K$
  abgeschlossen ist für jedes Kompaktum $K \subset X$.
\end{defn}
\begin{bem}
  Äquivalent dazu ist: eine Teilmenge $U \subset X$ ist offen, falls
  $U \cap K$ in $K$ offen ist für jedes Kompaktum $K \subset X$.
\end{bem}
Wir notieren volle Unterkategorie der kompakt erzeugten topologischen
Räume in den topologischen Räumen mit $\CG$ und der kompakt erzeugten
Hausdorffräume mit $\CGHaus$. Die kompakt erzeugten topologischen
Räume umfassen eine sehr große Klasse an relevanten topologischen
Räumen, etwa CW-Komplexe (nach dem nachfolgenden Lemma \ref{cg-crit})
oder erstabzählbare Räume (\cite{Steenrod}, 2.2).

Für uns relevant sind die folgenden Kriterien:
\begin{lemma} \label{cg-crit}
  \begin{enumerate}[label=(\roman*)]
    \item \label{itm:cg-crit-lc} Ist $X$ ein lokal kompakter
      topologischer Raum, so ist $X$ kompakt erzeugt.
    \item \label{itm:cg-crit-quot} Ist $p: X \surj Y$ eine
      Quotientenabbildung (d. h. $Y$ trägt die Finaltopologie
      bezüglich $p$) und $X$ kompakt erzeugt, so ist auch $Y$ kompakt
      erzeugt.
     \item \label{itm:cg-crit-cw} Ist $X$ ein CW-Komplex, so ist $X$
       kompakt erzeugt.
  \end{enumerate}
\end{lemma}
\begin{proof}
  \begin{enumerate}[label=(\roman*)]
    \item \label{itm:cg-crit-proof-lc} Sei $U \subset X$ mit $U \cap
      K$ offen für alle $K \subset X$ kompakt. Mit der
      Lokalkompaktheit von $X$ wählen wir zu jedem $x \in X$ eine
      kompakte Umgebung $K_x$ und erhalten, dass $U = \bigcup_{x \in
        X} U \cap K_x$ offen ist als Vereinigung offener Mengen.
    \item \label{itm:cg-crit-proof-quot} Sei $A \subset Y$ mit $A \cap
      K$ abgeschlossen für alle $K \subset Y$ kompakt. Mit der
      Finaltopologie auf $Y$ gilt, dass $A \subset Y$ genau dann
      abgeschlossen ist, wenn $p^{-1}(A) \subset X$ abgeschlossen
      ist. Sei nun $\tilde{K} \subset X$ kompakt. Es ist $p^{-1}(A)
      \cap \tilde{K}$ abgeschlossen, denn $p(p^{-1}(A) \cap \tilde{K})
      = p(\tilde{K}) \cap A$ ist abgeschlossen nach Voraussetzung
      ($p(\tilde{K}) \subset Y$ ist kompakt als stetiges Bild eines
      Kompaktums). Da $X$ kompakt erzeugt ist, folgt nun, dass
      $p^{-1}(A)$ abgeschlossen ist und somit auch $A \subset Y$.
    \item Die $n$-Bälle $D^n$ und $n-1$-Sphären $S^{n-1}$ sind kompakt
      erzeugt nach \ref{itm:cg-crit-proof-lc}. Disjunkte Vereinigungen
      kompakt erzeugter Räume sind kompakt erzeugt. Quotienten kompakt
      erzeugter Räume sind kompakt erzeugt nach
      \ref{itm:cg-crit-proof-quot}. Da kompakte Teilmengen von
      CW-Komplexen nur endlich viele offene Zellen treffen, ist auch
      die Vereinigung über alle $n$-Skelette kompakt erzeugt.
  \end{enumerate}
\end{proof}

Daraus folgt sofort unser Ziel.
\begin{kor}
  Die geometrische Realisierung $|X|$ einer simplizialen Menge $X \in
  \s\Ens$ ist ein kompakt erzeugter Hausdorffraum.
\end{kor}
\begin{proof}
  Es handelt sich um einen Hausdorffraum nach \ref{real-hd}. Nach
  \ref{cg-crit} \ref{itm:cg-crit-lc} ist $\coprod_n X_n \times
  |\Delta^n|$ kompakt erzeugt und mit \ref{cg-crit}
  \ref{itm:cg-crit-quot} auch $|X|$. Das folgt auch aus \ref{real-hd}
  und \ref{cg-crit} \ref{itm:cg-crit-cw}.
\end{proof}
\begin{bem}
  Die geometrische Realisierung $|\cdot|: \s\Ens \to \CGHaus$
  vertauscht mit endlichen Limites. Ganz allgemein lassen sich
  endliche Limites als Egalisatoren endlicher Produkte darstellen. Der
  Egalisator in $\Top$ zweier stetiger Abbildungen zwischen kompakt
  erzeugten Hausdorffräumen ist dabei als abgeschlossener Unterraum
  eines kompakt erzeugten Hausdorffraums selbst wieder in $\CGHaus$
  und erfüllt die universelle Eigenschaft in dieser
  Unterkategorie. Das Vertauschen der Realisierung mit Egalisatoren
  folgt direkt aus der Definition. Für das Vertauschen mit endlichen
  Produkten verweisen wir auf \cite{Gabriel-Zisman}. Entscheidend ist,
  dass das Produkt in $\Top$ zweier kompakt erzeugter Hausdorffräume
  im Allgemeinen nicht wieder kompakt erzeugt ist. Das korrekte
  Produkt in $\CGHaus$ heißt auch \emph{Kelley-Produkt} und ist
  gegeben durch Anwenden des Linksadjungierten der Inklusion $\CG \to
  \Top$ auf das Produkt in $\Top$. Dieser Linksadjungierte $k: \Top
  \to \CG$ verfeinert die Topologie eines Raumes $X$ um alle Mengen $U
  \subset X$, deren Schnitte mit allen Kompakta $K \subset X$ in $K$
  offen sind, also genau um die für kompakte Erzeugtheit
  benötigten. Es gilt also:
  \[ |X \times Y| \iso k(|X| \times |Y|). \]
\end{bem}

% TODO Vertauschen mit Kelley-Produkt ausführen?
% TODO Stelle suchen in Gabriel-Zisman
% TODO Gegenbsp. vertauscht nicht mit beliebigen Limites

\begin{bsp} \label{ex:cg-products}
  % TODO Gegenbsp CG nicht abg. unter Produkten in Top.
\end{bsp}

\section{Geometrische Realisierung als Koende}

Die Sprache der Enden und Koenden ermöglicht uns eine allgemeinere
Sicht auf die geometrische Realisierung und wird sich später als
nützlich erweisen, um andere Formen der geometrischen Realisierung
einzuführen. Wir führen Koenden und ihre Eigenschaften ein, Enden sind
dazu formal dual. Die Darstellung folgt \cite{Lore}.
\begin{defn}[\cite{Lore}, 1.1 f.]
  Sei $F: C\op \times C \to D$ ein Funktor. Ein Objekt $K \in D$ mit
  Morphismen $\iota_c: F(c, c) \to K$ für alle $c \in C$ heißt Kokeil
  für $F$, falls für alle Morphismen $f: c \to d$ in $C$ das folgende
  Diagramm kommutiert:
  \[ \begin{tikzcd}
    F(d, c) \arrow{r}{F(f, \id)} \arrow{d}{F(\id, f)}
    & F(c, c) \arrow{d}{\iota_c} \\
    F(d, d) \arrow{r}{\iota_d} & K
  \end{tikzcd} \]
  Die Kokeile für $F$ bilden eine Kategorie (Morphismen verträglich
  mit den $\iota_c$). Ein initiales Objekt in der Kategorie der
  Kokeile für $F$ (ein universeller Kokeil) heißt das Koende von $F$.
\end{defn}
Ein Kokeil für einen Funktor $F: C\op \times C \to D\op$ heißt
entsprechend Keil für den assoziierten Funktor $F\op: C \times C\op
\to D$ und ein universeller Keil das Ende des Funktors $F\op$. Wir
notieren häufig das Koende von $F$ als Integral:
\[ \int^{c \in C} F(c, c). \]

Wir können Koenden als Koegalisator von Koprodukten darstellen:
\begin{lemma}[\cite{Lore}, 1.14] \label{coend-coeq}
  Sei $F: C\op \times C \to D$ ein Funktor und $K$ sein Koende. Dann
  ist das folgende Diagramm ein Koegalisator:
  \[ \coprod_{f: c \to d} F(d, c)
  \overset{F(f, \id}{\underset{F(\id, f)}{\rightrightarrows}}
  \coprod_{c \in C} F(c, c) \to K. \]
\end{lemma}
\begin{proof}
  Der Morphismus in das Koende ist von den $\iota_c: F(c, c) \to K$
  induziert, die Morphismen des Egalisators durch die angegebenen
  Morphismen auf den Komponenten $F(d, c) \xrightarrow{F(f, \id)} F(c,
  c)$ und $F(d, c) \xrightarrow{F(\id, f)} F(d, d)$. Die universellen
  Eigenschaften, die den Koegalisator und das Koende definieren, sind
  dann identisch.
\end{proof}

Sind $F: C\op \to D$ und $G: C \to D$ Funktoren und ist $D$ eine
monoidale Kategorie (Schmelzkategorie mit universellen
Verschmelzungen), so definieren sie einen Funktor
\begin{alignat*}{4}
  C\op \times C &\to D \times D &&\to D, \\
  (a, b) &\mapsto (Fa, Gb) &&\mapsto Fa \otimes Gb.
\end{alignat*}
Das Koende über diesen Funktor heißt das \emph{Tensorprodukt} $F
\otimes G$ von $F$ und $G$.

Wir erinnern an die Darstellung der geometrischen Realisierung mit
Koegalisator und Koprodukt:
\[ |X| \iso (\coprod_n X_n \times |\Delta^n|) / \sim \]
mit dem Quotienten nach der Äquivalenzrelation
\[ (x, |f|(p)) \sim (f^* x, p) \]
für alle monotonen $f: [m] \to [n]$. Das ist gerade der Koegalisator
 \[ \coprod_{f: [n] \to [m]} X_n \times |\Delta^m|
 \overset{f^* \times \id}{\underset{\id \times |f|)}{\rightrightarrows}}
 \coprod_{[n]} X_n \times |\Delta^n| \to |X| \]
aus \ref{coend-coeq} und mithin, falls $\Top$ mit ihrer kartesischen
Schmelzstruktur durch Produkte versehen ist, das Tensorprodukt der
Funktoren
\begin{alignat*}{4}
  &X: \Delta\op \to &&\Ens \to &&\Top \qquad &&\text{und} \\
  &R: \Delta &&\to &&\Top, && [n] \mapsto |\Delta^n|,
\end{alignat*}
wobei die Mengen $X_n$ als diskrete topologische Räume aufgefasst
werden (\cite{Moer}, III.1):
\[ |X| = X \otimes R. \]

\begin{bem} \label{coend-correct-real}
  Rückblickend halten wir fest, dass während unsere durch die
  ``Kostetigkeitseigenschaft'' für die Simplexkategorie
  $\slicecat{\Delta}{r}{X}$ (Gl. \ref{eq:real-colim},
  \ref{eq:real-cocont}) definierte geometrische Realisierung in dieser
  Situation elegant ist, sie auf einer konkreten Eigenschaft der
  Situation simplizialer \emph{Mengen} beruht, die wir bei
  allgemeineren simplizialen Objekten nicht mehr gegeben
  haben. Konkret werden wir im nächsten Abschnitt einen gewissen
  Begriff einer Wirkung einer Kategorie auf einer anderen Kategorie
  erklären (Koexponentiale, siehe \ref{???}). Die Besonderheit
  simplizialer Mengen beruht nun darauf, dass sich die Wirkung von
  $\Ens$ auf $\Top$ $(X_n, |\Delta^n|) \mapsto \coprod_{X_n}
  |\Delta^n|$ als von der Wirkung von $\Ens$ auf $\s\Ens$, $(X_n,
  \Delta^n) \mapsto \coprod_{X_n} \Delta^n \cong \set{\Delta^n
    \xrightarrow{x} X}{x \in X_n}$ induziert darstellen lässt, was uns
  zum Begriff der Simplexkategorie von $X$ geführt hat. Bei Wirkungen
  wie von $\Top$ auf $\Top$ durch das kartesische Produkt wird es
  nicht mehr möglich sein, die $\Delta^n \to X$ aus der
  Simplexkategorie als topologischen Raum aufzufassen. Unsere
  Beschreibung der geometrischen Realisierung als Koende ist also die
  für allgemeinere Kontexte richtige Beschreibung.
\end{bem}

\section{Der (Ko-) Enden-Kalkül}

Die Sprache der (Ko-) Enden besitzt eine Reihe an Verträglichkeits-
und Umformungseigenschaften, die sie zu einem mächtigen Werkzeug für
eine Vielzahl formaler Rechnungen in Anwendungen der Kategorientheorie
machen (für Beispiele siehe \cite{Lore}). Im folgenden werden einige
Umformungsregeln dieses Kalküls \footnote{Bei der Übersetzung
  aus dem Englischen geht leider die Analogie zu den Regeln der
  Differentiations- und Integrationstheorie verloren.}
zusammengestellt.

% TODO Ninja Yoneda (density thm)

\begin{lemma}[Funktorialität, \cite{Lore}]
  Sind $F \xRightarrow{\eta} G \xRightarrow{\tau} H$ Transformationen
  von Funktoren $F, G, H: C\op \times C \to D$, deren Koenden
  existieren, so sind die auf den Koenden induzierten Morphismen
  \[ \int (\tau \circ \eta), \int \tau \circ \int \eta:
  \int^c F(c, c) \to \int^c H(c, c) \]
  durch die Transformationen auf den Koenden gegeben und damit
  verträglich: $\int (\tau \circ \eta), \int \tau \circ \int \eta$.
\end{lemma}
\begin{proof}
  Dies folgt sofort aus der Eindetigkeit dieses Morphismus in der
  universellen Definition von Koenden.
\end{proof}

\begin{lemma}[Fubini, \cite{Lore}, 1.9, \cite{ML}, IX, 8]
  Sei $F: C\op \times C \times D\op \times D \to E$ ein Funktor. Dann
  ist
  \[ \int^{c \in C} F(c, c, \cdot, -): D\op \times D \to E \]
  ein Funktor und es gilt
  \[ \int^{d \in D} \int^{c \in C} F(c, c, d, d)
  \iso \int^{(c, d) \in C \times D} F(c, c, d, d), \]
  wobei rechts $F$ als Funktor $F: (C \times D)\op \times (C \times
  D)$ aufgefasst wird.
\end{lemma}
\begin{proof}
  Sei $K: D\op \times D \to E$ ein Kokeil für $F: C\op \times C \to
  [D\op \times D, E]$ in der Funktorkategorie $D\op \times D \to E$
  und $L(K)$ ein Kokeil für $K$. Auch solche iterierten Kokeile $L(K)$
  bestehen aus Morphismen $F(c, c, d, d) \to L(K)$ für alle $c, d$ als
  Komposition von $F(c, c, \cdot, \cdot) \Implies K$ ausgewertet auf
  $(d, d) \in D\op \times D$ mit $K(d, d) \to L(K)$. Sie erfüllen die
  Kommutativitätseigenschaft von Kokeil-Quadraten für alle Morphismen
  der Form $(f, \id_d)$ in $C \times D$ (die Transformation aus der
  Kokeil-Eigenschaft von $K$ ausgewertet in $(d, d) \in D\op \times
  D$) sowie $(\id_c, g)$ in $C \times D$ (durch Vorschalten der zur
  Transfromation $F(c, c, \cdot, \cdot) \Implies K$ gehörigen
  Morphismen vor das Kokeil-Quadrat von $L(K)$).

  Nach denselben Argumenten definiert ein Kokeil $M$ für $F: (C \times
  D)\op \times (C \times D)$ einen iterierten Kokeil: Wir können einen
  iterierten Kokeil ``sparsam'' angeben, indem wir die inneren Kokeile
  nicht für alle $(d, d') \in D\op \times D$, sondern nur für
  Diagonalelemente $(d, d) \in D\op \times D$ sowie
  ``Morphismenobjekte'' $(d', d) \in D\op \times D$ für einen
  Morphismus $f: d \to d'$ angeben. Der iterierte Kokeil ist dadurch
  vollständig festgelegt, da ja nur diese Kokeile in den Daten eines
  iterierten Kokeils eine Rolle spielen. Nun ist $M$ zunächst wegen
  der Kommutativität von
  \[ \begin{tikzcd}
    F(c', c, d, d) \arrow{r} \arrow{d} & F(c, c, d, d) \arrow{d} \\
    F(c', c', d, d) \arrow{r} & M
  \end{tikzcd} \]
  ein $(d, d)$-Kokeil und dann durch Setzen der Inklusionen $F(c, c,
  d', d) \to M$ als Komposition $F(c, c, d', d) \to F(c, c, d', d')
  \to M = F(c, c, d', d) \to F(c, c, d, d) \to M$ wegen der
  Kommutativität von
  \[ \begin{tikzcd}
    F(c', c, d', d) \arrow{rr} \arrow{dd} & & F(c, c, d', d) \arrow{d} \\
    & & F(c, c, d, d) \arrow{d} \\
    F(c', c', d', d) \arrow{r} & F(c', c', d', d') \arrow{r} & M
  \end{tikzcd} \]
  auch ein $(d', d)$-Morphismenkokeil. Klarerweise ist dann auch $M$
  ein iterierter Kokeil über sich selbst als $(d, d)$- und $(d',
  d)$-Kokeil.

  Morphismen von Kokeilen entsprechen nun Morphismen iterierter
  Kokeile und damit entsprechen sich auch Koenden alias die initialen
  Objekte der beiden Kategorien.
\end{proof}
\begin{bem}
  Unser Argument verbessert die Aussage von \cite{ML}, IX.8, geringfügig, da
  wie angedeutet nicht alle $(d', d)$-Koenden existieren müssen.
  
  Ein weniger ad hoc formuliertes Argument würde Kokeile wie in als
  Kolimites über die zugehörige \emph{twisted arrow category}
  (\cite{Lore}, 1.12 f.) bzw. \emph{subdivision category} (\cite{ML},
  IX.5) beschreiben und die Aussage damit auf die für iterierte
  Kolimites und die jeweiligen Pfeilkategorien zurückführen.
\end{bem}

\begin{defn}
  Sei $F: C \to D$ ein Funktor. $F$ heißt \emph{stetig}, falls er mit
  allen Limites vertauscht. $F$ heißt \emph{kostetig}, falls er mit
  allen Kolimites vertauscht.
\end{defn}

\begin{lemma}[Koenden und kostetige Funktoren, \cite{Lore}, 1.16]
  \label{coend-cocont}
  Ist $F: D \to E$ ein kostetiger Funktor und $T: C\op \times C \to D$
  ein Funktor, dann ist der natürliche Morphismus
  \[ F \int^c T(c, c) \iso \int^c F \circ T(c, c), \]
  ein Isomorphismus, wann immer eines der beiden Koenden existiert.
\end{lemma}
\begin{proof}
  Das folgt direkt aus unserer Beschreibung von Koenden als Kolimites
  in \ref{coend-coeq}.
\end{proof}
\begin{bem}
  Das wichtigste Beispiel für diesen Fall ist der
  $\Hom$-Funktor. Praktisch nach Definition von Limes und Kolimes sind
  $D(d, \cdot): D \to \Ens$ und $D(\cdot, d): D\op \to \Ens$
  stetig. Wir erhalten für $F: C\op \times C \to D$:
  \begin{align*}
    D(d, \int_c F(c, c)) & \iso \int_c D(d, F(c, c)) \\
    D(\int^c F(c, c), d) &\iso \int_c D(F(c, c), d).
  \end{align*}
\end{bem}

\begin{lemma}[Transformationen als Ende, \cite{Lore}, 1.18]
  \label{trans-end}
  Seien $F, G: C \to D$ zwei Funktoren. Dann gibt es einen natürlichen
  Isomorphismus
  \[ [C, D](F, G) \iso \int_c D(Fc, Gc). \]
\end{lemma}
\begin{proof}
  Transformationen $F \Implies G$ sind genau diejenigen Tupel aus
  $\prod_c D(Fc, Gc)$, für die für jedes $f: c \to d$ in $C$ die
  Bilder rechts unten im Diagramm übereinstimmen:
  \[ \begin{tikzcd}
    & D(Fc, Gc) \arrow{d}{Gf \circ} \\
    D(Fd, Gd) \arrow{r}{\circ Ff} & D(Fc, Gd)
  \end{tikzcd}. \]
  Damit ist $[C, D](F, G)$ der Egalisator aus der Dualisierung von
  \ref{coend-coeq} und folglich das angegebene Ende.
\end{proof}

\subsection{Angereicherte Kategorien}

In der ursprünglichen Definition einer Kategorie tragen die Morphismen
$C(x, y)$ die Struktur einer Menge. Der Wunsch, Begriffe wie ``die
Summe zweier Morphismen'' oder ``der Nullmorphismus'' aus der
Kategorie der $R$-Moduln auch kategorientheoretisch zu erfassen,
brachte uns zum Begriff der additiven Kategorien, bei denen
Morphismenmengen abelsche Gruppen sind und Funktoren
Gruppenhomomorphismen auf den Morphismengruppen induzieren. Diese
Konstruktion möchten wir auch für Kategorien zur Verfügung stehen
haben, bei denen die Morphismenmengen eine andere Struktur tragen
(etwa die eines topologischen Raums) oder gar keine Mengen sind,
sondern Objekte einer anderen Kategorie $V$. Solche Kategorien
bezeichnet man als über $V$ angereicherte (engl. \emph{enriched})
Kategorien. Wir imitieren die herkömmliche Definition von Kategorien:
\begin{defn}[Angereicherte Kategorie]
  Sei $V$ eine monoidale Kategorie (Schmelzkategorie mit universellen
  Verschmelzungen). Eine über $V$ angereicherte Kategorie $C$
  ($V$-Kategorie) besteht aus:
  \begin{enumerate}
  \item einer Klasse von Objekten $\Ob(C)$
  \item für jedes Paar von Objekten $(x, y)$ ein Objekt $C(x,y) \in V$
  \item einer Komposition $C(x, y) \otimes C(y, z) \to C(x, z)$
  \item für jedes Objekt $x \in \Ob(C)$ einem Morphismus in $V$
    $\id_x: I \to C(x, x)$ für $I$ die universelle 0-Verschmelzung,
  \end{enumerate}
  sodass die Komposition assoziativ ist (bezüglich des eindeutigen
  Morphismus $C(w,x) \otimes (C(x,y) \otimes C(y,z)) \iso (C(w,x)
  \otimes C(x,y)) \otimes C(y,z)$) und verträglich mit den Identitäten
  (bezüglich der eindeutigen Morphismen $C(x, y) \otimes I \iso C(x,
  y)$ und $I \otimes C(x,y) \iso C(x,y)$).

  Ein Funktor zwischen $V$-Kategorien $F: C \to D$ ist eine Zuordnung
  auf Objekten $F: c \mapsto Fc$ zusammen mit Morphismen $C(x, y) \to
  D(Fx, Fy)$ in $V$.
\end{defn}
\begin{bsp}
  Eine Kategorie ist eine über $\Ens$ angereicherte Kategorie
  bezüglich der kartesischen monoidalen Struktur auf $\Ens$. Eine
  Kategorie mit additiver Struktur ist eine über $\Ab$ angereicherte
  Kategorie bezüglich des Tensorprodukts auf $\Ab$. Jede monoidale
  Kategorie mit adjungiertem internem Hom (monoidale geschlossene
  Kategorie) $V$ ist eine $V$-Kategorie über sich selbst.
\end{bsp}

In einer monoidalen geschlossenen Kategorie $V$ haben wir die Adjunktion
\[ V(a \otimes b, c) \iso V(a, V(b, c)) \]
mit $V(b, c) = (b \Implies c)$ dem internen Hom-Objekt. Das Objekt $a
\otimes b$ ist dann ein darstellendes Objekt für den Funktor $c
\mapsto V(a, V(b, c))$ und heißt manchmal auch Koexponential
(engl. \emph{copower}) von $a$ und $b$. Wir verallgemeinern diese
Situationen auf Koexponentiale in angereicherten Kategorien:
\begin{defn} \label{copower}
  Sei $C$ eine $V$-Kategorie. Ist $x \in C$ ein darstellendes Objekt
  für den Funktor $c \mapsto V(v, C(b, c))$ mit $v \in V, b \in C$,
  gilt also
  \[ C(x, c) \iso V(v, C(b, c)) \]
  natürlich in $c \in C$, so heißt $x = v \odot b$ das Koexponential
  von $v$ und $b$.  
\end{defn}
In diesem Fall nennt man die Kategorie $C$ auch \emph{tensoriert über
  $V$}.
\begin{bsp}
  Jede Kategorie mit beliebigen Koprodukten hat Koexponentiale über
  $\Ens$. Mit $V \odot B = \coprod_V B$ gilt:
  \[ C(V \odot B, C) \iso \Ens(V, C(B, C)). \]
\end{bsp}

Wir können nun auch die geometrische Realisierung in der Sprache
angereicherter Kategorien formulieren. Ist $C$ eine $V$-Kategorie mit
Koexponentialen und sind Funktoren
\begin{align*}
  &R: \Delta &\to C \qquad{und} \\
  &X: \Delta\op &\to V
\end{align*}
gegeben, so erklären wir die geometrische Realisierung von $X$ als das
Koende über den Funktor
\begin{align*}
  \Delta\op \times S &\to C, \\
  [n], [m] &\mapsto X[n] \odot R[m].
\end{align*}
Der Fall simplizialer Mengen ist $C = \Top$ und $V = \Ens$ mit dem
kanonischem Koexponential.

\end{document}
