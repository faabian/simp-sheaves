% Emacs mode: -*-latex-*-

\chapter{Schwach konstruierbare Garben auf simplizialen Mengen}
\chaptermark{Schwach konstruierbare Garben}
\label{ch:simp-set-sk}

In diesem Kapitel möchten wir die Aussagen aus
\autoref{sec:simp-comp-sheaf} und \autoref{sec:simp-comp-sk}
übertragen auf den Fall, dass es sich bei dem Basisraum um die
Realisierung einer simplizialen Menge anstelle eines
Simplizialkomplexes handelt. Simplizialkomplexe sind halbgeordnete
Mengen und unsere Technik verwendete die Ordnungstopologie
halbgeordneter Mengen. Simplizial konstante Garben auf simplizialen
Mengen entsprechen hingegen Diagrammen von Mengen, in denen es auch
mehrere Pfeile zwischen zwei zu nichtdegenerierten Simplizes gehörigen
Punkten geben kann. Solche kategoriellen Realisierungen simplizialer
Mengen werden wir in diesem Abschnitt konstruieren.

Konkret ist für $X \in \s\Ens$ eine simpliziale Menge eine Kategorie
$D_X$ gesucht, für die es eine Äquivalenz von Kategorien
\[ \Ens_{/|X|}^{\sk} \qiso {[D_X\op, \Ens]} \]
gibt. Dabei steht $\Ens_{/|X|}^{\sk}$ für die \emph{simplizial
  konstanten Garben} auf $|X|$, d. h. Garben, deren Einschränkungen
auf die Zellen der geometrischen Realisierung aus \ref{cw-cells}
konstant sind. Wir diskutieren die in Frage kommenden Kategorien $D_X$
an einem einfachen Beispiel. Betrachte die Standarddarstellung von
$S^1$ als simpliziale Menge $X$ mit einem nichtdegenerierten
$1$-Simplex und einem nichtdegenerierten $0$-Simplex aus
\ref{real-sphere}. Als kategorielle Realisierung $D_X$ von $X$ können
folgende Diagramme in Frage kommen:
\begin{enumerate}
\item Das Diagramm $D_X = (\bullet \to \bullet)$. Es handelt sich um
  das Diagramm der nichtdegenerierten Simplizes von $X$ mit der
  Angabe, welche Simplizes im Abschluss welcher Simplizes
  liegen. Dieses Diagramm ist zu grob, um simplizial konstanten Garben
  auf $S^1$ zu entsprechen, wie die Realisierung von $D_X$ mit der
  Ordnungstopologie und dann die erste Garbenkohomologie zeigt. In
  \autoref{sec:real-poset} werden wir diese kategorielle Realisierung
  mit der geometrischen Realisierung durch ordnungstopologische
  Simplizes in Beziehung setzen.
\item Das Diagramm $D_X = \slicecat{\Delta}{r}{X}$. Es handelt sich um
  das Diagramm aller Simplizes von $X$ mit der Angabe von Rändern und
  Degenerationen. Diese Diagramm ist zu fein, um von einer simplizial
  konstanten Garbe auf $S^1$ eindeutig bestimmt zu werden, denn diese
  trägt keine Informationen über die degenerierten Simplizes. Gleich
  im Anschluss werden wir diese kategorielle Realisierung mit den noch
  nicht realisierten simplizialen Objekten in $\EnsssTop$ mit
  diskreten Basisräumen $X$ in Beziehung setzen.
\item Das Diagramm $D_X = (\bullet \rightrightarrows \bullet)$. Dieses
  Diagramm würden wir anschaulich erwarten. Es handelt sich um die
  nichtdegenerierten Simplizes von $X$ mit der Angabe von Rändern,
  nicht aber von Degenerationen. Im Abschitt
  \ref{sec:real-directed-cat} zeigen wir die versprochene Aussage,
  dass Prägarben auf diesem Diagramm simplizial konstanten Garben auf
  $|X|$ entsprechen.
\end{enumerate}

Während wir für die anderen beiden kategoriellen Realisierungen mehr
ausholen müssen, zeigen wir die Interpretation der Prägarben auf der
Simplexkategorie von $X$ direkt.

Betrachte für eine feste simpliziale Menge $X \in \s\Ens$ die volle
Unterkategorie $(\s\EnsssTop)_X \subset \s\EnsssTop$ der simplizialen
Garben über topologischen Räumen mit Komorphismen, für die die
Basisräume diskret und als simplizialer topologischer Raum isomorph
zur simplizialen Menge $X$ sind. Für eine Garbe $F \in
(\s\EnsssTop)_X$ und $\sigma \in X_n$ einen $n$-Simplex schreiben wir
$F(\sigma)$ für die Menge $(F_n)_\sigma$, den Halm bei $\sigma$. Für
$f: [n] \to [m]$ monoton bestehen die Komorphismen $Ff^* F_n \to F_m$
aus Abbildungen $F(f(\sigma)) \to F(\sigma)$ für $\sigma \in X_m$. Da
die Simplexkategorie $\slicecat{\Delta}{r}{X}$ von $X$ gerade aus
allen Simplizes $\sigma \in \coprod_n X_n$ besteht mit Morphismen
$\sigma \to f(\sigma)$ für alle $f: [n] \to [m]$ und $\sigma \in X_m$,
haben wir gezeigt:
\begin{prop} \label{real-simplex-cat}
  Die obige Zuordnung liefert für $X \in \s\Ens$ eine simpliziale
  Menge eine Äquivalenz von Kategorien
  \[ [\slicecat{\Delta}{r}{X}\op, \Ens] \qiso (\s\EnsssTop)_X. \]
\end{prop}

\section{Kategorielle Realisierungen}
\label{sec:real-cat}

In diesem Abschnitt geben wir die kategoriellen Realisierungen einer
simplizialen Menge als gerichtete Kategorie bzw. als halbgeordnete
Menge an.

Wir benötigen die Begriffe für nichtdegenerierte Simplizes
(vgl. \ref{def:delta}, \ref{def:comb-standard-simplex}).
\begin{defn}
  Die Unterkategorie der endlichen nichtleeren Ordinalzahlen mit
  injektiven monotonen Abbildungen $\Delta^+ \subset \Delta$ heißt
  \emph{nichtdegenerierte Simplexkategorie}.
\end{defn}
Wir wiederholen die Begriffe für simpliziale Mengen für Prägarben auf
$\Delta^+$.
\begin{defn}
  Die darstellbare Prägarbe auf $\Delta^+$
  \[ \Delta^{+n} :=  \Delta^+(\cdot, [n]) \]
  heißt der \emph{nichtdegenerierte Standard-$n$-Simplex}.
\end{defn}
Diese Zuordnung liefert einen Funktor $r: \Delta^+ \to [{\Delta^+}\op,
  \Ens]$. Wir erhalten unsere für die kategorielle Realisierung
gewünschte kosimpliziale Kategorie durch den Funktor der
nichtdegenerierten Simplizes des nichtdegenerierten
Standard-$n$-Simplex. Bezeichne dazu $\iota: \Delta^+ \inj \Delta$ den
Inklusionsfunktor und
\[ \iota^*: [\Delta\op, \Ens] \to [{\Delta^+}\op, \Ens] \]
den Rückzugsfunktor auf Prägarben.
\begin{defn}
  Der \emph{Stufenfunktor} ist der Funktor
  \[N: \slicecat{\Delta^+}{r}{\iota^* \Delta^n}
  \to \slicecat{\Delta^+}{r}{\Delta^{+n}},
  \]
  gegeben durch das kommutative Quadrat
  \[ 
  \begin{tikzcd}
    f: \Delta^{+m} \to \iota^* \Delta^n \dar[mapsto]{\sim} \rar[mapsto]
    & N(f):  \Delta^{+k}  \to \iota^* \Delta^{+n} \dar[mapsto]{\sim} \\
    f \in \Delta([m], [n]) \rar[mapsto]
    & N(f) \in \Delta^+([k], [n]),
  \end{tikzcd}
  \]
  in dem die Vertikalen die eindeutigen Zuordnungen aus dem
  Yoneda-Lemma sind und die untere Horizontale die Abbildung, die eine
  monotone Abbildung $f$ auf die eindeutige monotone Injektion $N(f)$
  aus \ref{degen-repr} mit demselben Bild (und anderem
  Definitionsbereich $[k]$) schickt.
\end{defn}
\begin{bem}
  Der Name ``Stufenfunktor'' rührt daher, dass die Werte einer
  monotonen Funktion die Stufen in ihrem Graphen beschreiben.
\end{bem}
Das Vorschalten von monotonen Injektionen vor $f \in \Delta([m], [n])$
(Morphismen in $\slicecat{\Delta^+}{r}{\iota^* \Delta^n}$) induziert
auf der zugehörigen monotonen Injektion $\hat{f} \in \Delta^+([k],
[n])$ ebenfalls Morphismen durch Vorschalten von Injektionen, denn das
Einschränken von Funktionen auf Teilmengen verkleinert auch die
Bildmengen. Dies zeigt die Funktorialität.
\begin{prop}
  Die Zuordnung
  \[ \begin{tikzcd}
    {[n]} \arrow{dd}{f} \rar[mapsto]
    & \slicecat{\Delta^+}{r}{\Delta^{+n}} \dar{f \circ} \\
    & \slicecat{\Delta^+}{r}{\iota^* \Delta^m} \dar{N} \\
    {[m]} \rar[mapsto]
    & \slicecat{\Delta^+}{r}{\Delta^{+m}}
  \end{tikzcd} \]
  ist ein Funktor $R: \Delta \to \Cat$, genannt die
  \emph{kosimpliziale Standard-Kategorie}.
\end{prop}
\begin{bem}
  Es handelt sich um die opponierten Standard"=$n$"=Simplizialkomplexe
  aufgefasst als Kategorien. Die Konstruktion über die
  nichtdegenerierten Simplizes des Standard-Simplex liefert die
  benötigten Rand- und Degenerationsabbildungen auf natürliche Weise
  mit.
\end{bem}
\begin{proof}
  Wir müssen zeigen, dass für monotone $f: [m] \to [n]$ und $g: [l]
  \to [m]$ gilt
  \[ N ((f \circ g)\circ) = N (f \circ) \: N (g \circ) \]
  für $(f \circ)$ den Nachschaltefunktor und $N$ den
  Stufenfunktor. Das folgt aber daraus, dass beide Funktoren eine
  monotone Injektion $h: [k] \to [l]$ auf die Injektion auf $\im (f
  \circ g \circ h)$ schicken. Diese Entsprechnung ist verträglich mit
  Einschränkungen von $h$ (Vorschalten von monotonen Injektionen), ist
  also eine Transformation.

  Es handelt sich bei den Kategorien
  $\slicecat{\Delta^+}{r}{\Delta^{+n}}$ um ``gerichtete Kategorien'',
  bei denen es keine Kreise von Pfeilen außer den Identitäten gibt,
  denn die nichttrivialen Morphismen sind das Vorschalten von echten
  Injektionen und senken somit den Grad eines Simplex.
\end{proof}

\begin{prop} \label{cat-cocomplete}
  Die Kategorie der kleinen Kategorien $\Cat$ ist kovollständig.
\end{prop}
\begin{proof}
  Koprodukte in $\Cat$ sind die Koprodukte der zugrundeliegenden
  Köcher, d. h. die disjunkte Vereinigung über die Objektmengen und
  aus den Ausgangskategorien übernommene Morphismenmengen.

  Die Koegalisatoren in $\Cat$ sind schwieriger, vergleiche
  \cite{BBP}. Wir geben die Konstruktion kurz an. Betrachte kleine
  Kategorien mit Funktoren $A
  \mathrel{\mathop{\rightrightarrows}^{F}_{G}} B$. Die dem
  Koegalisator $B \to C$ zugrundeliegende mengentheoretische Abbildung
  ist der Koegalisator der den Funktoren $F$ und $G$ zugrundeliegenden
  mengentheoretischen Abbildungen. Durch diese Identifikationen in $C$
  werden Morphismen neu komponierbar, deren Kompositionen den durch
  die Identifikationen verschmolzenen Morphismenmengen hinzugefügt
  werden. Weiter müssen wie im folgenden Diagramm Morphismen $Ff$ und
  $Gf$ identifiziert werden, was auf die Kompositionen fortgesetzt
  wird.
  \[ 
  \begin{tikzcd}
    a \ar{dd}{f}
    & Fa \rar[phantom]{\sim} \ar{dd}{Ff} & Ga \ar{dd}{Gf}  \\[-10pt]
    \rar[mapsto, shorten <= 2em, xshift = -1em]
    & \rar[phantom]{\sim} & \lar[phantom] \\[-10pt]
    b & Fb \rar[phantom]{\sim} & Gb
  \end{tikzcd}
  \]  
  Die Identifikationen $Fa \sim Ga$ für $a \in A$ und $Ff \sim Gf$ für
  $f \in A(a, b)$ sind notwendig für einen Koegalisator $B \to C$,
  damit $A \mathrel{\mathop{\rightrightarrows}^{F}_{G}} B \to C$
  übereinstimmen. Die weiteren Schritte machen ``minimalinvasiv'' $B$
  mit diesen Identifikationen wieder zu einer Kategorie.
\end{proof}
\begin{bem}
  Der Ansatz, Limites und Kolimites in $Cat$ mittels
  \ref{refl-sub-complete} über die Einbettung $\Cat \subset
  \mathrm{Quiv}$ in die Kategorie der Köcher zu konstruieren,
  funktioniert \emph{nicht}. Jene ist als Prägarbenkategorie
  tatsächlich vollständig und kovollständig und die Inklusion hat mit
  der freien Pfadkategorie über einem Köcher tatsächlich einen
  Linksadjungierten; allerdings handelt es sich nicht um eine volle
  (dann also reflektive) Unterkategorie, weshalb die Limites und
  Kolimites von denen in $\mathrm{Quiv}$ bzw. ihren freien
  Pfadkategorien abweichen.
\end{bem}
\begin{defn}
  Die \emph{kategorielle Realisierung} einer simplizialen Menge $X$
  ist definiert als das Tensorprodukt von Funktoren $X \otimes R \in
  \Cat$ für $R$ die kosimpliziale Standard-Kategorie und die
  natürliche $\Ens$-tensorierte Struktur auf $\Cat$
  (\ref{ens-tensored}).
\end{defn}
\begin{bsp} \label{ex:s1-diagram}
  Betrachte die Standarddarstellung von $S^1$ als simpliziale Menge
  $X$ mit einem nichtdegenerierten $1$-Simplex und einem
  nichtdegenerierten $0$-Simplex aus \ref{real-sphere}. Wie
  angekündigt ist die kategorielle Realisierung von $X$ das Diagramm
  \[ \bullet \rightrightarrows \bullet \]
  mit zwei Objekten und zwei parallelen Pfeilen dazwischen, sowie
  nicht eingezeichneten Identitäten.
\end{bsp}

Hieraus lässt sich in einem zweiten Schritt die andere diskutierte
kategorielle Realisierung als halbgeordnete Menge konstruieren. Wir
erhalten die kosimpliziale halbgeordnete Menge $P: \Delta \to \poset$
aus $R: \Delta \to \Cat$ durch Anwenden eines Reflektors $\Cat \to
\poset$.

\begin{prop} \label{poset-cocomplete}
  Die Kategorie $\poset$ der halbgeordneten Mengen ist kovollständig.
\end{prop}
\begin{proof}
  Die halbgeordneten Mengen sind eine volle Unterkategorie $\poset
  \subset \Cat$. Wir können daher \ref{refl-sub-complete} verwenden,
  mit dem Reflektor $\Cat \to \poset$, der im folgenden Lemma
  konstruiert wird.
\end{proof}
\begin{defn}
  Eine Kategorie heißt \emph{dünn}, falls jede Morphismenmenge
  höchs\-tens einelementig ist. Eine Kategorie heißt
  \emph{Skelettkategorie}, falls in ihr jeder Isomorphismus eine
  Identität ist.
\end{defn}
\begin{lemma} \label{poset-reflective}
  Die vollen Unterkategorien $\mathrm{thinCat}, \mathrm{skelCat}
  \subset \Cat$ der dünnen bzw. Skelettkategorien sind reflektiv. Der
  Reflektor $\Cat \to \mathrm{skelCat}$ macht aus dünnen Kategorien
  dünne Kategorien und liefert durch Komposition mit dem Reflektor
  $\Cat \to \mathrm{thinCat}$ einen Reflektor $\pos: \Cat \to \poset$.
\end{lemma}
\begin{proof}
  Der Linksadjungierte zu $\mathrm{thinCat} \inj \Cat$ ist gegeben
  durch die Identifikation aller nichtleeren Morphismenmengen zu
  einelementigen Morphismenmengen. Der Linksadjungierte zu
  $\mathrm{skelCat} \inj \Cat$ ist die zu einer kleinen Kategorie mit
  dem Auswahlaxiom konstruierte Unterkategorie, die Isomorphieklassen
  von Objekten durch ein Objekt aus diesen ersetzt. Klar ist, dass
  Funktoren $F: C \to D$ in eine dünne Kategorie $D$ Abbildungen auf
  Objekten sind mit der Zusatzeigenschaft, dass es einen Morphismus
  $Ff: Fx \to Fy$ in $D$ geben muss, wann immer es einen Morphismus
  $f: x \to y$ in $C$ gibt. Das ist unerheblich davon, wie viele
  Morphismen $x \to y$ es in $C$ gibt und zeigt die erste
  Adjunktion. Ein Funktor in eine Skelettkategorie schickt isomorphe
  Objekte auf dasselbe Objekt, wird also schon durch das Bild eines
  Objekts jeder Isomorphieklasse eindeutig festgelegt. Dies zeigt die
  zweite Adjunktion.

  Der Reflektor $\Cat \to \mathrm{skelCat}$ ordnet einer Kategorie
  eine Unterkategorie zu und erhält deshalb Dünnheit. Halbgeordnete
  Mengen sind nach Definition dünne Skelettkategorien.
\end{proof}

Bezeichne nun $P = \pos R: \Delta \to \poset$ die kosimpliziale
halbgeordnete Menge zur kosimplizialen gerichteten Kategorie $R$. Die
halbgeordneten Mengen
\[ P[n] = \pos \slicecat{\Delta^+}{r}{\Delta^{+n}}
\]
sind dann die opponierten Standard-$n$-Simplizialkomplexe.

\section{Realisierung als halbgeordnete Menge}
\label{sec:real-poset}

Der kombinatorische topologische Raum $\blacktriangle X$ einer
simplizialen Menge eignet sich \emph{nicht} für die Übertragung der
Aussagen zu schwach konstruierbaren Garben auf Simplizialkomplexen auf
die Situation simplizialer Mengen, denn diese geometrische
Realisierung sieht nicht mehrfache Verklebungsabbildungen. Dies
möchten wir präzise machen.

\begin{prop} \label{clumsy-order-top}
  Sei $X \in \s\Ens$ eine simpliziale Menge. Dann gibt es einen
  Homöomorphismus $\blacktriangle X \iso \Ord((X \otimes P)\op)$.
\end{prop}
\begin{proof}
  Der Funktor $\Ord: \poset \to \Top$ ist nach dem nachgestellten
  Lemma kostetig. Daher reicht es mit \ref{coend-cocont} (und der
  Kostetigkeit des Opponierens), einen Isomorphismus kosimplizialer
  topologischer Räume $\blacktriangle \to \Ord P\op$ zu finden. Beide
  bestehen aus einem Punkt pro nichtdegeneriertem Simplex
  (\ref{cw-cells}) von $\Delta^n$ und haben als offene Mengen nach
  oben abgeschlossene Mengen. Randabbildungen $d_i$ sind Inklusionen
  in die Ränder, Degenerationen Kollapse von Kanten: nach Definition
  schickt $Ps_i: P[n] \to P[n-1]$ einen nichtdegenerierten Simplex $f:
  [m] \to [n]$ monoton und injektiv auf $N(s_i \circ f)$, den
  nichtdegenerierten Simplex, der zum Kollaps von $i$ und $i+1$ in $f$
  gehört.
\end{proof}
\begin{lemma}
  Der Funktor $\Ord: \poset \to \Top$, der eine halbgeordneten Menge
  mit der Ordnungstopologie versieht, ist kostetig.
\end{lemma}
\begin{proof}
    Klar ist, dass $\Ord$ mit Koprodukten vertauscht. Sei nun $A
    \mathrel{\mathop{\rightrightarrows}^{F}_{G}} B \to C$ ein
    Koegalisator in den halbgeordneten Mengen. Dann ist nach
    \ref{cat-cocomplete} und \ref{poset-reflective} die
    zugrundeliegende mengentheoretische Abbildung von $q: B \to C$ der
    mengentheoretische Koegalisator: nur der Reflektor $\Cat \to
    \mathrm{skelCat}$ könnte die zugrundeliegende Menge ändern, wird
    aber bereits auf eine Skelettkategorie angewandt, denn ein
    Kategorienkolimes über halbgeordnete Mengen enthält keine
    Morphismen in entgegengesetzte Richtungen. Wir müssen noch zeigen,
    dass $\Ord(q): \Ord B \to \Ord C$ final ist. Ist $U \subset C$
    eine Menge mit offenem Urbild $q^{-1}(C)$, so ist ein Morphismus
    $x \to y$ in $C$ mit $x \in U$ ein Pfad
    \[ x = v_0 \to v_1 \sim w_1 \to w_2 \sim v_2 \to \cdots \to y \]
    bestehend aus Morphismen in $B$, die sich nach den Identifaktionen
    durch $q$ verknüpfen lassen. Induktiv liegen nun nach der
    Abgeschlossenheit nach oben von $q^{-1}(U)$ alle $v_i$, $w_i$ in
    $q^{-1}(U)$ und somit auch $y$. Es folgt die Offenheit von $U$.
\end{proof}

Mit \ref{sheaf-order-top} und \ref{clumsy-order-top} erhalten wir
sofort die folgende kombinatorische Charakterisierung von Garben auf
der ordnungstopologischen Realisierung:
\begin{prop} \label{clumsy-comb}
  Sei $X \in \s\Ens$ eine simpliziale Menge. Dann gibt es eine
  Äquivalenz von Kategorien $\Ens_{/\blacktriangle X} \qiso [(X
    \otimes P)\op, \Ens]$.
\end{prop}

\section{Realisierung als gerichtete Kategorie}
\label{sec:real-directed-cat}

Stattdessen lässt sich die Äquivalenz von simplizial konstanten Garben
auf einer simplizialen Menge $X$ zu Diagrammen von Mengen für die
kategorielle Realisierung $X \otimes R$ durch gerichtete Kategorien
formulieren.

Als zweite Zutat für diese Aussage benötigen wir eine Beschreibung von
Garben auf Kolimites. Die Argumentation ist recht einfach, sofern die
richtigen Begriffe von Limites und Kolimites von großen Kategorien zur
Verfügung stehen. Problematisch dabei ist, dass wir selten
tatsächliche Gleichheit von Funktoren und Isomorphismen von Kategorien
haben (und benötigen), sondern Isotransformationen von Funktoren und
Äquivalenzen von Kategorien. Ein Limes-Begriff für Kategorien muss
insofern die \emph{2-kategorielle Struktur} der ``Kategorie'' der
Kategorien berücksichtigen. Wir erhalten die richtigen Formulierungen
wie in \cite{nlab:2-limit} erklärt anhand des Übersetzungsschemas:
\renewcommand{\arraystretch}{1.5}
\begin{center}
    \begin{tabulary}{\textwidth}{RCL}
    1-Kategorie & $\leftrightarrow$
    & 2-Kategorie \\
    Gleichheit von Morphismen & $\leftrightarrow$
    & 2-Isomorphismus von Morphismen (Isotransformation) \\
    kommutieren & $\leftrightarrow$
    & kommutieren bis auf 2-Isomorphismus \\
    Isomorphismus & $\leftrightarrow$
    & Quasi-Isomorphismus (Äquivalenz von Kategorien)  
  \end{tabulary}
\end{center}
Ein 2-Limes über ein über $I$ indiziertes System von Kategorien $C_i$
ist also etwa eine Kategorie $C$, für die es für jede Testkategorie
$T$ eine Äquivalenz von Kategorien gibt zwischen der Funktorkategorie
$[T, C]$ und der Kategorie bestehend aus (großen) Tupeln von Funktoren
$F_i \in [T, C_i]$ für alle $i \in I$, für die es Isotransformationen
$Cf \circ F_i \Isotrafo F_j$ für alle $f: i \to j$ in $I$ gibt. Mit
dieser Definition folgt automatisch die Exaktheit des Kategorien-Homs:
\begin{align*}
  [C, \lim_i D_i] &\qiso \lim_i [C, D_i], \\
  [\col_i C_i, D] &\qiso \lim_i [C_i, D].
\end{align*}
2-Limites von Kategorien sind dann eindeutig bis auf Isomorphismus im
2-Kategorie-Sinne, d. h. bis auf Äquivalenz von Kategorien.

Nun können wir die Beschreibung von Garben auf Kolimites formulieren.
\begin{satz} \label{sheaf-col}
  Sei $X = \col_i X_i$ ein Kolimes topologischer Räume. Dann ist der
  Limes von Funktoren
  \[ \kappa = \lim_i \ins_i^*: \EnsX \to \lim_i \ins_i^* \EnsX \]
  eine Äquivalenz von Kategorien. Dabei ist $\ins_i^* \EnsX$ das
  wesentliche Bild des Rückzugs entlang den Abbildungen des Kolimes
  $\ins_i: X_i \to X$ und die Systemmorphismen sind das Nachschalten
  von $f^*: \Ens_{/X_j} \to \Ens_{/X_i}$ für $f: i \to j$ in $I$.
\end{satz}
\begin{proof}
  Ist $X = \coprod_i X_i$ ein Koprodukt (d. h. $I$ diskret), so sind
  die $\ins_i^*$ Restriktionen, das Bild $\ins_i^* \EnsX \subset
  \Ens_{/X_i}$ ist dicht und die Äquivalenz ist die bekannte Aussage
  $\EnsX \qiso \prod_i \Ens_{/X_i}$.

  Im Fall eines Koegalisators
  \[ Z \overset{f}{\underset{g}{\rightrightarrows}} Y \xrightarrow{q} X \]
  besteht eine Garbe $G$ in der Limeskategorie aus einer Garbe $G$ auf
  $Y$, einem Isomorphismus $G \iso q^* F$ für $F$ eine Garbe auf $X$
  und einem Isomorphismus $\tau: f^* G \iso g^* G$ von Garben auf
  $Z$. Wir behaupten, dass ein Quasiinverser zu $\kappa$ durch eine
  Abwandlung $q_*$ des direkten Bilds gegeben ist mit
  \[ (q_* G)(U) := \set{s \in G(q^{-1}(U))}{\tau_x(s_{f(x)}) = s_{g(x)}
    \text{ für alle } x \in (q \circ f)^{-1}(U) }
  \]
  und der Abbildung $\tau_x$ definiert durch das kommutative Quadrat
  \[
  \begin{tikzcd}
    (f^* G)_x \dar{\sim} \rar{\textstyle\tau} & (g^* G)_x \dar{\sim} \\
    G_{f(x)} \rar{\textstyle\tau_x} & G_{g(x)}.
  \end{tikzcd}
  \]
  Der Isomorphismus $q_* \, \kappa \, F \iso F$ folgt nun mit der
  Finalität und Surjektivität von $q$ ähnlich wie im Beweis der
  Aussage zum finalen Rückzug mit zusammenhängenden Fasern
  (\ref{final-pullback}). Dort benötigten wir für $x \in X$ die
  Aussage, dass die Verknüpfung $U \to \kappa \,F \to F$ eines
  Schnitts über $U$ mit der Abbildung des Rückzugs $\kappa \, F \to F$
  über $X$ (mengentheoretisch) faktorisiert, also konstant ist in der
  Einschränkung auf die Faser $q^{-1}(x) \to F_x$. Ist nun $f(x) \sim
  g(x)$ eine Identifikation in der Faser, so stellt unsere
  Konstruktion sicher, dass ein $q_*$-Schnitt der Limesgarbe $\kappa
  \, F$ gerade die von $F_{q(f(x))} = F_{q(g(x))}$ herrührenden
  Identifikationen berücksichtigen muss und wieder über $X$
  faktorisiert.

  Der umgekehrte Isomorphismus $\kappa \, q_* \, G \iso G$ folgt aus
  diesem, da die Limeskategorie gerade das wesentliche Bild von
  $\kappa$ ist.
\end{proof}
\begin{bem}
  Entscheidend in diesem Beweis ist, dass die Isomorpismen $\tau$ zu
  den Daten von Limesgarben dazugehören. Im Fall von $X = S^1$ ist
  dies die Verklebung, die bestimmt, wie aus einer Garbe auf dem
  Einheitsintervall eine Garbe auf der Kreislinie gemacht wird.
\end{bem}
Diese Äquivalenz schränkt ein zu einer Äquivalenz der vollen
Unterkategorien simplizial konstanter Garben.
\begin{prop} \label{sheaf-col-sk}
 Für eine simpliziale Menge $X \in \s\Ens$ gilt:
  \[ \Ens_{/|X|}^{\sk} \qiso \lim_{\slicecat{\Delta}{r}{X}} \Ens_{/|\Delta^n|}^{\sk}. \]
\end{prop}
\begin{proof}
  Bei den simplizial konstanten Garben auf den Simplizes $\Delta^n \to
  X$ handelt es sich um die wesentlichen Bilder des Rückzugs entlang
  der Einbettungen $|\Delta^n| \to |X|$. Das folgt daraus, dass unter
  diesen Abbildungen das Urbild einer Zelle eine Vereinigung von
  Zellen ist.
\end{proof}
\begin{theorem} \label{sheaf-sset-sk}
  Sei $X \in \s\Ens$ eine simpliziale Menge. Dann gibt es eine
  Äquivalenz von Kategorien zwischen Prägarben auf der kategoriellen
  Realisierung von $X$ und simplizial konstanten Garben auf $|X|$:
  \[ [(X \otimes R)\op, \Ens] \qiso \Ens^{\sk}_{/|X|} .\]
\end{theorem}
\begin{proof}
  Wir nutzen, dass nach \ref{sheaf-col-sk} beide Seiten mit dem
  Kolimes
  \[ |X| \iso \col_{\slicecat{\Delta}{r}{X}} |\Delta^n| \]
  vertauschen:
  \begin{align*}
    [(X \otimes R)\op, \Ens]
    &\qiso [\col_{\slicecat{\Delta}{r}{X}} (\Delta^n \otimes R)\op, \Ens] \\
    &\qiso \lim_{\slicecat{\Delta}{r}{X}} [(\Delta^n \otimes R)\op, \Ens] \\
    &\qiso \lim_{\slicecat{\Delta}{r}{X}} [(\Delta^n \otimes P)\op, \Ens] \\
    &\qiso \lim_{\slicecat{\Delta}{r}{X}} \Ens_{/\Ord(\Delta^n)} \\
    &\qiso \lim_{\slicecat{\Delta}{r}{X}} \Ens_{/|\Delta^n|}^{\sk} \\
    &\qiso \Ens_{/|X|}^{\sk}.    
  \end{align*}
  Dabei verwendet der zweite Schritt die Stetigkeit des
  Kategorien-Homs, der dritte den Umstand, dass es in $\Delta^n$ keine
  Mehrfachkanten gibt, der vierte \ref{clumsy-comb} und der fünfte
  \ref{sk-char}.

  Beachte, dass der auftretende Kolimes zunächst ein (starker) Kolimes
  in der 1-Kategorie der kleinen Kategorien ist und mit dem
  (schwachen) 2-Kolimes für den zweiten Schritt übereinstimmt, weil es
  sich bei den beteiligten Kategorien um Skelettkategorien handelt.
\end{proof}
\begin{bem}
  Folgt man den Äquivalenzen aus dem Satz, so entsprechen offene
  Mengen in $|X|$ nach oben abgeschlossenen vollen Untersystemen $U
  \subset X \otimes R$ und Schnitte über offene Mengen Elementen des
  Limes $\lim_{\sigma \in U} F(\sigma)$ für $F$ eine Prägarbe auf der
  kategoriellen Realisierung von $X$. Schnitte und Vereinigungen
  offener Mengen entsprechen den Schnitten und Vereinigungen der
  zugehörigen nach oben abgeschlossenen Untersysteme und die
  Garbenbedingung ist in dieser Formulierung der Umstand, dass (auf
  Schnitten) kompatible Tupel kompatibler Tupel wieder kompatible
  Tupel auf der Vereinigung bilden.
\end{bem}
\begin{bem}
  Ein erster Ansatz für die Beschreibung von Garben auf Kolimites
  führt über ihre Beschreibung als gewisse volle Unterkategorie der
  Funktorkategorie $[\OffX\op, \Ens]$. Eine halbgeordnete Menge mit
  beliebigen Suprema (Vereinigungen), endlichen Infima (Schnitten) und
  einem Distributivgesetz heißt auch \emph{Locale}. Diese bilden mit
  Locale-Morphismen, opponierten Morphismen halbgeordneter Mengen, die
  Suprema und endliche Infima erhalten, eine Kategorie $\Loc$. Man
  könnte versuchen, den Funktor $\Off: \Top \to \Loc$ in Beziehung zu
  Kolimites zu setzen. Überraschend ist dabei, dass für Koprodukte die
  zugöhrige Locale erzeugt ist vom Koprodukt der Locales, für
  Koegalisatoren die zugehörige Locale jedoch der \emph{Egalisator}
  der zugehörigen Locales ist.
\end{bem}

Es schließt sich an diese Beschreibung simplizial konstanter Garben
auf simplizialen Mengen eine Reihe an Fragen an, die hier nurmehr
angerissen werden können.
\begin{enumerate}
\item Gibt es einen Rechtsadjungierten $p_*$ zur Verknüpfung
  \[p^*: [(X \otimes R)\op, \Ens] \qiso \Ens^{\sk}_{/|X|} \inj \Ens_{/|X|} \]
  der Äquivalenz mit der Einbettung?
\item Kann mit $p_*$ in Analogie zu \ref{dersk-eq}
  \[ \Der([(X \otimes R)\op, \Ab]) \qiso \Der^{\sk}(\Ab_{/|X|}) \]
  formuliert werden für $\Der^{\sk}(\Ab_{/|X|})$ die derivierten
  abelschen Garben auf $|X|$ mit simplizial konstanten
  Kohomologiegarben?
\item Wie lassen sich die gerichteten Kategorien charakterisieren, die
  als kategorielle Realisierung simplizialer Mengen auftreten?
  Beispielsweise ist $(\bullet \to \bullet)$ keine solche gerichtete
  Kategorie.
\item Lassen sich die Ergebnisse mittels des Nerven einer Kategorie
  auf beliebige Diagrammkategorien verallgemeinern?
\end{enumerate}

Wir können die Antworten nur kurz skizzieren:
\begin{enumerate}
\item Betrachte die stetigen Abbildungen
  \[ q: \coprod_n NX_n \times |\Delta^n| \to |X| \]
  und
  \[ r: \coprod_n NX_n \times |\Delta^n| \to
  \coprod_n NX_n \times \blacktriangle^n.
  \]
  Für $F \in \Ens_{/|X|}$ konstruieren wir $p_*F$ aus $r_* \,q^*\,
  F$. Ein nichtdegenerierter Simplex $\sigma$ von $X$ entspricht den
  Punkten
  \[ r(q^{-1}(|\sigma|)) \subset \coprod_n NX_n \times \blacktriangle^n.
  \]
  Nun setzen wir
  \[ p_*(\sigma) := \bigcap_{\tau \in r(q^{-1}(|\sigma|))} (r_*\,q^*\,F)_\tau , \]
  wobei der Schnitt wegen $r(q^{-1}(|\sigma|)) = r(q^{-1}(x))$ für $x
  \in |\sigma|$ mittels der Koeinheit der Adjunktion $(r^*, r_*)$ in
  $F_x$ gebildet werden kann, denn es gilt
  \[ (r_*\,q^*\,F)_{r(y)} \iso (r^*\,r_*\,q^*\,F)_y
  \to (q^*\, F)_y \iso F_{q(y)} = F_x
  \]
  für $r(y) \in r(q^{-1}(x))$. Die Generisierungsabbildungen
  $p_*F(\sigma) \to p_*F(\tau)$ sind dann die Einschränkungen der
  Abbildungen in $r_* \,q^*\, F$ aus \ref{simp-comp-sheaf}.

  Anschaulich ordnet der Funktor $p_*$ etwa einer Garbe $F$ auf $S^1$
  wie in \ref{ex:s1-diagram} die Prägarbe auf $(\bullet
  \rightrightarrows \bullet)$ zu, die dem zur Kreislinie gehörigen
  Punkt die Schnitte von $F$ über die Kreislinie zuordnet und dem
  Verklebungspunkt diejenigen Elemente des Halms von $F$ am
  Verklebungspunkt, die sich jeweils in beide Richtungen einseitig zu
  einem Schnitt über die Kreislinie fortsetzen lassen, dabei aber
  nicht übereinstimmen müssen. Wir sagen in einem solchen Fall, dass
  ein Schnitt entlang eines Morphismus in $X \otimes R$
  \emph{einseitig generisiert}.

  Anders gesprochen ist die Reihenfolge in \ref{clumsy-comb} die
  falsche: erst müssen die Generisierungsabbildungen bestimmt werden
  und erst danach darf verklebt werden.

  Die Adjunktion folgt im Wesentlichen daraus, dass ein Morphismus $F
  \to G$ von Garben über $|X|$ mit simplizial konstantem $F$ durch die
  Bilder der Schnitte $F(|\sigma|)$ für nichtdegeneriertes $\sigma$
  eindeutig bestimmt ist. Diese müssen, wenn ihr Urbild entlang $\tau
  \to \sigma$ einseitig generisiert, ebenfalls entlang $\tau \to
  \sigma$ einseitig generisieren, was zeigt, dass $F \to G$ bereits
  durch den simplizial konstanten Teil $p_* G$ von $G$ eindeutig
  festgelegt ist. Die Einheit der Adjunktion $\Id \Isotrafo p_* p^*$
  ist nach Konstruktion auf allen Objekten ein Isomorphismus.
\item Die Aussage gilt und folgt aus dem vorangegangenen Punkt und der
  Aussage zu allgemeiner schwacher Konstruierbarkeit \ref{gen-sk}.
\item Einem Objekt $\sigma$ einer gerichteten Kategorie kann eine
  Kardinalzahl zugeordnet werden, die angibt, wie viele nichttriviale
  Morphismen von $\sigma$ auslaufen. Sind es $n+1$ Stück, nennen wir
  $n$ die Dimension von $\sigma$. (Gibt es keine, nennen wir $\sigma$
  nulldimensional.) Eine gerichtete Kategorie $C$ tritt genau dann als
  kategorielle Realisierung einer simplizialen Menge auf, wenn alle
  Objekte endliche Dimension haben und die Dimensionen der Objekte
  eine Halbordnung auf $C$ definieren, die mit der Halbordnung durch
  Identifikation paralleler Morphismen in $C$ übereinstimmt.

  Alle gerichteten Kategorien lassen sich jedoch als Kolimes von zu
  simplizialen Mengen gehörigen gerichteten Kategorien
  auffassen. Anders ausgedrückt sind die gerichteten Kategorien die
  koreflektive Hülle in $\Cat$ der zu simplizialen Mengen gehörigen
  gerichteten Kategorien und die Aussage von \ref{sheaf-sset-sk}
  überträgt sich durch Kostetigkeit. Dies ist ein zur Konstruktion
  koreflektiver kartesisch abgeschlossener Kategorien aus
  Unterkategorien sich gut verhaltender topologischer Räume analoges
  Vorgehen. Die geometrische Realisierung einer gerichteten Kategorie
  $C = \col_i (C_i \otimes R)$ ist dabei der Kolimes $|C| = \col_i
  |X_i|$ für $C_i$ zu simplizialen Mengen $X_i$ gehörige gerichtete
  Kategorien. Beispielsweise gehört der Sierpinski-Raum zum Quotienten
  von $S^1$ nach der Identifikation der Bilder zweier gegenläufiger
  Wege $|\Delta^1| \to S^1$.
\item Ist $I$ eine kleine Kategorie, so ist der Nerv-Funktor gegeben
  durch 
  \begin{align*}
    N: \Cat &\to \s\Ens, \\
    I &\mapsto ([n] \mapsto [ [n], I]),
  \end{align*}
  wobei ein endliches nichtleeres Ordinal $[n]$ mittels
  \ref{poset-cat} als Kategorie aufgefasst wird. Eine Prägarbe $F \in
      [I\op, C]$ für $C$ eine Kategorie definiert trivial eine
      $C$-wertige Prägarbe auf $N(I) \otimes R$ durch
  \[ \begin{tikzcd}
    (F_0 \xrightarrow{f_1} F_1 \to \cdots \to F_n) \dar{d_0} \rar[mapsto]
    & F_0 \dar{f_1} \\
    (F_1 \to \cdots \to F_n) \rar[mapsto]
    & F_1
  \end{tikzcd} \]
  für die $0$-te Randabbildung, sowie Identitäten
  \[ \begin{tikzcd}
    (F_0 \to F_1 \to \cdots \to F_n) \dar{d_i} \rar[mapsto]
    & F_0 \dar{\id_{F_0}} \\
    (F_0 \to \cdots \to F_{i-1}
    \xrightarrow{f_{i+1} \circ f_i} F_{i+1} \to \cdots \to F_n) \rar[mapsto]
    & F_0
  \end{tikzcd} \]
  für die höheren Randabbilungen $i > 0$. Dies liefert einen Funktor
  \[ [I\op, \Ens] \to [(N(I) \otimes R)\op, \Ens] \qiso \Ens^{\sk}_{/|N(I)|}, \]
  unter welchem die Prägarben auf $I$ den \emph{geordnet simplizial
    konstanten} Garben auf $|N(I)|$ entsprechen: den simplizial
  konstanten Garben auf $|N(I)|$, für die höchstens die erste
  einseitige Generisierungsabbildung von der Identität abweicht. Auf
  einem Standard-$n$-Simplex bedeutet das etwa, dass die Garbe jeweils
  auf den Mengen
  \[ D^i = \set{(x_0, \cdots, x_n) \in |\Delta^n|}
     {x_j = 0 \text{ für } j < i, x_i > 0 } \qquad (0 \leq i \leq n)
    \]
  konstant ist.
\end{enumerate}
