% Emacs mode: -*-latex-*-
% include latex header (\usepackage, \newcommand etc.) 
\documentclass[a4paper]{article}
% \usepackage[left=3cm,right=3cm,top=3cm,bottom=2cm]{geometry} % page settings
\usepackage{amsmath}
\usepackage{amssymb}
\usepackage{amsthm}
\usepackage{etoolbox}
\usepackage[ngerman]{babel}
\usepackage[utf8]{inputenc}
\usepackage{mathtools}

\setlength{\parskip}{\medskipamount}
\setlength{\parindent}{0pt}

\newtheorem{theorem}{Theorem}
\newtheorem{lemma}[theorem]{Lemma}
\newtheorem{prop}[theorem]{Proposition}
\newtheorem{kor}[theorem]{Korollar}
\newtheorem{satz}[theorem]{Satz}
\newtheorem{defn}[theorem]{Definition}

\DeclareMathOperator{\Cat}{Cat}
\DeclareMathOperator{\poset}{poset}
\DeclareMathOperator{\EnsX}{Ens_{/X}}
\DeclareMathOperator{\pEnsX}{pEns_{/X}}
\DeclareMathOperator{\AbX}{Ab_{/X}}
\DeclareMathOperator{\pAbX}{pAb_{/X}}
\DeclareMathOperator{\OffX}{Off_X}
\DeclareMathOperator{\Ens}{Ens}
\DeclareMathOperator{\Ob}{Ob}
\DeclareMathOperator{\Der}{Der}
\DeclareMathOperator{\Ab}{Ab}
\DeclareMathOperator{\sKons}{s-Kons}
\DeclareMathOperator{\Ket}{Ket}
\DeclareMathOperator{\EnsB}{Ens_{/\B}}

\newcommand{\B}{\mathcal{B}}
\newcommand{\op}{^\mathrm{op}}
\newcommand{\iso}{\xrightarrow{\sim}}
\newcommand{\qiso}{\xrightarrow{\approx}}
\newcommand{\open}{\subset\kern-0.58em\circ}  % only possible in math mode
\newcommand{\K}{\mathcal{K}}
\newcommand{\Kreal}{|\K|}
\newcommand{\Z}{\mathbb{Z}}
\newcommand{\DerAbK}{\Der(\Ab_{/\Kreal})}
\newcommand{\DerskK}{\Der_{\mathrm{sk}}(\Kreal)}
\newcommand{\AbKr}{\Ab_{/\Kreal}}
\newcommand{\sKonsK}{\sKons(\K)}


\begin{document}

\title{Simpliziale Garben}
\author{Fabian Glöckle}
\date{\today}
% \maketitle

\section{Schwach konstruierbare Garben auf simplizialen Mengen}

In diesem Abschnitt möchten wir die Aussagen aus Kapitel \ref{??}
übertragen auf den Fall, dass es sich bei dem Basisraum um die
Realisierung einer simplizialen Menge anstelle eines
Simplizialkomplexes handelt. Wir gehen vor wie in den ersten beiden
Abschnitten.

Sei $X \in \s\Ens$ eine simpliziale Menge. Wir erinnern an die
eindeutige Darstellung eines Simplex $x$ als Degeneration $s^* y$ für
$y$ einen nichtdegenerierten Simplex und $s$ monoton und surjektiv aus
\ref{degen-repr}, notiert $y = N(x)$. Eine monotone Abbildung $f: [n]
\to [m]$ induziert dann eine partielle Abbildung $\tilde{f}: \coprod_n
NX_n \to \coprod_n NX_n$ auf den nichtdegenerierten Simplizes, gegeben
durch $N(x) \mapsto N(f^* x)$ für $x \in X_n$.
\begin{defn}
  Sei $X \in \s\Ens$ eine simpliziale Menge. Die $X$ zugeordnete
  Kategorie $C_X$ besitzt als Objekte $\Ob(C_X) := \coprod_n NX_n$ die
  Menge der nichtdegenerierten Simplizes von $X$ und für jedes $f: [n]
  \to [m]$ monoton und $\sigma \in NX_n$ einen Morphismus $\sigma \to
  \tilde{f}(\sigma))$ mit der Komposition $\sigma \to
  \tilde{f}(\sigma) \to \tilde{g}(\tilde{f}(\sigma)) = \sigma \to
  \widetilde{g \circ f}(\sigma)$.
\end{defn}

\begin{prop}
  Die kovariante Realisierung mittels plumper Simplizes von
  simplizialen Garben über diskreten topologischen Räumen mit
  Komorphismen \ref{real-ensx-cov} liefert eine Äquivalenz von
  Katgeorien %TODO
\end{prop}

Nach der Bemerkung \ref{sk-homalg} reicht es, die topologischen Teile
des Beweises aus \ref{ch2??} zu übertragen. Wir formulieren ganz
allgemein:
\begin{defn}
  Eine Konstruierbarkeitssituation ist eine stetige Abbildung $p: |\K|
  \to \K$ topologischer Räume, sodass gilt: Jeder Punkt $\sigma \in
  \K$ besitzt eine kleinste offene Umgebung $(\geq \sigma)$. Wir
  notieren $U(\sigma) = p^{-1}((\geq \sigma))$. Eine Garbe $F \in
  \AbKr$ heißt schwach $|\K|$-konstruierbar, falls die Koeinheit der
  Adjunktion auf $F$ einen Isomorphismus $p^* p_* F \iso F$ ist.
\end{defn}
In unserer Konstruierbarkeitssituation nennen wir $\K$ die
\emph{kombinatorische} und $|\K|$ die \emph{geometrische
  Realisierung}. Wir übertragen den Begriff derivierter schwach
$|\K|$-konstruierbarer Garben (mit schwach $|\K|$-konstruierbaren
Kohomologiegarben) und die Notationen $\sKons(\K)$ und $\DerskK$.

Wir können mit identischem Beweis den allgemeinen Teil von
\ref{sk-char} übertragen:
\begin{prop} \label{gensk-char}
  Für $F \in \AbKr$ sind äquivalent:
  \begin{enumerate}[label=(\arabic*)]
  \item \label{itm:gensk-char-sk} $F$ ist schwach $|\K|$-konstruierbar
  \item \label{itm:gensk-char-essim} $F$ liegt im wesentlichen Bild
    des Rückzugs $p^*$.
  \item \label{itm:gensk-char-res} Die Restriktion $F(U(\sigma)) \to
    F_x$ ist für alle $\sigma \in \K$ und alle $x \in |\sigma|$ ein
    Isomorphismus.
  \end{enumerate}
\end{prop}

In guten Konstruierbarkeitssituationen lässt sich auch eine
geometrische Formulierung schwacher $|\K|$-Konstruierbarkeit angeben:
\begin{defn}
  In einer Konstruierbarkeitssituation $p: |\K| \to \K$ heißt eine
  Garbe $F \in \AbKr$ geometrisch schwach $|\K|$-konstruierbar, falls
  die Einschränkungen $F|_{|\sigma|}$ konstant sind für alle Urbilder
  $|\sigma| = p^{-1}(\sigma)$ von Punkten $\sigma \in \K$.
\end{defn}
\begin{prop} \label{gensk-char-geom}
  Ist in einer Konstruierbarkeitssituation $p: |\K| \to \K$ ...  so
  ist für eine Garbe $F \in \AbKr$ äquivalent:
  \begin{enumerate}
    \item \label{itm:gensk-char-geom-sk} $F$ ist $|\K|$-konstruierbar.
    \item \label{itm:gensk-char-geom-geom} $F$ ist geometrisch schwach
      $|\K|$-konstruierbar.
  \end{enumerate}
  \end{prop}
Das ist der fehlende Teil von \ref{sk-char}.
\begin{proof}
  Die Richtung $\ref{itm:gensk-char-geom-sk} \Implies
  \ref{itm:gensk-char-geom-geom}$ folgt wieder aus \ref{gensk-char}
  \ref{itm:gensk-char-res} und \ref{const-stalk} wegen $|\sigma|
  \subset U(\sigma)$.

  Für die umgekehrte Richtung %TODO
\end{proof}

\end{document}
