% Emacs mode: -*-latex-*-
% include latex header (\usepackage, \newcommand etc.) 
\documentclass[a4paper]{article}
% \usepackage[left=3cm,right=3cm,top=3cm,bottom=2cm]{geometry} % page settings
\usepackage{amsmath}
\usepackage{amssymb}
\usepackage{amsthm}
\usepackage{etoolbox}
\usepackage[ngerman]{babel}
\usepackage[utf8]{inputenc}
\usepackage{mathtools}
\usepackage{tikz-cd}
\usepackage{enumitem}
\usepackage{hyperref}

\setlength{\parskip}{\medskipamount}
\setlength{\parindent}{0pt}

\theoremstyle{plain}
\newtheorem{theorem}{Theorem}
\newtheorem{lemma}[theorem]{Lemma}
\newtheorem{prop}[theorem]{Proposition}
\newtheorem{kor}[theorem]{Korollar}
\newtheorem{satz}[theorem]{Satz}
%% \providecommand*{\lemmaautorefname}{Lemma}
%% \providecommand*{\propautorefname}{Prop.}
%% \providecommand*{\korautorefname}{Korollar}
%% \providecommand*{\satzautorefname}{Satz}

\theoremstyle{definition}
\newtheorem{defn}[theorem]{Definition}

\theoremstyle{remark}
\newtheorem{bem}[theorem]{Bemerkung}

\DeclareMathOperator{\Cat}{Cat}
\DeclareMathOperator{\poset}{poset}
\DeclareMathOperator{\EnsX}{Ens_{/X}}
\DeclareMathOperator{\pEnsX}{pEns_{/X}}
\DeclareMathOperator{\AbX}{Ab_{/X}}
\DeclareMathOperator{\pAbX}{pAb_{/X}}
\DeclareMathOperator{\OffX}{Off_X}
\DeclareMathOperator{\Ens}{Ens}
\DeclareMathOperator{\Ob}{Ob}
\DeclareMathOperator{\Der}{Der}
\DeclareMathOperator{\Ab}{Ab}
\DeclareMathOperator{\sKons}{s-Kons}
\DeclareMathOperator{\Ket}{Ket}
\DeclareMathOperator{\EnsB}{Ens_{/\B}}
\DeclareMathOperator{\im}{im}
\DeclareMathOperator{\Id}{Id}
\DeclareMathOperator{\id}{id}
\DeclareMathOperator{\colf}{colf}
\DeclareMathOperator{\limf}{limf}
\DeclareMathOperator{\Top}{Top}

\newcommand{\etalespace}[1]{\overline{#1}}
\newcommand{\B}{\mathcal{B}}
\newcommand{\op}{^\mathrm{op}}
\newcommand{\iso}{\xrightarrow{\sim}}
\newcommand{\qiso}{\xrightarrow{\approx}}
\newcommand{\fromqiso}{\xleftarrow{\approx}}
\newcommand{\open}{\subset\kern-0.58em\circ}  % only possible in math mode
\newcommand{\K}{\mathcal{K}}
\newcommand{\Z}{\mathbb{Z}}
\newcommand{\R}{\mathbb{R}}
\newcommand{\DerAbK}{\Der(\Ab_{/|\K|})}
\newcommand{\DerskK}{\Der_{\mathrm{sk}}(|\K|)}
\newcommand{\DerpskK}{\Der^+_{\mathrm{sk}}(|\K|)}
\newcommand{\AbKr}{\Ab_{/|\K|}}
\newcommand{\sKonsK}{\sKons(\K)}
\newcommand{\inj}{\hookrightarrow}
\newcommand{\surj}{\twoheadrightarrow}
\newcommand{\Iff}{\Leftrightarrow}
\newcommand{\Implies}{\Rightarrow}
\newcommand{\cc}{^{\bullet}}  % chain complex
\newcommand{\from}{\leftarrow}


\begin{document}

\title{Simpliziale Garben}
\author{Fabian Glöckle}
\date{\today}
% \maketitle

\section{Schwach konstruierbare Garben auf simplizialen Mengen}

In diesem Abschnitt möchten wir die Aussagen aus Kapitel \ref{??}
übertragen auf den Fall, dass es sich bei dem Basisraum um die
Realisierung einer simplizialen Menge anstelle eines
Simplizialkomplexes handelt. Wir gehen vor wie in den ersten beiden
Abschnitten.

Für die Darstellung von Garben auf simplizialen Mengen als
Diagrammkategorien von Garben definieren wir eine \emph{kategorielle
  Realisierung} einer simplizialen Menge. Wir benötigen die Begriffe
für nichtdegenerierte Simplizes (vgl. \ref{def:delta},
\ref{def:comb-standard-simplex}).
\begin{defn}
  Die Unterkategorie der endlichen nichtleeren Ordinalzahlen mit
  injektiven monotonen Abbildungen $\Delta^+ \subset \Delta$ heißt
  \emph{nichtdegenerierte Simplexkategorie}.
\end{defn}
Wir wiederholen die Begriffe für simpliziale Mengen für Prägarben auf
$\Delta^+$.
\begin{defn}
  Die darstellbare Prägarbe auf $\Delta^+$
  \[ \Delta^{+n} :=  \Delta^+(\cdot, [n]) \]
  heißt \emph{nichtdegenerierter Standard-$n$-Simplex}.
\end{defn}
Diese Zurodnung liefert einen Funktor $r: \Delta^+ \to [\Delta^+\op,
  \Ens]$. Wir erhalten unsere für die kategorielle Realisierung
gewünschte kosimpliziale Kategorie durch den Funktor
\begin{align*}
  R: \Delta \to \Cat, \\
  [n] \mapsto \slicecat{\Delta^+}{r}{\Delta^{+n}}.
\end{align*}
% TODO zeigen, dass das durch perfide Tricks tatsächlich ein Funktor
% von \Delta aus ist!
\begin{defn}
  Die \emph{kategorielle Realisierung} einer simplizialen Menge $X \in
  \s\Ens$ ist definiert als das Tensorprodukt von Funktoren $X \times
  R \in \Cat$ für $R$ die kosimpliziale Kategorie.
\end{defn}

Betrachte nun für eine feste simpliziale Menge $X \in \s\Ens$ die
volle Unterkategorie $(\s\EnsssTop)_X \subset \s\EnsssTop$ der
simplizialen Garben über topologischen Räumen mit Komorphismen mit der
simplizialen Menge $X$ als diskreten Basisräumen.

Wir erhalten die folgende Übertragung von \ref{sheaf-simp-compl}:
\begin{prop}
  Die kovariante Realisierung mittels plumper Simplizes
  (\ref{real-enstop-cov}) liefert eine Äquivalenz von Kategorien
  \[ (\s\EnsssTop)_X \qiso \Ens_{/\blacktriangle X}. \]
  Weiter gibt es eine Äquivalenz
  \[ (\s\EnsssTop)_X \qiso [C_X\op, \Ens] \]
  gegeben durch den Funktor $F \mapsto (\sigma \mapsto (F_n)_\sigma$
  für $\sigma \in NX_n$.
\end{prop}
\begin{proof}
  Eine Garbe $F_n$ über einem diskreten Raum $X_n$ ist diskret und
  somit durch ihre Halme festgelegt. Wir können sie somit als einen
  Funktor der diskreten Kategorie $X_n$ in die Kategorie der Mengen
  $F_n: X_n \to \Ens$ auffassen. Ein Komorphismus über $Ff^*: X_m \to
  X_n$ sind dann Abbildungen zwischen den Halmen $(F_n)_{Ff(\sigma)}
  \to (F_m)_\sigma$ für $\sigma \in X_m$.
\end{proof}

\begin{lemma}
  Sei $X \in \s\Ens$ eine simpliziale Menge. Dann gibt es einen
  Homöomorphismus $\blacktriangle X \iso C_X$, wobei $C_X$ die
  Ordnungstopologie trägt. Insbesondere hat jeder Punkt in der plumpen
  Realisierung $\sigma \in \blacktriangle X$ eine kleinste offene
  Umgebung $(\geq \sigma)$.
\end{lemma}
\begin{proof}
  Jeder Punkt $\sigma \in \blacktriangle X$ ist der generische Punkt
  eines nichtdegenerierten $n$-Simplex.
  % TODO Zellen des CW-Komplexes haben disjunktes Inneres
  Dies liefert eine Bijektion $\blacktriangle X \to C_X$. 
\end{proof}

Nach der Bemerkung \ref{sk-homalg} reicht es, die topologischen Teile
des Beweises aus \ref{ch2??} zu übertragen. Wir formulieren ganz
allgemein:
\begin{defn}
  Eine Konstruierbarkeitssituation ist eine stetige Abbildung $p: |\K|
  \to \K$ topologischer Räume, sodass gilt: Jeder Punkt $\sigma \in
  \K$ besitzt eine kleinste offene Umgebung $(\geq \sigma)$. Wir
  notieren $U(\sigma) = p^{-1}((\geq \sigma))$. Eine Garbe $F \in
  \AbKr$ heißt schwach $|\K|$-konstruierbar, falls die Koeinheit der
  Adjunktion auf $F$ einen Isomorphismus $p^* p_* F \iso F$ ist.
\end{defn}
In unserer Konstruierbarkeitssituation nennen wir $\K$ die
\emph{kombinatorische} und $|\K|$ die \emph{geometrische
  Realisierung}. Wir übertragen den Begriff derivierter schwach
$|\K|$-konstruierbarer Garben (mit schwach $|\K|$-konstruierbaren
Kohomologiegarben) und die Notationen $\sKons(\K)$ und $\DerskK$.

Wir können mit identischem Beweis den allgemeinen Teil von
\ref{sk-char} übertragen:
\begin{prop} \label{gensk-char}
  Für $F \in \AbKr$ sind äquivalent:
  \begin{enumerate}[label=(\arabic*)]
  \item \label{itm:gensk-char-sk} $F$ ist schwach $|\K|$-konstruierbar
  \item \label{itm:gensk-char-essim} $F$ liegt im wesentlichen Bild
    des Rückzugs $p^*$.
  \item \label{itm:gensk-char-res} Die Restriktion $F(U(\sigma)) \to
    F_x$ ist für alle $\sigma \in \K$ und alle $x \in |\sigma|$ ein
    Isomorphismus.
  \end{enumerate}
\end{prop}

In guten Konstruierbarkeitssituationen lässt sich auch eine
geometrische Formulierung schwacher $|\K|$-Konstruierbarkeit angeben:
\begin{defn}
  In einer Konstruierbarkeitssituation $p: |\K| \to \K$ heißt eine
  Garbe $F \in \AbKr$ geometrisch schwach $|\K|$-konstruierbar, falls
  die Einschränkungen $F|_{|\sigma|}$ konstant sind für alle Urbilder
  $|\sigma| = p^{-1}(\sigma)$ von Punkten $\sigma \in \K$.
\end{defn}
\begin{prop} \label{gensk-char-geom}
  Ist in einer Konstruierbarkeitssituation $p: |\K| \to \K$ ...  so
  ist für eine Garbe $F \in \AbKr$ äquivalent:
  \begin{enumerate}
    \item \label{itm:gensk-char-geom-sk} $F$ ist $|\K|$-konstruierbar.
    \item \label{itm:gensk-char-geom-geom} $F$ ist geometrisch schwach
      $|\K|$-konstruierbar.
  \end{enumerate}
  \end{prop}
Das ist der fehlende Teil von \ref{sk-char}.
\begin{proof}
  Die Richtung $\ref{itm:gensk-char-geom-sk} \Implies
  \ref{itm:gensk-char-geom-geom}$ folgt wieder aus \ref{gensk-char}
  \ref{itm:gensk-char-res} und \ref{const-stalk} wegen $|\sigma|
  \subset U(\sigma)$.

  Für die umgekehrte Richtung %TODO
\end{proof}

\end{document}
