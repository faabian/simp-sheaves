% Emacs mode: -*-latex-*-
% include latex header (\usepackage, \newcommand etc.) 
\documentclass[a4paper]{article}
% \usepackage[left=3cm,right=3cm,top=3cm,bottom=2cm]{geometry} % page settings
\usepackage{amsmath}
\usepackage{amssymb}
\usepackage{amsthm}
\usepackage{etoolbox}
\usepackage[ngerman]{babel}
\usepackage[utf8]{inputenc}
\usepackage{mathtools}
\usepackage{tikz-cd}
\usepackage{enumitem}
\usepackage{hyperref}

\setlength{\parskip}{\medskipamount}
\setlength{\parindent}{0pt}

\theoremstyle{plain}
\newtheorem{theorem}{Theorem}
\newtheorem{lemma}[theorem]{Lemma}
\newtheorem{prop}[theorem]{Proposition}
\newtheorem{kor}[theorem]{Korollar}
\newtheorem{satz}[theorem]{Satz}
%% \providecommand*{\lemmaautorefname}{Lemma}
%% \providecommand*{\propautorefname}{Prop.}
%% \providecommand*{\korautorefname}{Korollar}
%% \providecommand*{\satzautorefname}{Satz}

\theoremstyle{definition}
\newtheorem{defn}[theorem]{Definition}

\theoremstyle{remark}
\newtheorem{bem}[theorem]{Bemerkung}

\DeclareMathOperator{\Cat}{Cat}
\DeclareMathOperator{\poset}{poset}
\DeclareMathOperator{\EnsX}{Ens_{/X}}
\DeclareMathOperator{\pEnsX}{pEns_{/X}}
\DeclareMathOperator{\AbX}{Ab_{/X}}
\DeclareMathOperator{\pAbX}{pAb_{/X}}
\DeclareMathOperator{\OffX}{Off_X}
\DeclareMathOperator{\Ens}{Ens}
\DeclareMathOperator{\Ob}{Ob}
\DeclareMathOperator{\Der}{Der}
\DeclareMathOperator{\Ab}{Ab}
\DeclareMathOperator{\sKons}{s-Kons}
\DeclareMathOperator{\Ket}{Ket}
\DeclareMathOperator{\EnsB}{Ens_{/\B}}
\DeclareMathOperator{\im}{im}
\DeclareMathOperator{\Id}{Id}
\DeclareMathOperator{\id}{id}
\DeclareMathOperator{\colf}{colf}
\DeclareMathOperator{\limf}{limf}
\DeclareMathOperator{\Top}{Top}

\newcommand{\etalespace}[1]{\overline{#1}}
\newcommand{\B}{\mathcal{B}}
\newcommand{\op}{^\mathrm{op}}
\newcommand{\iso}{\xrightarrow{\sim}}
\newcommand{\qiso}{\xrightarrow{\approx}}
\newcommand{\fromqiso}{\xleftarrow{\approx}}
\newcommand{\open}{\subset\kern-0.58em\circ}  % only possible in math mode
\newcommand{\K}{\mathcal{K}}
\newcommand{\Z}{\mathbb{Z}}
\newcommand{\R}{\mathbb{R}}
\newcommand{\DerAbK}{\Der(\Ab_{/|\K|})}
\newcommand{\DerskK}{\Der_{\mathrm{sk}}(|\K|)}
\newcommand{\DerpskK}{\Der^+_{\mathrm{sk}}(|\K|)}
\newcommand{\AbKr}{\Ab_{/|\K|}}
\newcommand{\sKonsK}{\sKons(\K)}
\newcommand{\inj}{\hookrightarrow}
\newcommand{\surj}{\twoheadrightarrow}
\newcommand{\Iff}{\Leftrightarrow}
\newcommand{\Implies}{\Rightarrow}
\newcommand{\cc}{^{\bullet}}  % chain complex
\newcommand{\from}{\leftarrow}


\begin{document}

\title{Simpliziale Garben}
\author{Fabian Glöckle}
\date{\today}
% \maketitle

\section{Schwach konstruierbare Garben auf simplizialen Mengen}

In diesem Abschnitt möchten wir die Aussagen aus Kapitel \ref{??}
übertragen auf den Fall, dass es sich bei dem Basisraum um die
Realisierung einer simplizialen Menge anstelle eines
Simplizialkomplexes handelt. Simplizialkomplexe sind halbgeordnete
Mengen, unsere Technik verwendete im Wesentlichen die
Ordnungstopologie halbgeordneter Mengen. Simplizial konstante Garben
auf simplizialen Mengen entsprechen hingegen Diagrammen von Mengen, in
denen es auch mehrere Pfeile zwischen zwei zu nichtdegenerierten
Simplizes gehörigen Punkten geben kann. Diese \emph{kategorielle
  Realisierung} simplizialer Mengen werden wir im folgenden
konstruieren.

Wir benötigen die Begriffe für nichtdegenerierte Simplizes
(vgl. \ref{def:delta}, \ref{def:comb-standard-simplex}).
\begin{defn}
  Die Unterkategorie der endlichen nichtleeren Ordinalzahlen mit
  injektiven monotonen Abbildungen $\Delta^+ \subset \Delta$ heißt
  \emph{nichtdegenerierte Simplexkategorie}.
\end{defn}
Wir wiederholen die Begriffe für simpliziale Mengen für Prägarben auf
$\Delta^+$.
\begin{defn}
  Die darstellbare Prägarbe auf $\Delta^+$
  \[ \Delta^{+n} :=  \Delta^+(\cdot, [n]) \]
  heißt der \emph{nichtdegenerierte Standard-$n$-Simplex}.
\end{defn}
Diese Zuordnung liefert einen Funktor $r: \Delta^+ \to [{\Delta^+}\op,
  \Ens]$. Wir erhalten unsere für die kategorielle Realisierung
gewünschte kosimpliziale Kategorie durch den Funktor der
nichtdegenerierten Simplizes des nichtdegenerierten
Standard-$n$-Simplex. Bezeichne dazu $\iota: \Delta^+ \inj \Delta$ den
Inklusionsfunktor und $\iota^*: [\Delta\op, \Ens] \to [{\Delta^+}\op,
  \Ens]$ den Rückzugsfunktor auf Prägarben.
\begin{defn}
  Der \emph{Stufenfunktor} ist der Funktor
  \[N: \slicecat{\Delta^+}{r}{\iota^* \Delta^n}
  \to \slicecat{\Delta^+}{r}{\Delta^{+n}},
  \]
  gegeben durch das kommutative Quadrat
  \[ 
  \begin{tikzcd}
    f: \Delta^{+m} \to \iota^* \Delta^n \dar[mapsto]{\sim} \rar[mapsto]
    & N(f):  \Delta^{+k}  \to \iota^* \Delta^{+n} \dar[mapsto]{\sim} \\
    f \in \Delta([m], [n]) \rar[mapsto]
    & N(f) \in \Delta^+([k], [n]),
  \end{tikzcd}
  \]
  in dem die Vertikalen die eindeutigen Zuordnungen aus dem
  Yoneda-Lemma sind und die untere Horizontale die Abbildung, die eine
  monotone Abbildung $f$ auf die eindeutige monotone Injektion $N(f)$
  aus \ref{degen-repr} mit demselben Bild (und anderem
  Definitionsbereich $[k]$) schickt.
\end{defn}
\begin{bem}
  Der Name ``Stufenfunktor'' rührt daher, dass die Werte einer
  monotonen Funktion die Stufen in ihrem Graphen beschreiben.
\end{bem}
Das Vorschalten von monotonen Injektionen vor $f \in \Delta([m], [n])$
(Morphismen in $\slicecat{\Delta^+}{r}{\iota^* \Delta^n}$) induziert
auf der zugehörigen monotonen Injektion $\hat{f} \in \Delta^+([k],
[n])$ ebenfalls Morphismen durch Vorschalten von Injektionen, denn das
Einschränken von Funktionen auf Teilmengen verkleinert auch die
Bildmengen. Dies zeigt die Funktorialität.
\begin{prop}
  Die Zuordnung
  \[ \begin{tikzcd}
    {[n]} \arrow{dd}{f} \rar[mapsto]
    & \slicecat{\Delta^+}{r}{\Delta^{+n}} \dar{f \circ} \\
    & \slicecat{\Delta^+}{r}{\iota^* \Delta^m} \dar{N} \\
    {[m]} \rar[mapsto]
    & \slicecat{\Delta^+}{r}{\Delta^{+m}}
  \end{tikzcd} \]
  ist ein Funktor $R: \Delta \to \Cat$, genannt die
  \emph{kosimpliziale Standard-Kategorie}.
\end{prop}
\begin{proof}
  Wir müssen zeigen, dass für monotone $f: [m] \to [n]$ und $g: [l]
  \to [m]$ gilt
  \[ N ((f \circ g)\circ) = N (f \circ) N (g \circ) \]
  für $(f \circ)$ den Nachschaltefunktor und $N$ den
  Stufenfunktor. Das folgt aber daraus, dass beide Funktoren eine
  monotone Injektion $h: [k] \to [l]$ auf die Injektion auf $\im (f
  \circ g \circ h)$ schicken. Diese Entsprechnung ist verträglich mit
  Einschränkungen von $h$ (Vorschalten von monotonen Injektionen), ist
  also eine Transformation.

  Es handelt sich bei den Kategorien
  $\slicecat{\Delta^+}{r}{\Delta^{+n}}$ um ``gerichtete Kategorien'',
  bei denen es keine Kreise von Pfeilen außer den Identitäten gibt,
  denn die nichttrivialen Morphismen sind das Vorschalten von echten
  Injektionen und senken somit den Grad eines Simplex.
\end{proof}

\begin{prop} \label{cat-cocomplete}
  Die Kategorie der kleinen Kategorien $\Cat$ ist kovollständig.
\end{prop}
\begin{proof}
  Koprodukte in $\Cat$ sind die Koprodukte der zugrundeliegenden
  Köcher, d. h. die disjunkte Vereinigung über die Objektmengen und
  aus den Ausgangskategorien übernommene Morphismenmengen.

  Die Koegalisatoren in $\Cat$ sind schwieriger, vergleiche
  \cite{BBP}. Wir geben die Konstruktion kurz an. Betrachte kleine
  Kategorien mit Funktoren $A
  \mathrel{\mathop{\rightrightarrows}^{F}_{G}} B$. Die dem
  Koegalisator $B \to C$ zugrundeliegende mengentheoretische Abbildung
  ist der Koegalisator der den Funktoren $F$ und $G$ zugrundeliegenden
  mengentheoretischen Abbildungen. Durch diese Identifikationen in $C$
  werden Morphismen neu komponierbar, deren Kompositionen den durch
  die Identifikationen verschmolzenen Morphismenmengen hinzugefügt
  werden. Weiter müssen wie im folgenden Diagramm Morphismen $Ff$ und
  $Gf$ identifiziert werden, was auf die Kompositionen fortgesetzt
  wird.
  \[ 
  \begin{tikzcd}
    a \ar{dd}{f}
    & Fa \rar[phantom]{\sim} \ar{dd}{Ff} & Ga \ar{dd}{Gf}  \\[-10pt]
    \rar[mapsto, shorten <= 2em, xshift = -1em]
    & \rar[phantom]{\sim} & \lar[phantom] \\[-10pt]
    b & Fb \rar[phantom]{\sim} & Gb
  \end{tikzcd}
  \]  
  Die Identifikationen $Fa \sim Ga$ für $a \in A$ und $Ff \sim Gf$ für
  $f \in A(a, b)$ sind notwendig für einen Koegalisator $B \to C$,
  damit $A \mathrel{\mathop{\rightrightarrows}^{F}_{G}} B \to C$
  übereinstimmen. Die weiteren Schritte machen ``minimalinvasiv'' $B$
  mit diesen Identifikationen wieder zu einer Kategorie.
\end{proof}
\begin{bem}
  Der Ansatz, Limites und Kolimites in $Cat$ mittels
  \ref{refl-sub-complete} über die Einbettung $\Cat \subset
  \mathrm{Quiv}$ in die Kategorie der Köcher zu konstruieren,
  funktioniert \emph{nicht}. Jene ist als Prägarbenkategorie
  tatsächlich vollständig und kovollständig und die Inklusion hat mit
  der freien Pfadkategorie über einem Köcher tatsächlich einen
  Linksadjungierten; allerdings handelt es sich nicht um eine volle
  (dann also reflektive) Unterkategorie, weshalb die Limites und
  Kolimites von denen in $\mathrm{Quiv}$ bzw. ihren freien
  Pfadkategorien abweichen.
\end{bem}
\begin{defn}
  Die \emph{kategorielle Realisierung} einer simplizialen Menge $X \in
  \s\Ens$ ist definiert als das Tensorprodukt von Funktoren $X \otimes
  R \in \poset$ für $R$ die kosimpliziale Standard-Kategorie und die
  natürliche $\Ens$-tensorierte Struktur auf $\poset$
  (\ref{ens-tensored}).
\end{defn}
\begin{bsp}
  Betrachte die Standarddarstellung von $S^1$ als simpliziale Menge
  $X$ mit einem nichtdegenerierten $1$-Simplex und einem
  nichtdegenerierten $0$-Simplex aus \ref{ex:real-sphere}. Dann ist
  die kategorielle Realisierung von $X$ das Diagramm
  \[ \bullet \rightrightarrows \bullet \]
  mit zwei Objekten und zwei parallelen Pfeilen dazwischen, sowie
  nicht eingezeichneten Identitäten.
\end{bsp}

\subsection{Realisierung als halbgeordnete Menge}

Der kombinatorische topologische Raum $\blacktriangle X$ einer
simplizialen Menge eignet sich \emph{nicht} für die Übertragung der
Aussagen zu schwach konstruierbaren Garben auf Simplizialkomplexen auf
die Situation simplizialer Mengen, denn diese geometrische
Realisierung sieht nicht mehrfache Verklebungsabbildungen. Dies
möchten wir präzise machen.

\begin{prop} \label{poset-cocomplete}
  Die Kategorie $\poset$ der halbgeordneten Mengen ist kovollständig.
\end{prop}
\begin{proof}
  Die halbgeordneten Mengen sind eine volle Unterkategorie $\poset
  \subset \Cat$. Wir können daher \ref{refl-sub-complete} verwenden,
  mit dem Reflektor $\Cat \to \poset$, der im folgenden Lemma
  konstruiert wird.
\end{proof}
\begin{defn}
  Eine Kategorie heißt \emph{dünn}, falls jede Morphismenmenge
  höchs\-tens einelementig ist. Eine Kategorie heißt
  \emph{Skelettkategorie}, falls in ihr jeder Isomorphismus eine
  Identität ist.
\end{defn}
\begin{lemma} \label{poset-reflective}
  Die vollen Unterkategorien $\mathrm{thinCat}, \mathrm{skelCat}
  \subset \Cat$ der dünnen bzw. Skelettkategorien sind reflektiv. Der
  Reflektor $\Cat \to \mathrm{skelCat}$ macht aus dünnen Kategorien
  dünne Kategorien und liefert durch Komposition mit dem Reflektor
  $\Cat \to \mathrm{thinCat}$ einen Reflektor $\pos: \Cat \to \poset$.
\end{lemma}
\begin{proof}
  Der Linksadjungierte zu $\mathrm{thinCat} \inj \Cat$ ist gegeben
  durch die Identifikation aller nichtleeren Morphismenmengen zu
  einelementigen Morphismenmengen. Der Linksadjungierte zu
  $\mathrm{skelCat} \inj \Cat$ ist die zu einer kleinen Kategorie mit
  dem Auswahlaxiom konstruierte Unterkategorie, die Isomorphieklassen
  von Objekten durch ein Objekt aus diesen ersetzt. Klar ist, dass
  Funktoren $F: C \to D$ in eine dünne Kategorie $D$ Abbildungen auf
  Objekten sind mit der Zusatzeigenschaft, dass es einen Morphismus
  $Ff: Fx \to Fy$ in $D$ geben muss, wann immer es einen Morphismus
  $f: x \to y$ in $C$ gibt. Das ist unerheblich davon, wie viele
  Morphismen $x \to y$ es in $C$ gibt und zeigt die erste
  Adjunktion. Ein Funktor in eine Skelettkategorie schickt isomorphe
  Objekte auf dasselbe Objekt, wird also schon durch das Bild eines
  Objekts jeder Isomorphieklasse eindeutig festgelegt. Dies zeigt die
  zweite Adjunktion.

  Der Reflektor $\Cat \to \mathrm{skelCat}$ liefert eine
  Unterkategorie und erhält deshalb Dünnheit. Halbgeordnete Mengen
  sind nach Definition dünne Skelettkategorien.
\end{proof}

Bezeichne dazu $P = \pos R: \Delta \to \poset$ die kosimpliziale
halbgeordnete Menge zu unseren Standardkategorien. Die halbgeordneten
Mengen $P[n] = \pos \slicecat{\Delta^+}{r}{\Delta^{+n}}$ sind dann die
opponierten Standard-$n$-Simplizialkomplexe.
\begin{prop} \label{clumsy-order-top}
  Sei $X \in \s\Ens$ eine simpliziale Menge. Dann gibt es einen
  Homöomorphismus $\blacktriangle X \iso \Ord((X \otimes P)\op)$.
\end{prop}
\begin{proof}
  Der Funktor $\Ord: \poset \to \Top$ ist nach dem nachgestellten
  Lemma kostetig. Daher reicht es mit \ref{coend-cocont} (und der
  Kostetigkeit des Opponierens), einen Isomorphismus kosimplizialer
  topologischer Räume $\blacktriangle \to \Ord P\op$ zu finden. Beide
  bestehen aus einem Punkt pro nichtdegeneriertem Simplex
  (\ref{cw-disjoint-inner}) von $\Delta^n$ und haben als offene Mengen
  nach oben abgeschlossene Mengen. Randabbildungen $d_i$ sind
  Inklusionen in die Ränder, Degenerationen Kollapse von Kanten. Dies
  begründen wir sorgfältiger: Unsere Definition von $Ps_i: P[n] \to
  P[n-1]$ schickt einen nichtdegenerierten Simplex $f: [m] \to [n]$
  monoton und injektiv auf $N(s_i \circ f)$, den nichtdegenerierten
  Simplex, der zum Kollaps von $i$ und $i+1$ in $f$ gehört.
\end{proof}
\begin{lemma}
  Der Funktor $\Ord: \poset \to \Top$, der eine halbgeordneten Menge
  mit der Ordnungstopologie versieht, ist kostetig.
\end{lemma}
\begin{proof}
    Klar ist, dass $\Ord$ mit Koprodukten vertauscht. Sei nun $A
    \mathrel{\mathop{\rightrightarrows}^{F}_{G}} B \to C$ ein
    Koegalisator in den halbgeordneten Mengen. Dann ist nach
    \ref{cat-cocomplete} und \ref{poset-reflective} die
    zugrundeliegende mengentheoretische Abbildung von $q: B \to C$ der
    mengentheoretische Koegalisator: nur der Reflektor $\Cat \to
    \mathrm{skelCat}$ könnte die zugrundeliegende Menge ändern, wird
    aber bereits auf eine Skelettkategorie angewandt, denn ein
    Kategorienkolimes über halbgeordnete Mengen enthält keine
    Morphismen in entgegengesetzte Richtungen. Wir müssen noch zeigen,
    dass $\Ord(q): \Ord B \to \Ord C$ final ist. Ist $U \subset C$
    eine Menge mit offenem Urbild $q^{-1}(C)$, so ist ein Morphismus
    $x \to y$ in $C$ mit $x \in U$ ein Pfad $x = v_0 \to v_1 \sim w_1
    \to w_2 \sim v_2 \to \cdots \to y$ bestehend aus Morphismen in
    $B$, die sich nach den Identifaktionen durch $q$ verknüpfen
    lassen. Induktiv liegen nun nach der Abgeschlossenheit nach oben
    von $q^{-1}(U)$ alle $v_i$, $w_i$ in $q^{-1}(U)$ und somit auch
    $y$. Es folgt die Offenheit von $U$.
\end{proof}

Mit \ref{sheaf-order-top} und \ref{clumsy-order-top} erhalten wir
sofort die folgende kombinatorische Charakterisierung von Garben auf
der plumpen Realisierung:
\begin{prop} \label{clumsy-comb}
  Sei $X \in \s\Ens$ eine simpliziale Menge. Dann gibt es eine
  Äquivalenz von Kategorien $\Ens_{/\blacktriangle X} \qiso [(X
    \otimes P)\op, \Ens]$.
\end{prop}
\begin{bem}
  Betrachte für eine feste simpliziale Menge $X \in \s\Ens$ die volle
  Unterkategorie $(\s\EnsssTop)_X \subset \s\EnsssTop$ der
  simplizialen Garben über topologischen Räumen mit Komorphismen, für
  die die Basisräume diskret und als simplizialer topologischer Raum
  isomorph zur simplizialen Menge $X$ sind. Man könnte eine Aussage
  wie die folgende erwarten:
  \begin{quote}
    Die kovariante Realisierung mittels plumper Simplizes
    (\ref{real-enstop-cov}) liefert eine Äquivalenz von Kategorien
    \[ (\s\EnsssTop)_X \qiso \Ens_{/\blacktriangle X}. \]
  \end{quote}
  Dies verhindern aber die degenerierten Simplizes $s^*(\sigma)$ für
  $s: [n] \to [m]$ monoton und surjektiv. Für diese enthält eine
  simpliziale Garbe $F$ auf der linken Seite beliebig wählbare Mengen,
  die Halme $(F_n)_{s^*(\sigma)}$, welche aus der geometrischen
  Realisierung nicht wiedergewonnen werden können, da sie mit ihren
  Bildern unter $Fs: (F_n)_{s^*(\sigma)} \to (F_m)_\sigma$
  identifiziert werden.

  Stattdessen setzen wir $(\s\EnsssTop)_X^{nd} \subset
  (\s\EnsssTop)_X$ die volle Unterkategorie der oben definierten
  simplizialen Garben $F$ über $X$, für die zudem die Degenerationen
  $Fs: (F_n)_{s^*(\sigma)} \to (F_m)_\sigma$ Bijektionen sind. Für
  diese erhalten wir den gewünschten quasiinversen Funktor, indem wir
  mit \ref{clumsy-comb} einer Garbe $F \in [(X \otimes R)\op, \Ens]$
  auf folgende Weise eine simpliziale Garbe $\hat{F} \in
  (\s\EnsssTop)_X^{nd}$ zuordnen: Schreibe für $\sigma \in X_n$ kurz
  $F(\sigma)$ für die Menge, die $F$ dem maximalen Element in
  $\{\sigma\} \times R[n]\op$ via $X_n \times R[n]\op \to (X \otimes
  R)\op$ zuordnet. Die Garben $\hat{F}_n \in \Ens_{/X_n}$ sind diskret
  und bestehen dann aus der Menge $F(N(\sigma))$ über $\sigma$. Für
  eine monotone Abbildung $f: [n] \to [m]$ erhalten wir die
  Abbildungen $(\hat{F}_n)_{f^*(\sigma)} \to (\hat{F}_{m})_\sigma$ für
  die Komorphismen von $\hat{F}$ über $f^*$ aus $F(N(f(\sigma))) \to
  F(N(\sigma))$. Insbesondere sind die Degenerationen
  $(\hat{F}_n)_{s^*(\sigma)} \to (\hat{F}_{n-1})_\sigma$ dann einfach
  Identitäten und unsere Abbildungen erfüllen die simplizialen
  Relationen.
\end{bem}

\subsection{??}

Wir kommen nun zur Übertragung der Ergebnisse aus \ref{ch02} auf die
Situation simplizialer Mengen. Dies ermöglicht etwa die Aussage auch
für Triangulierungen wie in Beispiel \ref{ex:real-sphere}.

Nach der Bemerkung \ref{sk-homalg} reicht es, die topologischen Teile
des Beweises zu übertragen. Wir sammeln die benötigten Axiome:
\begin{defn} \label{def:constr}
  Eine \emph{Konstruierbarkeitssituation} ist eine stetige Abbildung
  $p: |\K| \to \K$ mit den folgenden Eigenschaften:
  \begin{enumerate}
  \item \label{itm:constr-final} $p$ ist final, surjektiv und hat
    zusammenhängende Fasern.
  \item \label{itm:constr-comb-space} Jeder Punkt $\sigma \in \K$
    besitzt eine kleinste offene Umgebung $(\geq \sigma)$.
  \end{enumerate}
\end{defn}
Wir notieren $U(\sigma) = p^{-1}((\geq \sigma))$. In einer
Konstruierbarkeitssituation nennen wir $\K$ die \emph{kombinatorische}
und $|\K|$ die \emph{geometrische Realisierung}.

Der ``richtige'' äquivalente Begriff von schwacher Konstruierbarkeit
aus \ref{sk-char} wird zur allgemeinen Definition:
\begin{defn}
  Eine Garbe $F \in \AbKr$ heißt schwach $|\K|$-konstruierbar, falls
  die Koeinheit der Adjunktion auf $F$ ein Isomorphismus $p^* p_* F
  \iso F$ ist.
\end{defn}
Auch übertragen wir den Begriff derivierter schwach
$|\K|$-konstruierbarer Garben (mit schwach $|\K|$-konstruierbaren
Kohomologiegarben) und die Notationen $\sKons(\K)$ und $\DerskK$.

Mit identischem Beweis überträgt sich der allgemeine Teil von
\ref{sk-char} übertragen:
\begin{prop} \label{gensk-char}
  In einer Konstruierbarkeitssituation $p: |\K| \to \K$ sind für $F
  \in \AbKr$ sind äquivalent:
  \begin{enumerate}[label=(\arabic*)]
  \item \label{itm:gensk-char-sk} $F$ ist schwach $|\K|$-konstruierbar
  \item \label{itm:gensk-char-essim} $F$ liegt im wesentlichen Bild
    des Rückzugs $p^*$.
  \item \label{itm:gensk-char-res} Die Restriktion $F(U(\sigma)) \to
    F_x$ ist für alle $\sigma \in \K$ und alle $x \in |\sigma|$ ein
    Isomorphismus.
  \end{enumerate}
\end{prop}
Und es folgt sofort, in Anbetracht von \ref{dersk-eq-rel}:
\begin{theorem}
  Sei $p: |\K| \to \K$ eine Konstruierbarkeitssituation und $X \in
  \Top$.  Dann gibt es eine Äquivalenz von Kategorien
  \[ \Der(\sKons(\K \times X))
     \mathrel{\mathop{\rightleftarrows}^{\iota}_{R \beta}}
     \Der_{\mathrm{sk}}(|\K| \times X),
  \]
  wobei $\iota$ die Inklusion und $\beta = (p \times \id_X)^* (p
  \times \id_X)_*: \Ab_{|\K| \times X} \to \sKons(\K \times X)$ ist.
\end{theorem}
Es reicht also für den Fall simplizialer Mengen, die Axiome einer
Konstruierbarkeitssituation zu zeigen.
\begin{prop}
  Sei $X \in \s\Ens$ eine simpliziale Menge. Dann ist $p: |X| \to
  \blacktriangle X$ eine Konstruierbarkeitssituation.
\end{prop}
\begin{proof}
  Die Abbildung $p: |X| \to \blacktriangle X$ ist ein Kolimes über die
  Quotientenabbildungen $|\Delta^n| \to \blacktriangle^n$ mit
  zusammenhängenden Fasern und das Axiom \ref{def:constr}
  \ref{itm:constr-final} folgt aus dem nachgestellten Lemma. Das Axiom
  \ref{def:constr} \ref{itm:constr-comb-space} zur Existenz kleinster
  offener Umgebungen wurde in \ref{clumsy-comb} gezeigt.
\end{proof}
\begin{lemma}
  Sei $X_i \to Y_i$ ein Morphismus von Diagrammen von topologischen
  Räumen $[I, \Top]$ mit finalen, surjektiven Abbildungen $X_i \to
  Y_i$ mit zusammenhängenden Fasern. Dann ist die induzierte Abbildung
  $\col_i X_i \to \col_i Y_i$ final, surjektiv und hat
  zusammenhängende Fasern.
\end{lemma}
\begin{proof}
  Die Surjektivität ist offensichtlich (nimm ein Urbild unter einem
  geeigneten $Y_i \to \col_i Y_i$, dann unter $X_i \surj Y_i$ und dann
  dessen Inklusion nach $\col_i X_i$). Ist die Komposition $\col_i X_i
  \to \col_i Y_i \to Z$ stetig, so sind alle
  \[X_i \to \col_i X_i \to \col_i Y_i \to Z
  = X_i \to Y_i \to \col_i Y_i \to Z \] stetig, und die Stetigkeit von
  $g$ folgt daraus, dass die Kompositionen finaler Familien final ist
  und $\col_i Y_i$ folglich die Finaltopologie bezüglich aller $X_i
  \to \col_i Y_i$ trägt.

  Zum Zusammenhang der Fasern: Die beiden Kolimites sind Quotienten
  der disjunkten Vereinigung über das System nach einer von den
  Systemmorphismen herrührenden Äquivalenzrelation. Ist $y_i \sim
  Yf(y_i)$ mit $y_i \in F_i$ und $Yf: Y_i \to Y_j$ einem
  Systemmorphismus eine erzeugende Relation, so sind auch die Urbilder
  der Zusammenhangskomponenten von $y_i$ und $Yf(y_i)$ in $\col_i X_i$
  nicht disjunkt: ist etwa $x_i$ ein Urbild von $y_i$, so ist
  $Xf(x_i)$ ein Urbild von $Yf(y_i)$ und die Zusammenhangskomponenten
  treffen sich im Kolimes $\col_i X_i$ im Punkt $x_i \sim Xf(x_i)$.
\end{proof}
Bei der Übertragung von \ref{sk-char} ist die interessanteste
äquivalente Formulierung schwacher $|\K|$-Konstruierbarkeit bislang
unter den Tisch gefallen.
\begin{defn}
  In einer Konstruierbarkeitssituation $p: |\K| \to \K$ heißt eine
  Garbe $F \in \AbKr$ \emph{geometrisch schwach $|\K|$-konstruierbar},
  falls die Einschränkungen $F|_{|\sigma|}$ konstant sind für alle
  Urbilder $|\sigma| = p^{-1}(\sigma)$ von Punkten $\sigma \in \K$.
\end{defn}
Wir erhalten im Allgemeinen nur noch eine Implikation:
\begin{prop} \label{gensk-char-geom}
  Ist $p: |\K| \to \K$ eine Konstruierbarkeitssituation, so impliziert
  für eine Garbe $F \in \AbKr$ schwache $|\K|$-Konstruierbarkeit
  geometrisch schwache $|\K|$-Konstruierbarkeit.
\end{prop}
Das ist der fehlende Teil von \ref{sk-char}.
\begin{proof}
  Dies folgt wieder aus \ref{gensk-char} \ref{itm:gensk-char-res} und
  \ref{const-stalk} wegen $|\sigma| \subset U(\sigma)$.
\end{proof}
\begin{bsp}
  Die umgekehrte Richtung gilt im Allgemeinen nicht: Sei etwa $|X| =
  S^1$ mit der Triangulierung als simpliziale Menge aus
  \ref{ex:real-sphere} und $F \in \Ab_{/|X|}$ die nichtkonstante lokal
  konstante Garbe auf $S^1$ mit Halm $\Z$. Für $\sigma$ den
  $0$-Simplex ist dann $U(\sigma) = S^1$ und es ist $\Z \cong F_\sigma
  \ncong F(U(\sigma)) = \Gamma F = 0$.
\end{bsp}
\begin{bem}
  Für die umgekehrte Richtung würden wir für $x \in
  U(\sigma)$ eine stetige Zusammenziehung
  \[ h: (0, 1] \times U(\sigma) \to U(\sigma) \]
  benötigen, für die gilt:
  \begin{enumerate}
  \item Die Mengen $h({t} \times U(\sigma))$ bilden für $t \in (0, 1]$
    eine Umgebungsbasis von $x$.
  \item Es gilt $h(t, y) \in |\tau| \Iff y \in |\tau|$.
  \item $h$ ist surjektiv.
  \end{enumerate}
  (Dies sind die Eigenschaften aus dem Beweis von \ref{sk-char}.)  Die
  Existenz solcher Zusammenziehungen als Axiom zu setzen, bedeutet im
  Wesentlichen, nur Triangulierungen zu erlauben, bei denen die
  $U(\sigma)$-Mengen ``sich nicht selbst wieder treffen'' und damit im
  Wesentlichen wieder mit Simplizialkomplexen zu arbeiten.
\end{bem}

\end{document}
