Beweis Theorem ch02: Alles geht vmtl. auch mit $\Der^+$.

Nach der vorangegangenen Proposition ist das Bild der Einbettung
$\iota$ dicht in $\DerbskK$. Ein Dach $(A \from^{qi} X \to B)$ in der
Ore-Lokalisierung mit $A,B \in \sKons$ und $X \in \DerskK$ kann
ebenfalls mit der vorangegangenen Proposition mit $(Y \qi X),
Y \in \sKons$ erweitert werden, entspricht also einem Dach in
$\sKons$. Somit ist die Einbettung auch volltreu.

Wiki: am besten Umkehrfunktor schon wie KS mitnehmen, nicht kanonisch
aus obigem konstruierbar. Also:

Anders als KS arbeiten wir nicht mit KS, Rem. 1.7.12, das uns auf die
beschränkten derivierten Kategorien beschränken würde, sondern direkt
in $Der^+$. Mit der Exaktheit der Einbettung $\iota$ und der
Grothendieck-Spektralsequenz gilt:

\[ R \beta \circle \iota \cong R (\beta \circle \iota) \cong R id = id, \]

da welke Garben $\beta$-azyklisch sind, sowie

\[ \iota \circle R \beta \cong R(\iota \circle \beta) \cong id \]

nach der vorangegangenen Proposition.

--------------------------------------------------------------------------------

Die Aussage ist äquivalent zum folgenden Fortsetzungsresultat: Für
alle $U \open X$ ist die Restriktion ein Isomorphismus

\[ \Gamma(U \times I, F) \iso \Gamma(U \times \{t\}, F). \]

Denn ist die Koeinheit der Adjunktion ein Isomorphismus $\pi^* \pi_*
F \iso F$, so bestimmen wir die Schnitte über $U \times \{t\}$ wie
folgt: Sei $\iota: U \times \{t\} \inj X \times I$ die Inklusion. Wir
bemerken, dass $\pi \circ \iota$ die Inklusion von $U$ nach $X$ ist
und erhalten:

\[ \Gamma(U \times \{t\}, F)
   = \Gamma \iota^* F
   \iso \Gamma \iota^* \pi^* \pi_* F
   = \Gamma(U, \pi_* F)
   = \Gamma(U \times I, F). \]

Andersherum folgt die Bijektivität der Koeinheit der Adjunktion aber
auch aus dem Fortsetzungsresultat, denn wir können sofort den
Isomorphismus auf den Halmen über $(x, t) \in X \times I$ zeigen:

\[ (\pi^* \pi_* F)_{(x,t)}
   \iso (\pi_* F)_x
   = \colf\limits_{U \ni x} \Gamma(U \times I, F)
   \iso \colf\limits_{U \ni x} \Gamma(U \times \{t\}, F)
   \iso \colf\limits_{V \ni (x, t)} F(V)
   = F_{(x, t)}. \]

Dabei erhalten wir die Surjektivität von
$F(V) \to \Gamma(U \times \{t\}, F)$ aus der Bijektivität der
Verknüpfung
\[ \Gamma(U \times I, F) \to F(V) \to \Gamma(U \times \{t\}, F) \]
und die Injektivität aus der Eigenschaft, dass bereits die faserweise
stetige Fortsetzung eindeutig ist nach der Konstantheit der
Einschränkungen von $F$ auf die Fasern von $\pi$.

--------------------------------------------------------------------------------

% Weiter bei C-wertigen Garben:

\subsection{Diagrammkategorien von Garben}

Sei $I$ eine kleine Kategorie. Wir bezeichnen einen Funktor $F: I \to
C$ als Diagramm in $C$ der Form $I$. Wie können wir solche
Diagrammkategorien in $\EnsX$ charakterisieren?

\begin{prop}
  Sei $I$ eine kleine Kategorie. Dann gilt
  \[ \EnsX^I \qiso (\Ens^I)_{/X}. \]
\end{prop}
\begin{proof}
  Zunächst folgt aus dem Exponentialgesetz für Funktoren
  $\pEnsX^I \iso \p (\Ens^I)_{/X}$ und dann die Aussage mit der
  Garbenbedingung als Limes und der objektweisen Definition von
  Limites in Funktorkategorien.
\end{proof}

--------------------------------------------------------------------------------

% geom. Realisierung von simplizialen Garben

% TODO Def. simpliziales Objekt
% TODO Def. geom. Realisierung sEns \to CGHaus
% TODO CGHaus und Kelley-Produkt

% schnittweise (halmweise) geom. Realiserung von sEns/X \to CGHaus/X

% herkömmliche Garben als Top/X mit diskretem Halm
