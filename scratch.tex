Beweis Theorem ch02: Alles geht vmtl. auch mit $\Der^+$.

Nach der vorangegangenen Proposition ist das Bild der Einbettung
$\iota$ dicht in $\DerbskK$. Ein Dach $(A \from^{qi} X \to B)$ in der
Ore-Lokalisierung mit $A,B \in \sKons$ und $X \in \DerskK$ kann
ebenfalls mit der vorangegangenen Proposition mit $(Y \qi X),
Y \in \sKons$ erweitert werden, entspricht also einem Dach in
$\sKons$. Somit ist die Einbettung auch volltreu.

Wiki: am besten Umkehrfunktor schon wie KS mitnehmen, nicht kanonisch
aus obigem konstruierbar. Also:

Anders als KS arbeiten wir nicht mit KS, Rem. 1.7.12, das uns auf die
beschränkten derivierten Kategorien beschränken würde, sondern direkt
in $Der^+$. Mit der Exaktheit der Einbettung $\iota$ und der
Grothendieck-Spektralsequenz gilt:

\[ R \beta \circle \iota \cong R (\beta \circle \iota) \cong R id = id, \]

da welke Garben $\beta$-azyklisch sind, sowie

\[ \iota \circle R \beta \cong R(\iota \circle \beta) \cong id \]

nach der vorangegangenen Proposition.
