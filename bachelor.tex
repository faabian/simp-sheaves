\documentclass[a4paper]{article}
% \usepackage[left=3cm,right=3cm,top=3cm,bottom=2cm]{geometry} % page settings
\usepackage{amsmath}
\usepackage{amssymb}
\usepackage{amsthm}
\usepackage[textsize=small]{todonotes}
% \usepackage{etoolbox}
\usepackage[ngerman]{babel}
\usepackage[utf8]{inputenc}
% \usepackage{framed}
% \usepackage{fancyhdr}
% \usepackage{tikz}

\setlength{\parskip}{\medskipamount}
\setlength{\parindent}{0pt}

% \newcounter{satz}

\newtheorem{satz}{Satz}[section]
\newtheorem{defn}[satz]{Definition}
\newtheorem{theorem}[satz]{Theorem}
\newtheorem{prop}[satz]{Proposition}
\newtheorem{cor}[satz]{Korollar}


\newcommand{\N}{\mathbb{N}}
\newcommand{\R}{\mathbb{R}}
\newcommand{\op}{^{op}}
\newcommand{\del}{\partial}
\newcommand{\et}{\acute{e}t}

\begin{document}

\title{Geometrische Realisierung simplizialer Garben}
\author{Fabian Glöckle}
\date{\today}
\maketitle

\section{Einleitung}

Ziel der vorliegenden Arbeit ist es, eine geometrische Anschauung für
Diagrammkategorien Abelscher Garben zu entwickeln, wie sie etwa in der
Theorie der Derivatoren eine Rolle spielen.

\section{Simplizialkomplexe von Garben}

\section{Simpliziale Objekte in der Kategorie der Garben}

\section{Diagramme in der Kategorie der Garben}

\begin{thebibliography}{9}

%% \bibitem{lamport94}
%%   L. C. Grove, C. T. Benson,
%%   Finite Reflection Groups,
%%   Graduate Texts in Mathematics Vol. 99,
%%   2nd edition,
%%   Springer-Verlag (1985).

\end{thebibliography}

\end{document}
