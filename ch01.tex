% Emacs mode: -*-latex-*-

\chapter{Garben auf Simplizialkomplexen}
\label{ch:simp-comp}

\section{Simplizialkomplexe von Garben}
\label{sec:simp-comp-ord}

Bei Simplizialkomplexen handelt es sich um eine kombinatorisch
einfache und geometrisch zugängliche Klasse von
Diagrammkategorien. Ersteres liegt an der Interpretation von
Simplizialkomplexen als halbgeordnete Mengen, letzteres an der
Ordnungstopologie, mit der halbgeordnete Mengen versehen werden
können. Wir werden in diesem Abschnitt zeigen, inwiefern Diagramme von
Garben auf einem topologischen Raum $X$ von der Form eines
Simplizialkomplexes $\K$ als Garben auf dem Produktraum $\K \times X$
interpretiert werden können. Der Abschnitt dient auch als pädagogische
Einführung in die im weiteren Verlauf verwendeten Begriffe und
Methoden, und soll Modell und Anschauung für die späteren
Verallgemeinerungen sein.
\begin{defn}
  Ein \emph{Simplizialkomplex} ist eine Menge $E$, genannt die Ecken
  des Simplizialkomplexes, samt einem System $ \K \subset
  \mathcal{P}(E) $ von nichtleeren endlichen Teilmengen von $E$,
  genannt die Simplizes des Simplizialkomplexes, derart, dass gilt:
  \begin{itemize}
  \item jede einelementige Menge ist ein Simplex (d.h. $\{e\} \in
    \K \text{ für alle } e \in E$) und
  \item ist $L \in \K$ ein Simplex und $K \subset L$ eine
    nichtleere Teilmenge, so ist $K \in \K$ ein Simplex.
  \end{itemize}
\end{defn}
Ein Simplizialkomplex ist somit insbesondere eine halbgeordnete Menge
mit der Mengeninklusion als Halbordnung.

Wir erinnern an die Interpretation von halbgeordneten Mengen als
Kategorien. Gegeben eine halbgeordnete Menge $X$ definiere die
Kategorie $C_X$ bestehend aus Objekten $\Ob(C_X) = X$ mit einem
eindeutigen Morphismus $a \to b$ genau dann, wenn $a \leq b$ bezüglich
der Halbordnung. Funktoren der zugehörigen Kategorien $C_X \to C_Y$
entsprechen dann monotonen Abbildungen $X \to Y$. Wir erhalten:
\begin{lemma} \label{poset-cat}
  Obige Konstruktion liefert eine volltreue Einbettung
  \begin{align*}
    \poset &\to \Cat, \\
    X &\mapsto C_X.    
  \end{align*}
\end{lemma}
Hierbei bezeichnet $\Cat$ die Kategorie der kleinen Kategorien,
d.h. der Kategorien $C$, für die $\Ob(C)$ eine Menge ist und $\poset$
die Kategorie der halbgeordneten Mengen mit monotonen Abbildungen als
Morphismen.

Simplizialkomplexe können also als Kategorien aufgefasst werden und
definieren somit Diagrammkategorien.
\begin{defn}
  Sei $C$ eine Kategorie und $\K$ ein Simplizialkomplex
  aufgefasst als Kategorie. Wir nennen einen Funktor $\K \to
  C$ einen Simplizialkomplex in $C$ der Form $\K$.
\end{defn}
Die Simplizialkomplexe in $C$ bilden als Funktorkategorie eine
Kategorie. Wir notieren die Kategorie der Funktoren $F: A \to B$ mit
$B^A$ oder $[A, B]$. Funktorkategorien der Form $[C\op, \Ens]$ heißen
auch \emph{Prägarben} auf $C$.

Ein Simplizialkomplex in der terminalen Kategorie dasselbe wie ein
gewöhnlicher Simplizialkomplex. Im Folgenden interessieren wir uns für
Simplizialkomplexe von Garben oder Prägarben über einem topologischen
Raum $X$, d.~h. für die Fälle $C = \EnsX$ bzw. $C = \pEnsX$. Den Fall
der Prägarben verstehen wir sofort:
\begin{lemma}
  Das Exponentialgesetz von Funktoren liefert eine Äquivalenz von
  Kategorien
  \[
    [\K, \pEnsX] \qiso {[\K \times \OffX\op, \Ens]}
  \]    
  zwischen den Simplizialkomplexen von Prägarben auf $X$ und den
  Prägarben auf $\K\op \times \OffX$.
\end{lemma}
Hierbei bezeichnet $\OffX$ die durch Mengeninklusion halbgeordnete
Menge der offenen Mengen in $X$.
\begin{proof}
  Das folgt direkt aus $\pEnsX = [\OffX\op, \Ens]$ und dem
  Exponentialgesetz für Kategorien:
  \[ [A, [B, C]] \iso {[A \times B, C]}. \]
\end{proof}

Wir möchten Simplizialkomplexe von Garben auf $X$ als Garben auf einem
geeigneten topologischen Raum darstellen. Dies würde gelingen, wenn
$\K\op \times \OffX$ wieder die Kategorie der offenen Mengen eines
topologischen Raums ist. Es ist aber im Allgemeinen bereits in $\K$
die Vereinigungseigenschaft von Topologien verletzt. Stattdessen
erlaubt uns die Ordnungstopologie auf $\K$, das System $\K\op \times
\OffX$ als Basis der Topologie von $\K \times X$ aufzufassen.
\begin{defn}
  Sei $(X, \leq)$ eine halbgeordnete Menge. Die Topologie auf $X$,
  deren Basis aus Mengen der Form $(\geq \sigma) = \set{\tau \in
    X}{\tau \geq \sigma}$ (für $\sigma \in X$) besteht, heißt die
  \emph{Ordnungstopologie} auf $X$.
\end{defn}
Eine Menge $U \subset X$ ist bezüglich der Ordnungstopologie also
genau dann offen, wenn sie nach oben abgeschlossen ist, d. h. wenn
gilt
\[ U = \bigcup_{x \in U} (\geq x). \]
Wir prüfen, dass es sich um die Basis einer Topologie handeln kann:
\begin{lemma}
  Sei $(X, \leq)$ eine halbgeordnete Menge. Dann lassen sich endliche
  Schnitte im System der Mengen $(\geq \sigma)$, $\sigma \in X$, als
  Vereinigungen von Mengen in diesem System schreiben und $X$ wird
  durch die Mengen des Systems überdeckt.
\end{lemma}
\begin{proof}
  Es gilt $X = \bigcup_\sigma (\geq \sigma)$ und
  \[ (\geq \sigma) \cap (\geq \tau)
  = \bigcup_{x \in (\geq \sigma) \cap (\geq \tau)} (\geq x).
  \]
\end{proof}
Für $X = \K$ einen Simplizialkomplex vereinfachen sich die Schnitte:
\begin{equation} \label{eq:simp-comp-meets}
  (\geq \sigma) \cap (\geq \tau) = (\geq (\sigma \cup \tau))
  \text{ oder }
  (\geq \sigma) \cap (\geq \tau) = \emptyset.
\end{equation}

\begin{lemma} \label{ord-functor}
  Das Versehen mit der Ordnungstopologie definiert einen Funktor
  \[ \Ord: \poset \to \Top . \]
\end{lemma}
\begin{proof}
  Ist $f: X \to Y$ ein Morphismus halbgeordneter Mengen, so besteht
  das Urbild von $(\geq \sigma)$, $\sigma \in Y$ aus allen $\tau \in
  X$ mit $f(\tau) \geq \sigma$. Dies ist eine nach oben abgeschlossene
  Menge, also offen in $\Ord X$.
\end{proof}
Wir werden den Funktor $\Ord$ für Simplizialkomplexe $\K$ in der
Notation unterschlagen und sie direkt als topologische Räume
auffassen. Bezeichne nun $\B$ die Basis der Produkttopologie von $\K
\times X$ bestehend aus Produktmengen der Form $(\geq \sigma) \times
U$ mit $\sigma \in \K$, $U \open X$ als halbgeordnete Menge mit
Inklusionen als Morphismen. Der Funktor
\begin{align*} \label{eq:prod-base}
  \K\op \times \OffX &\to \B, \\
  (\sigma, U) &\mapsto (\geq \sigma) \times U
\end{align*}
ist dann ein Isomorphismus von Kategorien, denn ein Umkehrfunktor
projiziert eine Produktmenge auf die beiden Faktoren und wählt im
ersten Faktor das eindeutige minimale Element. Die Inklusion ist in
beiden Kategorien dieselbe, da $\sigma \geq \tau$ genau dann gilt,
wenn $(\geq \sigma) \subset (\geq \tau)$.

In $\K \times X$ gibt es zu viele offene Mengen für eine Aussage der
Art
\[
  [\K, \pEnsX] \qiso \p\Ens_{/\K \times X}.
\]
Allerdings können wir bei der Äquivalenz
\begin{equation} \label{eq:simp-comp-presheaf}
  [\K, \pEnsX]
  \qiso {[\K \times \OffX\op, \Ens]}
  \qiso {[\B\op, \Ens]}
\end{equation}
die rechte Seite als eine Garbenkategorie verstehen, wenn Garben schon
auf einer Basis der Topologie eindeutig festgelegt sind. Das ist wegen
der Verklebungseigenschaft von Garben tatsächlich der Fall.
\begin{defn}
  Sei $\B$ eine Basis eines topologischen Raumes $X$.  Wir bezeichnen
  die volle Unterkategorie der Prägarben auf $\B$, die die
  Verklebungseigenschaft von Garben für Überdeckungen in $\B$
  erfüllen, als die Kategorie der Garben auf $\B$ und notieren sie mit
  $\EnsB$.
\end{defn}
Konkret erfüllen Garben $F \in \EnsB$ also die folgende Eigenschaft:
\begin{quote}
  Ist $U = \bigcup_{i \in I} V_i$ eine Vereinigung mit $U, V_i \in
  \B$ und sind $s_i \in F(V_i)$ Schnitte mit
  übereinstimmenden Restriktionen $s_i |_{V_i \cap V_j} = s_j |_{V_i
    \cap V_j}$ für alle $i, j$, so gibt es genau einen Schnitt $s \in
  F(U)$ mit $s |_{V_i} = s_i$.
\end{quote}
Oder äquivalent, falls wir für $(V_i)_{i \in I}$ eine gesättigte
Überdeckung ist und wir als Systemmorphismen die Restriktionen wählen:
\[ F(U) = \lim_{i \in I} F(V_i) . \]

\begin{satz} \label{sheaf-on-basis}
  Sei $X$ ein topologischer Raum mit Basis $\B$. Dann gibt es eine
  Äquivalenz von Kategorien
  \[ \EnsX \qiso \EnsB , \]
  gegeben durch die Einschränkung auf $\B \subset \OffX$.
\end{satz}
\begin{proof}
  Wir konstruieren einen Quasi-Inversen: Sei dazu $F \in \EnsB$ und $U
  = \bigcup_{i \in I} U_i$ eine gesättigte Überdeckung von $U \open X$
  durch Basismengen $U_i \in \B$. Wir setzen $\hat{F}(U) = \lim
  F(U_i)$ und prüfen die Wohldefiniertheit. Sei $U = \bigcup_{j \in J}
  V_j$ eine weitere solche Überdeckung von $U$ durch Basismengen $V_j
  \in \B$. Dann gilt nach der Garbeneigenschaft auf Basismengen:
  \[ \lim_i F(U_i)
  \iso \lim_i \lim_j F(U_i \cap V_j)
  \iso \lim_j \lim_i F(U_i \cap V_j)
  \iso \lim_j F(V_j) .
  \]
  Die Zuordnung $F \mapsto \hat{F}$ ist also wohldefiniert. Das Bild
  $\hat{F}$ ist tatsächlich eine Garbe, denn falls $U = \bigcup_{i \in
    I} U_i$ eine unter endlichen Schnitten stabile Überdeckung durch
  offene Mengen und $U_i = \bigcup_{j} V_{ij}$ jeweils eine unter
  endlichen Schnitten stabile Überdeckung durch Basismengen $V_{ij}
  \in \B$ ist, so gilt
  \[ \hat{F}(U) = \lim_{i,j} F(V_{ij}) \iso \lim_i \hat{F}(U_i) \]
  zuerst nach der Definition von $\hat{F}(U)$ und dann wieder nach der
  Transitivität von Limites und der Definition der $\hat{F}(U_i)$.
  
  Die Funktorialität unserer Zuordnung folgt direkt aus der
  Funktorialität des Limes. Da für eine Basismenge $U \in \B$ mit der
  offensichtlichen Überdeckung natürlich $\hat{F}(U) = \lim F(U) =
  F(U) $ gilt, handelt es sich tatsächlich um einen Quasi-Inversen.
\end{proof}

Damit können wir die Hauptaussage des Abschnitts beweisen.
\begin{satz} \label{simp-comp-sheaf}
  Sei $\K$ ein Simplizialkomplex, $X$ ein topologischer Raum und $\B$
  die Basis der Produkttopologie von $\K \times X$ aus
  \autoref{eq:prod-base}.

  Die Äquivalenz $[\K, \pEnsX] \qiso [\B\op, \Ens]$ aus
  \autoref{eq:simp-comp-presheaf} schränkt ein zu einer Äquivalenz
  voller Unterkategorien
  \[
    [\K, \EnsX] \qiso \EnsB
  \]
  Insbesondere liefert die Verknüpfung mit \ref{sheaf-on-basis} eine
  Äquivalenz von Kategorien
  \[ [\K, \EnsX] \qiso \Ens_{/\K \times X} . \]
\end{satz}
\begin{proof}
  Wir müssen zeigen, dass $[\K, \EnsX]$ und $\EnsB$ durch äquivalente
  Bedingungen definierte volle Unterkategorien von $[\K, \pEnsX]$
  bzw. $[\B\op, \Ens]$ sind.

  In $[\K, \pEnsX]$ wird die Untekategorie der Simplizialkomplexe von
  Garben dadurch definiert, dass für festes $\sigma \in \K$ die
  Garbenbedingung für die $U \open X$ erfüllt sein muss, während in
  $[\B\op, \Ens]$ die Garbenbedingung für beliebige Basismengen $(\geq
  \sigma) \times U$ gefordert wird. Tatsächlich sind aber beide
  äquivalent, da im Fall einer Überdeckung $\mathcal{U}$ einer
  Basismenge $(\geq \sigma) \times U$ durch Basismengen $(\geq \tau_i)
  \times U_i$ eine Teilüberdeckung $\mathcal{V} \subset \mathcal{U}$
  aus Produktmengen mit $\tau_i = \sigma$ gewählt werden kann. Ein
  verträgliches Tupel aus Schnitten über Mengen aus $\mathcal{U}$
  entspricht dann einem verträglichem Tupel aus Schnitten über Mengen
  aus $\mathcal{V}$ und die Garbenbedingung für Basismengen folgt aus
  der für festes $\sigma \in \K$.
\end{proof}
Wir geben zum Abschluss die nicht-relative Version dieser Aussage an,
mit einem Beweis, der das einfache Vorgehen des vorangehenden Beweis
noch einmal systematischer darstellt. Mit den Techniken aus
\autoref{sec:gen-sheaves} folgt daraus umgekehrt auch die relative
Version und man mag bereits erkennen, wie das generelle Vorgehen zur
``Relativierung'' solcher Aussagen ist.
\begin{defn}
  Sei $(X, \leq)$ eine halbgeordnete Menge und $S \subset X$ eine
  Teilmenge. Ein Element $m \in X$ heißt Infimum von $S$, falls es
  eine größte untere Schranke von $S$ ist, d. h. $m \leq x$ für alle
  $x \in S$ und falls $m' \leq x$ für alle $x \in S$, dann $m' \leq
  m$.
\end{defn}
Infima sind Limites in der halbgeordneten Menge aufgefasst als
Kategorie. Das Infimum einer Teilmenge $S \in X$ muss nicht
existieren, ist in diesem Fall allerdings eindeutig. Ein
Simplizialkomplex besitzt binäre Infima. Die Basis aus $(\geq
\sigma)$-Mengen der Ordnungstopologie einer halbgeordneten Menge mit
binären Infima ist schnittstabil, vergleiche Gleichung
\ref{eq:simp-comp-meets}.
\begin{satz} \label{sheaf-order-top}
  Sei $\K \in \poset$ mit binären Infima. Dann ist der Funktor
  \[
  [\K, \Ens] \to \EnsB \qiso \Ens_{/\Ord \K}
  \]
  für $\B$ die Basis der Ordnungstopologie eine Äquivalenz von
  Kategorien.
\end{satz}
\begin{proof}
   Da $\sigma \leq \tau$ die umgekehrte Inklusion $(\geq \sigma)
   \supset (\geq \tau)$ impliziert, gilt $\K \iso \B\op$. Die
   Garbenbedingung für $\B$ ist leer, denn jede Überdeckung von $(\geq
   \sigma)$ durch Basismengen enthält $(\geq \sigma)$, $(\geq \sigma)$
   ist also initial in einer solchen Überdeckung und die Limites über
   die beiden Systeme stimmen überein.
\end{proof}
