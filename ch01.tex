% Emacs mode: -*-latex-*-
% include latex header (\usepackage, \newcommand etc.) 
\documentclass[a4paper]{article}
% \usepackage[left=3cm,right=3cm,top=3cm,bottom=2cm]{geometry} % page settings
\usepackage{amsmath}
\usepackage{amssymb}
\usepackage{amsthm}
\usepackage{etoolbox}
\usepackage[ngerman]{babel}
\usepackage[utf8]{inputenc}
\usepackage{mathtools}

\setlength{\parskip}{\medskipamount}
\setlength{\parindent}{0pt}

\newtheorem{theorem}{Theorem}
\newtheorem{lemma}[theorem]{Lemma}
\newtheorem{prop}[theorem]{Proposition}
\newtheorem{kor}[theorem]{Korollar}
\newtheorem{satz}[theorem]{Satz}
\newtheorem{defn}[theorem]{Definition}

\DeclareMathOperator{\Cat}{Cat}
\DeclareMathOperator{\poset}{poset}
\DeclareMathOperator{\EnsX}{Ens_{/X}}
\DeclareMathOperator{\pEnsX}{pEns_{/X}}
\DeclareMathOperator{\AbX}{Ab_{/X}}
\DeclareMathOperator{\pAbX}{pAb_{/X}}
\DeclareMathOperator{\OffX}{Off_X}
\DeclareMathOperator{\Ens}{Ens}
\DeclareMathOperator{\Ob}{Ob}
\DeclareMathOperator{\Der}{Der}
\DeclareMathOperator{\Ab}{Ab}
\DeclareMathOperator{\sKons}{s-Kons}
\DeclareMathOperator{\Ket}{Ket}
\DeclareMathOperator{\EnsB}{Ens_{/\B}}

\newcommand{\B}{\mathcal{B}}
\newcommand{\op}{^\mathrm{op}}
\newcommand{\iso}{\xrightarrow{\sim}}
\newcommand{\qiso}{\xrightarrow{\approx}}
\newcommand{\open}{\subset\kern-0.58em\circ}  % only possible in math mode
\newcommand{\K}{\mathcal{K}}
\newcommand{\Kreal}{|\K|}
\newcommand{\Z}{\mathbb{Z}}
\newcommand{\DerAbK}{\Der(\Ab_{/\Kreal})}
\newcommand{\DerskK}{\Der_{\mathrm{sk}}(\Kreal)}
\newcommand{\AbKr}{\Ab_{/\Kreal}}
\newcommand{\sKonsK}{\sKons(\K)}


\begin{document}

\title{Simpliziale Garben}
\author{Fabian Glöckle}
\date{\today}
% \maketitle

\chapter{Simplizialkomplexe von Garben}

\begin{defn}
  Ein Simplizialkomplex ist eine Menge $E$, genannt die Ecken des
  Simplizialkomplexes, samt einem System $ \mathcal{K} \subset
  \mathcal{P}(E) $ von nichtleeren endlichen Teilmengen von $E$,
  genannt die Simplizes des Simplizialkomplexes, derart, dass gilt:
  \begin{itemize}
  \item Jede einelementige Menge ist ein Simplex (d.h. $\{e\} \in
    \mathcal{K} \text{ für alle } e \in E$) und
  \item ist $L \in \mathcal{K}$ ein Simplex, $K \subset L$ nichtleere
    Teilmenge, so $K \in \mathcal{K}$.
  \end{itemize}
\end{defn}

Ein Simplizialkomplex ist somit insbesondere eine halbgeordnete Menge
mit der Mengeninklusion als Halbordnung.

Wir erinnern an die Interpretation von halbgeordneten Mengen als
Kategorien. Gegeben eine halbgeordnete Menge $X$ definiere die
Kategorie $C_X$ bestehend aus Objekten $\Ob(C_X) = X$ mit einem
eindeutigen Morphismus $a \to b$ genau dann, wenn $a \leq b$ bezüglich
der Halbordnung. Wir erhalten:

\begin{lemma}
  Obige Konstruktion liefert eine volltreue Einbettung
  \begin{align*}
    \poset &\to \Cat, \\
    X &\mapsto C_X.    
  \end{align*}
\end{lemma}

Hierbei bezeichnet $\Cat$ die Kategorie der kleinen Kategorien,
d.h. der Kategorien $C$, für die $\Ob(C)$ eine Menge ist und $\poset$
die Kategorie der halbgeordneten Mengen mit monotonen Abbildungen als
Morphismen.

\begin{proof}
  Offenbar ist $C_X$ tatsächlich eine kleine Kategorie. Funktoren $C_X
  \to C_Y$ sind Abbildungen $f$ auf Objekten, mit der Eigenschaft,
  dass es einen Morphismus $f(a) \leftarrow f(b)$ in $C_Y$ gibt, wann
  immer $a \leftarrow b$ in $C_X$. Das ist aber gerade die Eigenschaft
  einer monotonen Abbildung.
\end{proof}

Nun können wir Simplizialkomplexe von Garben definieren.

\begin{defn}
  Sei $C$ eine Kategorie, $\mathcal{K}$ ein Simplizialkomplex
  aufgefasst als Kategorie. Wir nennen einen Funktor $\mathcal{K} \to
  C$ einen Simplizialkomplex in $C$ der Form $\mathcal{K}$.
\end{defn}

Die Simplizialkomplexe in $C$ bilden wie jede Funktorenkategorie eine
Kategorie (Beweis einfach, vgl. ??). Wir notieren die Kategorie der
Funktoren $F: A \to B$ häufig mit $B^A$ oder auch $[A, B]$.

Allgemeiner nennen wir Funktorkategorien der Form $[C\op, \Ens]$ auch
Prägarben auf $C$.

Insbesondere ist ein Simplizialkomplex in der terminalen Kategorie
dasselbe wie ein gewöhnlicher Simplizialkomplex. Im Folgenden
interessieren wir uns besonders für die Fälle von Garben- und
Prägarbenkategorien $C = \EnsX$ bzw. $C = \pEnsX$ für $X$ einen
topologischen Raum.

\begin{lemma}
  Wir haben einen Isomorphismus von Kategorien
  \[
    [\mathcal{K}, \pEnsX] \iso {[\mathcal{K} \times \OffX\op, \Ens]}
  \]    
  zwischen den Simplizialkomplexen von Prägarben auf $X$ und den
  Prägarben auf $\mathcal{K}\op \times \OffX$.
\end{lemma}
Hierbei bezeichnet $\OffX$ die durch Mengeninklusion halbgeordnete
Menge der offenen Mengen in $X$.
\begin{proof}
  Das folgt direkt aus $\pEnsX = [\OffX\op, \Ens]$ und dem
  Exponentialgesetz für Kategorien:
  \[ [A, [B, C]] \iso {[A \times B, C]}. \]
\end{proof}

Unser Ziel wäre es nun, $\mathcal{K} \times \OffX$ wieder als
Kategorie von offenen Mengen eines topologischen Raums zu realisieren,
der funktoriell von $\mathcal{K}$ und $X$ abhängt. Das ist aber
natürlich im Allgemeinen nicht möglich, da bereits in $\mathcal{K}$
die Vereinigungseigenschaft von Topologien in der Regel verletzt
ist. Allerdings können wir $\mathcal{K} \times \OffX$ recht leicht zur
Basis einer Topologie von $\mathcal{K} \times X$ machen.
\begin{defn}
  Sei $(X, \leq)$ eine halbgeordnete Menge.  Wir bezeichnen die
  Topologie mit Basis den Mengen der Form $(\geq \sigma) = \{\tau \in
  X | \tau \geq \sigma\}$ (für $\sigma \in X$) als die
  Ordnungstopologie auf $X$.
\end{defn}
Wir prüfen, dass es sich um die Basis einer Topologie handeln kann:
\begin{lemma}
  Sei $(X, \leq)$ eine halbgeordnete Menge. Dann lassen sich endliche
  Schnitte im System der Mengen $(\geq \sigma)$, $\sigma \in X$ als
  Vereinigungen von Mengen in diesem System schreiben und $X$ wird
  durch die Mengen des Systems überdeckt.
\end{lemma}
\begin{proof}
  Offenbar gilt $X = \cup_\sigma (\geq \sigma)$ und $(\geq \sigma)
  \cap (\geq \tau) = \cup_{x \in (\geq \sigma) \cap (\geq \tau)} (\geq
  x)$.
\end{proof}
Eine Menge $U \subset X$ ist bezüglich der Ordnungstopologie also
genau dann offen, wenn sie nach oben abgeschlossen ist, d. h. wenn
gilt
\[ U = \cup_{x \in U} (\geq x). \]

Für $X = \mathcal{K}$ einen Simplizialkomplex vereinfachen sich die
Schnitte:
\[(\geq \sigma) \cap (\geq \tau) = (\geq (\sigma \cup \tau))
\text{ oder }
(\geq \sigma) \cap (\geq \tau) = \emptyset. \]

\begin{lemma} \label{ord-functor}
  Das Versehen mit der Ordnungstopologie definiert einen Funktor
  $\Ord: \poset \to \Top$.
\end{lemma}
\begin{proof}
  Ist $f: X \to Y$ ein Morphismus halbgeordneter Mengen, so besteht
  das Urbild von $(\geq \sigma)$, $\sigma \in Y$ aus allen $\tau \in
  X$ mit $f(\tau) \geq \sigma$. Dies ist eine nach oben abgeschlossene
  Menge, also offen in $\Ord X$.
\end{proof}
Wir werden den Funktor $\Ord$ für Simplizialkomplexe $\mathcal{K}$ in
der Notation unterschlagen und sie direkt als topologische Räume
auffassen.

Bezeichne nun $\B$ die Basis der Produkttopologie von
$\mathcal{K} \times X$ bestehend aus Produktmengen der Form $(\geq
\sigma) \times U$ mit $\sigma \in \mathcal{K}, U \open X$. Präziser
ist der Funktor
\begin{align*}
  \mathcal{K}\op \times \OffX &\to \B, \\
  (\sigma, U) &\mapsto (\geq \sigma) \times U
\end{align*}
ein Isomorphismus von Kategorien, denn ein Umkehrfunktor wird durch
die Projektionen einer Produktmenge auf ihre Faktoren und Wahl des
eindeutigen minimalen Elements im ersten Faktor gegeben. Die Inklusion
ist in beiden Kategorien dieselbe, da $\sigma \geq \tau$ genau dann,
wenn $(\geq \sigma) \subset (\geq \tau)$ gilt.

Nun gibt es zu viele offene Mengen in $\mathcal{K} \times X$ für eine
Aussage der Art
\[
  [\mathcal{K}, \pEnsX] \iso \mathrm{pEns}_{/\mathcal{K} \times X}.
\]
Allerdings können wir beim Isomorphismus
\[
  [\mathcal{K}, \pEnsX]
  \iso {[\mathcal{K} \times \OffX\op, \Ens]}
  \iso {[\B\op, \Ens]}
\]
die rechte Seite als eine Garbenkategorie verstehen, wenn in dieser
Objekte schon auf einer Basis der Topologie eindeutig festgelegt
sind. Das ist natürlich nicht der Fall für Prägarben, wohl aber durch
die Verklebungseigenschaft für Garben.
\begin{defn}
  Sei $\B$ eine Basis eines topologischen Raumes $X$.  Wir
  bezeichnen die volle Unterkategorie der Prägarben auf $\B$,
  die die Verklebungseigenschaft von Garben für Überdeckungen in
  $\B$ erfüllen, als die Kategorie der Garben auf
  $\B$ und notieren sie mit $\EnsB$.
\end{defn}

Konkret erfüllen Garben $F \in \EnsB$ also die folgende
Eigenschaft:
\begin{quote}
  Ist $U = \bigcup_{i \in I} V_i$ eine Vereinigung mit $U, V_i \in
  \B$ und sind $s_i \in F(V_i)$ Schnitte mit
  übereinstimmenden Restriktionen $s_i |_{V_i \cap V_j} = s_j |_{V_i
    \cap V_j}$ für alle $i, j$, so gibt es genau einen Schnitt $s \in
  F(U)$ mit $s |_{V_i} = s_i$.
\end{quote}
Oder äquivalent, falls wir für $(V_i)_{i \in I}$ eine unter endlichen
Schnitten stabile Überdeckung und als Systemmorphismen die
Restriktionen wählen:
\[ F(U) = \lim_{i \in I} F(V_i) . \]

\begin{satz}
  Sei $X$ ein topologischer Raum mit Basis $\B$. Dann gibt es eine
  Äquivalenz von Kategorien
  \[ \EnsX \qiso \EnsB \]
  gegeben durch die Einschränkung auf $\B \subset \OffX$.
\end{satz}
\begin{proof}
  Wir konstruieren einen Quasi-Inversen: Sei dazu $F \in \EnsB$ und $U
  = \bigcup_{i \in I} U_i$ eine unter endlichen Schnitten stabile
  Überdeckung von $U \open X$ durch Basismengen $U_i \in \B$. Wir
  setzen $\hat{F}(U) = \lim F(U_i)$ und prüfen die
  Wohldefiniertheit. Sei $U = \bigcup_{j \in J} V_j$ eine weitere
  solche Überdeckung von $U$ durch Basismengen $V_j \in \B$. Nun gilt
  nach der Garbeneigenschaft auf den Basismengen:
  \[ \lim_i F(U_i)
  \iso \lim_i \lim_j F(U_i \cap V_j)
  \iso \lim_j \lim_i F(U_i \cap V_j)
  \iso \lim_j F(V_j) .\]
  Unsere Zurdnung $F \mapsto \hat{F}$ ist also wohldefiniert. Das Bild
  $\hat{F}$ ist tatsächlich eine Garbe, denn falls $U = \bigcup_{i \in
    I} U_i$ eine unter endlichen Schnitten stabile Überdeckung durch
  offene Mengen und $U_i = \bigcup_{j} V_{ij}$ jeweils eine unter endlichen
  Schnitten stabile Überdeckung durch Basismengen $V_{ij} \in \B$ ist,
  so gilt
  \[ \hat{F}(U) = \lim_{i,j} F(V_{ij}) \iso \lim_i \hat{F}(U_i) \]
  zuerst nach der Definition von $\hat{F}(U)$ und dann wieder nach der
  Transitivität von Limites und der Definition der $\hat{F}(U_i)$.
  
  Die Funktorialität unserer Zuordnung folgt direkt aus der
  Funktorialität des Limes. Da für eine Basismenge $U \in \B$ mit der
  offensichtlichen Überdeckung natürlich $\hat{F}(U) = \lim F(U) =
  F(U) $ gilt, handelt es sich tatsächlich um einen Quasi-Inversen.
\end{proof}

\begin{satz} \label{sheaf-simp-compl}
  Der Funktor
  \[
  [\mathcal{K}, \EnsX] \to \EnsB \qiso \Ens_{/\mathcal{K} \times X}
  \]
  ist eine Äquivalenz von Kategorien.
\end{satz}
\begin{proof}
  Wir haben schon den Isomorphismus auf den Prägarbenkategorien
  \[ [\mathcal{K}, \pEnsX] \iso {[\B\op, \Ens]} \]
  und müssen nur noch zeigen, dass $[\mathcal{K}\op, \EnsX]$ und
  $\EnsB$ durch äquivalente Bedingungen definierte volle
  Unterkategorien sind.

  In $[\mathcal{K}\op, \pEnsX]$ wird die Untekategorie der
  Simplizialkomplexe von Garben dadurch definiert, dass für festes
  $\sigma \in \mathcal{K}$ die Garbenbedingung für die $U \open X$
  erfüllt sein muss, während in $[\B\op, \Ens]$ die Garbenbedingung
  für beliebige Basismengen $(\geq \sigma) \times U$ gefordert
  wird. Tatsächlich sind aber beide äquivalent, da im Fall einer
  Überdeckung $\mathcal{U}$ einer Basismenge $(\geq \sigma) \times U$
  durch Basismengen $(\geq \tau_i) \times U_i$ eine Teilüberdeckung
  $\mathcal{V} \subset \mathcal{U}$ aus Produktmengen mit $\tau_i =
  \sigma$ gewählt werden kann. Ein verträgliches Tupel aus Schnitten
  über Mengen aus $\mathcal{U}$ entspricht dann einem verträglichem
  Tupel aus Schnitten über Mengen aus $\mathcal{V}$ und die
  Garbenbedingung für Basismengen folgt aus der für festes
  $\sigma \in \mathcal{K}$.
\end{proof}

% TODO nichtrelative Version (Garben auf \Ord X, X \poset) erklären
% \label{sheaf-order-top}

\end{document}
