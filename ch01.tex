\documentclass[a4paper]{article}
% \usepackage[left=3cm,right=3cm,top=3cm,bottom=2cm]{geometry} % page settings
\usepackage{amsmath}
\usepackage{amssymb}
\usepackage{amsthm}
\usepackage{etoolbox}
\usepackage[ngerman]{babel}
\usepackage[utf8]{inputenc}


\setlength{\parskip}{\medskipamount}
\setlength{\parindent}{0pt}

\newtheorem{theorem}{Theorem}
\newtheorem{lemma}[theorem]{Lemma}
\newtheorem{prop}[theorem]{Proposition}
\newtheorem{kor}[theorem]{Korollar}
\newtheorem{satz}[theorem]{Satz}
\newtheorem{defn}[theorem]{Definition}

\DeclareMathOperator{\Cat}{Cat}
\DeclareMathOperator{\poset}{poset}
\DeclareMathOperator{\EnsX}{Ens_{/X}}
\DeclareMathOperator{\pEnsX}{pEns_{/X}}
\DeclareMathOperator{\AbX}{Ab_{/X}}
\DeclareMathOperator{\pAbX}{pAb_{/X}}
\DeclareMathOperator{\OffX}{Off_X}
\DeclareMathOperator{\Ens}{Ens}

\newcommand{\op}{\mathrm{op}}
\newcommand{\iso}{\xrightarrow{\sim}}
\newcommand{\open}{\subset\kern-0.58em\circ}  % only possible in math mode

\begin{document}

\title{Simpliziale Garben}
\author{Fabian Glöckle}
\date{\today}
% \maketitle

\section{Simplizialkomplexe von Garben}

\begin{defn}
  Ein Simplizialkomplex ist eine Menge $E$, genannt die Ecken des
  Simplizialkomplexes, samt einem System $ \mathcal{K} \subset
  \mathcal{P}(E) $ von nichtleeren endlichen Teilmengen von $E$,
  genannt die Simplizes des Simplizialkomplexes, derart, dass gilt:
  \begin{itemize}
  \item Jede einelementige Menge ist ein Simplex (d.h. $\{e\} \in
    \mathcal{K} \text{ für alle } e \in E$) und
  \item ist $L \in \mathcal{K}$ ein Simplex, $K \subset L$ nichtleere
    Teilmenge, so $K \in \mathcal{K}$.
  \end{itemize}
\end{defn}

Ein Simplizialkomplex ist somit insbesondere eine halbgeordnete Menge
mit der Mengeninklusion als Halbordnung.

Wir erinnern an die Interpretation von halbgeordneten Mengen als
Kategorien. Gegeben eine halbgeordnete Menge $X$ definiere die
Kategorie $C_X$ bestehend aus Objekten $Ob(C_X) = X$ mit einem
eindeutigen Morphismus $a \leftarrow b$ genau dann, wenn $a \leq b$ bezüglich
der Halbordnung. Wir erhalten:

\begin{lemma}
  Obige Konstruktion liefert eine volltreue Einbettung
  \begin{align*}
    \poset &\to \Cat, \\
    X &\mapsto C_X.    
  \end{align*}
\end{lemma}


Hierbei bezeichnet $\Cat$ die Kategorie der kleinen Kategorien,
d.h. der Kategorien $C$, für die $Ob(C)$ eine Menge ist und $\poset$
die Kategorie der halbgeordneten Mengen mit monotonen Abbildungen als
Morphismen.

\begin{proof}
  Offenbar ist $C_X$ tatsächlich eine kleine Kategorie. Funktoren $C_X
  \to C_Y$ sind Abbildungen $f$ auf Objekten, mit der Eigenschaft,
  dass es einen Morphismus $f(a) \leftarrow f(b)$ in $C_Y$ gibt, wann immer
  $a \leftarrow b$ in $C_X$. Das ist aber gerade die Eigenschaft einer
  monotonen Abbildung.
\end{proof}

Nun können wir Simplizialkomplexe von Garben definieren.

\begin{defn}
  Sei $C$ eine Kategorie, $\mathcal{K}$ ein Simplizialkomplex
  aufgefasst als Kategorie. Wir nennen einen Funktor $\mathcal{K}^\op \to
  C$ einen Simplizialkomplex in $C$ der Form $\mathcal{K}$.
\end{defn}

Die Simplizialkomplexe in $C$ bilden wie jede Funktorenkategorie eine
Kategorie (Beweis einfach, vgl. ??). Wir notieren die Kategorie der
Funktoren $F: A \to B$ häufig mit $B^A$ oder auch $[A, B]$.

Allgemeiner nennen wir Funktorkategorien der Form $[C^\op, \Ens]$ auch
Prägarben auf $C$.

Insbesondere ist ein Simplizialkomplex in der terminalen Kategorie
dasselbe wie ein gewöhnlicher Simplizialkomplex. Im Folgenden
interessieren wir uns besonders für die Fälle von Garben- und
Prägarbenkategorien $C = \EnsX$ bzw. $C = \pEnsX$ für $X$ einen
topologischen Raum.

\begin{lemma}
  Wir haben einen Isomorphismus von Kategorien
  \[
    [\mathcal{K}^\op, \pEnsX] \iso [(\mathcal{K} \times \OffX)^\op, \Ens]
  \]    
  zwischen den Simplizialkomplexen von Prägarben auf $X$ und den
  Prägarben auf $\mathcal{K} \times \OffX$.
\end{lemma}

Hierbei bezeichnet $\OffX$ die durch Mengeninklusion halbgeordnete
Menge der offenen Mengen in $X$.

\begin{proof}
  Das folgt direkt aus $\pEnsX = [\OffX^\op, \Ens]$ und dem
  Exponentialgesetz für Kategorien:
  \[ [A, [B, C]] \iso [A \times B, C]. \]
\end{proof}

Unser Ziel wäre es nun, $\mathcal{K} \times \OffX$ wieder als
Kategorie von offenen Mengen eines topologischen Raums zu realisieren,
der funktoriell von $\mathcal{K}$ und $X$ abhängt. Das ist aber
natürlich im Allgemeinen nicht möglich. (Beweis??) Allerdings können
wir $\mathcal{K} \times \OffX$ recht leicht zur Basis einer Topologie
von $\mathcal{K} \times X$ machen.

\begin{defn}
  Die Ordnungstopologie auf einer halbgeordneten Menge $X$ hat als
  abgeschlossene Mengen alle ``nach unten abgeschlossenen'' Mengen,
  d.h. Mengen $A \subset X$ mit der Eigenschaft, dass falls $b \in A$
  und $a \leq b$, so auch $a \in A$.
\end{defn}

Wir schreiben $\geq \sigma$ ... offene ...

\begin{lemma}
  Die Ordnungstopologie auf $X$ ist eine Topologie.
\end{lemma}

\begin{proof}
  ... schnittstabil ...
\end{proof}

Damit haben wir insbesondere auch Simplizialkomplexe mit einer
Ordnungstopologie versehen. Bezeichne dazu $\mathcal{B}$ die Basis der
Produkttopologie von $\mathcal{K} \times X$ bestehend aus
Produktmengen der Form $(\geq \sigma) \times U$ mit $\sigma \in
\mathcal{K}, U \open X$. Präziser ist der Funktor
  
\begin{align*}
  \mathcal{K} \times \OffX &\to \mathcal{B}, \\
  (\sigma, U) &\mapsto (\geq \sigma) \times U
\end{align*}

ein Isomorphismus von Kategorien, da auch der Begriff einer Inklusion
in beiden Katgeorien derselbe ist.

Es gibt zu viele offene Mengen in $\mathcal{K} \times X$ für eine
Aussage der Art
\[
  [\mathcal{K}^\op, \pEnsX] \iso \mathrm{pEns}_{/\mathcal{K} \times X}.
\]
Allerdings können wir beim Isomorphismus
\[
  [\mathcal{K}^\op, \pEnsX] \iso [(\mathcal{K} \times \OffX)^\op, \Ens] \iso [B^\op, \Ens]
\]
die rechte Seite als eine Garbenkategorie verstehen, wenn in dieser
Objekte schon auf einer Basis der Topologie eindeutig festgelegt
sind. Das ist natürlich nicht der Fall für $\pEnsX$, wohl aber durch
die Verklebungseigenschaft von Garben für $\EnsX$.

\begin{satz}
  Der Funktor
  \[
  [\mathcal{K}^\op, \EnsX] \to [B^\op, \Ens] \to \EnsX
  \]
  ist eine Äquivalenz von Kategorien.
\end{satz}
\begin{proof}
  Wir müssen nur noch zeigen, dass der hintere Funktor eine Äquivalenz
  von Kategorien ist.

  
\end{proof}

\end{document}
