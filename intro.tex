% Emacs mode: -*-latex-*-

\section*{Einleitung}
\addcontentsline{toc}{chapter}{Einleitung}

Diagrammkategorien von Garben auf topologischen Räumen treten immer
dann natürlich auf, wenn nicht nur das Verhalten einer einzelnen Garbe
unter einer bestimmten Operation von Interesse ist, sondern auch das
``Beziehungsgeflecht'' der beteiligten Garben untereinander, etwa bei
der Betrachtung kartesischer Quadrate von Garben auf einem
topologischen Raum. Ziel der vorliegenden Arbeit ist es, für einfache
Diagramme eine geometrische Interpretation von Diagrammkategorien von
Garben von Mengen auf einem topologischen Raum herzustellen. Dies wird
uns für zwei Klassen von Diagrammen gelingen: halbgeordnete Mengen vom
Typ eines Simplizialkomplexes sowie gerichtete Kategorien vom Typ
einer simplizialen Menge.

In \ref{ch:simp-comp} wird eine Interpretation von Diagrammen von
Mengen von der Form eines Simplizialkomplexes als Garben auf dem
zugehörigen topologischen Simplizialkomplex mit der Ordnungstopologie
vorgestellt. Uns interessieren auch die relativen Aussagen, inwiefern
Diagramme von Garben auf einem topologischen Raum $X$ von der Form
eines Simplizialkomplexes eine Garbe auf dem topologischen Produkt von
$X$ mit dem ordungstopologischen Simplizialkomplex beschreiben. Dies
wird in \ref{sec:simp-comp-ord} explizit vorgeführt. Der folgende
Abschnitt liefert eine geometrischere Charakterisierung dieser
Diagramme als die simplizial konstanten Garben auf der geometrischen
Realisierung des Simplizialkomplexes und die zugehörige Aussage für
die derivierten Kategorien, zunächst im nicht relativen Fall. Diesen
beantwortet \ref{sec:gen-sheaves}, indem eine allgemeine Technik zur
Relativierung solcher Aussagen vorgestellt wird.

Um diese Aussagen auf ihre Varianten für simpliziale Mengen und die
zugehörigen Diagramme zu übertragen, werden in \ref{ch:simp-sets}
simpliziale Mengen und ihre geometrische Realisierung
eingeführt. Dabei werden die grundlegenden Eigenschaften der
geometrischen Realisierung, ihre Formulierung mittels
nichtdegenerierter Simplizes und ihre Interpretation als iteratives
Verkleben von Zellen, ausführlich dargestellt. Für die Formulierung
der Exaktheitseigenschaften der Realisierung wird zudem gezeigt, dass
es sich um kompakt erzeugte Hausdorffräume handelt.

Die geometrische Realisierung simplizialer Mengen lässt sich
darstellen als Koende, eine universelle Konstruktion der
Kategorientheorie. Diese werden in \ref{ch:coends} eingeführt und als
eigenständige Objekte untersucht. Die Rechenregeln für Koenden, der
Koendenkalkül aus \ref{sec:coend-calc} wird sich im folgenden Kapitel
als wichtiges Hilfsmittel für ``abstract nonsense''-Beweise erweisen.

In \ref{ch:simp-sheaves} wird der Faden der Diagrammkategorien
wiederaufgenommen. Gesucht wird zunächst ein Adjungierter zur
geometrischen Realisierung simplizialer Garben aus
\ref{sec:simp-sheaves-real}. Dies gelingt über die allgemeine Dualität
zwischen Nerv und Realisierung, wenn die beteiligte Garbenkategorie
kartesisch abgeschlossen ist.  Für die Kategorie der Garben auf einem
topologischen Raum $X$ wird dies in \ref{sec:ensx-cart-closed}
gezeigt, für die Kategorie von Garben über variablen Basisräumen wird
in \ref{sec:enstop-cart-closed} ein einschränkender Grund formuliert,
wieso dies vermutlich nicht gilt. Für die korrekte Übertragung der
Aussagen über Diagrammkategorien von Simplizialkomplexen ist die
angesprochene Adjunktion allerdings nicht relevant. Vielmehr muss der
Begriff des einer simplizialen Menge zugeordneten Diagramms geklärt
werden, dies geschieht in \ref{sec:simp-set-sk}. Die Verallgemeinerung
der Aussagen über Diagrammkategorien kann dann für den Fall
gerichteter Kategorien vom Typ einer simplizialen Menge mittels
2-Limites von Kategorien bewiesen werden.

In der Notation folgt die Arbeit im Wesentlichen den Notationen aus
\cite{TG}. So steht der französischen Tradition folgend $\Ens$ für die
Kategorie der Mengen und $\EnsX$ für die Kategorie der Garben von
Mengen auf dem topologischen Raum $X$.

Die Darstellung der Aussagen zu schwach konstruierbaren Garben auf
Simplizialkomplexen folgt \cite{KS} und \cite{WS}, die zu simplizialen
Mengen \cite{GJ} und \cite{GM}. Das Kapitel zu Koenden ist stark an
\cite{Lore} angelehnt. Nicht mit Verweisen auf Literatur markierte
Aussagen sind eigenständig erarbeitet. Dies trifft insbesondere auf
weite Teile \ref{sec:simp-comp-ord}, \ref{sec:gen-sheaves} und
\ref{ch:simp-sheaves} zu. Beim Finden von Aussagen und Gegenbeispielen
und dem generellen Vertrautwerden mit dem Gebiet bin ich dennoch der
Autorschaft der Online-Enzyklopädie \emph{Nlab} und des
Frage-und-Antwort-Forums \emph{Mathoverflow} zu Dank
verpflichtet. Danken möchte ich auch meinem Betreuer Prof. Wolfgang
Soergel für die Stellung des interessanten Themas und die Freiheiten
bei der Bearbeitung und meinen Eltern für die persönliche und
finanzielle Unterstützung meines Studiums. Ich danke meinem Bruder
Jonathan Glöckle für die Formulierung von \ref{compact-open-discrete}
und Prof. Huber-Klawitter für die Beantwortung meiner Frage zu
\ref{sec:gen-sheaves}.
