% Emacs mode: -*-latex-*-

\section*{Einleitung}
\addcontentsline{toc}{chapter}{Einleitung}

Diagrammkategorien von Garben auf topologischen Räumen treten immer
dann natürlich auf, wenn nicht nur das Verhalten einer einzelnen Garbe
unter einer bestimmten Operation von Interesse ist, sondern auch das
``Beziehungsgeflecht'' der beteiligten Garben untereinander, etwa bei
der Betrachtung kartesischer Quadrate von Garben auf einem
topologischen Raum. Ziel der vorliegenden Arbeit ist es, für einfache
Diagramme eine geometrische Interpretation von Diagrammkategorien von
Garben von Mengen auf einem topologischen Raum herzustellen. Dies wird
uns für zwei Klassen von Diagrammen gelingen: halbgeordnete Mengen vom
Typ eines Simplizialkomplexes sowie gerichtete Kategorien vom Typ
einer simplizialen Menge.

In \autoref{ch:simp-comp} wird eine Interpretation von Diagrammen von
Mengen von der Form eines Simplizialkomplexes als Garben auf dem
zugehörigen topologischen Simplizialkomplex mit der Ordnungstopologie
vorgestellt. Uns interessieren auch die relativen Aussagen, inwiefern
Diagramme von Garben auf einem topologischen Raum $X$ von der Form
eines Simplizialkomplexes eine Garbe auf dem topologischen Produkt von
$X$ mit dem ordungstopologischen Simplizialkomplex beschreiben. Dies
wird in \autoref{sec:simp-comp-ord} explizit vorgeführt. Der folgende
Abschnitt liefert eine geometrische Charakterisierung dieser Diagramme
als die simplizial konstanten Garben auf der geometrischen
Realisierung des Simplizialkomplexes. Im Fall der zugehörigen
derivierten Kategorien verbessert sich die Aussage zu einer Äquivalenz
zu den derivierten Garben auf der geometrischen Realisierung mit
simplizial konstanter Kohomologie. Dies gelingt mittels der Technik
homotopieinjektiver Auflösungen auch für die unbeschränkte derivierte
Kategorie. Der relative Fall über einem topologischen Raum $X$ wird in
\autoref{sec:gen-sheaves} durch eine allgemeine Technik zur
Relativierung solcher Aussagen beantwortet.

Um diese Aussagen für Simplizialkomplexe auf simpliziale Mengen und
die zugehörigen Diagramme zu übertragen, werden in
\autoref{ch:simp-sets} simpliziale Mengen und ihre geometrische
Realisierung eingeführt. Dabei werden die grundlegenden Eigenschaften
der geometrischen Realisierung, ihre Formulierung mittels
nichtdegenerierter Simplizes und ihre Interpretation als iteratives
Verkleben von Zellen, ausführlich dargestellt. Die
Exaktheitseigenschaften der Realisierung werden in der Kategorie der
kompakt erzeugten Hausdorffräume behandelt.

Die geometrische Realisierung simplizialer Mengen lässt sich als ein
Koende darstellen, eine universelle Konstruktion der
Kategorientheorie. Koenden werden in \autoref{ch:coends} eingeführt
und als eigenständige Objekte untersucht. Die Rechenregeln für
Koenden, der Koendenkalkül aus \autoref{sec:coend-calc} wird sich im
folgenden Kapitel als wichtiges Hilfsmittel für \emph{``abstract
  nonsense''}-Beweise erweisen.

In \autoref{ch:simp-sheaves} wird der Faden der Diagrammkategorien
wiederaufgenommen. Gesucht wird zunächst ein Adjungierter zur
geometrischen Realisierung simplizialer Garben aus
\autoref{sec:simp-sheaves-real}. Dies gelingt über die allgemeine
Dualität zwischen Nerv und Realisierung, wenn die beteiligte
Garbenkategorie kartesisch abgeschlossen ist.  Für die Kategorie der
Garben auf einem topologischen Raum $X$ wird dies in
\autoref{sec:ensx-cart-closed} gezeigt, für die Kategorie von Garben
über variablen Basisräumen wird in \autoref{sec:enstop-cart-closed}
ein einschränkender Grund formuliert, wieso vermutlich keine solche
Struktur zu erwarten ist. Für die korrekte Übertragung der Aussagen
über Diagrammkategorien von Simplizialkomplexen ist die angesprochene
Adjunktion allerdings nicht relevant. Vielmehr muss der Begriff des
einer simplizialen Menge zugeordneten Diagramms geklärt werden, was in
\autoref{sec:simp-set-sk} geschieht. Die Verallgemeinerung der
Aussagen über Diagrammkategorien kann dann für den Fall gerichteter
Kategorien vom Typ einer simplizialen Menge mittels 2-Limites von
Kategorien bewiesen werden. Die Arbeit schließt mit Notizen zur
Übertragung der Ergebnisse über die derivierten Kategorien sowie zur
geometrischen Interpretation beliebiger Diagrammkategorien.

In der Notation folgt die Arbeit im Wesentlichen den Notationen aus
\cite{TG}. Insbesondere steht der französischen Tradition folgend
$\Ens$ für die Kategorie der Mengen und $\EnsX$ für die Kategorie der
Garben von Mengen auf dem topologischen Raum $X$. Die Äquivalenz von
Kategorien zwischen Garben von Mengen auf einem topologischen Raum $X$
und étalen Räumen $E \to X$ über $X$ wird in der Notation meistens
unterschlagen.

Die Darstellung der Aussagen zu schwach konstruierbaren Garben auf
Simplizialkomplexen folgt \cite{KS} und \cite{WS}, die zu simplizialen
Mengen \cite{GJ} und \cite{GM}. Das Kapitel zu Koenden ist stark an
\cite{Lore} angelehnt. Nicht mit Verweisen auf Literatur markierte
Aussagen sind eigenständig erarbeitet. Dies trifft insbesondere auf
weite Teile von\autoref{sec:simp-comp-ord}, \autoref{sec:gen-sheaves}
und \autoref{ch:simp-sheaves} zu. Beim Finden von Aussagen und
Gegenbeispielen und dem generellen Vertrautwerden mit dem Gebiet bin
ich dennoch der Autorschaft der Online-Enzyklopädie \emph{nLab} und
des Frage-und-Antwort-Forums \emph{Mathoverflow} zu Dank
verpflichtet. Danken möchte ich auch meinem Betreuer Prof. Wolfgang
Soergel für die Stellung des interessanten Themas, die inhaltlichen
Anregungen und die Freiheiten bei der Bearbeitung sowie meinen Eltern
für die persönliche und finanzielle Unterstützung meines Studiums. Ich
danke meinem Bruder Jonathan Glöckle für die Formulierung von
\ref{compact-open-discrete} und Prof. Huber-Klawitter für die
Beantwortung meiner Frage zu \autoref{sec:gen-sheaves}.
