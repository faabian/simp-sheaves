% include latex header (\usepackage, \newcommand etc.) 
\documentclass[a4paper]{article}
% \usepackage[left=3cm,right=3cm,top=3cm,bottom=2cm]{geometry} % page settings
\usepackage{amsmath}
\usepackage{amssymb}
\usepackage{amsthm}
\usepackage{etoolbox}
\usepackage[ngerman]{babel}
\usepackage[utf8]{inputenc}
\usepackage{mathtools}

\setlength{\parskip}{\medskipamount}
\setlength{\parindent}{0pt}

\newtheorem{theorem}{Theorem}
\newtheorem{lemma}[theorem]{Lemma}
\newtheorem{prop}[theorem]{Proposition}
\newtheorem{kor}[theorem]{Korollar}
\newtheorem{satz}[theorem]{Satz}
\newtheorem{defn}[theorem]{Definition}

\DeclareMathOperator{\Cat}{Cat}
\DeclareMathOperator{\poset}{poset}
\DeclareMathOperator{\EnsX}{Ens_{/X}}
\DeclareMathOperator{\pEnsX}{pEns_{/X}}
\DeclareMathOperator{\AbX}{Ab_{/X}}
\DeclareMathOperator{\pAbX}{pAb_{/X}}
\DeclareMathOperator{\OffX}{Off_X}
\DeclareMathOperator{\Ens}{Ens}
\DeclareMathOperator{\Ob}{Ob}
\DeclareMathOperator{\Der}{Der}
\DeclareMathOperator{\Ab}{Ab}
\DeclareMathOperator{\sKons}{s-Kons}
\DeclareMathOperator{\Ket}{Ket}
\DeclareMathOperator{\EnsB}{Ens_{/\B}}

\newcommand{\B}{\mathcal{B}}
\newcommand{\op}{^\mathrm{op}}
\newcommand{\iso}{\xrightarrow{\sim}}
\newcommand{\qiso}{\xrightarrow{\approx}}
\newcommand{\open}{\subset\kern-0.58em\circ}  % only possible in math mode
\newcommand{\K}{\mathcal{K}}
\newcommand{\Kreal}{|\K|}
\newcommand{\Z}{\mathbb{Z}}
\newcommand{\DerAbK}{\Der(\Ab_{/\Kreal})}
\newcommand{\DerskK}{\Der_{\mathrm{sk}}(\Kreal)}
\newcommand{\AbKr}{\Ab_{/\Kreal}}
\newcommand{\sKonsK}{\sKons(\K)}


\begin{document}

\title{Simpliziale Garben}
\author{Fabian Glöckle}
\date{\today}
% \maketitle

\section{Schwach konstruierbare Garben auf Simplizialkomplexen}

Ziel dieses Abschnitts ist die Charakterisierung schwach
konstruierbarer Garben auf Simplizialkomplexen und ihrer derivierten
Kategorie. Die Darstellung folgt im Wesentlichen \cite{KS} und \cite{WS}.

In diesem Abschnitt bezeichne $(V, \K)$ einen lokal-endlichen
Simplizialkomplex mit Eckenmenge $V$. Für einen Simplex $\sigma \in
\K$ definieren wir seine geometrische Realisierung $|\sigma| \subset
\R^V = \Ens(V, \R)$:

\[ |\sigma| = \{ x \in \R^V | x(v) = 0 \text{ für } v \notin \sigma,
x(v) > 0 \text{ für } v \in \sigma,
\sum_{v \in V} x(v) = 1 \}, \]

sowie die geometrische Realisierung $|\K| \subset \R^V$ von $\K$

\[ \bigcup_{\sigma \in \K} |\sigma|, \]

jeweils versehen mit der induzierten Topologie von $\R^V$.

Wir erhalten eine Abbildung

\[ p: |\mathcal{K}| \to \mathcal{K}, \]

genannt Simplexanzeiger oder Indikatorabbildung, der einem Punkt $x
\in |\mathcal{K}|$ in der geometrischen Realisierung den eindeutigen
Simplex $\sigma \in \mathcal{K}$ mit $x \in |\sigma|$ zuordnet.

\begin{lemma}
  Der Simplexanzeiger $p: |\K| \to \K$ ist stetig.
\end{lemma}
\begin{proof}
  Das Urbild einer Basismenge $(\geq \sigma)$ ist

  \[ p^{-1}((\geq \sigma)) = |\K| \cap \{x \in \R^V | x(v) > 0 \text{ für }
  v \in \sigma \}, \]

  der offene Stern um $\sigma$, den wir auch als $U(\sigma)$ notieren.
\end{proof}

% TODO: wo benötigen wir lok endl?

\begin{defn}
  Eine Garbe $F \in \AbKr$ heißt schwach $|\K|$-konstruierbar (oder
  kurz: schwach konstruierbar), falls für alle $\sigma \in \K$, die
  Einschränkungen $F|_{|\sigma|}$ konstante Garben sind. Wir
  bezeichnen die volle Unterkategorie der schwach konstruierbaren
  Garben in $\AbKr$ mit $\sKonsK$.

  Eine derivierte Garbe $F \in \DerAbK$ heißt schwach
  $|\K|$-konstruierbar, falls für alle $j \in \Z$ die
  Kohomologiegarben $H^j(F)$ schwach konstruierbar sind. Wir
  bezeichnen die volle Unterkategorie der schwach konstruierbaren
  derivierten Garben in $\DerAbK$ mit $\DerskK$.
\end{defn}

Wir bemerken zunächst:

\begin{lemma}
  Die Kategorie $\sKonsK$ ist abelsch.
\end{lemma}
\begin{proof}
  Durch den offensichtlichen Isomorphismus zur Kategorie der abelschen
  Gruppen (durch den Funktor der globalen Schnitte) ist die Kategorie
  der konstanten abelschen Garben auf einem topologischen Raum $X$
  eine abelsche Kategorie. Nun folgt die Aussage aus der Exaktheit des
  Pullbacks $i_{\sigma}^*$ entlang den Inklusionen $i_{\sigma}:
  |\sigma| \xhookrightarrow{} |\K|$.
\end{proof}

Entscheidend ist die folgende Charakterisierung schwach
$|\K|$-konstruierbarer derivierter Garben:

\begin{prop}[\cite{WS}]
  Für $F \in \AbKr$ sind äquivalent:
  \begin{enumerate}
  \item $F$ ist schwach $|\K|$-konstruierbar
  \item Die Koeinheit der Adjunktion ist auf $F$ ein Isomorphismus
    $p^*p_*F \iso F$.
  \item $F$ liegt im wesentlichen Bild des Rückzugs $p^*$.
  \item Die Restriktion $F(U(\sigma)) \to F_x$ ist für alle $\sigma
    \in \K$ und $x \in |\sigma|$ ein Isomorphismus.
  \end{enumerate}
\end{prop}
\begin{proof}
  Die Äquivalenz $(2) \Iff (3)$ ist allgemein kategorientheoretischer
  Natur. Dabei ist $(3) \Implies (2)$ offensichtlich (nimm $p_* F$)
  und $(2) \Implies (3)$ folgt aus den Dreiecksidentitäten.

  Die Äquivalenz $(2) \Iff (4)$ folgt aus der Bestimmung der Halme von
  $p^*p_* F$. Zunächst bemerken wir, dass in $\K$ die Menge $(\geq
  \sigma)$ die kleinste offene Umgebung von $\sigma$ ist, und wir also
  $p_*F((\geq \sigma)) \iso (p_*F)_{\sigma}$ erhalten. Somit gilt für
  $x \in |\sigma|$:
  \[ (p^*p_*F)_x \iso (p_*F)_{\sigma} \iso p_*F((\geq \sigma)) \iso F(U(\sigma)). \]
  Dabei wurde die Beschreibung der Halme des Rückzugs (mit $p(x) =
  \sigma$), obige Darstellung der Halme auf $\K$ und die Definition
  des Vorschubs (mit $p^{-1}((\geq \sigma)) = U(\sigma)$) verwendet.

  Die Implikation $(4) \Implies (1)$ folgt direkt aus dem
  nachgestellten Lemma, angewandt auf die Einschränkung von $F$ auf
  $U(\sigma)$, und der Tatsache, dass beliebige Einschränkungen
  konstanter Garben wieder konstant sind.

  Für die umgekehrte Richtung reicht es, die Aussage für die
  Einschränkung von $F$ auf $U(\sigma)$ zu zeigen. Wir betrachten für
  $x \in |\sigma|$ die Zusammenziehung
  \begin{align*}
    h: (0, 1] \times U(\sigma) &\to U(\sigma), \\
    (t, y) &\mapsto h(t, y) = t y + (1 - t) x.
  \end{align*}
  
  Die Mengen $h(\{t\} \times U(\sigma))$ bilden für $t \in (0, 1]$
    eine Umgebungsbasis von $x$, wir müssen also nur noch den Kolimes
    der Schnitte über diese Mengen bestimmen. Bezeichne $\pi: (0, 1]
      \times U(\sigma) \to U(\sigma)$ die Projektion auf den zweiten
      Faktor. Nach der simplizialen Konstanz von $F$ und wegen $h(t,
      y) \in |\tau| \Iff y \in |\tau|$ ist der Rückzug $h^* F$
      konstant auf den Fasern von $\pi$ und lässt sich somit nach dem
      zweiten nachgestellten Lemma schreiben als $h^* F \iso \pi^*
      \pi_* h^* F$. Damit erhalten wir wie gewünscht
      
      \[ F_x
      \iso \colf_{t \in (0, 1]} F(h(\{t\} \times U(\sigma)))
        \iso \colf_t \Gamma \iota_t^* h^* F
        \iso \colf_t \Gamma \iota_t^* \pi^* pi_* h^* F
        \iso \Gamma \id^* \pi_* h^* F
        \iso \Gamma h^* F
        \iso \Gamma F = F(U(\sigma)),
        \]

        im letzten Schritt nach der Surjektivität von $h$.

\end{proof}

Wir tragen die benötigten Lemmata nach.

\begin{lemma}[\cite{TG}, 2.1.41]
  Sei $X$ ein topologischer Raum, $F \in \EnsX$ eine Garbe auf $X$,
  für die die Restriktion $\Gamma F \iso F_x$ für alle $x \in X$
  bijektiv ist. Dann ist $F$ eine konstante Garbe auf $X$ mit Halm
  $\Gamma F$.
\end{lemma}
\begin{proof}
  Bezeichne $c: X \to \mathrm{top}$ die konstante Abbildung. Die
  Koeinheit der Adjunktion $c^* c_* F \to F$ induziert auf den Halmen
  gerade die vorausgesetzten Bijektionen, ist also ein
  Garben-Isomorphismus.
\end{proof}

\begin{lemma}
  q^*q_* Iso falls q surjektiv, Garbe konstant auf Halmen und ex. stetiger Schnitt der abg einbettung ist
  bew: brauche GK 4.3.19
\end{lemma}

Wir bezeichnen den Funktor $p^*p_*: \AbX \to \sKons$ kurz mit $\beta$
und bemerken, dass er nach obiger Proposition ein Rechtsadjungierter
zur Inklusion $\iota: \sKons \to \AbKr$ ist:

% TODO: wie?

Als Komposition zweier linksexakter Funktoren ist $\beta$ natürlich
wieder linksexakt. Der allgemeinen Terminologie folgend bezeichnen wir
eine Garbe $F \in \AbKr$ $\beta$-azyklisch, falls ihre höheren
Derivierten von $\beta$ verschwinden, also falls

\[ R^k\beta F = 0 \text{ für alle } k > 0.  \]

Später benötigen wir die folgende Charakterisierung
$\beta$-azyklischer Garben:

\begin{prop} \label{sk-ist-azyk}
  Eine Garbe $F \in \AbKr$ ist $\beta$-azyklisch genau dann, wenn
  $H^k(U(\sigma); F) = 0$ für alle $\sigma \in \K, k >
  0$.

  Insbesondere gilt:
  \begin{enumerate}
  \item Schwach $|\K|$-konstruierbare Garben sind $\beta$-azyklisch.
  \item Welke Garben sind $\beta$-azyklisch.
  \end{enumerate}  
\end{prop}
\begin{proof}
  Nach der Charakterisierung höherer direkter Bilder ist $R^q \beta F
  = p^* R^q p_* F$ für $F \in \AbKr$ isomorph zur Garbifizierung der
  Prägarbe

  \[ (\geq \sigma) \mapsto H^q(p^{-1}((\geq \sigma)); F) = H^q(p^{-1}(U(\sigma)); F). \]

  ...
\end{proof}

% Garbifizierung der Kohomologie-von-U-Prägarbe

\begin{prop}
  Sei $F \in \Ket^+(\AbKr)$ ein gegen die Richtung der Pfeile
  beschränkter Kettenkomplex aus $\beta$-azy\-klischen Garben mit
  schwach $|\K|$-kon\-stru\-ier\-bar\-en Kohomologiegarben $H^q(F)$. Dann
  ist $\beta F \to F$ ein Quasi-Iso\-mor\-phis\-mus.
\end{prop}
\begin{proof}
  Wir schneiden aus dem Kettenkomplex $(F^n, d^n)$ kurze exakte
  Sequenzen aus:
  
  \begin{alignat*}{4}
    H^0 = \ker d^0 \inj F^0 \surj &\im d^0 & & \\
    &\im d^0& {} \inj {} &\ker d^1& {}\surj {} &H^1& \\
    && &\ker d^1& {} \inj {} &F^1& {} \surj {} \im d^1 \\
    \vdots
  \end{alignat*}
  
  Sind in einer kurzen exakten Sequenz zwei der drei Objekte
  azyklisch, so nach dem Fünferlemma auch das dritte. Da nach
  Voraussetzung und \ref{sk-ist-azyk} $F^q$ und $H^q$
  $\beta$-azyklisch sind, sind alle oben betrachteten Objekte
  $\beta$-azyklisch und die kurzen exakten Sequenzen bleiben exakt
  nach Anwendung von $\beta$. Es folgt $H^q(\beta F) \iso \beta (H^q
  F)$ und weiter $\beta (H^q F) \iso H^q F$ nach der schwachen
  Konstruierbarkeit von $H^q F$.
\end{proof}

Damit ist der entscheidende Schritt für unser Ziel gezeigt. Wir
erhalten:

\begin{theorem}
  Sei $\K$ ein lokal-endlicher Simplizialkomplex. % ???

  Die oben definierten Funktoren $\iota, \beta$ induzieren auf den
  derivierten Kategorien eine Äquivalenz

  \[ \Der^+(\sKons(\K)) \mathrel{\mathop{\rightleftarrows}^{\iota}_{R \beta}} \DerpskK. \]
\end{theorem}
\begin{proof}
  Die Kategorien $\Der^+(\sKons(\K))$ und $\DerpskK$ haben genug
  Injektive.  Mit der Grothendieck-Spektralsequenz für derivierte
  Kategorien (\cite{TD}, 3.4.18) und der Exaktheit von $\iota$
  erhalten wir somit

  \[ R\beta \circ R\iota \iso R(\beta \circ \iota) \iso R \Id = \Id \]

  auf $\Der^+(\sKons(\K))$, da welke Garben $\beta$-azyklisch sind, sowie

  \[ \iota \circ R \beta \iso \Id \]

  auf $\DerpskK$ nach der vorangegangenen Proposition.
\end{proof}
\begin{bem}
  Kashiwara und Schapira beschränken sich auf die Äquivalenz der
  beschränkten derivierten Kategorien, allerdings mit der gleichen
  Argumentation. In \cite{KS} 1.7.12 wird eine allgemeine Aussage für
  solche Situationen gezeigt, die hier aber m.E. nicht benötigt wird.
\end{bem}

\end{document}
