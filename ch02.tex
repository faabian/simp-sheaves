% include latex header (\usepackage, \newcommand etc.) 
\documentclass[a4paper]{article}
% \usepackage[left=3cm,right=3cm,top=3cm,bottom=2cm]{geometry} % page settings
\usepackage{amsmath}
\usepackage{amssymb}
\usepackage{amsthm}
\usepackage{etoolbox}
\usepackage[ngerman]{babel}
\usepackage[utf8]{inputenc}
\usepackage{mathtools}

\setlength{\parskip}{\medskipamount}
\setlength{\parindent}{0pt}

\newtheorem{theorem}{Theorem}
\newtheorem{lemma}[theorem]{Lemma}
\newtheorem{prop}[theorem]{Proposition}
\newtheorem{kor}[theorem]{Korollar}
\newtheorem{satz}[theorem]{Satz}
\newtheorem{defn}[theorem]{Definition}

\DeclareMathOperator{\Cat}{Cat}
\DeclareMathOperator{\poset}{poset}
\DeclareMathOperator{\EnsX}{Ens_{/X}}
\DeclareMathOperator{\pEnsX}{pEns_{/X}}
\DeclareMathOperator{\AbX}{Ab_{/X}}
\DeclareMathOperator{\pAbX}{pAb_{/X}}
\DeclareMathOperator{\OffX}{Off_X}
\DeclareMathOperator{\Ens}{Ens}
\DeclareMathOperator{\Ob}{Ob}
\DeclareMathOperator{\Der}{Der}
\DeclareMathOperator{\Ab}{Ab}
\DeclareMathOperator{\sKons}{s-Kons}
\DeclareMathOperator{\Ket}{Ket}
\DeclareMathOperator{\EnsB}{Ens_{/\B}}

\newcommand{\B}{\mathcal{B}}
\newcommand{\op}{^\mathrm{op}}
\newcommand{\iso}{\xrightarrow{\sim}}
\newcommand{\qiso}{\xrightarrow{\approx}}
\newcommand{\open}{\subset\kern-0.58em\circ}  % only possible in math mode
\newcommand{\K}{\mathcal{K}}
\newcommand{\Kreal}{|\K|}
\newcommand{\Z}{\mathbb{Z}}
\newcommand{\DerAbK}{\Der(\Ab_{/\Kreal})}
\newcommand{\DerskK}{\Der_{\mathrm{sk}}(\Kreal)}
\newcommand{\AbKr}{\Ab_{/\Kreal}}
\newcommand{\sKonsK}{\sKons(\K)}


\begin{document}

\title{Simpliziale Garben}
\author{Fabian Glöckle}
\date{\today}
% \maketitle

\section{Schwach konstruierbare Garben auf Simplizialkomplexen}

Ziel dieses Abschnitts ist die Charakterisierung schwach
konstruierbarer Garben auf Simplizialkomplexen und ihrer derivierten
Kategorie. Die Darstellung folgt Kashiwara-Schapira.

In diesem Abschnitt bezeichne $\mathcal{K}$ einen lokal-endlichen
Simplizialkomplex und $|\mathcal{K}|$ seine geometrische
Realisierung. Wir erhalten eine stetige Abbildung

\[ p: |\mathcal{K}| \to \mathcal{K}, \]

genannt Simplexanzeiger oder Indikatorabbildung, der einem Punkt $x
\in |\mathcal{K}|$ in der geometrischen Realisierung den eindeutigen
Simplex $\sigma \in \mathcal{K}$ mit $x \in |\sigma|$ zuordnet.

% TODO: def geom Realisierung wie in KS mit |\sigma|
% TODO: Stetigkeit von p bzgl Ordnungstopologie
% TODO: wo benötigen wir lok endl?

\begin{defn}
  Eine Garbe $F \in \AbKr$ heißt schwach $\Kreal$-konstruierbar (oder
  kurz: schwach konstruierbar), falls für alle $\sigma \in \K$, die
  Einschränkungen $F|_{|\sigma|}$ konstante Garben sind. Wir
  bezeichnen die volle Unterkategorie der schwach konstruierbaren
  Garben in $\AbKr$ mit $\sKonsK$.

  Eine derivierte Garbe $F \in \DerAbK$ heißt schwach
  $\Kreal$-konstruierbar, falls für alle $j \in \Z$ die
  Kohomologiegarben $H^j(F)$ schwach konstruierbar sind. Wir
  bezeichnen die volle Unterkategorie der schwach konstruierbaren
  derivierten Garben in $\DerAbK$ mit $\DerskK$.
\end{defn}

Wir bemerken zunächst:

\begin{lemma}
  Die Kategorie $\sKonsK$ ist abelsch.
\end{lemma}
\begin{proof}
  Durch den offensichtlichen Isomorphismus zur Kategorie der abelschen
  Gruppen (durch den Funktor der globalen Schnitte) ist die Kategorie
  der konstanten abelschen Garben auf einem topologischen Raum $X$
  eine abelsche Kategorie. Nun folgt die Aussage aus der Exaktheit des
  Pullbacks $i_{\sigma}^*$ entlang den Inklusionen $i_{\sigma}:
  |\sigma| \xhookrightarrow{} \Kreal$.
\end{proof}

Entscheidend ist die folgende Charakterisierung schwach
$\Kreal$-konstruierbarer derivierter Garben:

\begin{prop}
  Für $F \in \DerAbK$ sind äquivalent:
  %  TODO: oder F \in \AbKr?
  \begin{enumerate}
  \item F ist schwach $\Kreal$-konstruierbar
  \item Die Koeinheit der Adjunktion ist auf $F$ ein Isomorphismus
    $p^*p_*F \iso F$.
  \end{enumerate}
\end{prop}
\begin{proof}
  % TODO
\end{proof}

% beta-azyklisch

Wir bezeichnen den Funktor $p^*p_*: \AbX \to \sKons$ kurz mit $\beta$
und bemerken, dass er nach obiger Proposition ein Rechtsadjungierter
zur Inklusion $\iota: \sKons \to \AbKr$ ist:

% TODO: wie?

Als Komposition zweier linksexakter Funktoren ist $\beta$ natürlich
wieder linksexakt. 

\begin{prop}
  Sei $F \in \Ket^+(\AbKr)$ ein gegen die Richtung der Pfeile
  beschränkter Kettenkomplex aus $\beta$-azyklischen Garben mit
  ebenfalls schwach $\Kreal$-konstruierbaren Kohomologiegarben
  $H^q(F)$. Dann ist $\beta F \to F$ ein Quasi-Isomorphismus.
\end{prop}
\begin{proof}
  Wir schneiden aus dem Kettenkomplex $(F^n, d^n)$ kurze exakte
  Sequenzen aus:
  
  \begin{alignat*}{4}
    H^0 = \ker d^0 \inj F^0 \surj &\im d^0 & & \\
    &\im d^0& {} \inj {} &\ker d^1& {}\surj {} &H^1& \\
    && &\ker d^1& {} \inj {} &F^1& {} \surj {} \im d^1 \\
    \vdots
  \end{alignat*}
  
  Sind in einer kurzen exakten Sequenz zwei der drei Objekte
  azyklisch, so nach dem Fünferlemma auch das dritte. Da nach
  Voraussetzung $F^q$ und $H^q$ $\beta$-azyklisch sind, sind alle oben
  betrachteten Objekte $\beta$-azyklisch und die kurzen exakten
  Sequenzen bleiben exakt nach Anwendung von $\beta$. Es folgt
  $H^q(\beta F) \iso \beta (H^q F)$ und weiter $\beta (H^q F) \iso H^q
  F$ nach der schwachen Konstruierbarkeit von $H^q F$.
\end{proof}

Damit ist der entscheidende Schritt für unser Ziel gezeigt. Wir
erhalten:

\begin{theorem}
  $\K$ lokal endlicher Simplizialkomplex beschränkter Dimension ??

  Die oben definierten Funktoren $\iota, \beta$ induzieren auf den
  derivierten Kategorien eine Äquivalenz

  \[ \Der^b(\sKons(\K)) \mathrel{\mathop{\rightleftarrows}^{\iota}_{R \beta}} \DerbskK. \]
\end{theorem}
\begin{proof}
  
\end{proof}

\end{document}
