% include latex header (\usepackage, \newcommand etc.) 
\documentclass[a4paper]{article}
% \usepackage[left=3cm,right=3cm,top=3cm,bottom=2cm]{geometry} % page settings
\usepackage{amsmath}
\usepackage{amssymb}
\usepackage{amsthm}
\usepackage{etoolbox}
\usepackage[ngerman]{babel}
\usepackage[utf8]{inputenc}
\usepackage{mathtools}
\usepackage{tikz-cd}
\usepackage{enumitem}
\usepackage{hyperref}

\setlength{\parskip}{\medskipamount}
\setlength{\parindent}{0pt}

\theoremstyle{plain}
\newtheorem{theorem}{Theorem}
\newtheorem{lemma}[theorem]{Lemma}
\newtheorem{prop}[theorem]{Proposition}
\newtheorem{kor}[theorem]{Korollar}
\newtheorem{satz}[theorem]{Satz}
%% \providecommand*{\lemmaautorefname}{Lemma}
%% \providecommand*{\propautorefname}{Prop.}
%% \providecommand*{\korautorefname}{Korollar}
%% \providecommand*{\satzautorefname}{Satz}

\theoremstyle{definition}
\newtheorem{defn}[theorem]{Definition}

\theoremstyle{remark}
\newtheorem{bem}[theorem]{Bemerkung}

\DeclareMathOperator{\Cat}{Cat}
\DeclareMathOperator{\poset}{poset}
\DeclareMathOperator{\EnsX}{Ens_{/X}}
\DeclareMathOperator{\pEnsX}{pEns_{/X}}
\DeclareMathOperator{\AbX}{Ab_{/X}}
\DeclareMathOperator{\pAbX}{pAb_{/X}}
\DeclareMathOperator{\OffX}{Off_X}
\DeclareMathOperator{\Ens}{Ens}
\DeclareMathOperator{\Ob}{Ob}
\DeclareMathOperator{\Der}{Der}
\DeclareMathOperator{\Ab}{Ab}
\DeclareMathOperator{\sKons}{s-Kons}
\DeclareMathOperator{\Ket}{Ket}
\DeclareMathOperator{\EnsB}{Ens_{/\B}}
\DeclareMathOperator{\im}{im}
\DeclareMathOperator{\Id}{Id}
\DeclareMathOperator{\id}{id}
\DeclareMathOperator{\colf}{colf}
\DeclareMathOperator{\limf}{limf}
\DeclareMathOperator{\Top}{Top}

\newcommand{\etalespace}[1]{\overline{#1}}
\newcommand{\B}{\mathcal{B}}
\newcommand{\op}{^\mathrm{op}}
\newcommand{\iso}{\xrightarrow{\sim}}
\newcommand{\qiso}{\xrightarrow{\approx}}
\newcommand{\fromqiso}{\xleftarrow{\approx}}
\newcommand{\open}{\subset\kern-0.58em\circ}  % only possible in math mode
\newcommand{\K}{\mathcal{K}}
\newcommand{\Z}{\mathbb{Z}}
\newcommand{\R}{\mathbb{R}}
\newcommand{\DerAbK}{\Der(\Ab_{/|\K|})}
\newcommand{\DerskK}{\Der_{\mathrm{sk}}(|\K|)}
\newcommand{\DerpskK}{\Der^+_{\mathrm{sk}}(|\K|)}
\newcommand{\AbKr}{\Ab_{/|\K|}}
\newcommand{\sKonsK}{\sKons(\K)}
\newcommand{\inj}{\hookrightarrow}
\newcommand{\surj}{\twoheadrightarrow}
\newcommand{\Iff}{\Leftrightarrow}
\newcommand{\Implies}{\Rightarrow}
\newcommand{\cc}{^{\bullet}}  % chain complex
\newcommand{\from}{\leftarrow}


\begin{document}

\title{Simpliziale Garben}
\author{Fabian Glöckle}
\date{\today}
% \maketitle

\section{Schwach konstruierbare Garben auf Simplizialkomplexen}

Ziel dieses Abschnitts ist die Charakterisierung schwach
konstruierbarer Garben auf Simplizialkomplexen und ihrer derivierten
Kategorie. Die Darstellung folgt im Wesentlichen \cite{KS} und \cite{WS}.

In diesem Abschnitt bezeichne $(V, \K)$ einen lokal-endlichen
Simplizialkomplex mit Eckenmenge $V$. Für einen Simplex $\sigma \in
\K$ definieren wir seine geometrische Realisierung $|\sigma| \subset
\R^V = \Ens(V, \R)$:
\[ |\sigma| = \{ x \in \R^V | x(v) = 0 \text{ für } v \notin \sigma,
   x(v) > 0 \text{ für } v \in \sigma,
   \sum_{v \in V} x(v) = 1 \}, \]
sowie die geometrische Realisierung $|\K| \subset \R^V$ von $\K$
\[ \bigcup_{\sigma \in \K} |\sigma|, \]
jeweils versehen mit der induzierten Topologie von $\R^V$.

Wir erhalten eine Abbildung
\[ p: |\mathcal{K}| \to \mathcal{K}, \]
genannt Simplexanzeiger oder Indikatorabbildung, der einem Punkt $x
\in |\mathcal{K}|$ in der geometrischen Realisierung den eindeutigen
Simplex $\sigma \in \mathcal{K}$ mit $x \in |\sigma|$ zuordnet.

\begin{lemma}
  Der Simplexanzeiger $p: |\K| \to \K$ ist stetig.
\end{lemma}
\begin{proof}
  Das Urbild einer Basismenge $(\geq \sigma)$ ist
  \[ p^{-1}((\geq \sigma)) = |\K| \cap \{x \in \R^V | x(v) > 0 \text{ für }
  v \in \sigma \}, \]
  der offene Stern um $\sigma$, den wir auch als $U(\sigma)$ notieren.
\end{proof}

% TODO: wo benötigen wir lok endl?

\begin{defn}
  Eine Garbe $F \in \AbKr$ heißt schwach $|\K|$-konstruierbar (oder
  kurz: schwach konstruierbar), falls für alle $\sigma \in \K$, die
  Einschränkungen $F|_{|\sigma|}$ konstante Garben sind. Wir
  bezeichnen die volle Unterkategorie der schwach konstruierbaren
  Garben in $\AbKr$ mit $\sKonsK$.

  Eine derivierte Garbe $F \in \DerAbK$ heißt schwach
  $|\K|$-konstruierbar, falls für alle $j \in \Z$ die
  Kohomologiegarben $H^j(F)$ schwach konstruierbar sind. Wir
  bezeichnen die volle Unterkategorie der schwach konstruierbaren
  derivierten Garben in $\DerAbK$ mit $\DerskK$.
\end{defn}

Wir bemerken zunächst:

\begin{lemma}[\cite{KS}, 8.1.3] \label{skons-abelian}
  Die Kategorie $\sKonsK$ ist abelsch.
\end{lemma}
\begin{proof}
  Durch den offensichtlichen Isomorphismus zur Kategorie der abelschen
  Gruppen (durch den Funktor der globalen Schnitte) ist die Kategorie
  der konstanten abelschen Garben auf einem topologischen Raum $X$
  eine abelsche Kategorie. Nun folgt die Aussage aus der Exaktheit des
  Pullbacks $i_{\sigma}^*$ entlang den Inklusionen $i_{\sigma}:
  |\sigma| \xhookrightarrow{} |\K|$.
\end{proof}

Entscheidend ist die folgende Charakterisierung schwach
$|\K|$-konstruierbarer Garben:

\begin{prop}[\cite{WS}, 8.4.6.3] \label{sk-char}
  Für $F \in \AbKr$ sind äquivalent:
  \begin{enumerate}[label=(\arabic*)]
  \item \label{itm:sk-char-sk} $F$ ist schwach $|\K|$-konstruierbar
  \item \label{itm:sk-char-counit} Die Koeinheit der Adjunktion ist
    auf $F$ ein Isomorphismus $p^*p_*F \iso F$.
  \item \label{itm:sk-char-essim} $F$ liegt im wesentlichen Bild des
    Rückzugs $p^*$.
  \item \label{itm:sk-char-res} Die Restriktion $F(U(\sigma)) \to F_x$
    ist für alle $\sigma \in \K$ und alle $x \in |\sigma|$ ein
    Isomorphismus.
  \end{enumerate}
\end{prop}
\begin{proof}
  Die Äquivalenz $\ref{itm:sk-char-counit} \Iff
  \ref{itm:sk-char-essim}$ ist allgemein kategorientheoretischer
  Natur. Dabei ist $\ref{itm:sk-char-counit} \Implies
  \ref{itm:sk-char-essim}$ offensichtlich (nimm $p_* F$) und
  $\ref{itm:sk-char-essim} \Implies \ref{itm:sk-char-counit}$ folgt
  aus den Dreiecksidentitäten.

  Die Äquivalenz $\ref{itm:sk-char-counit} \Iff \ref{itm:sk-char-res}$
  folgt aus der Bestimmung der Halme von $p^*p_* F$. Zunächst bemerken
  wir, dass in $\K$ die Menge $(\geq \sigma)$ die kleinste offene
  Umgebung von $\sigma$ ist, und wir also $p_*F((\geq \sigma)) \iso
  (p_*F)_{\sigma}$ erhalten. Somit gilt für $x \in |\sigma|$:
  \begin{equation}\label{eq:counit-sect}
    (p^*p_*F)_x \iso (p_*F)_{\sigma} \iso p_*F((\geq \sigma)) \iso F(U(\sigma)).  
  \end{equation}  
  Dabei wurde die Beschreibung der Halme des Rückzugs (mit $p(x) =
  \sigma$), obige Darstellung der Halme auf $\K$ und die Definition
  des Vorschubs (mit $p^{-1}((\geq \sigma)) = U(\sigma)$) verwendet.

  Die Implikation $\ref{itm:sk-char-res} \Implies
  \ref{itm:sk-char-sk}$ folgt direkt aus dem nachgestellten Lemma,
  angewandt auf die Einschränkung von $F$ auf $U(\sigma)$, und der
  Tatsache, dass beliebige Einschränkungen konstanter Garben wieder
  konstant sind.

  Für die umgekehrte Richtung reicht es, die Aussage für die
  Einschränkung von $F$ auf $U(\sigma)$ zu zeigen.  Wir betrachten für
  $x \in |\sigma|$ die Zusammenziehung
  \begin{align*}
    h: (0, 1] \times U(\sigma) &\to U(\sigma), \\
    (t, y) &\mapsto h(t, y) = t y + (1 - t) x.
  \end{align*}
  
  Die Mengen $h(\{t\} \times U(\sigma))$ bilden für $t \in (0, 1]$
    eine Umgebungsbasis von $x$, wir müssen also nur noch den Kolimes
    der Schnitte über diese Mengen bestimmen. Bezeichne $\pi: (0, 1]
      \times U(\sigma) \to U(\sigma)$ die Projektion auf den zweiten
      Faktor. Nach der simplizialen Konstanz von $F$ und wegen $h(t,
      y) \in |\tau| \Iff y \in |\tau|$ ist der Rückzug $h^* F$
      konstant auf den Fasern von $\pi$ und lässt sich somit nach dem
      zweiten nachgestellten Lemma schreiben als $\pi^* \pi_* h^* F
      \iso h^* F$. Bezeichne $\iota_t: U(\sigma) \xhookrightarrow{}
      (0, 1] \times U(\sigma)$ die Inklusion. Dann erhalten wir wie
        gewünscht mit der Funktorialität des Rückzugs, $\pi \circ
        \iota_t = \id_{U(\sigma)}$ sowie $\Gamma \pi_* = \Gamma$
   \begin{align*}
     F_x &\iso \colf\limits_{t \in (0, 1]} F(h(\{t\} \times U(\sigma))) \\
         &\iso \colf\limits_{t \in (0, 1]} \Gamma \iota_t^* h^* F \\
         &\iso \colf\limits_{t \in (0, 1]} \Gamma \iota_t^* \pi^* \pi_* h^* F \\
         &\iso \Gamma \id^* \pi_* h^* F \\
         &\iso \Gamma h^* F \\
         &\iso \Gamma F = F(U(\sigma)),
   \end{align*}
   im letzten Schritt nach der Surjektivität von $h$.
\end{proof}
\begin{bem} \label{beta-sect}
  Aus \autoref{eq:counit-sect} und \ref{itm:sk-char-res} folgt
  insbesondere auch $(p^* p_* F)(U(\sigma)) \iso F(U(\sigma))$ für
  alle $F \in \AbKr$ und alle $\sigma \in \K$.
\end{bem}

Wir tragen die benötigten Lemmata nach.

\begin{lemma}[\cite{TG}, 2.1.41]
  Sei $X$ ein topologischer Raum, $F \in \EnsX$ eine Garbe auf $X$,
  für die die Restriktion $\Gamma F \iso F_x$ für alle $x \in X$
  bijektiv ist. Dann ist $F$ eine konstante Garbe auf $X$ mit Halm
  $\Gamma F$.
\end{lemma}
\begin{proof}
  Bezeichne $c: X \to \mathrm{top}$ die konstante Abbildung. Die
  Koeinheit der Adjunktion $c^* c_* F \to F$ induziert auf den Halmen
  gerade die vorausgesetzten Bijektionen, ist also ein
  Garben-Isomorphismus.
\end{proof}

\begin{lemma}[\cite{TG}, 6.4.17]
  Sei $X$ ein topologischer Raum, $I \subset \R$ ein nichtleeres
  Intervall, $F \in \Ens_{/X \times I}$ eine Garbe und $\pi: X \times
  I \to X$ die Projektion auf den ersten Faktor. Ist $F$ konstant auf
  den Fasern von $\pi$, so ist die Koeinheit der Adjunktion auf $F$
  ein Isomorphismus $\pi^* \pi_* F \iso F$.
\end{lemma}
\begin{proof}
  Die Aussage ist äquivalent zum folgenden Fortsetzungsresultat:
  \begin{quote}
    Für alle $U \open X$ und $t \in I$ ist die Restriktion ein
    Isomorphismus
    \[ \Gamma(U \times I, F) \iso \Gamma(U \times \{t\}, F). \]
  \end{quote}
  Denn ist die Koeinheit der Adjunktion ein Isomorphismus $\pi^* \pi_*
  F \iso F$, so bestimmen wir die Schnitte über $U \times \{t\}$ wie
  folgt: Sei $\iota: U \times \{t\} \inj X \times I$ die Inklusion. Wir
  bemerken, dass $\pi \circ \iota$ die Inklusion von $U$ nach $X$ ist
  und erhalten:
  \[ \Gamma(U \times \{t\}, F)
      = \Gamma \iota^* F
      \iso \Gamma \iota^* \pi^* \pi_* F
      = \Gamma(U, \pi_* F)
      = \Gamma(U \times I, F). \]
  Andersherum folgt der Isomorphismus der Koeinheit der Adjunktion
  aber auch aus dem Fortsetzungsresultat, denn wir können sofort den
  Isomorphismus auf den Halmen über $(x, t) \in X \times I$ zeigen:
  \begin{align*}
    (\pi^* \pi_* F)_{(x,t)}
    &\iso (\pi_* F)_x \\
    &= \colf\limits_{U \ni x} \Gamma(U \times I, F) \\
    &\iso \colf\limits_{U \ni x} \Gamma(U \times \{t\}, F) \\
    &\iso \colf\limits_{V \ni (x, t)} F(V) \\
    &= F_{(x, t)}.
  \end{align*}
  Dabei erhalten wir die Surjektivität von $F(V) \to \Gamma(U \times
  \{t\}, F)$ aus der Bijektivität der Verknüpfung
  \[ \Gamma(U \times I, F) \to F(V) \to \Gamma(U \times \{t\}, F) \]
  und die Injektivität aus der Eigenschaft, dass bereits die faserweise
  stetige Fortsetzung eindeutig ist nach der Konstantheit der
  Einschränkungen von $F$ auf die Fasern von $\pi$.

  Nun können wir die Aussage zeigen. Zunächst folgt sie für $I$
  kompakt sofort aus eigentlichem Basiswechsel über dem kartesischen
  Diagramm mit eigentlichen und separierten Vertikalen
  \shorthandoff{"}
  \[ \begin{tikzcd}
    (x, t) \arrow[r, hook] \arrow[d] \arrow[dr, phantom, "\lrcorner"]
    & X \times I \arrow[d, "\pi"] \\
    x \arrow[r, hook]
    & X.
  \end{tikzcd} \]
  Ist nun $I \subset \R$ ein beliebiges Intervall, so können wir es
  als aufsteigende Vereinigung von Kompakta $I = \bigcup_j I_j$
  schreiben und erhalten für die Schnitte ebenfalls
  \begin{align*}
    \Gamma(U \times I, F)
     &= \Top(U \times I, \etalespace{F}) \\
     &\iso \colf\limits_j \Top(U \times I_j, \etalespace{F})
     = \colf\limits_j \Gamma(U \times I_j, F)
     \iso \Gamma(U \times \{t\}, F).
  \end{align*}
\end{proof}

Wir bezeichnen den Funktor $p^*p_*: \AbX \to \sKons$ kurz mit $\beta$
und bemerken, dass er nach obiger Proposition ein Rechtsadjungierter
zur Inklusion $\iota: \sKons \to \AbKr$ ist:

% TODO: wie?

Als Komposition zweier linksexakter Funktoren ist $\beta$ natürlich
wieder linksexakt. Der allgemeinen Terminologie folgend bezeichnen wir
eine Garbe $F \in \AbKr$ $\beta$-azyklisch, falls ihre höheren
Derivierten von $\beta$ verschwinden, also falls
\[ R^k\beta F = 0 \text{ für alle } k > 0.  \]

Später benötigen wir die folgende Charakterisierung
$\beta$-azyklischer Garben:

\begin{prop}[\cite{KS}, 8.1.8] \label{sk-ist-azyk}
  Sei $F \in \AbKr$. Dann gilt:
  \begin{enumerate}[label=(\roman*)]
  \item \label{itm:sect-comp} $\Gamma(U(\sigma), R^k\beta F) \iso
    H^k(U(\sigma); F)$ für alle $\sigma \in \K, k \geq 0$.
  \item \label{itm:char-acyc} $F$ ist $\beta$-azyklisch genau dann,
    wenn $H^k(U(\sigma); F) = 0$ für alle $\sigma \in \K, k > 0$.
  \end{enumerate}
  Insbesondere sind schwach $|\K|$-konstruierbare Garben und welke
  Garben $\beta$-azyklisch.
\end{prop}
\begin{bem}
  Man beachte, dass Aussage \ref{itm:char-acyc} eine allgemeine
  Verbesserung unserer allgemeinen Charakterisierung höherer direkter
  Bilder als Garbifizierungen der Prägarben der Kohomologien der
  Fasern ist für den Fall, dass die Schnittfunktoren
  $\Gamma(\pi^{-1}(U), \cdot)$ exakt sind.
  %% Nach der Charakterisierung höherer direkter Bilder ist $R^q \beta F
  %% = p^* R^q p_* F$ für $F \in \AbKr$ isomorph zur Garbifizierung der
  %% Prägarbe
  %% \[ (\geq \sigma) \mapsto H^q(p^{-1}((\geq \sigma)); F)
  %%    = H^q(p^{-1}(U(\sigma)); F). \]  
\end{bem}
\begin{proof}
  \ref{itm:char-acyc} folgt direkt aus \ref{itm:sect-comp} mit
  $R^k\beta F \in \sKonsK$ (vgl. \ref{skons-abelian}) und unserer
  Aussage über die Halme \ref{sk-char} \ref{itm:sk-char-res}.
  
  Mit der Exaktheit von Halmfunktoren $F \mapsto F_x$ und
  \ref{sk-char} \ref{itm:sk-char-res} sind die Funktoren
  $\Gamma(U(\sigma), \cdot)$ auf schwach $|\K|$-konstruierbaren Garben
  exakt und vertauschen folglich bei einem Komplex mit der Bildung der
  Kohomologie. Wir erhalten für $F \inj I\cc$ eine injektive Auflösung
  somit
  \[ \Gamma(U(\sigma), R^k\beta F)
  = \Gamma(U(\sigma), H^k(\beta I\cc))
  \iso H^k(\Gamma(U(\sigma), \beta I\cc))
  \iso H^k(\Gamma(U(\sigma), I\cc))
  = H^k(U(\sigma), F), \]
  wobei wir im dritten Schritt \ref{beta-sect} verwendet haben.
\end{proof}

% TODO überleiten

\begin{prop}
  Sei $F \in \Ket^+(\AbKr)$ ein gegen die Richtung der Pfeile
  beschränkter Kettenkomplex aus $\beta$-azy\-klischen Garben mit
  schwach $|\K|$-kon\-stru\-ier\-bar\-en Kohomologiegarben $H^q(F)$. Dann
  ist $\beta F \to F$ ein Quasi-Iso\-mor\-phis\-mus.
\end{prop}

\begin{proof}
  Wir schneiden aus dem Kettenkomplex $(F^n, d^n)$ kurze exakte
  Sequenzen aus:
  \begin{alignat*}{4}
    H^0 = \ker d^0 \inj F^0 \surj &\im d^0 & & \\
    &\im d^0& {} \inj {} &\ker d^1& {}\surj {} &H^1& \\
    && &\ker d^1& {} \inj {} &F^1& {} \surj {} \im d^1 \\
    \vdots
  \end{alignat*}
  
  Sind in einer kurzen exakten Sequenz zwei der drei Objekte
  azyklisch, so nach dem Fünferlemma auch das dritte. Da nach
  Voraussetzung und \ref{sk-ist-azyk} $F^q$ und $H^q$
  $\beta$-azyklisch sind, sind alle oben betrachteten Objekte
  $\beta$-azyklisch und die kurzen exakten Sequenzen bleiben exakt
  nach Anwendung von $\beta$. Es folgt $H^q(\beta F) \iso \beta (H^q
  F)$ und weiter $\beta (H^q F) \iso H^q F$ nach der schwachen
  Konstruierbarkeit von $H^q F$.
\end{proof}

Damit ist der entscheidende Schritt für unser Ziel gezeigt. Wir
erhalten:

\begin{theorem}
  Sei $\K$ ein lokal-endlicher Simplizialkomplex. % ???

  Die oben definierten Funktoren $\iota, \beta$ induzieren auf den
  derivierten Kategorien eine Äquivalenz

  \[ \Der^+(\sKons(\K)) \mathrel{\mathop{\rightleftarrows}^{\iota}_{R \beta}} \DerpskK. \]
\end{theorem}
\begin{proof}
  Die Kategorien $\Der^+(\sKons(\K))$ und $\DerpskK$ haben genug
  Injektive.  Mit der Grothendieck-Spektralsequenz für derivierte
  Kategorien (\cite{TD}, 3.4.18) und der Exaktheit von $\iota$
  erhalten wir somit

  \[ R\beta \circ R\iota \iso R(\beta \circ \iota) \iso R \Id = \Id \]

  auf $\Der^+(\sKons(\K))$, da welke Garben $\beta$-azyklisch sind, sowie

  \[ \iota \circ R \beta \iso \Id \]

  auf $\DerpskK$ nach der vorangegangenen Proposition.
\end{proof}
\begin{bem}
  Kashiwara und Schapira beschränken sich auf die Äquivalenz der
  beschränkten derivierten Kategorien, allerdings mit der gleichen
  Argumentation. In \cite{KS} 1.7.12 wird eine allgemeine Aussage für
  solche Situationen gezeigt, die hier aber m.E. nicht benötigt wird.
\end{bem}

\end{document}
