% include latex header (\usepackage, \newcommand etc.) 
\documentclass[a4paper]{article}
% \usepackage[left=3cm,right=3cm,top=3cm,bottom=2cm]{geometry} % page settings
\usepackage{amsmath}
\usepackage{amssymb}
\usepackage{amsthm}
\usepackage{etoolbox}
\usepackage[ngerman]{babel}
\usepackage[utf8]{inputenc}
\usepackage{mathtools}
\usepackage{tikz-cd}
\usepackage{enumitem}
\usepackage{hyperref}

\setlength{\parskip}{\medskipamount}
\setlength{\parindent}{0pt}

\theoremstyle{plain}
\newtheorem{theorem}{Theorem}
\newtheorem{lemma}[theorem]{Lemma}
\newtheorem{prop}[theorem]{Proposition}
\newtheorem{kor}[theorem]{Korollar}
\newtheorem{satz}[theorem]{Satz}
%% \providecommand*{\lemmaautorefname}{Lemma}
%% \providecommand*{\propautorefname}{Prop.}
%% \providecommand*{\korautorefname}{Korollar}
%% \providecommand*{\satzautorefname}{Satz}

\theoremstyle{definition}
\newtheorem{defn}[theorem]{Definition}

\theoremstyle{remark}
\newtheorem{bem}[theorem]{Bemerkung}

\DeclareMathOperator{\Cat}{Cat}
\DeclareMathOperator{\poset}{poset}
\DeclareMathOperator{\EnsX}{Ens_{/X}}
\DeclareMathOperator{\pEnsX}{pEns_{/X}}
\DeclareMathOperator{\AbX}{Ab_{/X}}
\DeclareMathOperator{\pAbX}{pAb_{/X}}
\DeclareMathOperator{\OffX}{Off_X}
\DeclareMathOperator{\Ens}{Ens}
\DeclareMathOperator{\Ob}{Ob}
\DeclareMathOperator{\Der}{Der}
\DeclareMathOperator{\Ab}{Ab}
\DeclareMathOperator{\sKons}{s-Kons}
\DeclareMathOperator{\Ket}{Ket}
\DeclareMathOperator{\EnsB}{Ens_{/\B}}
\DeclareMathOperator{\im}{im}
\DeclareMathOperator{\Id}{Id}
\DeclareMathOperator{\id}{id}
\DeclareMathOperator{\colf}{colf}
\DeclareMathOperator{\limf}{limf}
\DeclareMathOperator{\Top}{Top}

\newcommand{\etalespace}[1]{\overline{#1}}
\newcommand{\B}{\mathcal{B}}
\newcommand{\op}{^\mathrm{op}}
\newcommand{\iso}{\xrightarrow{\sim}}
\newcommand{\qiso}{\xrightarrow{\approx}}
\newcommand{\fromqiso}{\xleftarrow{\approx}}
\newcommand{\open}{\subset\kern-0.58em\circ}  % only possible in math mode
\newcommand{\K}{\mathcal{K}}
\newcommand{\Z}{\mathbb{Z}}
\newcommand{\R}{\mathbb{R}}
\newcommand{\DerAbK}{\Der(\Ab_{/|\K|})}
\newcommand{\DerskK}{\Der_{\mathrm{sk}}(|\K|)}
\newcommand{\DerpskK}{\Der^+_{\mathrm{sk}}(|\K|)}
\newcommand{\AbKr}{\Ab_{/|\K|}}
\newcommand{\sKonsK}{\sKons(\K)}
\newcommand{\inj}{\hookrightarrow}
\newcommand{\surj}{\twoheadrightarrow}
\newcommand{\Iff}{\Leftrightarrow}
\newcommand{\Implies}{\Rightarrow}
\newcommand{\cc}{^{\bullet}}  % chain complex
\newcommand{\from}{\leftarrow}


\begin{document}

\title{Simpliziale Garben}
\author{Fabian Glöckle}
\date{\today}
% \maketitle

\section{Schwach konstruierbare Garben auf Simplizialkomplexen}

In diesem Abschnitt bezeichne $\mathcal{K}$ einen lokal-endlichen
Simplizialkomplex und $|\mathcal{K}|$ seine geometrische
Realisierung. Wir erhalten eine stetige Abbildung

\[ p: |\mathcal{K}| \to \mathcal{K}, \]

genannt Simplexanzeiger oder Indikatorabbildung, der einem Punkt $x
\in |\mathcal{K}|$ in der geometrischen Realisierung den eindeutigen
Simplex $\sigma \in \mathcal{K}$ mit $x \in |\sigma|$ zuordnet.

% TODO: def geom Realisierung wie in KS mit |\sigma|
% TODO: Stetigkeit von p bzgl Ordnungstopologie
% TODO: wo benötigen wir lok endl?

\begin{defn}
  Eine Garbe $F \in \AbKr$ heißt schwach $\Kreal$-konstruierbar (oder
  kurz: schwach konstruierbar), falls für alle $\sigma \in \K$, die
  Einschränkungen $F|_{|\sigma|}$ konstante Garben sind. Wir
  bezeichnen die volle Unterkategorie der schwach konstruierbaren
  Garben in $\AbKr$ mit $\sKonsK$.

  Eine derivierte Garbe $F \in \DerAbK$ heißt schwach
  $\Kreal$-konstruierbar, falls für alle $j \in \Z$ die
  Kohomologiegarben $H^j(F)$ schwach konstruierbar sind. Wir
  bezeichnen die volle Unterkategorie der schwach konstruierbaren
  derivierten Garben in $\DerAbK$ mit $\DerskK$.
\end{defn}



\begin{prop}
  Für $F \in \DerAbK$ sind äquivalent:
  \begin{enumerate}
  \item F ist schwach $\Kreal$-konstruierbar
  \item Die Koeinheit der Adjunktion ist auf $F$ ein Isomorphismus
    $p^*p_*F \iso F$.
  \end{enumerate}
\end{prop}
\begin{proof}
  % TODO
\end{proof}

\begin{lemma}
  Die Kategorie $\sKonsK$ ist abelsch.
\end{lemma}
\begin{proof}
  Durch den offensichtlichen Isomorphismus zur Kategorie der abelschen
  Gruppen (durch den Funktor der globalen Schnitte) ist die Kategorie
  der konstanten abelschen Garben auf einem topologischen Raum $X$
  eine abelsche Kategorie. Nun folgt die Aussage aus der Exaktheit des
  Pullbacks $i_{\sigma}^*$ entlang den Inklusionen $i_{\sigma}:
  |\sigma| \xhookrightarrow{} \Kreal$.
\end{proof}

\begin{prop}
  Sei $F \in \Ket^+(\AbKr)$ aus S-azyklischen mit schwach konstruierbaren Kohomologiegarben. Dann ist die Koeinheit ein Quasiisomorphismus. (???)
\end{prop}


\end{document}
