% Emacs mode: -*-latex-*-

\section{Schwach konstruierbare Garben auf Simplizialkomplexen}
\label{sec:simp-comp-sk}

Die Ordnungstopologie auf einem Simplizialkomplex $\K$ erlaubt es,
sich Simplizialkomplexe von Garben als Garben auf einem Produktraum
mit einem Simplizialkomplex vorzustellen. Nach wie vor ist der
verwendete ordnungstopologische Simplizialkomplex allerdings ein eher
kombinatorisches Objekt. Für unsere zu weiten Teilen auf
Hausdorffräumen beruhende geometrische Vorstellung geben wir daher
eine weitere geometrische Charakterisierung von Garben auf einem
Simplizialkomplex $\K$ an. Wir werden sehen, dass sie sich als
simplizial konstante Garben auf der geometrischen Realisierung von
$\K$ auffassen lassen. Für die zugehörigen derivierten Kategorien kann
diese Aussage noch verbessert werden: Simplizialkomplexe von Mengen
entsprechen dann derivierten Garben, deren Kohomologien simplizial
konstant sind. Die Darstellung folgt im Wesentlichen \cite{KS} und
\cite{WS}. Der relative Fall ist Gegenstand des folgenden
Abschnitts. Hier beschränken wir uns erst einmal auf
Simplizialkomplexe von Mengen.

% TODO: lokal-endlich?

In diesem Abschnitt bezeichne $\K$ stets einen Simplizialkomplex mit
Eckenmenge $E$. Ist $\K$ ein endlicher Simplizialkomplex (d.~h. $E$
endlich), so definieren wir seine geometrische Realisierung als die
Vereinigung seiner offenen Simplizes. Setze für einen Simplex $\sigma
\in \K$ seine Realisierung $|\sigma| \subset \R^V = \Ens(V, \R)$ zu
\[ |\sigma| = \set{x \in \R^E}{x(e) = 0 \text{ für } e \notin \sigma,
   x(e) > 0 \text{ für } e \in \sigma,
   \sum_{e \in E} x(e) = 1 },
\]
sowie die geometrische Realisierung $|\K| \subset \R^E$ von $\K$
\[ \bigcup_{\sigma \in \K} |\sigma|, \]
jeweils versehen mit der induzierten Topologie von $\R^E$. Ein
beliebiger Simplizialkomplex $\K$ kann als Vereinigung $\bigcup_i
\K_i$ seiner endlichen Teilsimplizes geschrieben werden und wir
definieren die geometrische Realisierung von $\K$ durch
\[ |\K| = \bigcup_i |\K_i| \]
mit den induzierten Inklusionen und der Kolimes-Topologie.

Wir erhalten eine Abbildung
\[ p: |\K| \to \K, \]
genannt Simplexanzeiger oder Indikatorabbildung, der einem Punkt $x
\in |\K|$ in der geometrischen Realisierung den eindeutigen Simplex
$\sigma \in \K$ mit $x \in |\sigma|$ zuordnet.

\begin{lemma}
  Der Simplexanzeiger $p: |\K| \to \K$ ist eine finale Surjektion mit
  zusammenhängenden Fasern.
\end{lemma}
\begin{proof}
  Die Abbildung lässt sich als die Identifikation der
  (zusammenhängenden) offenen Inneren der Standardsimplizes $|\sigma|$
  zu einem Punkt $\sigma$ beschreiben. Dies zeigt die Eigenschaft
  einer mengentheoretischen Quotientenabbildung.
  % TODO Quotiententopologie

  Für einen endlichen Simplizialkomplex $\K$ ist das Urbild einer
  Basismenge $(\geq \sigma)$ der ``offene Stern um $\sigma$''
  \[ U(\sigma) := p^{-1}((\geq \sigma))
  = |\K| \cap \set{x \in \R^E}{x(e) > 0 \text{ für } e \in \sigma }.
  \]
  Dies zeigt (auch im Kolimes) die Stetigkeit.
\end{proof}
\begin{lemma}[\cite{TG}, 4.3.22] \label{final-pullback}
  Sei $p: X \to Y$ eine finale Surjektion mit zusammenhängenden
  Fasern. Dann ist die Einheit der Adjunktion $\Id \Trafo p_* p^*$ auf
  allen $G \in \Ens_{/Y}$ ein Isomorphismus.
\end{lemma}
\begin{proof}
  Wir zeigen, dass die Einheit der Adjunktion Bijektionen auf allen
  Schnitten über $U \open Y$ induziert. Ohne Einschränkung reicht der
  Fall der globalen Schnitte $U = Y$, denn die Einschränkung von $p$
  auf $p^{-1}(U) \to U$ erfüllt ebenfalls die Voraussetzungen des
  Lemmas. Zu zeigen ist also, dass die Einheit der Adjunktion
  Bijektionen $\Gamma G \iso \Gamma p_* p^* G = \Gamma p^* G$
  induziert. Nach der universellen Eigenschaft des Rückzugs stehen im
  Diagramm
  \[ 
  \begin{tikzcd}
    X \arrow[ddr]{\id} \arrow[drr]{t} \arrow[dr]{s}
    p^* G \rar \dar & G \dar \\
    X \rar{p} & Y
  \end{tikzcd}
  \]
  globale Schnitte $s: X \to p^* G$ von $p^* G \to X$ in Bijektion zu
  Schnitten $t: X \to G$ von $G \to Y$ über $p$. Eine solche Abbildung
  $t$ faktorisiert nun nach der Surjektivität von $p$ zunächst
  mengentheoretisch über $p$. Diese Faktorisierung ist eindeutig, denn
  $t$ ist konstant auf den Fasern $p^{-1}(y)$ von $y \in Y$ als
  stetige Abbildung von einem zusammenhängenden Raum in den diskreten
  Raum $G_y$. Die Stetigkeit dieser mengentheoretischen Abbildung
  folgt nun aus der Finalität von $p$.

  Die umgekehrte Zuordnung schaltet einem globalen Schnitt von $G \to
  Y$ über $\id_Y$ einen Schnitt über $p$ durch Vorschalten von $p$ zu.
\end{proof}
\begin{bem} \label{connected-fibers-subst}
  Die Voraussetzung zusammenhängender Fasern kann fallen gelassen
  werden, wenn stattdessen in $p^* G$ nur diejenigen Garbenschnitte
  als Schnitte zugelassen werden, welche gewisse zu den Daten von $p^*
  G$ gehörige Verträglichkeiten erfüllen müssen, die die Abbildung
  $p^{-1}(y) \to G_y$ wieder konstant machen. Dies werden wir in
  \ref{sheaf-col} benötigen und dort auch präzisieren.
\end{bem}
\begin{kor} \label{unit-iso}
  Für $p: |\K| \to \K$ und eine Garbe $F \in \Ens_{/\K}$ ist die
  Einheit der Adjunktion ein Isomorphismus
  \[ F \iso p_* p^* F . \]
\end{kor}
Anders ausgedrückt können wegen
\[ \Ens_{/\K}(F, G) \iso \Ens_{/\K}(F, p_* p^* G)
\iso \Ens_{/|\K|}(p^* F, p^* G)
\]
die Garben auf $\K$ als äquivalent zu einer vollen Unterkategorie der
Garben auf $|\K|$ aufgefasst werden. Wie können diese Garben im
wesentlichen Bild des Rückzugs $p^*$ charakterisiert werden?
\begin{defn}
  Eine Garbe $F \in \AbKr$ heißt \emph{schwach $\K$-konstruierbar}
  (oder kurz: schwach konstruierbar), falls für alle $\sigma \in \K$,
  die Einschränkungen $F|_{|\sigma|}$ konstante Garben sind. Wir
  bezeichnen die volle Unterkategorie der schwach konstruierbaren
  Garben in $\AbKr$ mit $\AbKrsk$.

  Eine derivierte Garbe $F \in \DerAbK$ heißt \emph{schwach
    $\K$-konstruierbar}, falls für alle $j \in \Z$ die
  Kohomologiegarben $H^j(F)$ schwach konstruierbar sind. Wir
  bezeichnen die volle Unterkategorie der schwach konstruierbaren
  derivierten Garben in $\DerAbK$ mit $\DerskK$.
\end{defn}

Wir bemerken zunächst:
\begin{lemma}[\cite{KS}, 8.1.3] \label{skons-abelian}
  Die Kategorie $\AbKrsk$ ist abelsch.
\end{lemma}
\begin{proof}
  Zunächst ist die Kategorie der konstanten abelschen Garben auf einem
  topologischen Raum $X$ eine abelsche Kategorie. Die Aussage folgt
  dann aus der Exaktheit und Additivität des Pullbacks $i_{\sigma}^*$
  entlang den Inklusionen $i_{\sigma}: |\sigma| \xhookrightarrow{}
  |\K|$.
\end{proof}

Das wesentliche Bild des Rückzugs $p^*$ kann nun als die Kategorie der
schwach $\K$-konstruierbaren Garben auf $|\K|$ charakterisiert
werden. Dies ist der wesentliche topologische Schritt in unserer
Argumentation.
\begin{prop}[\cite{WS}, 8.4.6.3] \label{sk-char}
  Für $F \in \AbKr$ sind äquivalent:
  \begin{enumerate}[label=(\arabic*)]
  \item \label{itm:sk-char-sk} $F$ ist schwach $\K$-konstruierbar
  \item \label{itm:sk-char-counit} Die Koeinheit der Adjunktion ist
    auf $F$ ein Isomorphismus $p^*p_*F \iso F$.
  \item \label{itm:sk-char-essim} $F$ liegt im wesentlichen Bild des
    Rückzugs $p^*$.
  \item \label{itm:sk-char-res} Die Restriktion $F(U(\sigma)) \to F_x$
    ist für alle $\sigma \in \K$ und alle $x \in |\sigma|$ ein
    Isomorphismus.
  \end{enumerate}
\end{prop}
\begin{proof}
  Die Äquivalenz $\ref{itm:sk-char-counit} \Iff
  \ref{itm:sk-char-essim}$ ist allgemein kategorientheoretischer
  Natur. Dabei ist $\ref{itm:sk-char-counit} \Implies
  \ref{itm:sk-char-essim}$ offensichtlich (nimm $p_* F$) und
  $\ref{itm:sk-char-essim} \Implies \ref{itm:sk-char-counit}$ folgt
  aus den Dreiecksidentitäten und \ref{unit-iso}.

  Die Äquivalenz $\ref{itm:sk-char-counit} \Iff \ref{itm:sk-char-res}$
  folgt aus der Bestimmung der Halme von $p^*p_* F$. Zunächst bemerken
  wir, dass in $\K$ die Menge $(\geq \sigma)$ die kleinste offene
  Umgebung von $\sigma$ ist, und wir also $p_*F((\geq \sigma)) \iso
  (p_*F)_{\sigma}$ erhalten. Somit gilt für $x \in |\sigma|$:
  \begin{equation}\label{eq:counit-sect}
    (p^*p_*F)_x \iso (p_*F)_{\sigma} \iso p_*F((\geq \sigma)) \iso F(U(\sigma)).  
  \end{equation}  
  Dabei wurde die Beschreibung der Halme des Rückzugs (mit $p(x) =
  \sigma$), obige Darstellung der Halme auf $\K$ und die Definition
  des Vorschubs (mit $p^{-1}((\geq \sigma)) = U(\sigma)$) verwendet.

  Die Implikation $\ref{itm:sk-char-res} \Implies
  \ref{itm:sk-char-sk}$ folgt direkt aus dem nachgestellten Lemma,
  angewandt auf die Einschränkung von $F$ auf $U(\sigma)$, und der
  Tatsache, dass beliebige Einschränkungen konstanter Garben wieder
  konstant sind.

  Für die umgekehrte Richtung reicht es, die Aussage für die
  Einschränkung von $F$ auf $U(\sigma)$ zu zeigen.  Wir betrachten für
  $x \in |\sigma|$ die Zusammenziehung
  \begin{align*}
    h: (0, 1] \times U(\sigma) &\to U(\sigma), \\
    (t, y) &\mapsto h(t, y) = t y + (1 - t) x.
  \end{align*}
  
  Die Mengen $h(\{t\} \times U(\sigma))$ bilden für $t \in (0, 1]$
  eine Umgebungsbasis von $x$, wir müssen also nur noch den Kolimes
  der Schnitte über diese Mengen bestimmen. Bezeichne $\pi: (0, 1]
  \times U(\sigma) \to U(\sigma)$ die Projektion auf den zweiten
  Faktor. Nach der simplizialen Konstanz von $F$ und wegen
  \[ h(t, y) \in |\tau| \Iff y \in |\tau| \]
  ist der Rückzug $h^* F$ konstant auf den Fasern von $\pi$ und lässt
  sich somit nach dem zweiten nachgestellten Lemma schreiben als
  $\pi^* \pi_* h^* F \iso h^* F$. Bezeichne $\iota_t: U(\sigma)
  \xhookrightarrow{} (0, 1] \times U(\sigma)$ die Inklusion in der
    ersten Komponente. Dann erhalten wir wie gewünscht mit der
    Funktorialität des Rückzugs und $\pi \circ \iota_t =
    \id_{U(\sigma)}$

  \begin{align*}
     F_x &\iso \colf\limits_{t \in (0, 1]} F(h(\{t\} \times U(\sigma))) \\
         &\iso \colf\limits_{t \in (0, 1]} \Gamma \, \iota_t^* \, h^* \, F \\
         &\iso \colf\limits_{t \in (0, 1]}
           \Gamma \, \iota_t^* \, \pi^* \, \pi_* \, h^* \, F \\
           &\iso \colf\limits_{t \in (0, 1]}
             \Gamma \, \iota_1^* \, \pi^* \, \pi_* \, h^* \, F \\
         &\iso \Gamma \, \iota_1^* \, h^* \, F \\    
         &\iso F(U(\sigma)).
  \end{align*}
\end{proof}
\begin{bem} \label{beta-sect}
  Aus Gleichung \ref{eq:counit-sect} und \ref{itm:sk-char-res} folgt
  insbesondere auch $(p^* p_* F)(U(\sigma)) \iso F(U(\sigma))$ für
  alle $F \in \AbKr$ und alle $\sigma \in \K$.
\end{bem}

Wir tragen die benötigten Lemmata nach.
\begin{lemma}[\cite{TG}, 2.1.41 (Var.)] \label{const-stalk}
  Sei $X$ ein topologischer Raum, $F \in \EnsX$ eine Garbe auf $X$,
  für die die Restriktion $\Gamma F \iso F_x$ für alle $x \in X$
  bijektiv ist. Dann ist $F$ eine konstante Garbe auf $X$ mit Halm
  $\Gamma F$.
\end{lemma}
\begin{proof}
  Bezeichne $c: X \to \mathrm{top}$ die konstante Abbildung. Die
  Koeinheit der Adjunktion $c^* c_* F \to F$ induziert auf den Halmen
  gerade die vorausgesetzten Bijektionen, ist also ein
  Garben-Isomorphismus.
\end{proof}

\begin{lemma}[\cite{TG}, 6.4.17] \label{constant-on-fibers}
  Sei $X$ ein topologischer Raum, $I \subset \R$ ein nichtleeres
  Intervall, $F \in \Ens_{/X \times I}$ eine Garbe und $\pi: X \times
  I \to X$ die Projektion auf den ersten Faktor. Ist $F$ konstant auf
  den Fasern von $\pi$, so ist die Koeinheit der Adjunktion auf $F$
  ein Isomorphismus $\pi^* \pi_* F \iso F$.
\end{lemma}
\begin{proof}
  Die Aussage ist äquivalent zum folgenden Fortsetzungsresultat:
  \begin{quote}
    Für alle $U \open X$ und $t \in I$ ist die Restriktion ein
    Isomorphismus
    \[ \Gamma(U \times I, F) \iso \Gamma(U \times \{t\}, F). \]
  \end{quote}
  Denn ist die Koeinheit der Adjunktion ein Isomorphismus $\pi^* \pi_*
  F \iso F$, so bestimmen wir die Schnitte über $U \times \{t\}$ wie
  folgt: Sei $\iota: U \times \{t\} \inj X \times I$ die Inklusion. Wir
  bemerken, dass $\pi \circ \iota$ die Inklusion von $U$ nach $X$ ist
  und erhalten:
  \[ \Gamma(U \times \{t\}, F)
      = \Gamma \iota^* F
      \iso \Gamma \iota^* \pi^* \pi_* F
      = \Gamma(U, \pi_* F)
      = \Gamma(U \times I, F). \]
  Andersherum folgt der Isomorphismus der Koeinheit der Adjunktion
  aber auch aus dem Fortsetzungsresultat, denn wir können sofort den
  Isomorphismus auf den Halmen über $(x, t) \in X \times I$ zeigen:
  \begin{align*}
    (\pi^* \pi_* F)_{(x,t)}
    &\iso (\pi_* F)_x \\
    &= \colf\limits_{U \ni x} \Gamma(U \times I, F) \\
    &\iso \colf\limits_{U \ni x} \Gamma(U \times \{t\}, F) \\
    &\iso \colf\limits_{V \ni (x, t)} F(V) \\
    &= F_{(x, t)}.
  \end{align*}
  Dabei erhalten wir die Surjektivität von $F(V) \to \Gamma(U \times
  \{t\}, F)$ aus der Bijektivität der Verknüpfung
  \[ \Gamma(U \times I, F) \to F(V) \to \Gamma(U \times \{t\}, F) \]
  und die Injektivität aus der Eigenschaft, dass bereits die faserweise
  stetige Fortsetzung eindeutig ist nach der Konstantheit der
  Einschränkungen von $F$ auf die Fasern von $\pi$.

  Nun können wir die Aussage zeigen. Zunächst folgt sie für $I$
  kompakt sofort aus eigentlichem Basiswechsel über dem kartesischen
  Diagramm mit eigentlichen und separierten Vertikalen
  % TODO korrigieren Basiswechsel
  \shorthandoff{"}
  \[ \begin{tikzcd}
    (x, t) \rar[hook] \dar \arrow[dr, phantom, "\lrcorner"]
    & X \times I \dar["\pi"] \\
    x \rar[hook]
    & X.
  \end{tikzcd} \]
  Ist nun $I \subset \R$ ein beliebiges Intervall, so können wir es
  als aufsteigende Vereinigung von Kompakta $I = \bigcup_j I_j$
  schreiben und erhalten durch das Verkleben von Schnitten ebenfalls
  \[ \Gamma(U \times I, F)
  \iso \colf_j \Gamma(U \times I_j, F)
  \iso \Gamma(U \times \{t\}, F).
  \]
\end{proof}

Wir bezeichnen den Funktor $p^*p_*: \AbKr \to \sKons(\K)$ kurz mit
$\beta$. Es handelt sich in der später eingeführten Terminologie um
einen Koreflektor (vgl. \ref{corefl}).
\begin{prop} \label{beta-adjoint}
  Der Funktor $\beta: \AbKr \to \AbKrsk$ ist ein Rechtsadjungierter
  zur Inklusion $\iota: \AbKrsk \to \AbKr$.
\end{prop}
\begin{proof}
  Für $S \in \AbKrsk$ und $F \in \AbKr$ gilt:
  \begin{align*}
    & \AbKrsk(S, \beta F) \\
    \iso & \AbKrsk(\beta S, \beta F) \\
    \iso & \AbKr(p_* S, p_* p^* p_* F) \\
    \iso & \AbKr(p_* S, p_* F) \\
    \iso & \AbKr(\beta S, F) \\
    \iso & \AbKr(S, F).
  \end{align*}
  Dabei wurden die Adjunktion $(p^*, p_*)$ und im dritten Schritt
  \ref{unit-iso} benutzt.
\end{proof}

Als Komposition zweier linksexakter Funktoren ist $\beta$ wieder
linksexakt. Der allgemeinen Terminologie folgend bezeichnen wir eine
Garbe $F \in \AbKr$ als $\beta$-azyklisch, falls ihre höheren
Derivierten von $\beta$ verschwinden, also falls
\[ R^k\beta F = 0 \qquad \text{für alle } k > 0.  \]

Später benötigen wir die folgende Charakterisierung
$\beta$-azyklischer Garben:
% TODO benötigt?
\begin{prop}[\cite{KS}, 8.1.8] \label{sk-is-acyc}
  Sei $F \in \AbKr$. Dann gilt:
  \begin{enumerate}[label=(\roman*)]
  \item \label{itm:sk-is-acyc-comp} $\Gamma(U(\sigma), R^k\beta F)
    \iso H^k(U(\sigma); F)$ für alle $\sigma \in \K, k \geq 0$.
  \item \label{itm:sk-is-acyc-beta} $F$ ist $\beta$-azyklisch genau dann,
    wenn $H^k(U(\sigma); F) = 0$ für alle $\sigma \in \K, k > 0$.
  \end{enumerate}
  Insbesondere sind schwach $\K$-konstruierbare Garben und welke
  Garben $\beta$-azyklisch.
\end{prop}
\begin{bem}
  Man beachte, dass Aussage \ref{itm:sk-is-acyc-beta} eine allgemeine
  Verbesserung unserer allgemeinen Charakterisierung höherer direkter
  Bilder als Garbifizierungen der Prägarben der Kohomologien der
  Fasern ist für den Fall, dass die Schnittfunktoren
  $\Gamma(\pi^{-1}(U), \cdot)$ exakt sind.
\end{bem}
\begin{proof}
  \ref{itm:sk-is-acyc-beta} folgt direkt aus \ref{itm:sk-is-acyc-comp} mit
  $R^k\beta F \in \AbKrsk$ (vgl. \ref{skons-abelian}) und unserer
  Aussage über die Halme \ref{sk-char} \ref{itm:sk-char-res}.
  
  Mit der Exaktheit von Halmfunktoren $F \mapsto F_x$ und
  \ref{sk-char} \ref{itm:sk-char-res} sind die Funktoren
  $\Gamma(U(\sigma), \cdot)$ auf schwach $\K$-konstruierbaren Garben
  exakt und vertauschen folglich bei einem Komplex mit der Bildung der
  Kohomologie. Wir erhalten für $F \inj I\cc$ eine injektive Auflösung
  somit
  \[ \Gamma(U(\sigma), R^k\beta F)
  = \Gamma(U(\sigma), H^k(\beta I\cc))
  \iso H^k(\Gamma(U(\sigma), \beta I\cc))
  \iso H^k(\Gamma(U(\sigma), I\cc))
  = H^k(U(\sigma), F), \]
  wobei wir im dritten Schritt \ref{beta-sect} verwendet haben.
\end{proof}

Damit können wir uns nun den derivierten Kategorien zuwenden.
\begin{satz} \label{beta-qi}
  Sei $F \in \Ket(\AbKr)$ ein Kettenkomplex aus $\beta$-azyklischen
  Garben mit schwach $\K$-konstruierbaren Kohomologiegarben
  $H^q(F)$. Dann ist $\beta F \to F$ ein Quasi-Isomorphismus.
\end{satz}
\begin{proof}
  Wir müssen zeigen, dass $\beta F \to F$ Isomorphismen $\H^q(\beta F)
  \iso H^q(F)$ auf den Kohomologien induziert. Da aber alle
  beteiligten Garben $\beta$-azyklisch sind und die
  $\beta$-azyklischen Garben eine abelsche Unterkategorie bilden, auf
  der $\beta$ exakt ist, vertauscht $\beta$ mit dem Bilden der
  Kohomologie und es folgt
  \[ H^q(\beta F) \iso \beta (H^q F) \iso H^q F \]
  wegen der schwachen Konstruierbarkeit der Kohomologiegarben $H^q F$.
\end{proof}

Damit ist der entscheidende Schritt für unser Ziel gezeigt. Wir
erhalten:
\begin{theorem} \label{dersk-eq}
  Sei $\K$ ein Simplizialkomplex. Dann definieren die oben definierten
  Funktoren $\iota, \beta$ auf den derivierten Kategorien eine
  Äquivalenz
  \[ \Der(\AbKrsk)
  \mathrel{\mathop{\rightleftarrows}^{\iota}_{R \beta}}
  \DerskK. \]
\end{theorem}
\begin{proof}
  %% Die Funktoren $(\iota, \beta)$ bleiben eine Adjunktion nach
  %% Übergang zu den derivierten Kategorien nach \cite{TD}, 3.2.25.
  % TODO adj. bleibt

  Mit \cite{TD}, 3.6.10, finden wir für $F \in \Ket(\AbKr)$ eine
  homotopieinjektive Auflösung $F \qiso I$ durch injektive Garben, die
  also entfaltet ist für $\beta$. Mit dem vorangegangenen Satz
  \ref{beta-qi} und der $\beta$-Azyklizität injektiver Garben erhalten
  wir für $F \in \DerskK$:
  \[R \beta F \iso \beta I \iso I \iso F. \]
  Tatsächlich hat $R \beta$ Bild in $\Der(\AbKrsk)$ und unsere obige
  Aussage beinhaltet die Isotransformationen $\Id \Isotrafo R \beta
  \circ \iota$ und $\iota \circ R \beta \Isotrafo \Id$.
\end{proof}
\begin{bem}
  Kashiwara und Schapira beschränken sich auf die Äquivalenz der
  beschränkten derivierten Kategorien, allerdings mit der gleichen
  Argumentation. In \cite{KS} 1.7.12 wird eine allgemeine Aussage für
  solche Situationen gezeigt, die hier aber m.E. nicht benötigt wird.
\end{bem}
\begin{bem}
  Ist $\K$ lokal-endlich und von beschränkter Dimension, so erhalten
  wir dasselbe Ergebnis mit deutlich weniger Technik, nämlich ohne
  Rückgriff auf homotopieinjektive Auflösungen. In der Tat können wir
  nach dem Satz über das unbeschränkte Derivieren homologisch
  endlicher Funktoren (\cite{TG}, 3.7.4) mit beliebigen Auflösungen
  aus $\beta$-azyklischen Objekten arbeiten und dann einfach mit
  \ref{sk-is-acyc} \ref{itm:sk-is-acyc-beta} eine beliebige welke
  Auflösung wählen.

  Wir müssen also zeigen, dass $\beta$ homologisch rechtsendlich ist
  und sich jede Garbe $F \in \AbKr$ in eine $\beta$-azyklische
  einbetten lässt. Letzteres folgt wieder aus \ref{sk-is-acyc}
  \ref{itm:sk-is-acyc-beta}, ersteres zeigen wir im folgenden Lemma.

  %% In der Tat können wir die geometrische Realisierung eines solchen
  %% Simplizes lokal (d. h. auf $U(\sigma)$-Mengen) in einen $\R^n$
  %% einbetten. Nun kann ein Schnitt über $U(\sigma)$ (wieso?) auf
  %% eindeutige Weise ausgedehnt werden zu einem Schnitt über
  %% $\overline{U(\sigma)}$ und wir erhalten
  %% \[ H^q(U(\sigma), F) \] \iso H^q_!(\overline{U(\sigma)}, F)
  %% = 0 \quad \text{für } q > n \]

  %% nach der kohomologischen Dimension von $n$-Mannigfaltigkeiten
  %% \cite{TG} ??.
\end{bem}
\begin{lemma}
  Ist $\K$ ein lokal-endlicher Simplizialkomplex beschränkter
  Dimension, so ist $\beta: \AbKr \to \AbKr$ ein homologisch
  rechtsendlicher Funktor.

  %% der homologischen Dimension kleiner oder gleich $\dim \K$.
  %% stimmt das überhaupt?
\end{lemma}

%% Dabei definieren wir die Dimension eines nichtleeren
%% Simplizialkomplexes $\K$ als
%% \[ \dim \K := \sup_{\sigma \in \K} |\sigma| - 1 \]
%% und $\dim \K := - \infty$, falls $\K$ leer ist.

\begin{proof}
  Sei $F \in \AbKr$. Dann ist $R^k \beta F$ schwach
  $\K$-konstruierbar und wir bestimmen für $x \in |\sigma|$ die
  Halme mittels
  \[ (R^k \beta F)_x \iso (R^k \beta F)(U(\sigma)) \iso H^k(U(\sigma); F) \]
  unter Verwendung von \ref{sk-char} \ref{itm:sk-char-res} und
  \ref{sk-is-acyc} \ref{itm:sk-is-acyc-comp}. Das führt die Aussage
  auf die beschränkte homologische Dimension von
  $n$-Mannigfaltigkeiten im folgenden Lemma zurück, denn für $\K$
  lokal-endlich von beschränkter Dimension ist $U(\sigma)$ eine offene
  Teilmenge eines endlichdimensionalen reellen Vektorraums $\R^n$.
\end{proof}
\begin{lemma}
  Für $U \open \R^n$ und $F \in \Ab_{/U}$ gilt $H^q(U; F) = 0$ für
  alle $q > n$.
\end{lemma}
\begin{proof}
  Zunächst können wir ohne Einschränkung $U = \R^n$ annehmen, indem
  wir verwenden, dass die Fortsetzung durch Null entlang der
  Einbettung $j: U \inj \R^n$ Isomorphismen
  \[ H^q(U; F) \iso H^q(U; j_! F) \]
  liefert.

  Nach \cite{KS}, 3.2.2, gibt es für $F \in \Ab_{/\R^n}$ eine
  $n$-Schritt Auflösung durch kompaktweiche Garben. Wir müssen also
  zeigen, dass kompaktweiche Garben $\Gamma$-azyklisch sind
  (\cite{TG}, 4.8 zeigt das nur für $\Gamma_!$). Mit dem
  Azyklizitätskriterium \cite{TG}, 4.1.6, und den Aussagen für
  $\Gamma_!$ bleibt nur noch zu zeigen, dass eine kurze exakte Sequenz
  von Garben $F' \inj F \surj F''$ mit $F'$ kompaktweich exakt bleibt
  nach Anwenden von $\Gamma$. Dies folgern wir allerdings sofort aus
  der $\sigma$-Kompaktheit von $\R^n$ und den Aussagen für $\Gamma_!$:
  Sei dazu $\R^n = \bigcup_i K_i$ eine kompakte Ausschöpfung mit $K_i$
  kompakt und $K_i \subset \intr(K_{i+1})$. Dann gilt $\Gamma_!(K_i,
  F) = \Gamma(K_i, F)$, $F|_{K_i}$ ist kompaktweich und unsere kurze
  exakte Sequenz bleibt exakt nach Anwenden von $\Gamma(K_i, \cdot)$
  und somit auch im filtrierenden Kolimes nach Anwenden von $\Gamma =
  \colf_i \Gamma(K_i, \cdot)$.
\end{proof}
\begin{bem} \label{sk-homalg}
  Man beachte, dass der Beweis ab \ref{beta-adjoint} ausschließlich
  auf homologischer Algebra beruht, die konkreten topologischen
  Eigenschaften der Situation fließen im Wesentlichen in \ref{sk-char}
  ein. Dies wird sich in \autoref{sec:real-directed-cat} als nützlich
  erweisen, um die Aussage für simpliziale Mengen zu formulieren.
\end{bem}
Wir halten den allgemeinen Teil des Arguments fest:
\begin{satz} \label{gen-sk}
  Seien $C$ und $D$ abelsche Kategorien mit einem Paar adjungierter
  Funktoren $(p^*, p_*)$
  \[ C \mathrel{\mathop{\rightleftarrows}^{p^*}_{p_*}} D, \]
  für das die Einheit der Adjunktion $\Id \Isotrafo p_* p^*$ auf allen
  Objekten ein Isomorphismus ist. Bezeichne Objekte $d \in D$, auf
  denen die Koeinheit $p^* p_* \Trafo \Id$ ein Isomorphismus ist, als
  \emph{schwach konstruierbar}. Dann ist die volle Unterkategorie der
  schwach konstruierbaren Objekte $D^{\sk}$ von $D$ koreflektiv mit
  dem Koreflektor $\beta = p^* p_*$. Gibt es in $\Der(D)$ für
  beliebige unbeschränkte Kettenkomplexe homotopieinjektive
  Auflösungen durch injektive Objekte, so gibt es eine Äquivalenz von
  Kategorien
  \[ \Der(D^{\sk})
  \mathrel{\mathop{\rightleftarrows}^{\iota}_{R \beta}}
  \Der^{\sk}(D),
  \]
  wobei $\Der^{\sk}(D)$ die volle Unterkategorie der derivierten
  Kategorie von $D$ mit schwach konstruierbarer Kohomologie
  bezeichnet.
\end{satz}
