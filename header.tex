\documentclass[a4paper]{book}
% \usepackage[left=3cm,right=3cm,top=3cm,bottom=2cm]{geometry} % page settings
\usepackage{amsmath}
\usepackage{amssymb}
\usepackage{amsthm}
\usepackage{etoolbox}
\usepackage[ngerman]{babel}
\usepackage[utf8]{inputenc}
\usepackage{mathtools}
\usepackage{tikz-cd}
\usepackage{enumitem}
\usepackage{hyperref}
\usepackage{extpfeil}
\usepackage{tabulary}
\usepackage[T1]{fontenc}
%% \usepackage[backend=bibtex,
%%             dateabbrev=false,
%%             style=authoryear]{biblatex}

\setlength{\parskip}{\medskipamount}
\setlength{\parindent}{0pt}

\mathtoolsset{centercolon}  % for correct ':='

\theoremstyle{plain}
\newtheorem{theorem}{Theorem}[chapter]
\newtheorem{lemma}[theorem]{Lemma}
\newtheorem{prop}[theorem]{Proposition}
\newtheorem{kor}[theorem]{Korollar}
\newtheorem{satz}[theorem]{Satz}
\newtheorem{bsp}[theorem]{Beispiel}
\newcommand{\satzautorefname}{Satz}
\newcommand{\lemmaautorefname}{Lemma}
\newcommand{\propautorefname}{Prop.}
\newcommand{\korautorefname}{Korollar}
\newcommand{\bspautorefname}{Beispiel}
% \newcommand{\theoremautorefname}{Theorem}
%% \providecommand*{\lemmaautorefname}{Lemma}
%% \providecommand*{\propautorefname}{Prop.}
%% \providecommand*{\korautorefname}{Korollar}
%% \providecommand*{\satzautorefname}{Satz}

\theoremstyle{definition}
\newtheorem{defn}[theorem]{Definition}

\theoremstyle{remark}
\newtheorem{bem}[theorem]{Bemerkung}

\DeclareMathOperator{\pos}{pos}
\DeclareMathOperator{\Loc}{Loc}
\DeclareMathOperator{\Off}{Off}
\DeclareMathOperator{\Ord}{Ord}
\DeclareMathOperator{\Step}{Step}
\DeclareMathOperator{\Sub}{Sub}
\DeclareMathOperator{\TW}{TW}
\DeclareMathOperator{\Hom}{Hom}
\DeclareMathOperator{\Esp}{Esp}
\DeclareMathOperator{\Cat}{Cat}
\DeclareMathOperator{\poset}{poset}
\DeclareMathOperator{\EnsX}{Ens_{/X}}
\DeclareMathOperator{\pEnsX}{pEns_{/X}}
\DeclareMathOperator{\AbX}{Ab_{/X}}
\DeclareMathOperator{\pAbX}{pAb_{/X}}
\DeclareMathOperator{\OffX}{Off_X}
\DeclareMathOperator{\Ens}{Ens}
\DeclareMathOperator{\Ob}{Ob}
\DeclareMathOperator{\Der}{Der}
\DeclareMathOperator{\Ab}{Ab}
\DeclareMathOperator{\Ket}{Ket}
\DeclareMathOperator{\EnsB}{Ens_{/\B}}
\DeclareMathOperator{\im}{im}
\DeclareMathOperator{\ev}{ev}
\DeclareMathOperator{\Id}{Id}
\DeclareMathOperator{\id}{id}
\DeclareMathOperator{\colf}{colf}
\DeclareMathOperator{\col}{col}
\DeclareMathOperator{\limf}{limf}
\DeclareMathOperator{\Top}{Top}
\DeclareMathOperator{\CGHaus}{CGHaus}
\DeclareMathOperator{\CG}{CG}
\DeclareMathOperator{\point}{pt}
\DeclareMathOperator{\s}{s}  % simpliziales Objekt
\DeclareMathOperator{\p}{p}  % Prägarbe
\DeclareMathOperator{\lk}{lk}  % lokal konstant
\DeclareMathOperator{\intr}{int}
\DeclareMathOperator{\pr}{pr}
\DeclareMathOperator{\ins}{in}
\DeclareMathOperator{\sk}{sk}  % simplicially constant

\newcommand{\Trafo}{\Implies}
\newcommand{\Isotrafo}{\xRightarrow{\sim}}
\newcommand{\Ub}{\mathrm{\ddot{U}b}}
\newcommand{\etTop}{\mathrm{\acute{e}tTop}}
\newcommand{\et}{\mathrm{\acute{e}t}}
\newcommand{\etalespace}[1]{\overline{#1}}
\newcommand{\B}{\mathcal{B}}
\newcommand{\op}{^\mathrm{op}}
\newcommand{\iso}[1][]{\xrightarrow[#1]{\sim}}
\newcommand{\qiso}{\xrightarrow{\approx}}
\newcommand{\fromqiso}{\xleftarrow{\approx}}
\newcommand{\open}{\subset\kern-0.58em\circ}
\newcommand{\K}{\mathcal{K}}
\newcommand{\Z}{\mathbb{Z}}
\newcommand{\R}{\mathbb{R}}
\newcommand{\N}{\mathbb{N}}
\newcommand{\DerAbK}{\Der(\Ab_{/|\K|})}
\newcommand{\DerskK}{\Der^{\sk}(\AbKr)}
\newcommand{\AbKr}{\Ab_{/|\K|}}
\newcommand{\AbKrsk}{\Ab_{/|\K|}^{\sk}}
\newcommand{\inj}{\hookrightarrow}
\newcommand{\surj}{\twoheadrightarrow}
\newcommand{\Iff}{\Leftrightarrow}
\newcommand{\Implies}{\Rightarrow}
\newcommand{\iHom}{\Rightarrow}
\newcommand{\cc}{^{\bullet}}  % chain complex
\newcommand{\from}{\leftarrow}
\newcommand{\slicecat}[3]{#1 \downarrow \! {\scriptstyle #2} \, #3}
\newcommand{\del}{\partial}
\newcommand{\set}[2]{\left\{ #1 \,\middle|\, #2 \right\}}
\newcommand{\sslash}{\mathbin{/\mkern-6mu/}}  % double slash
\newcommand{\EnsTop}{\Ens_{/\Top}}
\newcommand{\EnsssTop}{\Ens_{\sslash \Top}}  % Ens slash slash Top
\renewcommand{\coprod}{\bigsqcup}

\hyphenation{
  Di-a-gramms
  }

% Allow for colons in tikz pictures. Code from
% https://tex.stackexchange.com/questions/89467/why-does-pdftex-hang-on-this-file

\makeatletter
\AtEndPreamble{
  \ifnum\mathcode`\:=\string"8000
    \begingroup\lccode`\~=`\:
    \lowercase{\endgroup\let\math@colon@meaning~}
  \else
    \expandafter\let\expandafter\math@colon@meaning\string:
  \fi
}
\AtBeginDocument{
  \ifnum\catcode`\:=\active
    \letcs\text@colon@meaning{active@char\string:}
  \else
    \expandafter\let\expandafter\text@colon@meaning\string:
  \fi
\protected\def\tikz@nonactivecolon{%
  \ifmmode
    \expandafter\math@colon@meaning
   \else
    \expandafter\text@colon@meaning
   \fi} 
\begingroup\lccode`\~=`\:
\lowercase{\endgroup\let~\tikz@nonactivecolon}
}
\makeatother

% Allow unaligning in align environment. Code from
% https://tex.stackexchange.com/questions/84345/how-do-i-get-this-strange-alignment
\makeatletter
\newcommand{\unalign}[1]{%
  \ifmeasuring@ \else
    \makebox[\ifcase 1\maxcolumn@widths \fi][l]{$\displaystyle#1$}
  \fi
}
\makeatother
