\documentclass[a4paper]{article}
% \usepackage[left=3cm,right=3cm,top=3cm,bottom=2cm]{geometry} % page settings
\usepackage{amsmath}
\usepackage{amssymb}
\usepackage{amsthm}
\usepackage[textsize=small]{todonotes}
% \usepackage{etoolbox}
\usepackage[ngerman]{babel}
\usepackage[utf8]{inputenc}
% \usepackage{framed}
% \usepackage{fancyhdr}
% \usepackage{tikz}

\setlength{\parskip}{\medskipamount}
\setlength{\parindent}{0pt}

% \newcounter{satz}

\newtheorem{satz}{Satz}[section]
\newtheorem{defn}[satz]{Definition}
\newtheorem{theorem}[satz]{Theorem}
\newtheorem{prop}[satz]{Proposition}
\newtheorem{cor}[satz]{Korollar}


\newcommand{\N}{\mathbb{N}}
\newcommand{\R}{\mathbb{R}}
\newcommand{\op}{^{op}}
\newcommand{\del}{\partial}
\newcommand{\et}{\acute{e}t}

% \makeatletter
% \@addtoreset{satz}{section}% Reset satz counter with every section
% \@addtoreset{theorem}{subsection}
% \@addtoreset{theorem}{subsubsection}
%% \newcommand{\theoremprefix}{}
%% \let\thetheoremsaved\thetheorem
%% \renewcommand{\thetheorem}{\theoremprefix\thetheoremsaved}
%% \let\sectionsaved\section
%% \patchcmd{\@startsection}{\par}{\renewcommand{\theoremprefix}{\csname the#1\endcsname.}}{}{}
%% \makeatother


%% \pagestyle{fancy}
%% \lhead{
%%     \begin{tabular}{ll}
%%     Proseminar Endliche Spiegelungsgruppen \\
%%     Prof. Dr. Katrin Wendland \\
%%     PD Dr. Emanuel Scheidegger
%%   \end{tabular}
%% }
%% \chead{}
%% \rhead{
%%   \begin{tabular}{ll}
%%     Fabian Glöckle \\
%%     24.01.2017
%%   \end{tabular}
%% }
%% \lfoot{}
%% \cfoot{}
%% \rfoot{} 


\begin{document}

\section{Diagramme in der Kategorie der Garben}

\begin{defn}
  Sei $I$ eine kleine Kategorie. Unter einem Diagramm der Form $I$ in
  $Ab/X$ bzw. einem $I$-System in $Ab/X$ verstehen wir einen Funktor $I
  \to Ab/X$.
\end{defn}

Sind $i \to j$ Objekte mit einem Morphismus in $I$, so notieren wir
für sein Bild in $Ab/X$ häufig $F_i \to F_j$.

Es bezeichne $\Delta$ die Kategorie der endlichen nichtleeren
Ordinalzahlen mit monotonen Abbildungen als Morphismen. Wir notieren
die $(n+1)$-elementige Ordinalzahl der Dimension ihrer zugehörigen
geometrischen Realisierung gemäß mit $[n]$.

Wie jede partiell geordnete Menge, kann ein Objekt $X$ von $\Delta$
eingebettet werden in die Kategorie der kleinen Kategorien $Cat$:
wähle dazu als Objektmenge $X$ selbst und statte mit einem einzigen
Morphismus $x \to y$ aus, wann immer $x \leq y$ in $X$. Wir notieren
diese Kategorie mit $i(X)$.

Eine simpliziale Menge $X$ ist ein Funktor $\Delta\op \to Ens$,
d.h. für jedes $n \in \N$ eine Menge $X_n$, mit Rand- und
Degenerationsabbildungen (denn solche Abbildungen erzeugen die
monotonen Abbildungen).

\begin{defn}
  Wir bezeichnen mit dem Nerv einer kleinen Kategorie $I$ das Bild von
  $I$ unter dem Funktor $N: Cat \to [\Delta]\op, Ens]$ von der
    Kategorie der kleinen Kategorien in die Kategorie der simplizialen
    Mengen, der gegeben ist durch $$N(I)([n]) = [i([n]), I]$$ für alle
    $I \in Cat$ und $[n] \in \Delta$.
\end{defn}

Hierbei steht $[A, B]$ für die Kategorie der Funktoren $A \to B$ mit
Transformationen als Morphismen.

Ausgeschrieben bedeutet diese Definition, dass der Nerv von $I$ die
simpliziale Menge ist, deren $n$-Simplizes Diagramme in $I$ der Form
$A_0 \to A_1 \to \dots \to A_n$ sind, mit Degenerationsabbildungen dem
Einfügen der Identität an der Stelle $A_i$ und Randabbildungen der
Komposition der in $A_i$ ein- und auslaufenden Abbildung
(bzw. Weglassen von $A_0$ oder $A_n$).

Wir erinnern an die Definition des geometrischen Standard-$n$-Simplex
$$ \Delta^n = \{(x_0, \dots, x_n) \in \R^{n+1} | x_i \geq 0 \forall
i \in \{0, \dots, n\}, \sum x_i = 1\} $$ und seine Degenerations- und
Seitenabbildungen $d_i$ und $s_i$. \todo{präzisieren}

Sind $\del_i$ und $\sigma_i$ die Degenerations- bzw. Seitenabbildungen
einer simplizialen Menge $X$, so erhalten wir eine geometrische
Realisierung von $X$ mittels

$$ |X| = \coprod_n X_n \times \Delta_n / \sim ,$$

wobei $\sim$ die Äquivalenzrelation beschreibt, die von $(s, d_i p) \sim
(\del_i s, p)$ und $(s, s_i p) \sim (\sigma_i s, p)$ erzeugt ist.

Ziel dieses Abschnitts ist nun, zu einem $I$-System von Garben auf
einem topologischen Raum $X$ eine Garbe über $X \times |N(I)|$ zu
finden, die diese auf eine ``möglichst natürliche'' Weise
repräsentiert.

Wir werden geometrisch vorgehen und die zu den Garben $F_i$ gehörigen
étalen Räume konstant auf Teilmengen der geometrischen Realisierung
des Nervs von $I$ erweitern, disjunkt vereinigen und mit einer
geeigneten Topologie verkleben.

Betrachte dazu die folgende ``geordnete Zerlegung des
Standard-$n$-Simplex''

$$\Delta^n = \coprod_{i=0}^n M_i ,$$

wobei

$$M_i = \{(x_0, \dots, x_n) \in \Delta^n | x_j = 0 \text{für} j < i,
x_i \neq 0 \} . $$

\todo{Bildchen}

Ist nun $F_0 \to F_1 \to \dots \to F_n$ ein $n$-Simplex in $N(I)$, so
ziehen wir $F_i$ längs der Projektion $X \times M_i \to X$ zurück und
erhalten als étalen Raum das Produkt $\et(F_i) \times M_i$. Wir setzen

$$E = \coprod \et(F_i) \times M_i ,$$

wobei über alle $n \in \N$ und alle $n$-Simplizes $F_0 \to F_1 \to
\dots \to F_n$ in $N(I)_n$ vereinigt wird und $\Delta^n$ wie oben
zerlegt wird.

\todo{passt das auch mit degenerierten?}

Zur Konstruktion der Topologie auf $E$ zunächst eine Definition.

\begin{defn}
  Ist $F_i$ ein $I$-System von Garben auf einem topologischen Raum
  $X$, so heiße $I$-Schnitt über eine offene Menge $U \subset X$ ein
  Tupel $(s_i)_{i \in J \subset I}$ von Schnitten $s_i \in F_i(U)$
  derart, dass $s_i \mapsto s_j$ unter $F_i(U) \to F_j(U)$, wann immer
  $i \to j$ in einer vollen Unterkategorie $J$ von $I$.

  \todo{J \emph{volle} Unterkategorie?}
\end{defn}

Ist $s$ ein $I$-Schnitt, so erhalten wir wie bei der Konstruktion des
étalen Raums zu einer einzelnen Garbe eine Abbildung $\overline{s}: X
\times |N(I)| \to E, (x, p) \mapsto (s_i)_x$ falls $p \in M_i$ und die
zu $M_i$ gehörige Garbe in der obigen Konstruktion von $E$ $F_i$ ist.

Damit können wir die folgende Topologie auf $E$ definieren.

\begin{defn}
  Wir bezeichnen die oben konstruierte Menge $E$ mit der
  Finaltopologie bezüglich allen $\overline{s}$ für einen $I$-Schnitt
  $s$ den étalen Raum des I-Systems von Garben $F_i$.
\end{defn}

Beachte dabei, dass insbesondere alle in $\et(F_i) \times M_i$ offenen
Mengen wieder offen sind. \todo{stimmt vmtl. nicht, wenn J voll}

Ist allgemein $i \to j, j \to k, i \to k$ ein nicht notwendigerweise
kommutatives Dreieck in $I$, so haben wir zwei Verklebungen $F_i \to
F_k$ und $F_i \to F_j \to F_k$. Das Entscheidende an der Konstruktion
über den Nerv ist nun, dass gerade nur solche Dreiecke in $I$
2-Simplizes sind, bei denen beide übereinstimmen, d.h. die Komposition
der ersten Abbildungen gerade die letzte ist bzw. das Dreieck
kommutiert. So können die étalen Räume auch geometrisch
verkleben. \todo{besser erklären}

Dass es sich bei diesem topologischen Raum tatsächlich um den étalen
Raum einer Garbe auf $X \times |N(I)|$ handelt, zeigt der folgende
Satz.

\begin{satz}
  Die Abbildung $E \to X \times |N(I)|, (s_x, p) \mapsto (x, p)$ ist
  étale.
\end{satz}

\begin{proof}
  Folgt. \todo{...} Idee: um die offene Umgebung für den lokalen
  Homöomorphismus zu konstruieren, müssen wir im Wesentlichen diesen
  Halm in $E$ zu einem $I$-Schnitt fortsetzen. Das ist aber nicht
  schwer, denn dazu müssen wir nur den Schnitt $s_i \in F_i(U)$, der
  unseren Halm darstellt, mit den Nerv-Simplex-Abbildungen $F_i \to
  F_{i+1} \to \dots$ abbilden.
\end{proof}

Inwiefern diese Konstruktion die gewünschten Eigenschaften einer
geometrsichen Realisierung von Diagrammen von Garben hat, soll im Rest
dieses Abschnitts untersucht werden. Es stellen sich folgende Fragen:

\begin{enumerate}
\item Ist diese Abbildung $[I, Ab/X] \to Ab/(X \times |N(I)|)$
  funktoriell?
\item Wie lässt sich das wesentliche Bild charakterisieren?
\item Kann ein Umkehrfunktor konstruiert werden, d.h. gibt es eine
  Äquivalenz von Kategorien von $[I, Ab/X]$ zu einer Unterkategorie
  von $Ab/(X \times |N(I)|)$?
\item Wie verhält sich diese Äquivalenz bei Übergang zur derivierten
  Kategorie der Abelschen Garben auf $X$?
\end{enumerate}

Meine Antworten bisher:

\begin{enumerate}
\item Ja.
\item Etwas kompliziert, da die geordnete Zerlegung der Simplizes dann
  auch ``geordnet simplizial konstante Garben'' ergibt; weiter fordert
  die Art der Verklebung automatisch bestimmte
  Fortsetzungseigenschaften von Schnitten.
\item Ja, die komplizierten Bedingungen machen diesen dafür einfach.
\item ???
\end{enumerate}

\end{document}
