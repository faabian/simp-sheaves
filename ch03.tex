% include latex header (\usepackage, \newcommand etc.) 
\documentclass[a4paper]{article}
% \usepackage[left=3cm,right=3cm,top=3cm,bottom=2cm]{geometry} % page settings
\usepackage{amsmath}
\usepackage{amssymb}
\usepackage{amsthm}
\usepackage{etoolbox}
\usepackage[ngerman]{babel}
\usepackage[utf8]{inputenc}
\usepackage{mathtools}
\usepackage{tikz-cd}
\usepackage{enumitem}
\usepackage{hyperref}

\setlength{\parskip}{\medskipamount}
\setlength{\parindent}{0pt}

\theoremstyle{plain}
\newtheorem{theorem}{Theorem}
\newtheorem{lemma}[theorem]{Lemma}
\newtheorem{prop}[theorem]{Proposition}
\newtheorem{kor}[theorem]{Korollar}
\newtheorem{satz}[theorem]{Satz}
%% \providecommand*{\lemmaautorefname}{Lemma}
%% \providecommand*{\propautorefname}{Prop.}
%% \providecommand*{\korautorefname}{Korollar}
%% \providecommand*{\satzautorefname}{Satz}

\theoremstyle{definition}
\newtheorem{defn}[theorem]{Definition}

\theoremstyle{remark}
\newtheorem{bem}[theorem]{Bemerkung}

\DeclareMathOperator{\Cat}{Cat}
\DeclareMathOperator{\poset}{poset}
\DeclareMathOperator{\EnsX}{Ens_{/X}}
\DeclareMathOperator{\pEnsX}{pEns_{/X}}
\DeclareMathOperator{\AbX}{Ab_{/X}}
\DeclareMathOperator{\pAbX}{pAb_{/X}}
\DeclareMathOperator{\OffX}{Off_X}
\DeclareMathOperator{\Ens}{Ens}
\DeclareMathOperator{\Ob}{Ob}
\DeclareMathOperator{\Der}{Der}
\DeclareMathOperator{\Ab}{Ab}
\DeclareMathOperator{\sKons}{s-Kons}
\DeclareMathOperator{\Ket}{Ket}
\DeclareMathOperator{\EnsB}{Ens_{/\B}}
\DeclareMathOperator{\im}{im}
\DeclareMathOperator{\Id}{Id}
\DeclareMathOperator{\id}{id}
\DeclareMathOperator{\colf}{colf}
\DeclareMathOperator{\limf}{limf}
\DeclareMathOperator{\Top}{Top}

\newcommand{\etalespace}[1]{\overline{#1}}
\newcommand{\B}{\mathcal{B}}
\newcommand{\op}{^\mathrm{op}}
\newcommand{\iso}{\xrightarrow{\sim}}
\newcommand{\qiso}{\xrightarrow{\approx}}
\newcommand{\fromqiso}{\xleftarrow{\approx}}
\newcommand{\open}{\subset\kern-0.58em\circ}  % only possible in math mode
\newcommand{\K}{\mathcal{K}}
\newcommand{\Z}{\mathbb{Z}}
\newcommand{\R}{\mathbb{R}}
\newcommand{\DerAbK}{\Der(\Ab_{/|\K|})}
\newcommand{\DerskK}{\Der_{\mathrm{sk}}(|\K|)}
\newcommand{\DerpskK}{\Der^+_{\mathrm{sk}}(|\K|)}
\newcommand{\AbKr}{\Ab_{/|\K|}}
\newcommand{\sKonsK}{\sKons(\K)}
\newcommand{\inj}{\hookrightarrow}
\newcommand{\surj}{\twoheadrightarrow}
\newcommand{\Iff}{\Leftrightarrow}
\newcommand{\Implies}{\Rightarrow}
\newcommand{\cc}{^{\bullet}}  % chain complex
\newcommand{\from}{\leftarrow}


\begin{document}

\title{Simpliziale Garben}
\author{Fabian Glöckle}
\date{\today}
% \maketitle

\section{Schwach konstruierbare Garben auf Simplizialkomplexen, relative Version}

In diesem Abschnitt sollen die Beobachtungen der letzten beiden
Abschnitte vereint werden. Nach Abschnitt \autoref{sec:simp-comp-sheaf}
sind Garben auf einem Simplizialkomplex $\K$ nichts anderes als ein
Simplizialkomplex von Garben über dem einpunktigen Raum. In Abschnitt
\autoref{sec:sk-pt} haben wir diese Garben auf $\K$ geometrisch
charakterisiert als die simplizial konstanten Garben auf der
geometrischen Realisierung $|\K|$ von $\K$. Wir erwarten daher auch
eine relative Version dieser Aussage über einem beliebigen
topologischen Raum $X$, die die Simplizialkomplexe von Garben auf $X$
alias Garben auf $\K \times X$ geometrisch beschreibt.

Wir geben zunächst eine leichte Verallgemeinerung der Aussage von \autoref{sec:simp-comp-sheaf} an.

Wir definieren Garben mit Werten in beliebigen Kategorien $C$ mit der
schon in \autoref{sec:simp-comp-sheaf} verwandten allgemeinen
Abstiegsbedingung.
\begin{defn}[\cite{TG}, 2.1.5]
  Sei $C$ eine vollständige Kategorie und $X$ ein topologischer
  Raum. Eine $C$-wertige Prägarbe $F \in [\OffX\op, C]$ auf $X$ heißt
  $C$-wertige Garbe auf $X$, falls sie die Abstiegsbedingung erfüllt:
  \begin{quote}
    Für alle unter endlichen Schnitten stabilen offenen Überdeckungen
    $U = \bigcup_i U_i$ gilt $F(U) = \lim_i F(U_i)$.
  \end{quote}
\end{defn}
Für $C$ die Kategorien der Mengen oder der abelschen Gruppen ist diese
Definition äquivalent zur bekannten Definition über die eindeutige
Verklebbarkeit von verträglichen Schnitten.

Wir prüfen, dass wir so insbesondere ``garbenwertige Garben'' definieren
können:
\begin{lemma}
  Die Kategorien $\EnsX$ und $\AbX$ der (abelschen) Garben auf einem
  topologischen Raum $X$ sind vollständig.
\end{lemma}
\begin{proof}
  Die Kategorien der Mengen $\Ens$ ist vollständig. Somit ist auch die
  Kategorie der Prägarben auf $X$ $[\OffX\op, \Ens]$ vollständig nach
  der Beschreibung von Limites in Funktorkategorien als objektweisen
  Limites. Tatsächlich ist aber in $\EnsX$ die Limes-Prägarbe eines
  Systems von Garben bereits eine Garbe nach unserer Fromulierung der
  Garbenbedingung als Limes und der Kommutativität von
  Limites. Derselbe Beweis gilt für $\AbX$ unter Verwendung der
  Vollständigkeit der abelschen Gruppen.
\end{proof}

Bezeichne wieder $\B$ die Kategorie der Basis der Topologie auf einem
Produkt topologischer Räume $X \times Y$ mit Inklusionen als
Morphismen.
\begin{satz}
  Seien $X$ und $Y$ topologische Räume. Dann gibt es eine Äquivalenz
  von Kategorien
  \[ (\EnsX)_{/Y} \qiso \Ens_{/\B} \fromqiso \Ens_{/X \times Y} \]
  gegeben durch
  \begin{align*}
    U \times V &\mapsto (F(V))(U) \quad
    \text{für } F \in (\EnsX)_{/Y} \text{ und} \\
     U \times V &\mapsto F(U \times V) \quad
     \text{die Restriktion für } F \in \Ens_{/X \times Y}
  \end{align*}
  für $U \open X$ und $V \open Y$.
\end{satz}
\begin{proof}
  Die zweite Äquivalenz ist \ref{sheaf-on-basis}. Für die erste
  Äquivalenz bemerken wir wie in \ref{simp-comp-sheaf}, dass die
  zugrundeliegenden Prägarbenkategorien übereinstimmen. Nun fordert
  die Garbenbedingung für $\Ens_{/\B}$ die Verklebungseigenschaft für
  beliebige Überdeckungen von Basismengen durch Basismengen, während
  die Garbenbedingungen für $(\EnsX)_{/Y}$ die Verklebungseigenschaft
  für ``Produkt-Überdeckungen'' von Basismengen fordert, d.~h. für
  Überdeckungen der Form $U \times V = \bigcup_{i,j} U_i \times V_j$
  für $U = \bigcup_i U_i$ eine Überdeckung von $U$ und $V = \bigcup_j
  V_j$ eine Überdeckung von $V$. Wir rechnen dies nach:
  \begin{align*}
    (F(V))(U)
    &= (\lim_j F(V_j))(U) \\
    &= \lim_j F(V_j)(U) \\
    &= \lim_j \lim_i F(V_j)(U_i)
  \end{align*}
  wobei im ersten Schritt die Garbenbedingung von $F \in
  (\EnsX)_{/Y}$, und im dritten die von $F(V_j) \in \EnsX$ verwendet
  wurde. Der zweite Schritt ist die Beschreibung von Limites in
  Funktorkategorien als objektweise Limites.

  %% Im zweiten Schritt benutzen wir, dass für $\EnsX$
  %% Schnittfunktoren $\Gamma(U, \cdot) = \Ens_X(\point_U, \cdot)$
  %% darstellbar sind für $\point_U$ die Einschränkung auf $U$ der
  %% konstanten einpunktigen Garbe auf $X$ und somit mit Limites
  %% vertauschen.

  Beide Verklebungsbedingungen sind aber äquivalent, da eine beliebige
  Überdeckung von $U \times V$ natürlich mit den Projektionen auf $X$
  und $Y$ Überdeckungen von $U$ und $V$ induziert und die
  Verträglichkeitsvoraussetzung für die Garbenbedingung von
  $\Ens_{/\B}$ bezüglich der Produktüberdeckung genau dann erfüllt
  ist, wenn sie für die ursprüngliche Überdeckung erfüllt ist.
\end{proof}
\begin{bem}
  Für abelsche Garben erhalten wir die analoge Aussage $(\AbX)_{/Y}
  \qiso \Ab_{/X \times Y}$.
  %% da auch die Schnittfunktoren $\Gamma(U, \cdot) = \AbX(\Z_U, \cdot)$
  %% darstellbar sind
\end{bem}

Wir bezeichnen als eine relativ zu $X$ schwach $|\K|$-konstruierbare
Garbe eine Garbe $F \in \Ab_{|\K| \times X}$, für die die
Einschränkungen $F|_{|\sigma| \times X}$ Rückzüge von Garben auf $X$
sind, es also ein $G \in \AbX$ gibt mit
\[ F|_{|\sigma| \times X} \iso \pi^* G \]
für $\pi: |\sigma| \times X \to X$ die Projektion.

Wir könnten nun für diesen Begriff dieselben Aussagen wie im
vorangegangen Abschnitt mit vollkommen analogen Argumenten erneut
beweisen. Ein bisschen Kategorientheorie und unsere obige
Charakterisierung ermöglichen uns aber ein einfacheres Vorgehen. Wir
bemerken, dass für die Konstruktion unserer Kategorienbifaserung
$\Ab_{/\Top} \to \Top$ mitsamt ihren bekannten Eigenschaften und die
darauf aufbauende Argumentation im vorangegangenen Abschnitt der
Umstand keine Rolle gespielt hat, dass unsere Garben Werte in den
abelschen Gruppen annnehmen. Dieselben Konstruktionen funktionieren
für $\mathcal{A}_{/\Top}$ eine Garbenkategorie mit Werten in einer
beliebigen vollständigen abelschen Kategorie. Davon überzeugt man sich
im Zweifel auch explizit zunächst durch Übertragung auf
Garbenkategorien mit Werten in $R$-Linksmoduln und dann durch den
Einbettungssatz von Mitchell auf beliebige abelsche Kategorien.

Wir erhalten:
\begin{theorem}
  % Formulierung des Theorems in ch02 in seiner endgültigen Form,
  % übertragen auf den relativen Fall
\end{theorem}

\end{document}
