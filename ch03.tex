% Emacs mode: -*-latex-*-
% include latex header (\usepackage, \newcommand etc.) 
\documentclass[a4paper]{article}
% \usepackage[left=3cm,right=3cm,top=3cm,bottom=2cm]{geometry} % page settings
\usepackage{amsmath}
\usepackage{amssymb}
\usepackage{amsthm}
\usepackage{etoolbox}
\usepackage[ngerman]{babel}
\usepackage[utf8]{inputenc}
\usepackage{mathtools}
\usepackage{tikz-cd}
\usepackage{enumitem}
\usepackage{hyperref}

\setlength{\parskip}{\medskipamount}
\setlength{\parindent}{0pt}

\theoremstyle{plain}
\newtheorem{theorem}{Theorem}
\newtheorem{lemma}[theorem]{Lemma}
\newtheorem{prop}[theorem]{Proposition}
\newtheorem{kor}[theorem]{Korollar}
\newtheorem{satz}[theorem]{Satz}
%% \providecommand*{\lemmaautorefname}{Lemma}
%% \providecommand*{\propautorefname}{Prop.}
%% \providecommand*{\korautorefname}{Korollar}
%% \providecommand*{\satzautorefname}{Satz}

\theoremstyle{definition}
\newtheorem{defn}[theorem]{Definition}

\theoremstyle{remark}
\newtheorem{bem}[theorem]{Bemerkung}

\DeclareMathOperator{\Cat}{Cat}
\DeclareMathOperator{\poset}{poset}
\DeclareMathOperator{\EnsX}{Ens_{/X}}
\DeclareMathOperator{\pEnsX}{pEns_{/X}}
\DeclareMathOperator{\AbX}{Ab_{/X}}
\DeclareMathOperator{\pAbX}{pAb_{/X}}
\DeclareMathOperator{\OffX}{Off_X}
\DeclareMathOperator{\Ens}{Ens}
\DeclareMathOperator{\Ob}{Ob}
\DeclareMathOperator{\Der}{Der}
\DeclareMathOperator{\Ab}{Ab}
\DeclareMathOperator{\sKons}{s-Kons}
\DeclareMathOperator{\Ket}{Ket}
\DeclareMathOperator{\EnsB}{Ens_{/\B}}
\DeclareMathOperator{\im}{im}
\DeclareMathOperator{\Id}{Id}
\DeclareMathOperator{\id}{id}
\DeclareMathOperator{\colf}{colf}
\DeclareMathOperator{\limf}{limf}
\DeclareMathOperator{\Top}{Top}

\newcommand{\etalespace}[1]{\overline{#1}}
\newcommand{\B}{\mathcal{B}}
\newcommand{\op}{^\mathrm{op}}
\newcommand{\iso}{\xrightarrow{\sim}}
\newcommand{\qiso}{\xrightarrow{\approx}}
\newcommand{\fromqiso}{\xleftarrow{\approx}}
\newcommand{\open}{\subset\kern-0.58em\circ}  % only possible in math mode
\newcommand{\K}{\mathcal{K}}
\newcommand{\Z}{\mathbb{Z}}
\newcommand{\R}{\mathbb{R}}
\newcommand{\DerAbK}{\Der(\Ab_{/|\K|})}
\newcommand{\DerskK}{\Der_{\mathrm{sk}}(|\K|)}
\newcommand{\DerpskK}{\Der^+_{\mathrm{sk}}(|\K|)}
\newcommand{\AbKr}{\Ab_{/|\K|}}
\newcommand{\sKonsK}{\sKons(\K)}
\newcommand{\inj}{\hookrightarrow}
\newcommand{\surj}{\twoheadrightarrow}
\newcommand{\Iff}{\Leftrightarrow}
\newcommand{\Implies}{\Rightarrow}
\newcommand{\cc}{^{\bullet}}  % chain complex
\newcommand{\from}{\leftarrow}


\begin{document}

\title{Simpliziale Garben}
\author{Fabian Glöckle}
\date{\today}
% \maketitle

\chapter{Verallgemeinerte Garben}

In diesem Abschnitt sollen die Beobachtungen der letzten beiden
Abschnitte vereint werden. Nach Abschnitt
\autoref{sec:simp-comp-sheaf} sind Garben auf einem Simplizialkomplex
$\K$ nichts anderes als ein Simplizialkomplex von Garben über dem
einpunktigen Raum. In Abschnitt \autoref{sec:sk-pt} haben wir diese
Garben auf $\K$ geometrisch charakterisiert als die simplizial
konstanten Garben auf der geometrischen Realisierung $|\K|$ von
$\K$. Wir erwarten daher auch eine relative Version dieser Aussage
über einem beliebigen topologischen Raum $X$, die die
Simplizialkomplexe von Garben auf $X$ alias Garben auf $\K \times X$
geometrisch beschreibt.

Wir geben zunächst eine leichte Verallgemeinerung der Aussage von
\autoref{sec:simp-comp-sheaf} an.

Wir definieren Garben mit Werten in beliebigen Kategorien $C$ mit der
schon in \autoref{sec:simp-comp-sheaf} verwandten allgemeinen
Abstiegsbedingung.
\begin{defn}[\cite{TG}, 2.1.5]
  Sei $C$ eine Kategorie und $X$ ein topologischer Raum. Eine
  $C$-wertige Prägarbe $F \in [\OffX\op, C]$ auf $X$ heißt $C$-wertige
  Garbe auf $X$, falls sie die Abstiegsbedingung erfüllt:
  \begin{quote}
    Für alle unter endlichen Schnitten stabilen offenen Überdeckungen
    $U = \bigcup_i U_i$ gilt $F(U) = \lim_i F(U_i)$.
  \end{quote}
\end{defn}
Wir notieren die Kategorie der $C$-wertigen Prägarben auf $X$ mit $\p
C_{/X}$ und die der $C$-wertigen Garben auf $X$ mit $C_{/X}$.

Für $C$ die Kategorien der Mengen oder der abelschen Gruppen ist obige
Definition äquivalent zur bekannten Definition über die eindeutige
Verklebbarkeit von verträglichen Schnitten.

Auch das Konzept der Garbifizierung können wir auf $C$-wertige
Prägarben verallgemeinern.
\begin{satz}
  Sei $C$ eine vollständige und kovollständige Kategorie. Dann hat der
  Vergissfunktor $C_{/X} \to \p C_{/X}$ einen Linksadjungierten,
  genannt (verallgemeinerte) Garbifizierung.
\end{satz}
\begin{proof}
  Sei $F \in \p C_{/X}$. Wir behaupten, dass die Prägarbe
  \[ F^+(U) :=   \colf\limits_{\mathcal{U}/U} \lim_j F(U_j) \]
  mit den induzierten Restriktionen eine Garbe ist und die
  Adjunktionseigenschaft erfüllt. Dabei steht $\mathcal{U}/U$ für das
  filtrierende System aller gesättigten Überdeckungen $\mathcal{U}$
  von $U$. Die Konstruktion ist funktoriell.

  Für die Garbeneigenschaft konstruieren wir für eine gesättigte
  offene Überdeckung $U = \bigcup U_i$ inverse Morphismen
  \[ \colf\limits_{\mathcal{U}/U} \lim_{V \in \mathcal{U}} F(V)
  \rightleftarrows
  \lim_i \colf\limits_{\mathcal{U}_i/U_i} \lim_{V \in \mathcal{U}_i} F(V). \]
  Diese erhalten wir einerseits aus den natürlichen Projektionen des
  Limes und Inklusionen des Kolimes und andererseits aus der
  Konstruktion verträglicher Familien von Abbildungen, deren
  Verträglichkeit sich jeweils sofort aus der Eindeutigkeit der
  Restriktionen ergibt. Dabei werden einer Überdeckung in
  $\mathcal{U}/U$ die Überdeckungen in $\mathcal{U}_i/U_i$ zugeordnet,
  die sich durch Schneiden mit $U_i$ ergeben, und umgekehrt Familien
  von Überdeckungen der $U_i$ die ihrer Vereinigung zugeordnete
  gesättigte Überdeckung von $U$. Dass beide Morphismen invers
  zueinander sind, ergibt sich erneut aus der Eindeutigkeit der
  Restriktionen.

  Auch für die Adjunktionseigenschaft gehen wir wie bei mengenwertigen
  Garben vor und erhalten zunächst den natürlichen Prägarbenmorphismus
  $F \to F^+$, der zur einelementigen Überdeckung von $U$ gehört. Wir
  möchten nun zeigen, dass jeder Prägarbenmorphismus $F \to G$ in eine
  Garbe $G$ eindeutig über unseren natürlichen Morphismus $F \to F^+$
  faktorisiert. Die Morphismen unseres Prägarbenmorphismus $F(V) \to
  G(V)$ induzieren nun nach der Restriktionsverträglichkeit Morphismen
  auf den Limites, und somit auch im Kolimes Morphismen in
  \[ C(F^+(U), G(U) =
     C(\colf\limits_{\mathcal{U}/U} \lim_{V \in \mathcal{U}} F(V),
     \lim_{V \in \mathcal{U}} G(V)) \iso
     \colf\limits_{\mathcal{U}/U}
     C(\lim_{V \in \mathcal{U}} F(V), \lim_{V \in \mathcal{U}} G(V)), \]
  für alle $U \open X$, die ebenfalls mit den Restriktionen kompatibel
  sind.

  Dass $F \to G$ nun mittels dieses Morphismus $F^+ \to G$ über $F \to
  F^+$ faktorisiert, folgt nun aber direkt, da der Morphismus für
  einelementige Überdeckungen der aus dem Prägarbenmorpismus ist.

  Zur Eindeutigkeit dieser Faktorisierung halten wir fest, dass der
  Morphismus $F^+ \to G$ auf einelementigen Überdeckungen schon
  aufgrund der Faktorisierungseigenschaft und dann auf größeren
  Überdeckung durch die Garbeneigenschaften von $F^+$ und $G$
  eindeutig festgelegt ist.
\end{proof}

Wir prüfen, dass wir so insbesondere eine Garbifizierung zu
``garbenwertigen Garben'' erhalten.
\begin{lemma} \label{ensx-complete}
  Die Kategorien $\EnsX$ und $\AbX$ der (abelschen) Garben auf einem
  topologischen Raum $X$ sind vollständig und kovollständig.
\end{lemma}
\begin{proof}
  Die Kategorien der Mengen $\Ens$ ist vollständig und
  kovollständig. Somit ist auch die Kategorie der Prägarben auf $X$
  $[\OffX\op, \Ens]$ vollständig und kovollständig nach der
  Beschreibung von Limites in Funktorkategorien als objektweise
  Limites. Als Rechtsadjungierter der Garbifizierung vertauscht nun
  der Inklusionsfunktor $\iota: \EnsX \to \pEnsX$ mit Limites,
  d. h. die Limites in $\EnsX$ sind die Limites der zugehörigen
  Prägarben. Die Kolimites sind gegeben durch die Garbifizierungen der
  Prägarben-Kolimites. Es gilt mit dem Prägarbenkolimes $\col_i \iota
  F_i$ über ein System von Garben $F_i \in \EnsX$ für eine Garbe $G
  \in \EnsX$:
  \begin{align*}
    \EnsX((\col_i \iota F_i)^+, G)
    &\iso \pEnsX(\col_i \iota F_i, \iota G) \\
    &\iso \col_i \pEnsX(\iota F_i, \iota G) \\
    &\iso \col_i \EnsX(F_i, G).
  \end{align*}
  Derselbe Beweis gilt für $\AbX$ unter Verwendung der Vollständigkeit
  und Kovollständigkeit der abelschen Gruppen.
\end{proof}
Der Beweis hat nur benutzt, dass $\pEnsX$ vollständig und
kovollständig ist und $\EnsX$ eine volle Unterkategorie mit
Linksadjungiertem zur Inklusion.
\begin{defn} \label{refl-sub}
  Eine volle Unterkategorie einer Kategorie, für die die Inklusion
  einen Linksadjungierten besitzt, heißt \emph{reflektive}
  Unterkategorie. In diesem Fall heißt der Linksadjungierte
  \emph{Reflektor}.
\end{defn}
Wir halten fest:
\begin{prop} \label{refl-sub-complete}
  Sei $C \subset D$ eine reflektive Unterkategorie. Ist $D$
  vollständig, so ist auch $C$ vollständig mit denselben Limites. Ist
  $D$ kovollständig, so ist auch $C$ kovollständig mit den
  Reflektionen der Kolimites in $D$ als Kolimites.
\end{prop}

Bezeichne wieder $\B$ die Kategorie der Basis der Topologie auf einem
Produkt topologischer Räume $X \times Y$ mit Inklusionen als
Morphismen.
\begin{satz} \label{sheaves-prod-topos}
  Seien $X$ und $Y$ topologische Räume. Dann gibt es eine Äquivalenz
  von Kategorien
  \[ (\EnsX)_{/Y} \qiso \Ens_{/\B} \fromqiso \Ens_{/X \times Y} \]
  gegeben durch
  \begin{align*}
    U \times V &\mapsto (F(V))(U) \quad
    \text{für } F \in (\EnsX)_{/Y} \text{ und} \\
     U \times V &\mapsto F(U \times V) \quad
     \text{die Restriktion für } F \in \Ens_{/X \times Y}
  \end{align*}
  für $U \open X$ und $V \open Y$.
\end{satz}
\begin{proof}
  Die zweite Äquivalenz ist \ref{sheaf-on-basis}. Für die erste
  Äquivalenz bemerken wir wie in \ref{simp-comp-sheaf}, dass die
  zugrundeliegenden Prägarbenkategorien übereinstimmen. Nun fordert
  die Garbenbedingung für $\Ens_{/\B}$ die Verklebungseigenschaft für
  beliebige Überdeckungen von Basismengen durch Basismengen, während
  die Garbenbedingungen für $(\EnsX)_{/Y}$ die Verklebungseigenschaft
  für ``Produkt-Überdeckungen'' von Basismengen fordert, d.~h. für
  Überdeckungen der Form $U \times V = \bigcup_{i,j} U_i \times V_j$
  für $U = \bigcup_i U_i$ eine Überdeckung von $U$ und $V = \bigcup_j
  V_j$ eine Überdeckung von $V$. Wir rechnen dies nach:
  \begin{align*}
    (F(V))(U)
    &= (\lim_j F(V_j))(U) \\
    &= \lim_j F(V_j)(U) \\
    &= \lim_j \lim_i F(V_j)(U_i)
  \end{align*}
  wobei im ersten Schritt die Garbenbedingung von $F \in
  (\EnsX)_{/Y}$, und im dritten die von $F(V_j) \in \EnsX$ verwendet
  wurde. Der zweite Schritt ist die Beschreibung von Limites in
  Funktorkategorien als objektweise Limites.

  %% Im zweiten Schritt benutzen wir, dass für $\EnsX$
  %% Schnittfunktoren $\Gamma(U, \cdot) = \Ens_X(\point_U, \cdot)$
  %% darstellbar sind für $\point_U$ die Einschränkung auf $U$ der
  %% konstanten einpunktigen Garbe auf $X$ und somit mit Limites
  %% vertauschen.

  Beide Verklebungsbedingungen sind aber äquivalent, da eine beliebige
  Überdeckung von $U \times V$ natürlich mit den Projektionen auf $X$
  und $Y$ Überdeckungen von $U$ und $V$ induziert und die
  Verträglichkeitsvoraussetzung für die Garbenbedingung von
  $\Ens_{/\B}$ bezüglich der Produktüberdeckung genau dann erfüllt
  ist, wenn sie für die ursprüngliche Überdeckung erfüllt ist.
\end{proof}
\begin{bem}
  Für abelsche Garben erhalten wir die analoge Aussage $(\AbX)_{/Y}
  \qiso \Ab_{/X \times Y}$.
  %% da auch die Schnittfunktoren $\Gamma(U, \cdot) = \AbX(\Z_U, \cdot)$
  %% darstellbar sind
% TODO
\end{bem}

\subsection{Exkurs: Überlagerungen von Produkträumen}

Nachdem der vorangegangene Satz die étalen Räume über einem
Produktraum $X \times Y$ beschreibt, möchten wir nun die
Überlagerungen eines solchen Produktraums beschreiben. Wir können die
Frage in unserer allgemeineren Terminologie formulieren.
\begin{prop}
  Die Äquivalenz von Kategorien
  \[ \EnsX \rightleftarrows \etTop_X \]
  induziert eine Äquivalenz der vollen Unterkategorien
  \[ \EnsX^{\lk} \rightleftarrows \Ub_X, \]
  wobei $\EnsX^{\lk}$ die lokal konstanten Garben auf $X$ bezeichnet.
\end{prop}
\begin{proof}
  Die Äquivalenz induziert zunächst die Äquivalenz der vollen
  Unterkategorien der konstanten Garben und der trivialen
  Überlagerungen und dann bei lokaler Forderung der jeweiligen
  Eigenschaften die Aussage des Satzes.
\end{proof}

Wir erinnern an die überlagerungstheoretische Definition einfachen
Zusammenhangs in der Sprache von Garben.
\begin{defn}
  Ein topologischer Raum $Y$ heißt einfach zusammenhängend, falls jede
  lokal konstante Garbe auf $Y$ konstant ist.
\end{defn}
Einfach zusammenhängende Räume sind somit insbesondere
zusammenhängend, da wir sonst auf den Zusammenhangskomponenten
triviale Überlagerungen wählen können, deren Fasern verschiedene
Kardinalitäten haben.

\begin{satz}
  Seien $X$ und $Y$ topologische Räume, $Y$ einfach
  zusammenhängend. Bezeichne $\pi: X \times Y \to X$ die
  Projektion. Dann induzieren die Funktoren  
  \[ \EnsX^{\lk}
  \mathrel{\mathop{\rightleftarrows}^{\pi^*}_{\pi_*}}
  \Ens_{/X \times Y}^{\lk} \]
  eine Äquivalenz von Kategorien.
\end{satz}
\begin{proof}
  Der Isomorphismus $F \iso \pi_* \pi^* F$ für $F \in \EnsX^{\lk}$ ist
  gerade die Aussage zu finalem Rückzug mit zusammenhängender
  Faser \ref{??}, da $Y$ zusammenhängend ist als einfach
  zusammenhängender Raum.

  Der Isomorphismus $\pi^* \pi_* F \iso F$ für $F \in \Ens_{/X \times
  Y}^{\lk}$ folgt aus der Aussage zu faserkonstanten Garben \ref{??},
  falls wir zeigen können, dass $F$ konstant ist auf den Fasern von
  $\pi$. Tatsächlich ist aber $F|_{\pi^{-1}(x)}$ eine lokal konstante
  Garbe auf $Y$ und mithin konstant wegen des einfachen Zusammenhangs
  von $Y$. 
\end{proof}

\subsection{Anwendung auf relativ schwach konstruierbare Garben} 

Wir bezeichnen als eine relativ zu $X$ schwach $|\K|$-konstruierbare
Garbe eine Garbe $F \in \Ab_{|\K| \times X}$, für die die
Einschränkungen $F|_{|\sigma| \times X}$ Rückzüge von Garben auf $X$
sind, es also ein $G \in \AbX$ gibt mit
\[ F|_{|\sigma| \times X} \iso \pi^* G \]
für $\pi: |\sigma| \times X \to X$ die Projektion.

Wir könnten nun für diesen Begriff dieselben Aussagen wie im
vorangegangen Abschnitt mit vollkommen analogen Argumenten erneut
beweisen. Kategorientheorie und unsere obige Charakterisierung von
Garben auf Produkträumen ermöglichen uns aber ein einfacheres
Vorgehen. Wir bemerken, dass für die Konstruktion unserer
Kategorienbifaserung $\Ab_{\sslash \Top} \to \Top$ mitsamt ihren
bekannten Eigenschaften und die darauf aufbauende Argumentation im
vorangegangenen Abschnitt der Umstand keine Rolle gespielt hat, dass
unsere Garben Werte in den abelschen Gruppen annnehmen. Dieselben
Konstruktionen funktionieren für $\mathcal{A}_{\sslash \Top}$ eine
Garbenkategorie mit Werten in einer beliebigen vollständigen abelschen
Kategorie. Davon überzeugt man sich im Zweifel auch explizit zunächst
durch Übertragung auf Garbenkategorien mit Werten in $R$-Linksmoduln
und dann durch den Einbettungssatz von Mitchell auf beliebige abelsche
Kategorien.

Um den relativen Fall abzuschließen, werden diese Argumentation auf
den Fall $\mathcal{A} = \AbX$ anwenden. Die benötigten
homotopieinjektiven Auflösungen im Beweis erhalten wir dann durch
unsere Äquivalenz $(\AbX)_{/Y} \qiso \Ab_{/X \times Y}$. Wir erhalten:
\begin{theorem} \label{dersk-eq-rel}
  Sei $\K$ ein Simplizialkomplex. Dann gibt es eine Äquivalenz von
  Kategorien
  \[ \Der(\sKons(\K \times X))
     \mathrel{\mathop{\rightleftarrows}^{\iota}_{R \beta}}
     \Der_{\mathrm{sk}}(|\K| \times X),
  \]
  wobei $\iota$ die Inklusion und $\beta = (p \times \id_X)^* (p
  \times \id_X)_*: \Ab_{|\K| \times X} \to \sKons(\K \times X)$ ist.
\end{theorem}
\begin{bem}
  Der tieferstehende Grund für \ref{sheaves-prod-topos} und die sich
  daraus ergebende Möglichkeit, alle über $X = \point$ gezeigten
  Aussagen auch zu relativieren, ergibt sich daraus, dass es sich bei
  $\Ens$ und $\EnsX$ beiden um \emph{elementare Topoi} handelt, also
  Kategorien, die die wichtigsten Eigenschaften der Kategorie der
  Mengen verallgemeinern. Für die Übertragung von Aussagen von $\Ens$
  auf andere Topoi ist entscheidend, dass diese über eine konstruktive
  interne Logik verfügen, die etwa das Auswahlaxiom oder oder den Satz
  vom ausgeschlossenen Dritten aus der klassischen Logik und
  Mengenlehre nicht kennt. Argumente, die diese Eigenschaften der
  klassischen Logik nicht benutzen, übertragen sich sofort auf andere
  Topoi. Siehe \cite{MoerTopoi} für eine Darstellung dieser Ideen.
\end{bem}

\end{document}
