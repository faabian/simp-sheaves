% Emacs mode: -*-latex-*-
% include latex header (\usepackage, \newcommand etc.) 
\documentclass[a4paper]{article}
% \usepackage[left=3cm,right=3cm,top=3cm,bottom=2cm]{geometry} % page settings
\usepackage{amsmath}
\usepackage{amssymb}
\usepackage{amsthm}
\usepackage{etoolbox}
\usepackage[ngerman]{babel}
\usepackage[utf8]{inputenc}
\usepackage{mathtools}
\usepackage{tikz-cd}
\usepackage{enumitem}
\usepackage{hyperref}

\setlength{\parskip}{\medskipamount}
\setlength{\parindent}{0pt}

\theoremstyle{plain}
\newtheorem{theorem}{Theorem}
\newtheorem{lemma}[theorem]{Lemma}
\newtheorem{prop}[theorem]{Proposition}
\newtheorem{kor}[theorem]{Korollar}
\newtheorem{satz}[theorem]{Satz}
%% \providecommand*{\lemmaautorefname}{Lemma}
%% \providecommand*{\propautorefname}{Prop.}
%% \providecommand*{\korautorefname}{Korollar}
%% \providecommand*{\satzautorefname}{Satz}

\theoremstyle{definition}
\newtheorem{defn}[theorem]{Definition}

\theoremstyle{remark}
\newtheorem{bem}[theorem]{Bemerkung}

\DeclareMathOperator{\Cat}{Cat}
\DeclareMathOperator{\poset}{poset}
\DeclareMathOperator{\EnsX}{Ens_{/X}}
\DeclareMathOperator{\pEnsX}{pEns_{/X}}
\DeclareMathOperator{\AbX}{Ab_{/X}}
\DeclareMathOperator{\pAbX}{pAb_{/X}}
\DeclareMathOperator{\OffX}{Off_X}
\DeclareMathOperator{\Ens}{Ens}
\DeclareMathOperator{\Ob}{Ob}
\DeclareMathOperator{\Der}{Der}
\DeclareMathOperator{\Ab}{Ab}
\DeclareMathOperator{\sKons}{s-Kons}
\DeclareMathOperator{\Ket}{Ket}
\DeclareMathOperator{\EnsB}{Ens_{/\B}}
\DeclareMathOperator{\im}{im}
\DeclareMathOperator{\Id}{Id}
\DeclareMathOperator{\id}{id}
\DeclareMathOperator{\colf}{colf}
\DeclareMathOperator{\limf}{limf}
\DeclareMathOperator{\Top}{Top}

\newcommand{\etalespace}[1]{\overline{#1}}
\newcommand{\B}{\mathcal{B}}
\newcommand{\op}{^\mathrm{op}}
\newcommand{\iso}{\xrightarrow{\sim}}
\newcommand{\qiso}{\xrightarrow{\approx}}
\newcommand{\fromqiso}{\xleftarrow{\approx}}
\newcommand{\open}{\subset\kern-0.58em\circ}  % only possible in math mode
\newcommand{\K}{\mathcal{K}}
\newcommand{\Z}{\mathbb{Z}}
\newcommand{\R}{\mathbb{R}}
\newcommand{\DerAbK}{\Der(\Ab_{/|\K|})}
\newcommand{\DerskK}{\Der_{\mathrm{sk}}(|\K|)}
\newcommand{\DerpskK}{\Der^+_{\mathrm{sk}}(|\K|)}
\newcommand{\AbKr}{\Ab_{/|\K|}}
\newcommand{\sKonsK}{\sKons(\K)}
\newcommand{\inj}{\hookrightarrow}
\newcommand{\surj}{\twoheadrightarrow}
\newcommand{\Iff}{\Leftrightarrow}
\newcommand{\Implies}{\Rightarrow}
\newcommand{\cc}{^{\bullet}}  % chain complex
\newcommand{\from}{\leftarrow}


\begin{document}

\title{Simpliziale Garben}
\author{Fabian Glöckle}
\date{\today}
% \maketitle

\section{Realisierung simplizialer Garben}

Die Beschreibung der geometrischen Realisierung als Tensorprodukt von
Funktoren eröffnet uns eine Reihe weiterer geometrischer
Realisierungen, die diejenige simplizialer Mengen verallgemeinern.

Zunächst stellen wir fest, dass wir in unserer Konstruktion
simpliziale Mengen immer als diskrete simpliziale topologische Räume
betrachtet haben, und die Diskretheit genauso gut auch fallen lassen
können. Wir erhalten die geometrische Realisierung $|X| = X \otimes R$
eines simplizialen topologischen Raums $X: \Delta\op \to \Top$.
\begin{bsp}
  Wir betrachten den simplizialen topologischen Raum $X: \Delta\op \to
  \Top$, den wir aus dem (kombinatorischen) Standard-1-Simplex
  $\Delta^1$ erhalten, indem wir disjunkte Vereinigungen von Punkten
  durch disjunkte Vereinigungen von Intervallen $I = [0, 1]$ mit von
  den Identitäten induzierten Abbildungen ersetzen. Offenbar ist die
  geometrische Realiserung das Produnkt $I \times
  |\Delta^1|$. Ersetzen wir $X_0$ wieder durch zwei Punkte ${0, 1}$
  mit beliebigen Degenerationen, so erhalten wir eine zu einer
  Kreisscheibe verdickte Linie zwischen den beiden Punkten als
  Realisierung. Ersetzen wir die höheren $X_n, n \geq 2$ ebenfalls
  wieder durch Punkte mit beliebigen Randabbildungen, so sorgen deren
  Identifikationen dafür, dass die geometrische Realisierung wieder
  $|\Delta^1|$ wird.
  % TODO Interessantere Beispiele? RP^n oder so?
\end{bsp}

Weiter verallgemeinert die Konstruktion auch auf
Diagrammkategorien. Ist $I$ eine kleine Kategorie und $X: \Delta\op
\to \Top^I$ ein simpliziales $I$-System topologischer Räume, so
erhalten wir eine geometrische Realisierung $|X| = X \otimes R$ von
$X$, wenn wir $R: \Delta \to \Top \to \Top^I$ mittels des Funktors der
konstanten Darstellung auf die Diagrammkategorie
fortsetzen. Insbesondere erhalten wir eine geometrische Realisierung
für die Kategorie der Paare topologischer Räume mit stetiger
Abbildung, d. h. die Diagrammkategorie $\Top^{[1]}$ für $[1]$ die
Kategorie des Ordinals $[1] \in \Delta$, die Pfeilkategorie $\{
\bullet \to \bullet \}$. Eine Realisierung für Garben über
topologischen Räumen erhalten wir so aber nicht: Sind in einem
simplizialen $\Top^{[1]}$ alle Abbildungen étale, bedeutet das noch
nicht, dass auch die induzierte Abbildung in der geometrischen
Realisierung étale ist. Sind etwa alle Basisräume $X_n$ einpunktig, so
ist die Realisierung $|X|$ ebenfalls einpunktig. Étale Räume $F_n \to
X_n$ sind diskret, d. h. die geometrische Realisierung $|F|$ ist die
einer simplizialen Menge, also im Allgemeinen ein höherdimensionaler
CW-Komplex. Ein solcher ist nicht diskret, also nicht étale über
$|X|$. Wir erhalten die korrekte Realisierung für $\Ens_{/\Top}$,
indem wir den Kolimes in der Garbenkategorie statt in den
topologischen Räumen bilden.
\begin{prop}
  Sei $F \in [\Delta\op, \Ens_{/\Top}]$ eine simpliziale Garbe über
  topologischen Räumen mit Morphismen und $R: \Delta \to \Ens_{/\Top}$
  eine kosimpliziale Garbe über topologischen Räumen. Dann ist die
  Realisierung $F \otimes R \in \Ens_{/\Top}$ eine Garbe über der
  geometrischen Realisierung der Basisräume.
\end{prop}
\begin{proof} \label{real-enstop}
  Die Existenz der Realisierung folgt direkt aus der Beschreibung von
  Koenden als Kolimites \ref{coend-col} und der Kovollständigkeit von
  $\Ens_{/\Top}$ nach \ref{enstop-complete}. Dort wird auch der
  Basisraum des Kolimites zum Kolimes der Basisräume bestimmt.
\end{proof}
% TODO stimmt nicht
\begin{bem}
  Eine kanonische Wahl für $R$ ist etwa die konstant einelementige
  Garbe $|\Delta^n| \to |\Delta^n|$ oder ihre plumpe Variante
  $\blacktriangle^n \to \blacktriangle^n$.
\end{bem}
\begin{bem} \label{real-ensx}
  Diese geometrische Realisierung simplizialer Garben auf
  topologischen Räumen spezialisiert zu einer geometrischen
  Realisierung simplizialer Garben auf $X$: Ist
  $F: \Delta\op \to \EnsX$ eine simpliziale Garbe auf $X$, so ist ihre
  geometrische Realisierung aus \ref{real-enstop} eine Garbe
  $|F| \in \EnsX$, da die geometrische Realisierung des konstanten
  simplizialen topologischen Raums $X: [n] \mapsto X$ selbst $X$ ist.

  Für Garben $E_n$ auf diskreten Räumen $D_n$ handelt es sich um die
  geometrische Realisierung eines Pfeils simplizialer Mengen. Die
  relative Version über $X$ hiervon ist die folgende: für Garben $E_n$
  auf $X \times D_n$ für diskrete $D_n$ und zu für $f: [m] \to [n]$
  monoton von $D_n \to D_m$ induzierten Basen $D_n \times X \to
  D_m \times X$ der Garbenmorphismen ist die geometrische Realisierung
  eine Garbe über $X \times |D|$, für $|D|$ die Realisierung der
  simplizialen Menge $[n] \mapsto D_n$.
\end{bem}

Ziel unserer Überlegungen wird es sein, die Aussagen zu simplizial
konstanten Garben auf der geometrischen Realisierung eines
Simplizialkomplexes $\K$ als Garben auf dem topologischen Raum $\K$
auf die Situation simplizialer Mengen zu übertragen. Die
angesprochenen Realisierungen in \ref{real-enstop} und \ref{real-ensx}
sind dafür nicht geeignet. Das liegt daran, dass wir, um aus Garben
auf der Realisierung wieder ein Diagramm von Garben zu erhalten,
generisierende Randabbildungen benötigen. Im Fall einer simplizialen
Garbe $\Delta\op \to \Ens_{/\Top}$ sind die Randabbildungen im
Garbensystem dagegen gegenläufig zu den generisierenden Einbettungen
$|d_i|: |\Delta^{n-1}| \inj |\Delta^n|$ der Basisräume.

Wir erklären eine neue Realisierung, die diesem Anspruch gerecht wird.
Sei dazu $R: \Delta \to \Ens_{/\Top}$ eine kosimplizialer
topologischer Raum und $F: \Delta\op \to \Ens_{\sslash\Top}, [n]
\mapsto F_n \in Ens_{/X_n}$ eine simpliziale Garbe über topologischen
Räumen mit Komorphismen. Für $f: [n] \to [m]$ monoton gibt es also
eine stetige Abbildung $Ff: X_m \to X_n$ und einen Morphismus von
Garben über $X_m$: $Ff^* F_n \to F_m$. Wir erhalten einen Funktor $K$
von der Unterteilungskategorie $\Sub(\Delta)$ von $\Delta$ in die
Garben über topologischen Räumen
\begin{equation} \label{dg:cov-real}
  \begin{tikzcd}
    \Sub(\Delta) \arrow{dd}
    & {[n]}^\S \dar[mapsto, shorten <= 3em, yshift=1.5em]
    & \lar f^\S \dar[mapsto, shorten <= 3em, yshift=1.5em] \rar
    & {[m]}^\S \dar[mapsto, shorten <= 3em, yshift=1.5em] \\[25]
    & F_n \times R[n]
    & \lar Ff^* F_n \times R[n] \rar
    & F_m \times R[m] \\[-2em]
    \Ens_{/\Top}
    & \dar[shorten <= -1em]
    & \dar[shorten <= -1em]
    & \dar[shorten <= -1em] \\[-2em]
    & X_n \times R[n]
    & \lar X_m \times R[n] \rar
    & X_m \times R[m],
  \end{tikzcd}
\end{equation}
der für $f: [n] \to [m]$ in $\Delta$ auf Morphismen $f^\S \to [n]^\S$
vom universellen Morphismus $Ff^* F_n \to F_n$ über $Ff$ induziert ist
und auf Morphismen $f^\S \to [m]^\S$ durch die Morphismen $Ff^* F_n
\to F_m$ in $\Ens_{/X_m}$ sowie $Rf$. Letzterer ist tatsächlich ein
Morphismus in $\Ens_{/\Top}$, denn es kommutiert
\[ \begin{tikzcd}
  Ff^* F_n \times R[n] \dar \rar
  & F_m \times R[n] \dar \rar
  & F_m \times R[m] \dar \\
  X_m \times R[n] \rar
  & X_m \times R[n] \rar
  & X_m \times R[m].
\end{tikzcd} \]
Wir erhalten die folgende \emph{kovariante Realisierung}:
\begin{prop} \label{real-enstop-cov}
  Sei $F \in [\Delta\op, \Ens_{\sslash \Top}]$ eine simpliziale Garbe
  über topologischen Räumen mit Komorphismen und $R: \Delta \to \Top$
  ein kosimplizialer topologischer Raum. Dann ist der Kolimes $|F|$
  über den oben definierten zugehörigen Funktor $K: \Sub(\Delta) \to
  \Ens_{/\Top}$ eine Garbe über der geometrischen Realisierung $X
  \otimes R$ der Basisräume.
\end{prop}
\begin{bem} \label{real-ensx-cov}
  Diese geometrische Realisierung simplizialer Garben auf
  topologischen Räumen mit Komorphismen spezialisiert zu einer
  geometrischen Realisierung simplizialer Garben auf $X$: Ist $F:
  \Delta\op \to \Ens_{\sslash X}$ eine simpliziale Garbe auf
  topologischen Räumen mit Komorphismen und konstantem Basisraum $X$
  alias eine kosimpliziale Garbe $F\op: \Delta \to \EnsX$, so
  vereinfacht das Diagramm \ref{dg:cov-real} zu
  \[
  \begin{tikzcd}
    \Sub(\Delta) \arrow{dd}
    & {[n]}^\S \dar[mapsto, shorten <= 3em, yshift=1.5em]
    & \lar f^\S \dar[mapsto, shorten <= 3em, yshift=1.5em] \rar
    & {[m]}^\S \dar[mapsto, shorten <= 3em, yshift=1.5em] \\[25]
    & F_n \times R[n]
    & \lar[equal] Ff^* F_n \times R[n] \rar
    & F_m \times R[m] \\[-2em]
    \Ens_{/\Top}
    & \dar[shorten <= -1em]
    & \dar[shorten <= -1em]
    & \dar[shorten <= -1em] \\[-2em]
    & X \times R[n]
    & \lar[equal] X \times R[n] \rar
    & X \times R[m],
  \end{tikzcd}
  \]
  und ihre geometrische Realisierung aus \ref{real-enstop-cov} ist
  eine Garbe $|F| \in \EnsX$, der Kolimes über den Funktor $\Delta \to
  \Ens_{/\Top}$, der $f: [n] \to [m]$ monoton auf den Morphismus
  \[ \begin{tikzcd}
    F_n \times R[n] \dar \rar{F\op f \times |f|}
    & F_m \times R[m] \dar \\
    X \times R[n] \rar
    & X \times R[m]
  \end{tikzcd} \]
  schickt.

  Sind alle $X_n$ diskret, so bestimmt für $\sigma \in X_n$ der Halm
  $(F_m)_\sigma$ die konstante Garbe $(F_m)_\sigma \times R[n] \to
  {\sigma} \times R[n]$ und wir erhalten für monotones $f: [n] \to
  [m]$ Abbildungen $(F_n)_{f(\sigma)} \to (F_m)_\sigma$, die diese
  konstanten Garben verkleben. Insbesondere sind für Randabbildungen
  $d_i$ die Verklebungen Generisierungen, die angeben, wie ein Element
  des Halms am Rand eines Simplex einen Schnitt über eine Umgebung
  dieses Punkts (auch im Inneren des Simplex) definiert. Wir werden
  diese Beobachtungen in \ref{real-enstop-cov-equiv} präzisieren und
  die zunächst seltsam anmutende Konstruktion als natürlich
  wahrnehmen.

  Für die relative Version betrachten wir Basisräume $X_n = X \times
  D_n$ mit diskreten $D_n$ und von $D_m \to D_n$ induzierten
  Abbildungen. Zu $\sigma \in D_n$ gehört dann eine Garbe $F_\sigma :=
  F_n|_{\sigma \times X} \in \EnsX$ und wir erhalten für monotones $f:
  [n] \to [m]$ Garbenmorphismen $F_{f(\sigma)} \to F_\sigma$, die
  diese Garben verkleben. Wieder sind diese generisierend, erlauben
  also die Ausweitung eines $U$-Schnitts von einem Randpunkt auf einen
  $U$-Schnitt im Inneren.
\end{bem}
\begin{bem}
  Eine weitere Idee zur geometrischen Realisierung simplizialer Garben
  ist die folgende: Ist $F$ eine Prägarbe auf $I$ mit Werten in $C$
  und $G: C \to D$ ein Funktor, so ist $GF: I\op \to D$ eine Prägarbe
  mit Werten in $D$. Nach dem Exponentialgesetz von Funktoren ist eine
  simpliziale Prägarbe über $X$ dasselbe wie eine Prägarbe mit Werten
  in den simplizialen Mengen und wir können die geometrische
  Realisierung $|\cdot|: \s\Ens \to \CGHaus$ nachschalten. Dies
  liefert eine Prägarbe topologischer Räume auf $X$. Allerdings
  schränkt diese Konstruktion nicht auf die vollen Unterkategorien der
  Garben ein: die Garbenbedingung ist als Limes über ein im
  Allgemeinen unendliches System formuliert, die geometrische
  Realisierung vertauscht aber im Allgemeinen nicht mit beliebigen
  Limites.
\end{bem}

\subsection{Die Dualität von Nerv und Realisierung}

Wir suchen Rechtsadjungierte für unsere geometrischen
Realisierungen. Für die Realisierung simplizialer Mengen gelingt uns
das einfach.
\begin{satz}
  Der Funktor der singulären Ketten $S: \Top \to \s\Ens$, $SY
  = \Top(R \, \cdot, Y): [n] \mapsto \Top(|\Delta^n|, Y)$ ist
  rechtsadjungiert zur geometrischen Realisierung
  $|\cdot|: \s\Ens \to \Top$.
\end{satz}
\begin{proof}
  Die Rand- und Degenerationsabbildungen von $SY$ sind für $f: [n] \to
  [m]$ gegeben durch Vorschalten von $|f|: |\Delta^n| \to
  |\Delta^m|$. Wir berechnen
  \begin{align*}
    \Top(|X|, Y)
    & = \Top(\col_{\slicecat{\Delta}{r}{X}} |\Delta^n|, Y) \\
    & \iso \col_{\slicecat{\Delta}{r}{X}} \Top(|\Delta^n|, Y) \\
    & \iso \col_{\slicecat{\Delta}{r}{X}} \s\Ens(\Delta^n, \Top(R \, \cdot, Y)) \\
    & \iso \s\Ens(\col_{\slicecat{\Delta}{r}{X}} \Delta^n, \Top(R \, \cdot, Y)) \\
    & \iso \s\Ens(X, SY)
  \end{align*}  
  mit der Definition der geometrischen Realisierung im ersten Schritt
  (Gl. \ref{eq:real-colim}), der Verträglichkeit von $\Hom:
  C\op \times C \to \Ens$ mit Limites im zweiten und vierten Schritt,
  unserer Bestimmung der $n$-Simplizes als Morphismenmenge
  (Gl. \ref{eq:simp-as-hom}) im dritten Schritt und unserer
  Beschreibung einer simplizialen Menge als Kolimes über ihre
  Simplexkategorie (\ref{sset-colim}) im letzten Schritt.
\end{proof}
Während dieses Argument wieder ein sehr anschauliches ist, möchten wir
wie in \ref{coend-correct-real} erklärt, unser Argument mit den
Begriffen und Techniken von Koenden führen, um es automatisch
verallgemeinern zu können. Wir geben hier noch einmal die direkte
Übersetzung obigen Beweises in die Sprache der Koenden an, und dann
sofort die Verallgemeinerung.
\begin{proof} (\cite{Lore}, 3.2)
  Wir berechnen mit den Regeln des Koenden-Kalküls:
  \begin{align*}
     \Top(|X|, Y)
     &= \Top \left( \int^{[n]} X[n] \times R[n], Y \right) \\
     & \iso[\ref{coend-cocont}]
       \int_{[n]} \Top \big( X[n] \times R[n], Y \big) \\
     & \iso[\ref{copower}]
       \int_{[n]} \Ens \big( X[n], \Top(R[n], Y) \big) \\
     & \iso[\ref{trans-end}]
       [\Delta\op, \Ens] \big( X, \Top(R \, \cdot, Y) \big) \\
     &= \s\Ens(X, SY).
  \end{align*}
\end{proof}

\begin{theorem} [Allgemeine Nerv-Realisierungs-Dualität, \cite{Lore}, 3.2]
  \label{nerve}
  Seien $C$ eine $V$-Kategorie mit Koexponentialen $\odot$ und ein
  Funktor $R: S \to C$ gegeben. Dann gibt es eine Adjunktion
  $(|\cdot|, N)$
  \[ C \xtofrom[N]{|\cdot|} [S\op, V] \]
  mit
  \begin{alignat*}{3}
    &|\cdot|: && X &&\mapsto \int^{s \in S} X(s) \odot R(s) \qquad \text{und} \\
    &N: && Y &&\mapsto C(R \, \cdot, Y).
  \end{alignat*}
\end{theorem}
\begin{proof}
  In wörtlicher Verallgemeinerung des Vorangegangenen:
  \begin{align*}
     C(|X|, Y)
     &= C \left( \int^s X(s) \odot R(s), Y \right) \\
     & \iso[\ref{coend-cocont}]
       \int_s C \big( X(s) \odot R(s), Y \big) \\
     & \iso[\ref{copower}]
       \int_s V \big( X(s), C(R(s), Y) \big) \\
     & \iso[\ref{trans-end}]
       [S\op, V] \big( X, C(R \, \cdot, Y) \big) \\
     &= [S\op, V] (X, NY).
  \end{align*}
\end{proof}

\subsection{Die kartesisch abgeschlossene Struktur der Garben auf $X$}

Für unsere allgemeine Dualität von Nerv und Realisierung \ref{nerve}
benötigen wir also eine bessere $V$-angereichterte Struktur auf
$C$. Wenn wir uns auf $\EnsX$ beschränken, erhalten wir sogar die
Struktur einer kartesisch abgeschlossenen Kategorie.
\begin{defn}
  Eine Kategorie $C$ mit endlichen Produkten heißt \emph{kartesisch
    abgeschlossen}, falls es ein internes Hom für die kartesische
  Schmelzstruktur durch Produkte gibt.
\end{defn}
Das bedeutet konkret, dass es eine Adjunktion $(\times Y, Y \Implies)$
gibt, also eine in allen Variablen natürliche Bijektion
\[ C(X \times Y, Z) \iso C(X, Y \Implies Z). \]

\begin{prop} \label{ensx-cart-closed}
  Die Kategorie $\EnsX$ ist kartesisch abgeschlossen mit Produkt
  \[ (F \times G)(U) = F(U) \times G(U) \]
  und internem Hom
  \[ (F \Implies G)(U) = \Ens_{/U}(F|_U, G|_U) \]
  jeweils mit den von den Restriktionen von $F$ und $G$ induzierten
  Restriktionen. Der étale Raum des Produkts ist gegeben durch das
  Faserprodukt über $X$:
  \[ \etalespace{F \times G}
     \iso \etalespace{F} \times_X \etalespace{G}. \]
\end{prop}
\begin{proof}
  Das Produkt erfüllt offenbar die universelle Eigenschaft in $\pEnsX$
  und ist eine Garbe, da Produkte mit dem Limes der Garbeneigenschaft
  vertauschen (Spezialfall von \ref{ensx-complete}). Das interne Hom
  besteht aus stetigen Abbildungen über $U$ und erfüllt somit die
  Garbenbedingung, die ja sogar nach der Verklebbarkeit stetiger
  Abbildungen modelliert war. Für die Adjunktion müssen wir zeigen
  \[ \EnsX(F \times G, H) \iso \EnsX(F, G \Implies H). \]
  Links stehen restriktionsverträgliche Systeme $F(U) \times G(U) \to
  H(U)$ alias $F(U) \to \Ens(G(U), H(U))$, rechts
  restriktionsverträgliche Systeme $F(U) \to \Ens_{/U}(G|_U,
  H|_U)$. Wir erhalten eine Abbildung von rechts nach links durch den
  globalen Teil $G(U) \to H(U)$ des Garbenmorphismus $G|_U \to H|_U$
  und das Exponentialgesetz in $\Ens$ und von links nach rechts durch
  Ergänzen des globalen Teils $G(U) \to H(U)$ durch verträgliche $G(V)
  \to H(V)$ als die Bilder unter $F(U) \to F(V) \to \Ens(G(V),
  H(V))$. Diese Abbildungen sind zueinander invers.

  Für den étalen Raum des Produkts erhalten wir nach der universellen
  Eigenschaft des Faserprodukts eine stetige Abbildung über $X$
  \[ \etalespace{F \times G}
  \to \etalespace{F} \times_X \etalespace{G}.
  \]
  Diese induziert auf den Halmen die Bijektionen
  \[ (F \times G)_x \iso F_x \times G_x \]
  aus dem Vertauschen endlicher Limites mit filtrierenden Kolimites.
\end{proof}
Diese Struktur einer kartesisch abgeschlossenen Kategorie macht
$\EnsX$ insbesondere zu einer über sich selbst tensorierten Kategorie
im Sinne von \ref{copower}. Wir erhalten einen Nerv-Funktor für die
geometrische Realisierung simplizialer Garben auf $X$ aus \ref{nerve}.

Den obigen konkreten Beweis für das interne Hom der Prägarbenkategorie
$\pEnsX = [\OffX\op, \Ens]$ können wir mit einer Rechnung im
Koendenkalkül auf beliebige Prägarbenkategorien ausweiten.
\begin{prop}
  Ist $C$ eine kleine Kategorie, so ist die Prägarbenkategorie
  $\Ens^{C\op}$ kartesisch abgeschlossen.
\end{prop}
\begin{proof}
  Nach der objektweisen Berechnung von Limites in Funktorkategorien
  ist das Prägarbenprodukt gegeben durch $(F \times G)(c) = F(c)
  \times G(c)$ für $F, G \in \Ens^{C\op}$ und $c \in C$. Wir
  behaupten, dass das interne Hom in der Prägarbenkategorie die
  Prägarbe
  \[ (F \Implies G)(c) := \Ens^{C\op}(F \times C(\cdot, c), G), \]
  ist, die auf Morphismen $f: c \to d$ durch Vorschalten von
  Transformationen $\id_F \times (\circ f)$ gegeben ist, mit $(\circ
  f): C(\cdot, c) \to C(\cdot, d)$ dem Nachschalten von $f$. Mit
  \ref{trans-end} sind Morphismen in $\Ens^{C\op}$ darstellbar als
  Ende
  \[ \Ens^{C\op}(F, G) = \int_c \Ens(F(c), G(c)) \]
  und wir berechnen mit den Regeln des (Ko-) Endenkalküls für $F, G, H
  \in \Ens^{C\op}$:
  \begin{align*}
    \Ens^{C\op}(F, G \Implies H)
    &\iso[Def.] \Ens^{C\op}(F, \Ens^{C\op}(F \times C(\cdot, \bullet), G)) \\
    &\iso[\ref{trans-end}]
     \int_c \Ens(F(c), \int_d \Ens(G(d) \times C(d, c), H(d))) \\
    &\iso[\ref{coend-cocont}]
     \int_c \int_d \Ens(F(c), \Ens(G(d) \times C(d, c), H(d))) \\
    &\iso[\ref{coend-fubini}]
     \int_d \int_c \Ens(F(c), \Ens(G(d) \times  C(d, c), H(d))) \\
    &\iso[Adj.] \int_d \int_c \Ens(F(c) \times G(d) \times C(d, c), H(d)) \\
    &\iso[\ref{coend-cocont}]
     \int_d \Ens( \int^c F(c) \times G(d) \times C(d, c), H(d)) \\
    &\iso[\ref{coend-density}]
     \int_d \Ens( F(d) \times G(d), H(d)) \\
    &\iso[\ref{trans-end}]
     \Ens^{C\op}(F \times G, H).
  \end{align*}
  % TODO left und right setzen für bessere Klammern
\end{proof}
Die obere Aussage über Prägarben auf topologischen Räumen ergibt sich
daraus durch die Beobachtung, dass $F|_U = F \times \OffX(\cdot, U)$
ist, denn $\OffX$ ist halbgeordnet durch Inklusionen. Wir erhalten
auch die kartesisch abgeschlossene Struktur simplizialer Mengen, der
Prägarbenkategorie auf $\Delta$. Explizit ist für $X, Y \in \s\Ens$:
\[ (X \times Y)_n = X_n \times Y_n \]
und
\[ (X \Implies Y)_n = \s\Ens(X \times \Delta^n, Y). \]

Auch die Rolle von $\Ens$ kann verallgemeinert werden. Wir erhalten:
\begin{prop} \label{presheaf-cart-closed}
  Sei $E$ eine kartesisch abgeschlossene Kategorie und $C$ eine kleine
  Kategorie. Dann ist die Kategorie der Prägarben $E^{C\op}$
  angereichert über $E$ und kartesisch abgeschlossen.
\end{prop}
\begin{proof}
  Sind $F, G \in E^{C\op}$ Prägarben, so erhalten wir die
  angereicherte Struktur durch Übertragung der obigen Formulierung als
  $\Ens$-Ende:
  \[ E^{C\op}(F, G) := \int_c E(F(c), G(c)) \in E, \]
  für $E(\cdot, \cdot)$ das interne Hom in $E$. Damit funktioniert der
  Beweis oben auch für diesen Fall.
\end{proof}

\subsection{Kategorien von Garben über topologischen Räumen}

Wir betrachten die Kategorienfaserungen $\Ens_{/\Top} \to \Top$ mit
Morphismen den stetigen Abbildungen zwischen den étalen Räumen über
der stetigen Abbildung in der Basis sowie $\Ens{\sslash \Top} \to
\Top$ mit Opkomorphismen als Morphismen, d. h. für $F \in \EnsX$ und
$G \in \Ens_{/Y}$:
\[ \Ens_{\sslash \Top}(F, G) = \coprod_{f: X \to Y} \EnsX(f^* G, F). \]

Wir möchten einen Nerv-Funktor nicht nur für die Realisierung
simplizialer Garben über $X$ finden, sondern auch für simpliziale
Garben über variablen topologischen Räumen, also für simpliziale
Objekte in $\Ens_{/\Top}$ und $\Ens_{\sslash \Top}$. Dafür benötigen
wir wieder eine monoidal abgeschlossene Struktur auf diesen
Kategorien.

Die Kategorie $\Ens_{/\Top}$ besitzt endliche Produkte, die
algebraisch gegeben sind durch Rückzug und Produkt und topologisch
durch Bilden der Produkträume. Konkret:
\begin{prop} \label{enstop-prod}
  Seien $F_{1,2} \in \Ens_{/X_{1,2}}$ Garben über topologischen Räumen
  $X_1$ und $X_2$. Dann ist die Garbe
  \[ F_1 \times F_2 := \pr_1^* F_1 \times \pr_2^* F_2 \in \Ens_{/X_1 \times X_2} \]
  mit $\pr_{1,2}: X_1 \times X_2 \to X_{1,2}$ den Projektionen das
  Produkt von $F_1$ und $F_2$ in $\Ens_{/\Top}$. Ihr étaler Raum ist:
  \[ \etalespace{F_1 \times F_2} = \etalespace{F_1} \times \etalespace{F_2}
  \to X_1 \times X_2
  \]
  mit der von $\etalespace{F_{1,2}} \to X_{1,2}$ induzierten
  Produktabbildung.
\end{prop}
\begin{proof}
  Für ein Testobjekt $G \in \Ens_{/Y}$ prüft man leicht die Bijektion
  von Faserprodukten
  \begin{align*}
    \Top(\etalespace{G}, \etalespace{F_1} \times \etalespace{F_2})
    &\times_{\Top(\etalespace{G}, X_1 \times X_2)} \Top(Y, X_1 \times X_2) \\
    \unalign{\iso} \Top(\etalespace{G}, \etalespace{F_1})
    &\times_{\Top(\etalespace{G}, X_1)} \Top(Y, X_1) \\
    \times \Top(\etalespace{G}, \etalespace{F_2})
    &\times_{\Top(\etalespace{G}, X_2)} \Top(Y, X_2),
  \end{align*}
  dies zeigt die Aussage über den étalen Raum des Produkts. Die
  Abbildung $\etalespace{F_1} \times \etalespace{F_2} \to X_1 \times
  X_2$ ist étale und konkret ein Homöomorphismus auf der Produktmenge
  der Umgebungen, auf denen $\etalespace{F_{1,2}} \to X_{1,2}$
  Homöomorphismen sind.
  
  Für die algebraische Beschreibung erhalten wir mit der Offenheit der
  Projektionen $\pr_{1,2}$ und \ref{ensx-cart-closed} für die Schnitte
  über Basismengen $U_1 \times U_2$:
  \begin{align*}
    (F_1 \times F_2)(U_1 \times U_2)
    &\iso (\pr_1^* F_1)(U_1 \times U_2) \times (\pr_2^* F_2)(U_1 \times U_2) \\
    &\iso F_1(U_1) \times F_2(U_2).
  \end{align*}
  Wir erhalten also einen Garbenmorphismus über $X_1 \times X_2$ von
  der algeraischen zur topologischen Beschreibung, indem einem Paar
  $(s, t) \in F_1(U_1) \times F_2 \times U_2$ der Schnitt $s \times t:
  U_1 \times U_2 \to \etalespace{F_1} \times \etalespace{F_2}$
  zugeordnet wird. Dieser Morphismus induziert auf den Halmen die
  Bijektion $(F_1 \times F_2)_{x, y} \iso (F_1)_x \times (F_2)_y$ aus
  dem Vertauschen von endlichen Produkten mit filtrierenden Kolimites.
\end{proof}
\begin{bem}
  Auf ähnliche Weise kann man auch für $\Ens_{\sslash \Top}$ endliche
  Produkte konstruieren: es handelt sich (wegen der opponierten
  Fasern) um das \emph{Ko}produkt der mit den Projektionen auf den
  Produktraum zurückgezogenen Garben.
\end{bem}
Auch dieses Verfahren können wir für beliebige Limites und Kolimites
durchführen und so \ref{ensx-complete} übertragen:
\begin{satz} \label{enstop-fin-complete}
  Die Kategorie der Garben auf topologischen Räumen mit Morphismen
  $\Ens_{/\Top}$ besitzt endliche Limites.
\end{satz}
\begin{proof}
  Es reicht mit \ref{enstop-prod} die Existenz von Egalisatoren zu
  zeigen. Seien dazu $(F \to X) \rightrightarrows (G \to Y)$ zwei
  Morphismen. Wir zeigen, dass der Egalisator $E \to W$ aus
  $\Top^{[1]}$ eine Garbe ist. Sei $f$ die (übereinstimmende)
  Verknüpfung $W \to X \rightrightarrows Y$. Nun ist $E$ insbesondere
  ein Egalisator von Garben über $W \subset X$: $E \to F|_W
  \rightrightarrows f^* G$ und somit eine Garbe.
\end{proof}
\begin{bem}
  Die verbleibenden Fragen zur Vollständigkeit und Kovollständigkeit
  von $\EnsTop$ werden in \ref{enstop-not-complete} und
  \ref{enstop-no-coequalizers} negativ beantwortet.
\end{bem}

\subsection{Kartesisch abgeschlossene koreflektive Kategorien topologischer Räume}

Wir könnten erwarten, dass wie das Produkt auch das interne Hom von
$\EnsX$ in unsere relative Situation übertragen werden kann. Dies
gelingt tatsächlich aber im Allgemeinen nicht, denn in diesem Fall
erhielten wir durch Einschränken auf die Basis ein zum kartesischen
Produkt adjungiertes internes Hom in der Kategorie der topologischen
Räume (\ref{full-enstop-not-cart-closed}), was bekanntermaßen in
dieser Allgemeinheit nicht möglich ist (\cite{Borceux???}). Wir müssen
uns also wieder auf eine bequeme Kategorie topologischer Räume mit
internem Hom einschränken.

Die häufige Wahl $\CGHaus$ ist für uns ungeeignet, denn der étale Raum
einer Garbe über einem kompakt erzeugten Hausdorffraum ist im
Allgemeinen kein Hausdorffraum mehr (betrachte etwa die Garbe der
stetigen Funktionen nach $\R$). Abhilfe schafft uns eine Konstruktion
aus \cite{Vogt}, die die den kompakt erzeugten Räumen
zugrundeliegenden Gedanken verallgemeinert. Wir geben hier nur die
Ergebnisse an.

Äquivalent zu unserer (der \emph{point-set-}Topologie entspringenden)
Definition kompakt erzeugter Räume ist die folgende Charakterisierung:
\begin{lemma}[\ref{def:cg}, Variante]
  Ein topologischer Raum $X$ ist kompakt erzeugt genau dann, wenn
  gilt: Eine Teilmenge $U \subset X$ ist offen genau dann, wenn ihr
  Urbild unter allen stetigen Abbildungen $K \to X$, $K$ kompakt,
  offen ist.
\end{lemma}
\begin{proof}
  Unsere Bedingung besagt, dass $X$ die Finaltopologie bezüglich des
  Systems der $K \to X$, $K$ kompakt tragen soll. Die Bedingung aus
  der ursprünglichen Definition ist dieselbe für das System der
  Inklusionen kompakter Mengen $K \subset X$. Da jede stetige Abbilung
  $K \to X$, $K$ kompakt, über die Inklusion ihres kompakten Bilds
  faktorisiert, ist letzteres System in ersterem konfinal und die
  Finaltopologien stimmen überein.
\end{proof}
Der in \ref{real-prodcuts} angesprochene zur Inklusion
Rechtsadjungierte $k: \Top \to \CG$ lässt sich nun auch beschreiben
als das Versehen der $X$ zugrundeliegenden Menge mit der genannten
Finaltopologie. Der Raum $kX$ ist dann sogar ein Kolimes über das
System der $K \to X$, $K$ kompakt, mitsamt den Morphismen über $X$
(\cite{Vogt}, 1.1).

Nun verallgemeinern wir (\cite{Vogt}, 1): Sei $\mathcal{I}$ eine
nichtleere volle Unterkategorie von $\Top$ (für $\CG$ die kompakten
Räume). Betrachte die Kategorie $\slicecat{\mathcal{I}}{}{X}$ und $kX
:= \col_{\slicecat{mathcal{I}}{}{X}} X$. Bezeichne die volle
Unterkategorie der topologischen Räume $X$ mit $kX \cong X$ mit
$\mathcal{K}$. Dann ist $k: \Top \to \mathcal{K}$ ein Funktor und
rechtsadjungiert zur Inklusion $\mathcal{K} \to \Top$. Es gilt
$\mathcal{I} \subset \mathcal{K}$.
\begin{bem}
  Dual zu \ref{refl-sub} heißt eine volle Unterkategorie mit zur
  Inklusion Rechtsadjungiertem \emph{koreflektiv}, der
  Rechtsadjungierte heißt \emph{Koreflektor}. Die Konstruktion, die
  zur vollen Unterkategorie $\mathcal{I} \subset \Top$ eine
  koreflektive Unterkategorie $\mathcal{K} \subset \Top$ liefert,
  welche $\mathcal{I}$ umfasst, heißt auch Übergang zur
  \emph{koreflektiven Hülle}. Es handelt sich tatsächlich um eine
  idempotente Operation (\cite{Vogt}, Prop. 1.5).
\end{bem}
Die koreflektive Hülle besitzt die folgenden Stabilitätseigenschaften:
\begin{prop} \label{k-complete}
  Die koreflektive Hülle $\mathcal{K}$ ist vollständig und
  kovollständig. Die Kolimites stimmen mit den Kolimites aus $\Top$
  überein, die Limites entstehen durch Anwendung des Koreflektors $k$
  auf den Limes in $\Top$.
\end{prop}
Insbesondere ist $\mathcal{K}$ also stabil unter disjunkten Summen und
Quotientenbildung.
\begin{proof}
  Das ist die duale Aussage zu \ref{refl-sub-complete}. Die
  Vollständigkeit und Kovollständigkeit von $\Top$ durch Versehen der
  mengentheoretischen Limites bzw. Kolimites mit der Initial-
  bzw. Finaltopologie ist bekannt.
\end{proof}

Im allgemeinen kann man keine Aussage darüber treffen, ob mit der
Relativtopologie versehene Unterräume von Objekten in $\mathcal{K}$
wieder zu $\mathcal{K}$ gehören. Wir benötigen die folgende
Eigenschaft:
\begin{enumerate}
\item \label{itm:k-axiom-subspace} Ist $U \open X$ ein offener
  Unterraum eines Objekts $X \in \mathcal{I}$ versehen mit der
  Relativtopologie, so gilt $U \in \mathcal{K}$.
\end{enumerate}
In diesem Fall gilt bereits für Objekte $X \in \mathcal{K}$, dass
offene Unterräume $U \open X$ wieder Objekte von $\mathcal{K}$
sind. Dieselbe Aussage gilt, wenn man ``offen'' zweimal durch
``abgeschlossen'' ersetzt (\cite{Vogt}, Prop. 2.4).

Wir nehmen nun an, dass $\mathcal{I}$ die folgenden Axiome erfüllt
(\cite{Vogt}, Axiom 2):
\begin{enumerate}
  \setcounter{enumi}{1}
  \item \label{itm:k-axiom-prod} $\mathcal{I}$ ist abgeschlossen unter
    endlichen kartesischen Produkten (Produkten in $\Top$).
  \item \label{itm:k-axiom-ev} Sind $X, Y \in \mathcal{I}$, so ist die
    Auswertungsabbildung
    \begin{align*}
      \ev_{X, Y}: \Top_{co}(X, Y) \times X &\to Y,
      (f, x) &\mapsto f(x)
    \end{align*}
    stetig. Dabei ist $\Top_{co}(X, Y)$ die Morphismenmenge $\Top(X,
    Y)$ versehen mit der kompakt-offen Topologie.
\end{enumerate}
Dann besitzt $\mathcal{K}$ die Struktur einer kartesisch
abgeschlossenen Kategorie mit Produkten
\[ X \otimes Y \: := \: k(X \times Y) \]
den ``k-ifizierungen'' der Produkte in $\Top$ und internem Hom
\[ X \Implies Y \: := \: k(\Top_{co}(X, Y)) \] 
(\cite{Vogt}, 3).

\begin{defn}
  Ein topologischer Raum heißt \emph{lokalkompakt} (im starken Sinne),
  wenn jeder Punkt eine Umgebungsbasis aus kompakten Mengen besitzt.
\end{defn}
\begin{bem}
  Dies ist eine stärkere Bedingung als \emph{lokal kompakt} (im
  schwachen Sinne) wie in \ref{cg-crit} zu sein. Jene stimmt überein
  mit unserer Konvention für ``lokal Eigenschaft'' und wird daher
  getrennt geschrieben. Für Hausdorffräume stimmen beide Begriffe
  überein.
\end{bem}
\begin{prop}[\cite{Vogt}, 5]
  Die folgenden vollen Unterkategorien der Kategorie der topologischen
  Räume erfüllen die Axiome \ref{itm:k-axiom-subspace} -
  \ref{itm:k-axiom-ev}.
  \begin{enumerate}[label=(\roman*)]
  \item die Kategorie der kompakten Hausdorffräume $\mathcal{I}_K$,
  \item die Kategorie der lokalkompakten topologischen Räume
    $\mathcal{I}_L$.
  \end{enumerate}
\end{prop}
Für das Axiom \ref{itm:k-axiom-subspace} weisen wir das nach. Da es
sich um eine lokale Eigenschaft handelt, gilt die Aussage im Fall der
lokalkompakten Räume sofort. Für die kompakten Hausdorffräume bemerkt
man, dass nach dem folgenden Lemma eine offene Teilmenge eines
kompakten Hausdorffraums lokalkompakt ist und lokalkompakte
Hausdorffräume mit den kompakt erzeugten Hausdorffräumen allgemein
(vgl. \ref{cg-crit} \ref{itm:cg-crit-lc}) in der koreflektiven Hülle
der kompakten Hausdorffräume enthalten sind: in der Tat ist für diese
das System der Inklusionen kompakter Teilmengen konfinal im System der
von kompakten Hausdorffräumen ausgehenden stetigen Abbildungen, da das
Bild von Kompakta unter stetigen Abbildungen kompakt ist. Die
Bedingung, kompakt erzeugt zu sein, bedeutet aber gerade, die
Finaltopologie bezüglich dieser Inklusionen zu tragen.
\begin{lemma}
  Sei $K$ ein kompakter Hausdorffraum und $U \open K$ eine offene
  Teilmenge. Dann ist $U$ mit der induzierten Topologie lokalkompakt.
\end{lemma}
\begin{proof}
  Sei $V \open U$ eine offene Umgebung eines Punktes $x \in U$. Der
  Rand $\del V$ ist als abgeschlossene Teilmenge eines kompakten
  Hausdorffraums kompakt und kann somit durch endlich viele offene
  Mengen überdeckt werden, die disjunkt zu einer offenen Umgebung
  $W_0$ von $x$ sind. Bezeichne die Vereinigung dieser Mengen mit
  $W$. Wegen $W \supset \del V$ ist $V \setminus W = \overline{V}
  \setminus W$ abgeschlossen und somit eine kompakte Umgebung von $x$,
  die die offene Umgebung $W_0$ von $x$ enthält.
\end{proof}
Auch Axiom \ref{itm:k-axiom-prod} sieht man direkt: ein Produkt von
Hausdorffräumen ist bekanntermaßen wieder Hausdorffsch und ein Produkt
kompakter Räume wieder kompakt. Mit dieser Aussage finden wir auch bei
einem Produkt lokalkompakter Räume Umgebungsbasen aus Kompakta durch
die Umgebungsbasen aus Produktmengen.

Für das Axiom \ref{itm:k-axiom-ev} verweisen wir auf die Literatur,
siehe etwa \cite{??}.
% TODO finden

\begin{kor}
  Die koreflektiven Hüllen von $\mathcal{I}_K$ und $\mathcal{I}_L$
  sind kartesisch abgeschlossen und enthalten mit jedem Objekt $X$
  auch alle offenen und alle abgeschlossenen Unterräume $Y \subset X$.
\end{kor}

Damit können wir die für uns entscheidende Eigenschaft zeigen:
\begin{prop} \label{k-etale-closed}
  Ist $X \in \mathcal{K}$ für $\mathcal{K}$ die koreflektive Hülle von
  $\mathcal{I}_K$ bzw. $\mathcal{I}_L$ und $F \to X$ eine étale
  Abbildung, so ist auch $F \in \mathcal{K}$.
\end{prop}
\begin{proof}
  Wir können den étalen Raum $F \to X$ als Kolimes mittels der
  Schnitte $F(U)$ über offene Mengen $U \open X$ darstellen:
  \[ F \iso \coprod_{U \open X} F(U) \times U / \sim. \]
  Dabei läuft das Koprodukt über alle offenen Teilmengen von $X$ und
  ist die Äquivalenzrelation die Identifikation gleicher Keime, d. h.
  \[ (s, p) \sim (t, q) \Iff p = q \text{ und } s_p = t_p. \]
  Die étale Abbildung $F \to X$ ist dann von der Projektion auf die
  zweiten Faktoren induziert und wohldefiniert. Man erkennt leicht den
  Isomorphismus als die Koeinheit der Adjunktion $(\et, S)$ aus
  \cite{TG}, 2.1.24, eingeschränkt auf die Kategorie der étalen Räume
  über $X$.

  Nach den Stabilitätseigenschaften von $\mathcal{K}$ sind die offenen
  Teilmengen $U \subset X$ Objekte von $\mathcal{K}$ und dann auch der
  Kolimes $F$ bestehend aus Koprodukt und Koegalisator. Man beachte,
  dass es sich bei $F(U) \times U$ mit der diskreten Topologie auf
  $F(U)$ formal um das Koprodukt $\coprod_{F(U)} U$ handelt.
\end{proof}
\begin{bem}
  Der Beweis wiederholt bei genauerer Betrachtung die Aussage, dass
  jede Prägarbe auf $X$ ein Kolimes über darstellbare Prägarben
  $\OffX(\cdot, U)$ ist (\ref{presheaf-colimit-representable}).
\end{bem}

Wir können uns nun der Frage nach einer kartesisch abgeschlossenen
Struktur auf $\EnsTop$ zuwenden. Ganz allgemein gilt:
\begin{prop}
  Ist $C \subset D$ eine koreflektive Unterkategorie einer kartesisch
  abgeschlossenen Kategorie $C$ und stimmen die Produkte in $C$ und
  $D$ überein, dann ist $D$ kartesisch abgeschlossen.
\end{prop}
\begin{proof}
  In diesem Fall ist die Koreflektion $(X \Implies Y)^+$ des internen
  Hom in $D$ ein internes Hom in $C$.
\end{proof}

Damit können wir folgern:
\begin{prop} \label{full-enstop-not-cart-closed}
  Die Kategorie der Garben über topologischen Räumen mit Morphismen
  $\EnsTop$ ist nicht kartesisch abgeschlossen.
\end{prop}
\begin{proof}
  Betrachte die volle Unterkategorie $\Top \subset \EnsTop$ gegeben
  durch die Einbettung $X \mapsto (\varnothing \to X)$. Diese ist
  koreflektiv mit dem Koreflektor $(G \to Y) \mapsto (\varnothing \to
  Y)$. In der Tat gilt:
  \[ \EnsTop((\varnothing \to X), (G \to Y)) \iso \Top(X, Y)
  \iso \EnsTop((\varnothing \to X), (\varnothing \to Y)).
  \]
  Die Produkte in $\EnsTop$ und $\Top \subset \EnsTop$ stimmen nach
  \ref{enstop-prod} überein. Somit müsste laut
  \ref{coreflective-cart-closed} $\Top$ kartesisch abgeschlossen sein,
  ein Widerspruch zu der bekannten Aussage, dass dies für $\Top$ nicht
  möglich ist (\cite{??}).
\end{proof}

Da das einzige Problem die fehlende kartesisch abgeschlossene Struktur
in der Basis war, schränken wir uns auf eine bequemere Kategorie $\K$
ein. Zunächst betrachten wir den Fall von Paaren von $\K$-Räumen mit
stetiger, aber nicht notwendigerweise étaler, Abbildung.
\begin{lemma} \label{k-arrow-cart-closed}
  Sei $\K \subset \Top$ eine koreflektive, kartesisch abgeschlossene
  Kategorie topologischer Räume. Dann ist $\K^{[1]}$ kartesisch
  abgeschlossen.
\end{lemma}
\begin{proof}
  Es handelt sich um eine Prägarbenkategorie auf $[1]\op$ mit Werten in
  einer kartesisch abgeschlossenen Kategorie. Die Aussage folgt somit
  aus \ref{presheaf-cart-closed}. Expliziter ist das interne Hom von
  $F \to X$ mit $G \to Y$ das Paar
  \[ \K(F, G) \times_{\K(F, Y)} \K(X, Y) \to \K(X, Y),
  \]
  das die Menge der kommutativen Quadrate mit einer Topologie
  ausstattet, mit der Projektion auf den zweiten Faktor als
  Abbildung. Die Adjunktion $\K^{[1]}((F \to X) \times (G \to Y), (H
  \to Z)) \iso \K^{[1]}((F \to X), (G \to Y) \Implies (H \to Z))$ für
  $(F \to X), (G \to Y), (H \to Z) \in \K^{[1]}$ ist dann die
  Bijektion von Faserprodukten
  \begin{align*}
    &\K(F \times G, H) \times_{\K(F \times G, Z)} \K(X \times Y, Z) \\
    \iso \quad &\K(F, \K(G, H) \times_{\K(G, Z)} \K(Y, Z))
    \times_{\K(F, \K(Y, Z))} \K(X, \K(Y, Z)).
  \end{align*}
  % TODO left/right setzen
\end{proof}
Im allgemeinen ist das interne Hom in $\K^{[1]}$ für $(F \to X)$ und $G
\to Y)$ mit étalen Abbildungen nicht wieder étale. Wir können
versuchen, es zu ``étalisieren'':
\begin{lemma}[\cite{TG}, 2.1.40]
  Für $X$ einen topologischen Raum ist die volle Unterkategorie
  $\etTop_X \inj \Top_X$ koreflektiv. Der zur Inklusion
  Rechtsadjungierte heißt \emph{Étalisierung}.
\end{lemma}
\begin{proof} (\cite{TG}, 2.1.40)
  Wir erhalten die Étalisierung als die Verknüpfung $\et \circ S$ für
  $S$ den Funktor der Schnittgarbe und $\et$ den Funktor des étalen
  Raums einer Garbe. Es handelt sich um die Verknüpfung von
  Adjunktionen
  \[(\et, S) \circ (S, \et) = (\et \circ S, \et \circ S):
  \Top_X \rightleftarrows \EnsX \rightleftarrows \etTop_X,
  \]
  wobei letztere Adjunktion die bekannte Äquivalenz von Kategorien
  ist.
\end{proof}
Ist $\K \subset \Top$ nun eine Kategorie topologischer Räume, die mit
jedem Raum $X$ auch jeden étalen Raum über $X$ enthält (etwa wie in
\ref{k-etale-closed}), so können wir die Kategorie der étalen Räume
über $\K$ als volle Unterkategorie von $\K^{[1]}$ auffassen:
\[ \et \K^{[1]} := \Ens_{/\K} \subset \K^{[1]}. \]
Gäbe es nun einen Koreflektor $^+: \K^{[1]} \to \et \K^{[1]}$, so hätten wir
für $F, G, H \in \et \K^{[1]}$:
\begin{align*}
  \et \K^{[1]}(F \times G, H)
  &\iso \K^{[1]}(F \times G, H) \\
  &\iso \K^{[1]}(F, G \Implies H) \\
  &\iso \et \K^{[1]}(F, (G \Implies H)^+),
\end{align*}
mit der kartesisch abgeschlossenen Struktur aus
\ref{k-arrow-cart-closed} im zweiten Schritt. Die Produkte in $\K^{[1]}$
und $\et \K^{[1]}$ stimmen dabei nach dem folgenden Lemma
überein. Umgekehrt ist, falls ein internes Hom $(G \Implies H)^+$ in
$\et \K^{[1]}$ existiert, die Zuordnung $(G \Implies H) \mapsto (G
\Implies H)^+$ ein partiell definierter Koreflektor. Unsere Aufgabe
ist es nun zu zeigen, dass eine solche relative Form der Étalisierung
nicht möglich ist.

\begin{lemma}
  Ist $k: \Top \to \K$ ein Koreflektor und $p: F \to X$ étale in
  $\Top$, so ist $kp: kF \to kX$ étale.
\end{lemma}
\begin{bem}
  Die Bedingung, dass $\K$ koreflektiv sei, ist hier nur eine sehr
  schwache Einschränkung, denn sie ist äquivalent dazu, dass $\K$
  unter Kolimites in $\Top$ abgeschlossen ist (\cite{Herrlich},
  Thm. 37.3).
  % TODO benötigen wir col-Abschluss eh?
\end{bem}
\begin{proof}
  Dass $p$ étale ist, bedeutet, dass es für jedes $x \in F$ ein
  kommutatives Quadrat
  \[ 
  \begin{tikzcd}
    U \dar{\sim} \rar[hook] & F \dar{p} \\
    p(U) \rar[hook] & X
  \end{tikzcd}
  \]
  gibt mit $U \open F$ und $p(U) \open X$. Anwendung von $k$ ergibt
  das entsprechende Diagramm, in welchem $kU$ und $k(p(U))$ offen
  sind, da der Koreflektor die Topologien höchstens verfeinert und die
  Abbildungen injektiv bleiben, da $k$ die zugrundeliegenden
  mengentheoretischen Abbildungen erhält. Da $k$ ein Funktor ist,
  kommutiert das Quadrat und $kU \iso k(p(U)) = (kp)(KU)$ ist ein
  Homöomorphismus.
\end{proof}
Das Übereinstimmen der Produkte in $\K^{[1]}$ und $\et \K^{[1]}$ ergibt sich
nun daraus, dass für $(F \to X), (G \to Y) \in \et \K^{[1]}$ zunächst das
Produkt $(F \times G \to X \times Y)$ in $\Top$ étale ist, und dann
nach dem Lemma auch das Produkt $k(F \times G) \to k(X \times Y)$ in
$\K^{[1]}$ étale und somit in $\et \K^{[1]}$ ist.

\begin{satz} \label{enstop-missing-coreflector}
  Sei $\K \subset \Top$ eine volle Unterkategorie. Der partiell
  definierte Koreflektor $^+: \K^{[1]} \to \et \K^{[1]}$ ist nur auf
  $\et \K^{[1]}$ definiert.
\end{satz}
\begin{proof}
  Dies liegt daran, dass die Testobjekte, mit denen wir Objekte in
  $\K^{[1]}$ eindeutig festlegen können, bereits étale sind. Betrachte
  dazu die Einbettungen $\iota, \tau: \K \to \et \K^{[1]}$ durch über
  dem betreffenden Raum initiale bzw. terminale Garben:
  \begin{align*}
    \iota X := (\varnothing \to X), \\
    \tau X := (X \xrightarrow{\id} X.
  \end{align*}
  Gelte also
  \begin{equation} \label{eq:coreflection-adjunction}
    \K^{[1]}((F \to X), (G \to Y)) \iso \et \K^{[1]}((F \to X), (G \to Y)^+)
  \end{equation}
  für alle étalen $F \to X$. Ein Objekt $(H \to Z) \in \K^{[1]}$ ist
  nach dem Yoneda-Lemma (bis auf eindeutigen Isomorphismus) eindeutig
  festgelegt durch seine Yoneda-Einbettung $\K^{[1]}(\cdot, (H \to
  Z))$. Wir behaupten, dass sogar die Funktoren $\K^{[1]}(\iota \cdot,
  (H \to Z)) = \K(\cdot, Z)$ und $\K^{[1]}(\tau \cdot, (H \to Z)) =
  \K(\cdot, H): \K \to \Ens$ und ihre aus $\iota \cdot \Implies \tau
  \cdot$ entstehende Transformation ausreichen, um $H \to Z$ eindeutig
  festzulegen. In der Tat bestimmen die beiden Funktoren nach dem
  Yoneda-Lemma bereits die beteiligten Räume $H$ und $Z$ eindeutig und
  der Morphismus $H \to Z$ entspricht der Transformation. Nun sind
  $\iota X, \tau X \in \et \K^{[1]}$ für alle $X \in \K$, was den
  Beweis abschließt: dann müssen nämlich im Fall von
  \ref{eq:coreflection-adjunction} $(G \to Y)$ und $(G \to Y)^+$
  isomorph sein und die Koreflektion ist genau dann definiert, wenn
  $(G \to Y)$ sowieso schon étale ist.
\end{proof}
\begin{kor}
  Es gibt ein internes Hom für $F \to X$ und $G \to Y$ in $\et
  \K^{[1]}$ genau dann, wenn die Projektion auf den zweiten Faktor
  \[ \K(F, G) \times_{\K(F, Y)} \K(X, Y) \to \K(X, Y) \]
  étale ist.
\end{kor}
\begin{proof}
  Dies folgt aus \ref{k-arrow-cart-closed} und
  \ref{enstop-missing-coreflector}.
\end{proof}
\begin{bem}
  Diese Bedingung ist sehr restriktiv, denn sie besagt im Fall einer
  koreflektiven vollen Unterkategorie $\K$, dass es für eine Abbildung
  $f: F \to G$ über $g: X \to Y$ eine in der $k$-Verfeinerung der
  kompakt-offen Topologie offene Umgebung von $g$ gibt, über der $f$
  jeweils eine eindeutige darüberliegende Abbildung $F \to G$
  definiert.
\end{bem}

\subsection{Vollständigkeitseigenschaften von der Garben auf topologischen Räumen}

Mit denselben Techniken können wir auch zeigen, dass die Kategorie
$\EnsTop$ der Garben auf topologischen Räumen über die endlichen
Limites aus \ref{enstop-fin-complete} und die trivialen Koprodukte
hinaus weder vollständig noch kovollständig ist.

\begin{prop} \label{enstop-not-complete}
  Die Kategorie $\EnsTop$ der Garben über topologischen Räumen besitzt
  keine unendlichen Produkte.
\end{prop}
\begin{proof}
  Dies sieht man schon am Beispiel eines unendlichen Produkts von
  Inklusionen $\iota: U \inj X$ einer echten offenen Teilmenge $U \neq
  X$. Die Abbildung
  \[ \prod_\N \iota: \prod_n U \to \prod_\N X \]
  ist nicht étale, denn es gibt keine Schnitte $\prod_\N X \to
  \prod_\N U$, weil eine Menge in der Basis der Produkttopologie von
  $\prod_\N X$ in fast allen Faktoren $X$ die Projektion $X$ hat. Dies
  wäre aber wegen des Fehlens eines Koreflektors
  (\ref{enstop-missing-coreflector}) für die Existenz von unendlichen
  Produkten nötig.
\end{proof}
\begin{kor} \label{enstop-not-reflective}
  Die volle Unterkategorie $\EnsTop \subset \Top^{[1]}$ ist nicht
  reflektiv.
\end{kor}
\begin{proof}
  Gäbe es einen zur Inklusion Linksadjungierten, so wäre die Inklusion
  linksexakt, würde also Limites erhalten.
\end{proof}
\begin{prop} \label{enstop-no-coequalizers}
  Die Kategorie $\EnsTop$ der Garben über topologischen Räumen besitzt
  nicht alle Koegalisatoren.
\end{prop}
\begin{proof}
  Betrachte etwa $F \in \Ens_{/[0,1]}$ die konstant zweielementige
  Garbe und für $\point \in \Ens_{/\point}$ die konstant einelementige
  Garbe auf dem Punkt die Morphismen $\point \rightrightarrows F$, die
  den Punkt auf die beiden Endpunkte über $0$ bzw $1 \in [0, 1]$ eines
  festen globalen Schnitts von $F$ schicken. Der Koegalisator in
  $\Top^{[1]}$ ist ein topologischer Raum über $S^1$, dessen eine
  Zusammenhangskomponente $S^1 \xrightarrow{\id} S^1$ und dessen
  andere Zusammenhangskomponente eine \emph{abgeschlossene} Teilmenge
  einer Spirale über $S^1$ ist, die in genau einem Punkt
  zweielementiges Urbild hat. Der Koegalisator existiert also nach
  \ref{enstop-not-reflective} nicht, da die Abbildung im Koegalisator
  in $\Top^{[1]}$ nicht étale ist.
  % TODO der Reflektor existiert ja nur nicht überall...
  % TODO besseres Beispiel, komplett durchrechnen..
\end{proof}


% TODO Notationen vereinheitlichen in den Beweisen zum fehlenden
% kart. Abschluss

%% TODO:
%% * erwähnen, dass keine Einheit/Koeinheits-Isos zu erwarten, aber auf
%%   Homotopie: Quillen-Äq. von Modellkategorien



% TODO:
% * konkret machen (im Fall von Y_n diskret vllt)
% * charakterisiere das Bild -> kleiner machen mit kleinerem Intervall?
%   (Moer)
% * deriviert???

\end{document}
