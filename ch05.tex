% include latex header (\usepackage, \newcommand etc.) 
\documentclass[a4paper]{article}
% \usepackage[left=3cm,right=3cm,top=3cm,bottom=2cm]{geometry} % page settings
\usepackage{amsmath}
\usepackage{amssymb}
\usepackage{amsthm}
\usepackage{etoolbox}
\usepackage[ngerman]{babel}
\usepackage[utf8]{inputenc}
\usepackage{mathtools}
\usepackage{tikz-cd}
\usepackage{enumitem}
\usepackage{hyperref}

\setlength{\parskip}{\medskipamount}
\setlength{\parindent}{0pt}

\theoremstyle{plain}
\newtheorem{theorem}{Theorem}
\newtheorem{lemma}[theorem]{Lemma}
\newtheorem{prop}[theorem]{Proposition}
\newtheorem{kor}[theorem]{Korollar}
\newtheorem{satz}[theorem]{Satz}
%% \providecommand*{\lemmaautorefname}{Lemma}
%% \providecommand*{\propautorefname}{Prop.}
%% \providecommand*{\korautorefname}{Korollar}
%% \providecommand*{\satzautorefname}{Satz}

\theoremstyle{definition}
\newtheorem{defn}[theorem]{Definition}

\theoremstyle{remark}
\newtheorem{bem}[theorem]{Bemerkung}

\DeclareMathOperator{\Cat}{Cat}
\DeclareMathOperator{\poset}{poset}
\DeclareMathOperator{\EnsX}{Ens_{/X}}
\DeclareMathOperator{\pEnsX}{pEns_{/X}}
\DeclareMathOperator{\AbX}{Ab_{/X}}
\DeclareMathOperator{\pAbX}{pAb_{/X}}
\DeclareMathOperator{\OffX}{Off_X}
\DeclareMathOperator{\Ens}{Ens}
\DeclareMathOperator{\Ob}{Ob}
\DeclareMathOperator{\Der}{Der}
\DeclareMathOperator{\Ab}{Ab}
\DeclareMathOperator{\sKons}{s-Kons}
\DeclareMathOperator{\Ket}{Ket}
\DeclareMathOperator{\EnsB}{Ens_{/\B}}
\DeclareMathOperator{\im}{im}
\DeclareMathOperator{\Id}{Id}
\DeclareMathOperator{\id}{id}
\DeclareMathOperator{\colf}{colf}
\DeclareMathOperator{\limf}{limf}
\DeclareMathOperator{\Top}{Top}

\newcommand{\etalespace}[1]{\overline{#1}}
\newcommand{\B}{\mathcal{B}}
\newcommand{\op}{^\mathrm{op}}
\newcommand{\iso}{\xrightarrow{\sim}}
\newcommand{\qiso}{\xrightarrow{\approx}}
\newcommand{\fromqiso}{\xleftarrow{\approx}}
\newcommand{\open}{\subset\kern-0.58em\circ}  % only possible in math mode
\newcommand{\K}{\mathcal{K}}
\newcommand{\Z}{\mathbb{Z}}
\newcommand{\R}{\mathbb{R}}
\newcommand{\DerAbK}{\Der(\Ab_{/|\K|})}
\newcommand{\DerskK}{\Der_{\mathrm{sk}}(|\K|)}
\newcommand{\DerpskK}{\Der^+_{\mathrm{sk}}(|\K|)}
\newcommand{\AbKr}{\Ab_{/|\K|}}
\newcommand{\sKonsK}{\sKons(\K)}
\newcommand{\inj}{\hookrightarrow}
\newcommand{\surj}{\twoheadrightarrow}
\newcommand{\Iff}{\Leftrightarrow}
\newcommand{\Implies}{\Rightarrow}
\newcommand{\cc}{^{\bullet}}  % chain complex
\newcommand{\from}{\leftarrow}


\begin{document}

\title{Simpliziale Garben}
\author{Fabian Glöckle}
\date{\today}
% \maketitle

\section{Simpliziale Garben}

Die Beschreibung der geometrischen Realsierung als Tensorprodukt von
Funktoren eröffnet uns eine Reihe weiterer geometrischer
Realisierungen, die diejenige simplizialer Mengen verallgemeinern.

Zunächst stellen wir fest, dass wir in unserer Konstruktion
simpliziale Mengen immer als diskrete simpliziale topologische Räume
betrachtet haben, und die Diskretheit genauso gut auch fallen lassen
können. Wir erhalten die geometrische Realisierung $|X| = X \otimes R$
eines simplizialen topologischen Raums $X: \Delta\op \to \Top$.
\begin{bsp}
  Wir betrachten den simplizialen topologischen Raum $X: \Delta\op \to
  \Top$, den wir aus dem (kombinatorischen) Standard-1-Simplex
  $\Delta^1$ erhalten, indem wir disjunkte Vereinigungen von Punkten
  durch disjunkte Vereinigungen von Intervallen $I = [0, 1]$ mit von
  den Identitäten induzierten Abbildungen ersetzen. Offenbar ist die
  geometrische Realiserung das Produnkt $I \times
  |\Delta^1|$. Ersetzen wir $X_0$ wieder durch zwei Punkte ${0, 1}$
  mit beliebigen Degenerationen, so erhalten wir eine zu einer
  Kreisscheibe verdickte Linie zwischen den beiden Punkten als
  Realisierung. Ersetzen wir die höheren $X_n, n \geq 2$ ebenfalls
  wieder durch Punkte mit beliebigen Randabbildungen, so sorgen deren
  Identifikationen dafür, dass die geometrische Realisierung wieder
  $|\Delta^1|$ wird.
  % TODO Interessantere Beispiele? RP^n oder so?
\end{bsp}

Weiter verallgemeinert die Konstruktion auch auf
Diagrammkategorien. Ist $I$ eine kleine Kategorie und $X: \Delta\op
\to \Top^I$ ein simpliziales $I$-System topologischer Räume, so
erhalten wir eine geometrische Realisierung $|X| = X \otimes R$ von
$X$, wenn wir $R: \Delta \to \Top \to \Top^I$ mittels des Funktors der
konstanten Darstellung auf die Diagrammkategorie
fortsetzen. Insbesondere erhalten wir eine geometrische Realisierung
für die Kategorie der Paare topologischer Räume mit stetiger
Abbildung, d. h. die Diagrammkategorie $\Top^I$ für $I$ die von
$\{ \bullet \to \bullet \}$ erzeugte Kategorie. Uns interessiert der
Fall von Garben:
\begin{satz}
  Sei $I = \{ \bullet \to \bullet \}$, $X: \Delta\op \to \Top^I$ ein
  simpliziales Paar topologischer Räume mit stetiger Abbildung, für
  die $X_n: E_n \to Y_n$ étale ist für alle $[n]$. Dann ist auch die
  geometrische Realisierung $|X|: |E| \to |Y|$ étale.
\end{satz}
\begin{proof}
  Die Realisierung ist die Abbildung $|E| = \coprod_n E_n \times
  |\Delta^n| / \sim \to |Y| = \coprod_n Y_n \times |\Delta^n| / \sim$,
  die von den $X_n: E_n \to Y_n$ induziert wird. Sind die $X_n: E_n
  \to Y_n$ étale, so ist die Abbildung auf den Koprodukten étale und
  es reicht zu bemerken, dass die Wirkung von $f: [n] \to [m]$ auf den
  Koprodukten verträglich ist nach Definition eines Funktors nach
  $\Top^I$.
\end{proof}
In die Sprache der Garben zurückübersetzt bedeutet das, dass wir eine
geometrische Realisierung erklärt haben für simpliziale Garben über
topologischen Räumen $X: \Delta\op \to \Ens_{/\Top}$:
\[ E_n \in \Ens_{/Y_n} \qquad \rightsquigarrow \qquad |E| \in \Ens_{/|Y|} . \]
Man beachte, dass Morphismen étaler $\Top^I$ den ``Morphismen'' in
$\Ens_{/\Top}$ entsprechen, während sonst häufig mit Komorphismen
gearbeitet wird.

\subsection{Die Dualität von Nerv und Realisierung}

Wir suchen Rechtsadjungierte für unsere geometrischen
Realisierungen. Für die Realisierung simplizialer Mengen gelingt uns
das einfach.
\begin{satz}
  Der Funktor der singulären Ketten $S: \Top \to \s\Ens$, $SY
  = \Top(R \, \cdot, Y): [n] \mapsto \Top(|\Delta^n|, Y)$ ist
  rechtsadjungiert zur geometrischen Realisierung
  $|\cdot|: \s\Ens \to \Top$.
\end{satz}
\begin{proof}
  Die Rand- und Degenerationsabbildungen von $SY$ sind für $f: [n] \to
  [m]$ gegeben durch Vorschalten von $|f|: |\Delta^n| \to
  |\Delta^m|$. Wir berechnen
  \begin{align*}
    \Top(|X|, Y)
    & = \Top(\col_{\slicecat{\Delta}{r}{X}} |\Delta^n|, Y) \\
    & \iso \col_{\slicecat{\Delta}{r}{X}} \Top(|\Delta^n|, Y) \\
    & \iso \col_{\slicecat{\Delta}{r}{X}} \s\Ens(\Delta^n, \Top(R \, \cdot, Y)) \\
    & \iso \s\Ens(\iso \col_{\slicecat{\Delta}{r}{X}} \Delta^n, \Top(R \, \cdot, Y)) \\
    & \iso \s\Ens(X, SY)
  \end{align*}  
  mit der Definition der geometrischen Realisierung im ersten Schritt
  (Gl. \ref{eq:real-colim}), der Verträglichkeit von $\Hom:
  C\op \times C \to \Ens$ mit Limites im zweiten und vierten Schritt,
  unserer Bestimmung der $n$-Simplizes als Morphismenmenge
  (Gl. \ref{eq:simp-as-hom}) im dritten Schritt und unserer
  Beschreibung einer simplizialen Menge als Kolimes über ihre
  Simplexkategorie (\ref{sset-colim}) im letzten Schritt.
\end{proof}
Während dieses Argument wieder ein sehr anschauliches ist, möchten wie
wie in \ref{coend-correct-real} erklärt, unser Argument mit den
Begriffen und Techniken von Koenden führen, um es automatisch
verallgemeinern zu können. Wir geben hier noch einmal die direkte
Übersetzung obigen Beweises in die Sprache der Koenden an, und dann
sofort die Verallgemeinerung.
\begin{proof} (\cite{Lore}, 3.2)
  Wir berechnen mit den Regeln des Koenden-Kalküls:
  \begin{align*}
     \Top(|X|, Y)
     &= \Top \left( \int^{[n]} X[n] \times R[n], Y \right) \\
     & \iso[\ref{coend-cocont}]
       \int_{[n]} \Top \big( X[n] \times R[n], Y \big) \\
     & \iso[\ref{copower}]
       \int_{[n]} \Ens \big( X[n], \Top(R[n], Y) \big) \\
     & \iso[\ref{trans-end}]
       [\Delta\op, \Ens] \big( X, \Top(R \, \cdot, Y) \big) \\
     &= \s\Ens(X, SY).
  \end{align*}
\end{proof}

\begin{theorem} [Allgemeine Nerv-Realisierungs-Dualität, \cite{Lore}, 3.2]
  Seien $C$ eine $V$-Kategorie mit Koexponentialen und ein Funktor $R:
  S \to C$ gegeben. Dann gibt es eine Adjunktion $(|\cdot|, N)$  
  \[ C \xtofrom[N]{|\cdot|} [S\op, V] \]
  mit
  \begin{alignat*}{3}
    &|\cdot|: && X &&\mapsto \int^s X(s) \odot R(s) \qquad \text{und} \\
    &N: && Y &&\mapsto C(R \, \cdot, Y).
  \end{alignat*}
\end{theorem}
\begin{proof}
  In wörtlicher Verallgemeinerung des Vorangegangenen:
  \begin{align*}
     C(|X|, Y)
     &= C \left( \int^s X(s) \odot R(s), Y \right) \\
     & \iso[\ref{coend-cocont}]
       \int_s C \big( X(s) \odot R(s), Y \big) \\
     & \iso[\ref{copower}]
       \int_s V \big( X(s), C(R(s), Y) \big) \\
     & \iso[\ref{trans-end}]
       [S\op, V] \big( X, C(R \, \cdot, Y) \big) \\
     &= [S\op, V] (X, NY).
  \end{align*}
\end{proof}

% TODO:
% * erster Ansatz: halmweise Realisierung funktioniert nicht
% * erklären, wieso das für Ens/Top nicht funktioniert

\subsection{Die kartesisch abgeschlossene Struktur der Garben auf $X$}

Für unsere allgemeine Dualität von Nerv und Realisierung \ref{}
benötigen wir also eine bessere $V$-angereichterte Struktur auf
$C$. Wenn wir uns auf $\EnsX$ beschränken, erhalten wir sogar die
Struktur einer kartesisch abgeschlossenen Kategorie
(engl. \emph{cartesian closed category}), d. h. einer Kategorie mit
endlichen Produkten, für deren kartesische monoidale Struktur es ein
internes Hom gibt.

\begin{prop}
  Die Kategorie $\EnsX$ ist kartesisch abgeschlossen mit Produkt
  \[ (F \times G)(U) = F(U) \times G(U) \]
  und internem Hom
  \[ (F \Implies G)(U) = G(U)^{F(U)} \].
\end{prop}
\begin{proof}
  Das Produkt erfüllt offenbar die universelle Eigenschaft in $\pEnsX$
  und ist eine Garbe, da Produkte mit dem Limes der Garbeneigenschaft
  vertauschen. Das interne Hom besteht für $U \open X$ und $V \subset
  U$ offen aus verträglichen Abbildungen $F(V) \to G(V)$, mithin also
  aus Garbenmorphismen $F|_U \to G|_U$. Diese sind untereinander
  veträglich durch Einschränkung und erfüllen die Garbenbedingung, die
  ja gerade nach der Verklebbarkeit stetiger Abbildungen modelliert
  war. Für die Adjunktion $(\cdot \times G, G \Implies \cdot)$
  bemerken wir, dass sie nach dem Exponentialgesetz in $\Ens$ bereits
  für die Prägarbenkategorien gilt.
\end{proof}


% TODO: Problem? Garbenmorphismen, nicht wie sonst Komorphismen!

% TODO:
% * konkret machen (im Fall von Y_n diskret vllt)
% * Nerv und Realisierung, finde die Adjungierten
% * charakterisiere das Bild
% * deriviert???

% * Problem: was ist die V-enriched Struktur bei uns??

\end{document}
