% include latex header (\usepackage, \newcommand etc.) 
\documentclass[a4paper]{article}
% \usepackage[left=3cm,right=3cm,top=3cm,bottom=2cm]{geometry} % page settings
\usepackage{amsmath}
\usepackage{amssymb}
\usepackage{amsthm}
\usepackage{etoolbox}
\usepackage[ngerman]{babel}
\usepackage[utf8]{inputenc}
\usepackage{mathtools}
\usepackage{tikz-cd}
\usepackage{enumitem}
\usepackage{hyperref}

\setlength{\parskip}{\medskipamount}
\setlength{\parindent}{0pt}

\theoremstyle{plain}
\newtheorem{theorem}{Theorem}
\newtheorem{lemma}[theorem]{Lemma}
\newtheorem{prop}[theorem]{Proposition}
\newtheorem{kor}[theorem]{Korollar}
\newtheorem{satz}[theorem]{Satz}
%% \providecommand*{\lemmaautorefname}{Lemma}
%% \providecommand*{\propautorefname}{Prop.}
%% \providecommand*{\korautorefname}{Korollar}
%% \providecommand*{\satzautorefname}{Satz}

\theoremstyle{definition}
\newtheorem{defn}[theorem]{Definition}

\theoremstyle{remark}
\newtheorem{bem}[theorem]{Bemerkung}

\DeclareMathOperator{\Cat}{Cat}
\DeclareMathOperator{\poset}{poset}
\DeclareMathOperator{\EnsX}{Ens_{/X}}
\DeclareMathOperator{\pEnsX}{pEns_{/X}}
\DeclareMathOperator{\AbX}{Ab_{/X}}
\DeclareMathOperator{\pAbX}{pAb_{/X}}
\DeclareMathOperator{\OffX}{Off_X}
\DeclareMathOperator{\Ens}{Ens}
\DeclareMathOperator{\Ob}{Ob}
\DeclareMathOperator{\Der}{Der}
\DeclareMathOperator{\Ab}{Ab}
\DeclareMathOperator{\sKons}{s-Kons}
\DeclareMathOperator{\Ket}{Ket}
\DeclareMathOperator{\EnsB}{Ens_{/\B}}
\DeclareMathOperator{\im}{im}
\DeclareMathOperator{\Id}{Id}
\DeclareMathOperator{\id}{id}
\DeclareMathOperator{\colf}{colf}
\DeclareMathOperator{\limf}{limf}
\DeclareMathOperator{\Top}{Top}

\newcommand{\etalespace}[1]{\overline{#1}}
\newcommand{\B}{\mathcal{B}}
\newcommand{\op}{^\mathrm{op}}
\newcommand{\iso}{\xrightarrow{\sim}}
\newcommand{\qiso}{\xrightarrow{\approx}}
\newcommand{\fromqiso}{\xleftarrow{\approx}}
\newcommand{\open}{\subset\kern-0.58em\circ}  % only possible in math mode
\newcommand{\K}{\mathcal{K}}
\newcommand{\Z}{\mathbb{Z}}
\newcommand{\R}{\mathbb{R}}
\newcommand{\DerAbK}{\Der(\Ab_{/|\K|})}
\newcommand{\DerskK}{\Der_{\mathrm{sk}}(|\K|)}
\newcommand{\DerpskK}{\Der^+_{\mathrm{sk}}(|\K|)}
\newcommand{\AbKr}{\Ab_{/|\K|}}
\newcommand{\sKonsK}{\sKons(\K)}
\newcommand{\inj}{\hookrightarrow}
\newcommand{\surj}{\twoheadrightarrow}
\newcommand{\Iff}{\Leftrightarrow}
\newcommand{\Implies}{\Rightarrow}
\newcommand{\cc}{^{\bullet}}  % chain complex
\newcommand{\from}{\leftarrow}


\begin{document}

\title{Simpliziale Garben}
\author{Fabian Glöckle}
\date{\today}
% \maketitle

\section{Simpliziale Garben}

Die Beschreibung der geometrischen Realsierung als Tensorprodukt von
Funktoren eröffnet uns eine Reihe weiterer geometrischer
Realisierungen, die diejenige simplizialer Mengen verallgemeinern.

Zunächst stellen wir fest, dass wir in unserer Konstruktion
simpliziale Mengen immer als diskrete simpliziale topologische Räume
betrachtet haben, und die Diskretheit genauso gut auch fallen lassen
können. Wir erhalten die geometrische Realisierung $|X| = X \otimes R$
eines simplizialen topologischen Raums $X: \Delta\op \to \Top$.
\begin{bsp}
  Wir betrachten den simplizialen topologischen Raum $X: \Delta\op \to
  \Top$, den wir aus dem (kombinatorischen) Standard-1-Simplex
  $\Delta^1$ erhalten, indem wir disjunkte Vereinigungen von Punkten
  durch disjunkte Vereinigungen von Intervallen $I = [0, 1]$ mit von
  den Identitäten induzierten Abbildungen ersetzen. Offenbar ist die
  geometrische Realiserung das Produnkt $I \times
  |\Delta^1|$. Ersetzen wir $X_0$ wieder durch zwei Punkte ${0, 1}$
  mit beliebigen Degenerationen, so erhalten wir eine zu einer
  Kreisscheibe verdickte Linie zwischen den beiden Punkten als
  Realisierung. Ersetzen wir die höheren $X_n, n \geq 2$ ebenfalls
  wieder durch Punkte mit beliebigen Randabbildungen, so sorgen deren
  Identifikationen dafür, dass die geometrische Realisierung wieder
  $|\Delta^1|$ wird.
  % TODO Interessantere Beispiele? RP^n oder so?
\end{bsp}

Weiter verallgemeinert die Konstruktion auch auf
Diagrammkategorien. Ist $I$ eine kleine Kategorie und $X: \Delta\op
\to \Top^I$ ein simpliziales $I$-System topologischer Räume, so
erhalten wir eine geometrische Realisierung $|X| = X \otimes R$ von
$X$, wenn wir $R: \Delta \to \Top \to \Top^I$ mittels des Funktors der
konstanten Darstellung auf die Diagrammkategorie
fortsetzen. Insbesondere erhalten wir eine geometrische Realisierung
für die Kategorie der Paare topologischer Räume mit stetiger
Abbildung, d. h. die Diagrammkategorie $\Top^I$ für $I = \{ \bullet
\to \bullet \}$. Uns interessiert der Fall von Garben:
\begin{satz}
  Sei $I = \{ \bullet \to \bullet \}$, $X: \Delta\op \to \Top^I$ ein
  simpliziales Paar topologischer Räume mit stetiger Abbildung, für
  die $X_n: E_n \to Y_n$ étale ist für alle $[n]$. Dann ist auch die
  geometrische Realisierung $|X|: |E| \to |Y|$ étale.
\end{satz}
\begin{proof}
  Die Realisierung ist die Abbildung $|E| = \coprod_n E_n \times
  |\Delta^n| / \sim \to |Y| = \coprod_n Y_n \times |\Delta^n| / \sim$,
  die von den $X_n: E_n \to Y_n$ induziert wird. Sind die $X_n: E_n
  \to Y_n$ étale, so ist die Abbildung auf den Koprodukten étale und
  es reicht zu bemerken, dass die Wirkung von $f: [n] \to [m]$ auf den
  Koprodukten verträglich ist nach Definition eines Funktors nach
  $\Top^I$.
\end{proof}
In die Sprache der Garben zurückübersetzt bedeutet das, dass wir eine
geometrische Realisierung erklärt haben für simpliziale Garben über
topologischen Räumen $X: \Delta\op \to \Ens_{/\Top}$:
\[ E_n \in \Ens_{/Y_n} \qquad \rightsquigarrow \qquad |E| \in \Ens_{/|Y|} . \]

% TODO:
% * konkret machen (im Fall von Y_n diskret vllt)
% * Nerv und Realisierung, finde die Adjungierten
% * charakterisiere das Bild
% * deriviert???

\end{document}
