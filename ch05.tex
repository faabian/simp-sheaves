% Emacs mode: -*-latex-*-

\chapter{Simpliziale Garben über topologischen Räumen}
\chaptermark{Simpliziale Garben}
\label{ch:simp-sheaves}

Die allgemeine Formulierung der geometrischen Realisierung als Koende
aus \autoref{sec:coend-real} und die Dualität von Nerv und
Realisierung aus \autoref{sec:nerve} verleiten zu der Hoffnung,
simpliziale Objekte in einer Garbenkategorie über einem festen
topologischen Raum $X$ oder variablen Basisräumen auf eine ähnliche
Weise geometrisch realisieren zu können und dann mittels des
adjungierten Nerv-Funktors eine Äquivalenz bestimmter Unterkategorien
zu finden. Diese Hoffnung wird sich als verfrüht herausstellen, wir
werden allerdings im folgenden Abschnitt eine andere Variante der
geometrischen Realisierung vorstellen und im weiteren Verlauf die
Vollständigkeits- und Abschlusseigenschaften der angesprochenen
Garbenkategorien diskutieren.

\section{Realisierung simplizialer Garben}
\label{sec:simp-sheaves-real}

Die Beschreibung der geometrischen Realisierung als Tensorprodukt von
Funktoren eröffnet zunächst auch eine geometrische Realisierung
simplizialer topologischer Räume. Bei der geometrischen Realisierung
simplizialer Mengen haben wir die Simplexmengen $X_n$ als diskrete
topologische Räume betrachtet, die Konstruktion funktioniert aber
genauso für beliebige topologische Räume $X_n$. Wir erhalten die
geometrische Realisierung $|X| = X \otimes R$ eines simplizialen
topologischen Raums $X: \Delta\op \to \Top$.
\begin{bsp}
  Wir betrachten den simplizialen topologischen Raum $X$, den wir aus
  dem (kombinatorischen) Standard-1-Simplex $\Delta^1$ erhalten, indem
  wir disjunkte Vereinigungen von Punkten durch disjunkte
  Vereinigungen von Intervallen $I = [0, 1]$ mit von den Identitäten
  induzierten Abbildungen ersetzen. Offenbar ist die geometrische
  Realiserung das Produnkt $I \times |\Delta^1|$. Ersetzen wir $X_0$
  wieder durch zwei Punkte ${0, 1}$ mit beliebigen Degenerationen, so
  erhalten wir eine zu einer Kreisscheibe verdickte Linie zwischen den
  beiden Punkten als Realisierung. Ersetzen wir die höheren $X_n, n
  \geq 2$ ebenfalls wieder durch Punkte mit beliebigen
  Randabbildungen, so sorgen deren Identifikationen dafür, dass die
  geometrische Realisierung wieder $|\Delta^1|$ wird.

  Interessanter ist vielleicht der simpliziale topologische Raum, der
  in Grad 0 aus den Simplizes $S^1 \sqcup \point$ und in Grad 1 aus
  den nichtdegenerierten Simplizes $D^2$ besteht, und dessen
  Randabbildungen $S^1 \times {0}$ mit $\point$ und $S^1 \times {1}$
  mit $S^1 \subset D^2$ durch ``Wickeln mit doppelter
  Geschwindigkeit'' (komplex $x \mapsto x^2$) identifizieren. Dazu
  kommen degenerierte Simplizes, sodass die Degenerationsabbildungen
  keine weiteren Identifikationen verursachen. Wir erhalten so den
  zweidimensionalen reell-projektiven Raum.
\end{bsp}
Diese Konstruktion verallgemeinert auch auf Diagrammkategorien. Ist
$I$ eine kleine Kategorie und $X: \Delta\op \to \Top^I$ ein
simpliziales $I$-System topologischer Räume, so erhalten wir eine
geometrische Realisierung $|X| = X \otimes R$ von $X$, wenn wir $R:
\Delta \to \Top \to \Top^I$ mittels des Funktors der konstanten
Darstellung auf die Diagrammkategorie fortsetzen. Insbesondere
erhalten wir eine geometrische Realisierung für die Kategorie der
Paare topologischer Räume mit stetiger Abbildung, d. h. die
Diagrammkategorie $\Top^{[1]}$ für $[1] = (\bullet \to \bullet)$ die
Kategorie des Ordinals $[1] \in \Delta$. Eine Realisierung für Garben
über topologischen Räumen erhalten wir so aber nicht: Sind in einem
simplizialen $\Top^{[1]}$ alle Abbildungen étale, bedeutet das noch
nicht, dass auch die induzierte Abbildung in der geometrischen
Realisierung étale ist. Sind etwa alle Basisräume $X_n$ einpunktig, so
ist die Realisierung $|X|$ ebenfalls einpunktig. Étale Räume $F_n \to
X_n$ sind diskret, d. h. die geometrische Realisierung $|F|$ ist die
einer simplizialen Menge, also im Allgemeinen ein höherdimensionaler
CW-Komplex. Ein solcher ist nicht diskret, also nicht étale über
$|X|$. Wir können versuchen, den Kolimes des Koendes nicht in der
Kategorie der Paare topologischer Räume, sondern in der richtigen
Kategorie $\EnsTop$ zu bilden. Später werden wir jedoch sehen, dass
die beiden in ``nicht-kombinatorischen'' Fällen übereinstimmen müssen
und der benötigte Kolimes in $\EnsTop$ in den für uns relevanten
Fällen nicht existiert (\ref{enstop-coequalizers}).

Dass dieser naive Ansatz zur geometrischen Realisierung simplizialer
Garben nicht funktioniert, liegt daran, dass die Randabbildungen der
simplizialen Garben gegenläufig sind zu den Einbettungen der
Ränder. Um Garben geeignet verkleben zu können, benötigen wir aber
generisierende Randabbildungen, die in derselben Richtung verlaufen.

Wir erklären eine neue Realisierung, die diesem Anspruch gerecht wird.
Sei dazu $R: \Delta \to \Top$ ein kosimplizialer topologischer Raum
und
\begin{align*}
  F: \Delta\op &\to \EnsssTop, \\
  [n] &\mapsto F_n \in Ens_{/X_n}
   \end{align*}
eine simpliziale Garbe über topologischen Räumen mit Komorphismen. Für
$f: [n] \to [m]$ monoton gibt es also eine stetige Abbildung $Ff: X_m
\to X_n$ und einen Morphismus von Garben über $X_m$:
\[ Ff^* F_n \to F_m .\]
Wir erhalten einen Funktor $K$ von der Unterteilungskategorie
$\Sub(\Delta)$ von $\Delta$ in die Kategorie der Garben über
topologischen Räumen
\begin{equation} \label{dg:cov-real}
  \begin{tikzcd}
    \Sub(\Delta) \arrow{dd}
    & {[n]}^\S \dar[mapsto, shorten <= 3em, yshift=1.5em]
    & \lar f^\S \dar[mapsto, shorten <= 3em, yshift=1.5em] \rar
    & {[m]}^\S \dar[mapsto, shorten <= 3em, yshift=1.5em] \\[25]
    & F_n \times R[n]
    & \lar Ff^* F_n \times R[n] \rar
    & F_m \times R[m] \\[-2em]
    \Ens_{/\Top}
    & \dar[shorten <= -1em]
    & \dar[shorten <= -1em]
    & \dar[shorten <= -1em] \\[-2em]
    & X_n \times R[n]
    & \lar X_m \times R[n] \rar
    & X_m \times R[m],
  \end{tikzcd}
\end{equation}
der für $f: [n] \to [m]$ in $\Delta$ auf Morphismen $f^\S \to [n]^\S$
vom Rückzug $Ff^* F_n \to F_n$ über $Ff$ induziert ist und auf
Morphismen $f^\S \to [m]^\S$ durch die Morphismen $Ff^* F_n \to F_m$
in $\Ens_{/X_m}$ sowie $Rf$. Letzterer ist tatsächlich ein Morphismus
in $\Ens_{/\Top}$, denn es kommutiert
\[ \begin{tikzcd}
  Ff^* F_n \times R[n] \dar \rar
  & F_m \times R[n] \dar \rar
  & F_m \times R[m] \dar \\
  X_m \times R[n] \rar
  & X_m \times R[n] \rar
  & X_m \times R[m].
\end{tikzcd} \]
Wir erhalten die Vorform der \emph{kovarianten Realisierung}, den
Kolimes in $\EnsTop$ über den oben definierten $F$ zugeordneten
Funktor $K: \Sub(\Delta) \to \EnsTop$.

Wann existiert dieser Kolimes in $\EnsTop$? Schon im einfachsten Fall
diskreter Basisräume ist das im Allgemeinen nicht zu
erwarten. Betrachte etwa die simpliziale Menge aus zwei
nichtdegenerierten $1$-Simplizes $\sigma$ und $\tau$, die an einem
gemeinsamen $0$-Simplex $x$ zusammengeklebt sind. In der kovarianten
Realisierung einer Garbe $F$ über diesem Raum besteht (wegen
\ref{enstop-coequalizers} der Halm über dem Verklebungspunkt $x$ aus
der Vereinigung der Halme von $F_\sigma$, $F_\tau$ und $F_x$ mit den
durch die Verklebungen $F_x \to F_\sigma$ und $F_x \to F_\tau$
induzierten Identifikationen. Nur wenn diese beiden Abbildungen
surjektiv sind, ist tatsächlich jedes Element des Halms in beide
Richtungen fortsetzbar und die kovariante Realisierung eine Garbe.

Wir reparieren dieses Problem von Hand. Bezeichne dazu einen Punkt in
einem Raum $F \in \Top_X$ mit $p: F \to X$ als \emph{étale}, wenn er
eine offene Umgebung $U$ besitzt, für die $p(U) \subset X$ offen und
$p|_U: U \to p(U)$ ein Homöomorphismus ist.
\begin{satz} \label{real-enstop-cov}
  Sei $F \in [\Delta\op, \EnsssTop]$ eine simpliziale Garbe über
  topologischen Räumen mit Komorphismen und $R: \Delta \to \Top$ ein
  kosimplizialer topologischer Raum. Dann heißt die Menge der étalen
  Punkte des Kolimes über den oben definierten zugehörigen Funktor $K:
  \Sub(\Delta) \to \EnsTop \to \Top^{[1]}$ die \emph{kovariante
    Realisierung} $|F|$ von $F$ und ist eine Garbe über der
  geometrischen Realisierung $X \otimes R$ der Basisräume.
\end{satz}
\begin{proof}
  Nichtétale Punkte liegen nach Definition nicht in den étalen
  Umgebungen anderer Punkte, können also nach ihrem Vergessen nicht
  für deren étale Umgebungen fehlen.
\end{proof}
\begin{bem}
  Diese Konstruktion ist zunächst unbefriedigend, da das Vergessen
  nichtétaler Punkte keine guten Eigenschaften besitzt. Es ist besser
  sich die kovariante Realisierung als andere Formulierung der
  folgenden Konstruktion vorzustellen. Für eine Garbe $F \in
  \s\EnsssTop$ über Basisräumen $X \in \s\Top$ und $R$ den
  kosimplizialen Raum der Standardsimplizes betrachte das Diagramm von
  oben, in dem
  \[ F_n \times |\Delta^n| \to X_n \times |\Delta^n| \]
  durch
  \[ F_n \times \intr |\Delta^n| \to X_n \times |\Delta^n| \]
  ersetzt wird. Degenerierte Teile des Diagramms sollen mit denselben
  Abbildungen ``wegidentifiziert'' werden. Randabbildungen können
  hingegen nicht mehr als Identifikationen beschrieben werden, sondern
  bestehen darin, dass in der disjunkten Vereinigung (wie bei der
  Definition der Garbifizierung) Mengen offen gemacht werden, die
  Schnitte (also Verklebungen) repräsentieren sollen. Da in der
  ursprünglichen Formulierung nichtétale Punkte nur auf den
  Verklebungsrändern auftreten, sind diese gerade die Punkte, die in
  dieser Formulierung von vornherein nicht vorkommen.
\end{bem}
\begin{bem} \label{real-ensx-cov}
  Diese geometrische Realisierung simplizialer Garben auf
  topologischen Räumen mit Komorphismen spezialisiert zu einer
  geometrischen Realisierung simplizialer Garben auf $X$: Ist $F:
  \Delta\op \to \Ens_{\sslash X}$ eine simpliziale Garbe auf
  topologischen Räumen mit Komorphismen und konstantem Basisraum $X$
  alias eine kosimpliziale Garbe $F\op: \Delta \to \EnsX$, so
  vereinfacht Diagramm \ref{dg:cov-real} zu
  \[
  \begin{tikzcd}
    \Sub(\Delta) \arrow{dd}
    & {[n]}^\S \dar[mapsto, shorten <= 3em, yshift=1.5em]
    & \lar f^\S \dar[mapsto, shorten <= 3em, yshift=1.5em] \rar
    & {[m]}^\S \dar[mapsto, shorten <= 3em, yshift=1.5em] \\[25]
    & F_n \times R[n]
    & \lar[equal] Ff^* F_n \times R[n] \rar
    & F_m \times R[m] \\[-2em]
    \Ens_{/\Top}
    & \dar[shorten <= -1em]
    & \dar[shorten <= -1em]
    & \dar[shorten <= -1em] \\[-2em]
    & X \times R[n]
    & \lar[equal] X \times R[n] \rar
    & X \times R[m],
  \end{tikzcd}
  \]
  und ihre geometrische Realisierung aus \ref{real-enstop-cov} ist
  eine Garbe $|F| \in \EnsX$, der Kolimes über den Funktor $\Delta \to
  \Ens_{/\Top}$, der $f: [n] \to [m]$ monoton auf den Morphismus
  \[ \begin{tikzcd}
    F_n \times R[n] \dar \rar{F\op f \times |f|}
    & F_m \times R[m] \dar \\
    X \times R[n] \rar
    & X \times R[m]
  \end{tikzcd} \]
  schickt, wieder mit dem Vergessen nichtétaler Punkte.

  Sind alle $X_n$ diskret, so bestimmt für $\sigma \in X_m$ der Halm
  $(F_m)_\sigma$ die konstante Garbe $(F_m)_\sigma \times R[m] \to
  \{\sigma\} \times R[m]$ und wir erhalten für monotones $f: [n] \to
  [m]$ Abbildungen $(F_n)_{f(\sigma)} \to (F_m)_\sigma$, die diese
  konstanten Garben verkleben. Wir werden diese Beobachtungen in
  \ref{real-simplex-cat} präzisieren.
\end{bem}
\begin{bem}
  Eine weitere Idee zur geometrischen Realisierung simplizialer Garben
  ist die folgende: Ist $F$ eine Prägarbe auf $I$ mit Werten in $C$
  und ein Funktor $G: C \to D$ gegeben, so ist $GF: I\op \to D$ eine
  Prägarbe mit Werten in $D$. Nach dem Exponentialgesetz von Funktoren
  ist eine simpliziale Prägarbe über $X$ dasselbe wie eine Prägarbe
  mit Werten in den simplizialen Mengen und wir können die
  geometrische Realisierung $|\cdot|: \s\Ens \to \CGHaus$
  nachschalten. Dies liefert eine Prägarbe topologischer Räume auf
  $X$. Allerdings schränkt diese Konstruktion nicht auf die vollen
  Unterkategorien der Garben ein: die Garbenbedingung ist als Limes
  über ein im Allgemeinen unendliches System formuliert, die
  geometrische Realisierung vertauscht aber im Allgemeinen nicht mit
  beliebigen Limites (vgl. \ref{real-products}).
\end{bem}

\section[Die kartesisch abgeschlossene Struktur der Garben auf
  \texorpdfstring{$X$}{X}]
  {Die kartesisch abgeschlossene Struktur der Garben auf
  \texorpdfstring{$X$}{X}
  \sectionmark{Kartesischer Abschluss über $X$}}
\sectionmark{Kartesischer Abschluss über $X$}
\label{sec:ensx-cart-closed}

Für die allgemeine Dualität von Nerv und Realisierung \ref{nerve}
benötigen wir also eine $V$-angereichterte Struktur auf $C$. Wenn wir
uns auf $\EnsX$ beschränken, erhalten wir sogar die Struktur einer
kartesisch abgeschlossenen Kategorie.
\begin{defn}
  Eine Kategorie $C$ mit endlichen Produkten heißt \emph{kartesisch
    abgeschlossen}, falls es ein internes Hom für die kartesische
  Schmelzstruktur durch Produkte gibt.
\end{defn}
Das bedeutet konkret, dass es eine Adjunktion $(\times Y, Y \iHom)$
gibt, also eine in allen Variablen natürliche Bijektion
\[ C(X \times Y, Z) \iso C(X, Y \iHom Z). \]

\begin{prop} \label{ensx-cart-closed}
  Die Kategorie $\EnsX$ ist kartesisch abgeschlossen mit Produkt
  \[ (F \times G)(U) = F(U) \times G(U) \]
  und internem Hom
  \[ (F \iHom G)(U) = \Ens_{/U}(F|_U, G|_U) \]
  jeweils mit den von den Restriktionen von $F$ und $G$ induzierten
  Restriktionen. Der étale Raum des Produkts ist das Faserprodukt über
  $X$
  \[ \etalespace{F \times G}
     \iso \etalespace{F} \times_X \etalespace{G}. \]
\end{prop}
\begin{proof}
  Das Produkt erfüllt die universelle Eigenschaft in $\pEnsX$ und ist
  eine Garbe, da Produkte mit dem Limes der Garbeneigenschaft
  vertauschen (Spezialfall von \ref{ensx-complete}). Das interne Hom
  besteht aus stetigen Abbildungen über $U$ und erfüllt somit die
  Garbenbedingung, die ja sogar nach der Verklebbarkeit stetiger
  Abbildungen modelliert war. Für die Adjunktion müssen wir zeigen
  \[ \EnsX(F \times G, H) \iso \EnsX(F, G \iHom H). \]
  Links stehen restriktionsverträgliche Systeme
  \[ F(U) \times G(U) \to H(U) \quad \text{bzw.} \quad
  F(U) \to \Ens(G(U), H(U)),
  \]
  rechts restriktionsverträgliche Systeme
  \[ F(U) \to \Ens_{/U}(G|_U, H|_U) . \]
  Wir erhalten eine Abbildung von rechts nach links durch den globalen
  Teil $G(U) \to H(U)$ des Garbenmorphismus $G|_U \to H|_U$ und das
  Exponentialgesetz in $\Ens$ und von links nach rechts durch Ergänzen
  des globalen Teils des Morphismus $G(U) \to H(U)$ durch verträgliche
  $G(V) \to H(V)$ als die Bilder unter
  \[ F(U) \to F(V) \to \Ens(G(V), H(V)). \]
  Diese Abbildungen sind zueinander invers.

  Für den étalen Raum des Produkts erhalten wir nach der universellen
  Eigenschaft des Faserprodukts eine stetige Abbildung über $X$
  \[ \etalespace{F \times G}
  \to \etalespace{F} \times_X \etalespace{G}.
  \]
  Diese induziert auf den Halmen die Bijektionen
  \[ (F \times G)_x \iso F_x \times G_x \]
  aus dem Vertauschen endlicher Limites mit filtrierenden Kolimites.
\end{proof}
Diese Struktur einer kartesisch abgeschlossenen Kategorie macht
$\EnsX$ insbesondere zu einer über sich selbst tensorierten Kategorie
im Sinne von \ref{def:copower}. Gäbe es eine geometrische Realisierung
simplizialer Garben durch Koenden, hätten wir so mit \ref{nerve} einen
adjungierten Nerv-Funktor gefunden.

Den obigen konkreten Beweis für das interne Hom der Prägarbenkategorie
auf den offenen Mengen eines topologischen Raums können wir mit einer
Rechnung im Koendenkalkül auf beliebige Prägarbenkategorien ausweiten.
\begin{prop}
  Ist $C$ eine kleine Kategorie, so ist die Prägarbenkategorie
  $\Ens^{C\op}$ kartesisch abgeschlossen.
\end{prop}
\begin{proof}
  Nach der objektweisen Berechnung von Limites in Funktorkategorien
  ist das Prägarbenprodukt gegeben durch $(F \times G)(c) = F(c)
  \times G(c)$ für $F, G \in \Ens^{C\op}$ und $c \in C$. Testen mit
  darstellbaren Prägarben motiviert die Definition
  \[ (F \iHom G)(c) := \Ens^{C\op}(F \times C(\cdot, c), G), \]
  mit vom Nachschalten von $f: c \to d$ induzierten Restriktionen.
  Mit \ref{trans-end} sind Morphismen in $\Ens^{C\op}$ darstellbar als
  Ende
  \[ \Ens^{C\op}(F, G) = \int_c \Ens(F(c), G(c)) \]
  und wir berechnen mit den Regeln des (Ko-) Endenkalküls für $F, G, H
  \in \Ens^{C\op}$:
  \begin{align*}
    \Ens^{C\op}(F, G \iHom H)
    &\iso[Def.] \Ens^{C\op} (F, \Ens^{C\op}(F \times C(\cdot, -), G)) \\
    &\iso[\ref{trans-end}]
     \int_c \Ens \left( F(c), \int_d \Ens(G(d) \times C(d, c), H(d)) \right) \\
    &\iso[\ref{coend-cocont}]
     \int_c \int_d \Ens(F(c), \Ens(G(d) \times C(d, c), H(d))) \\
    &\iso[\ref{coend-fubini}]
     \int_d \int_c \Ens(F(c), \Ens(G(d) \times  C(d, c), H(d))) \\
    &\iso[Adj.] \int_d \int_c \Ens(F(c) \times G(d) \times C(d, c), H(d)) \\
    &\iso[\ref{coend-cocont}]
     \int_d \Ens \left( \int^c F(c) \times G(d) \times C(d, c), H(d) \right) \\
    &\iso[\ref{coend-density}]
     \int_d \Ens( F(d) \times G(d), H(d)) \\
    &\iso[\ref{trans-end}]
     \Ens^{C\op}(F \times G, H).
  \end{align*}
\end{proof}
Die obere Aussage über Prägarben auf topologischen Räumen ergibt sich
daraus durch die Beobachtung, dass $F|_U = F \times \OffX(\cdot, U)$
ist, denn $\OffX$ ist halbgeordnet durch Inklusionen. Wir erhalten
auch die kartesisch abgeschlossene Struktur simplizialer Mengen, der
Prägarbenkategorie auf $\Delta$. Explizit ist für $X, Y \in \s\Ens$:
\[ (X \times Y)_n = X_n \times Y_n \]
und
\[ (X \iHom Y)_n = \s\Ens(X \times \Delta^n, Y). \]

Auch die Rolle von $\Ens$ kann verallgemeinert werden. Wir erhalten:
\begin{prop} \label{presheaf-cart-closed}
  Sei $E$ eine kartesisch abgeschlossene Kategorie und $C$ eine kleine
  Kategorie. Dann ist die Kategorie der Prägarben $E^{C\op}$
  angereichert über $E$ und kartesisch abgeschlossen.
\end{prop}
\begin{proof}
  Sind $F, G \in E^{C\op}$ Prägarben, so erhalten wir die
  angereicherte Struktur durch Übertragung der obigen Formulierung als
  $\Ens$-Ende:
  \[ E^{C\op}(F, G) := \int_c E(F(c), G(c)) \in E, \]
  für $E(\cdot, \cdot)$ das interne Hom in $E$. Damit funktioniert der
  Beweis oben auch für diesen Fall.
\end{proof}

\section[Produkte von Garben über topologischen Räumen]
        {Produkte von Garben über topologischen Räumen
          \sectionmark{Produkte von Garben}}
\sectionmark{Produkte von Garben}

Wir betrachten die Kategorienfaserung $\EnsTop \to \Top$ mit
Morphismen den stetigen Abbildungen zwischen den étalen Räumen über
der stetigen Abbildung in der Basis. Diese Kategorie von Garben über
variabler Basis besitzt endliche Produkte, aber vermutlich kein
internes Hom. Ersteres zeigen wir in diesem Abschnitt, letzteres wird
uns bis zum Ende des Kapitels beschäftigen.

Die Produkte von $\EnsTop$ können algebraisch als Rückzug und Produkt
und topologisch als das Bilden des Produktraums beschrieben
werden. Konkret:
\begin{prop} \label{enstop-prod}
  Seien $F_{1,2} \in \Ens_{/X_{1,2}}$ Garben über topologischen Räumen
  $X_1$ und $X_2$. Dann ist die Garbe
  \[ F_1 \times F_2 := \pr_1^* F_1 \times \pr_2^* F_2 \in \Ens_{/X_1 \times X_2} \]
  mit $\pr_{1,2}: X_1 \times X_2 \to X_{1,2}$ den Projektionen das
  Produkt von $F_1$ und $F_2$ in $\Ens_{/\Top}$. Ihr étaler Raum ist:
  \[ \etalespace{F_1 \times F_2} = \etalespace{F_1} \times \etalespace{F_2}
  \to X_1 \times X_2
  \]
  mit der von $\etalespace{F_{1,2}} \to X_{1,2}$ induzierten
  Produktabbildung.
\end{prop}
\begin{proof}
  Für ein Testobjekt $G \in \Ens_{/Y}$ prüft man leicht die Bijektion
  von Faserprodukten
  \begin{align*}
    \Top(\etalespace{G}, \etalespace{F_1} \times \etalespace{F_2})
    &\times_{\Top(\etalespace{G}, X_1 \times X_2)} \Top(Y, X_1 \times X_2) \\
    \unalign{\iso} \Top(\etalespace{G}, \etalespace{F_1})
    &\times_{\Top(\etalespace{G}, X_1)} \Top(Y, X_1) \\
    \times \Top(\etalespace{G}, \etalespace{F_2})
    &\times_{\Top(\etalespace{G}, X_2)} \Top(Y, X_2),
  \end{align*}
  was die Aussage über den étalen Raum des Produkts zeigt. Die
  Abbildung $\etalespace{F_1} \times \etalespace{F_2} \to X_1 \times
  X_2$ ist étale und konkret ein Homöomorphismus auf der Produktmenge
  der Umgebungen, auf denen $\etalespace{F_{1,2}} \to X_{1,2}$
  Homöomorphismen sind.
  
  Für die algebraische Beschreibung prüft man unter Verwendung der
  Offenheit der Projektionen, dass die Prägarbenrückzüge Garben sind,
  und erhält mit \ref{ensx-cart-closed} für die Schnitte über
  Basismengen $U_1 \times U_2$:
  \[ (F_1 \times F_2)(U_1 \times U_2) \iso F_1(U_1) \times F_2(U_2).
  \]
  Wir erhalten also einen Garbenmorphismus über $X_1 \times X_2$ von
  der algeraischen zur topologischen Beschreibung, indem einem Paar
  $(s, t) \in F_1(U_1) \times F_2( U_2)$ der Schnitt $s \times t: U_1
  \times U_2 \to \etalespace{F_1} \times \etalespace{F_2}$ zugeordnet
  wird. Dieser Morphismus induziert auf den Halmen die Bijektion
  \[ (F_1)_x \times (F_2)_y \iso (F_1 \times F_2)_{x, y} \]
  aus dem Vertauschen von endlichen Produkten mit filtrierenden
  Kolimites.
\end{proof}
\begin{bem}
  Auf ähnliche Weise kann man auch für $\Ens_{\sslash \Top}$ endliche
  Produkte konstruieren: es handelt sich (wegen der opponierten
  Fasern) um das \emph{Ko}produkt der mit den Projektionen auf den
  Produktraum zurückgezogenen Garben.
\end{bem}
Weiter ist $\EnsTop$ auch unter Pullbacks abgeschlossen.
\begin{satz} \label{enstop-fin-complete}
  Die Kategorie der Garben auf topologischen Räumen mit Morphismen
  $\Ens_{/\Top}$ besitzt endliche Limites.
\end{satz}
\begin{proof}
  Es reicht mit \ref{enstop-prod} die Existenz von Egalisatoren zu
  zeigen. Seien dazu $(F \to X) \rightrightarrows (G \to Y)$ zwei
  Morphismen. Wir zeigen, dass der Egalisator $E \to W$ aus
  $\Top^{[1]}$ eine Garbe ist. Sei $f$ die (übereinstimmende)
  Verknüpfung $W \to X \rightrightarrows Y$. Dann ist $E$ auch ein
  Egalisator von Garben über $W \subset X$: $E \to F|_W
  \rightrightarrows f^* G$ und somit eine Garbe.
\end{proof}
\begin{bem}
  Die verbleibenden Fragen zur Vollständigkeit und Kovollständigkeit
  von $\EnsTop$ werden in \ref{enstop-not-complete} und
  \ref{enstop-coequalizers} negativ beantwortet.
\end{bem}

\section[Kartesisch abgeschlossene Kategorien topologischer Räume]
        {Kartesisch abgeschlossene Kategorien topologischer Räume
        \sectionmark{Kartesisch abgeschlossene Kategorien}}
\sectionmark{Kartesisch abgeschlossene Kategorien}

Wir könnten erwarten, dass wie das Produkt auch das interne Hom von
$\EnsX$ in die Situation variabler Basisräume übertragen werden
kann. Dies gelingt aber im Allgemeinen nicht, denn in diesem Fall
erhielten wir durch Einschränken auf die Basis ein zum kartesischen
Produkt adjungiertes internes Hom in der Kategorie der topologischen
Räume (\ref{full-enstop-not-cart-closed}), welches bekanntermaßen
nicht existiert (\cite{Borceux}, Prop. 7.1.2). Wir müssen uns also
wieder auf eine bequeme Kategorie topologischer Räume mit internem Hom
einschränken.

Die häufige Wahl $\CGHaus$ ist für uns ungeeignet, denn der étale Raum
einer Garbe über einem kompakt erzeugten Hausdorffraum ist im
Allgemeinen kein Hausdorffraum mehr (betrachte etwa die Garbe der
stetigen Funktionen nach $\R$). Abhilfe schafft die Konstruktion
gewisser koreflektiver Hüllen aus \cite{Vogt}, die die den kompakt
erzeugten Räumen zugrundeliegenden Gedanken verallgemeinert. Wir geben
hier nur die Ergebnisse an.

Äquivalent zur (der \emph{point-set-}Topologie entspringenden)
Definition kompakt erzeugter Räume ist die folgende Charakterisierung:
\begin{lemma}[\ref{def:cg}, Variante]
  Ein topologischer Raum $X$ ist kompakt erzeugt genau dann, wenn er
  die Finaltopologie bezüglich des Systems aller Abbildungen $K \to X$
  von kompakten Räumen $K$ nach $X$ trägt.
\end{lemma}
\begin{proof}
  Die Bedingung aus der ursprünglichen Definition ist dieselbe für das
  System der Inklusionen kompakter Mengen $K \subset X$. Da jede
  stetige Abbilung $K \to X$ mit $K$ kompakt über die Inklusion ihres
  kompakten Bilds faktorisiert, ist letzteres System in ersterem
  konfinal und die Finaltopologien stimmen überein.
\end{proof}
Der in \ref{real-products} angesprochene zur Inklusion
Rechtsadjungierte $k: \Top \to \CG$ lässt sich nun auch beschreiben
als das Versehen der $X$ zugrundeliegenden Menge mit der genannten
Finaltopologie. Der Raum $kX$ ist dann sogar ein Kolimes über das
System der $K \to X$, $K$ kompakt, mitsamt den Morphismen über $X$
(\cite{Vogt}, 1.1).

Nun verallgemeinern wir (\cite{Vogt}, 1): Sei $\mathcal{I}$ eine
nichtleere volle Unterkategorie von $\Top$ (für $\CG$ die kompakten
Räume). Betrachte die Kategorie $\slicecat{\mathcal{I}}{}{X}$ und $kX
:= \col_{\slicecat{\mathcal{I}}{}{X}} X$. Bezeichne die volle
Unterkategorie der topologischen Räume $X$ mit $kX \cong X$ mit
$\mathcal{K}$. Dann ist $k: \Top \to \mathcal{K}$ ein Funktor und
rechtsadjungiert zur Inklusion $\mathcal{K} \to \Top$. Es gilt
$\mathcal{I} \subset \mathcal{K}$.
\begin{bem} \label{corefl}
  Dual zu \ref{def:refl-sub} heißt eine volle Unterkategorie mit zur
  Inklusion Rechtsadjungiertem \emph{koreflektiv}, der
  Rechtsadjungierte heißt \emph{Koreflektor}. Die Konstruktion, die
  zur vollen Unterkategorie $\mathcal{I} \subset \Top$ eine
  koreflektive Unterkategorie $\mathcal{K} \subset \Top$ liefert,
  welche $\mathcal{I}$ umfasst, heißt auch Übergang zur
  \emph{koreflektiven Hülle}. Es handelt sich tatsächlich um eine
  idempotente Operation (\cite{Vogt}, Prop. 1.5), die Ergänzung von
  $\mathcal{I}$ um alle Kolimites von Objekten in $\mathcal{I}$
  (\cite{Herrlich}, 37).
\end{bem}
Die koreflektive Hülle besitzt die folgenden Stabilitätseigenschaften:
\begin{prop} \label{k-complete}
  Die koreflektive Hülle $\mathcal{K}$ ist vollständig und
  kovollständig. Die Kolimites stimmen mit den Kolimites aus $\Top$
  überein, die Limites entstehen durch Anwendung des Koreflektors $k$
  auf den Limes in $\Top$.
\end{prop}
Insbesondere ist $\mathcal{K}$ also stabil unter disjunkten Summen und
Quotientenbildung.
\begin{proof}
  Das ist die duale Aussage zu \ref{refl-sub-complete}. Die
  Vollständigkeit und Kovollständigkeit von $\Top$ durch Versehen der
  mengentheoretischen Limites bzw. Kolimites mit der Initial-
  bzw. Finaltopologie ist bekannt.
\end{proof}

Im allgemeinen kann man keine Aussage darüber treffen, ob mit der
Relativtopologie versehene Unterräume von Objekten in $\mathcal{K}$
wieder zu $\mathcal{K}$ gehören. Wir benötigen die folgende
Eigenschaft:
\begin{enumerate}
\item \label{itm:k-axiom-subspace} Ist $U \open X$ ein offener
  Unterraum eines Objekts $X \in \mathcal{I}$ versehen mit der
  Relativtopologie, so gilt $U \in \mathcal{K}$.
\end{enumerate}
In diesem Fall gilt bereits für Objekte $X \in \mathcal{K}$, dass
offene Unterräume $U \open X$ wieder Objekte von $\mathcal{K}$
sind. Dieselbe Aussage gilt, wenn man ``offen'' zweimal durch
``abgeschlossen'' ersetzt (\cite{Vogt}, Prop. 2.4).

Wir nehmen nun an, dass $\mathcal{I}$ die folgenden Axiome erfüllt
(\cite{Vogt}, Axiom 2):
\begin{enumerate}
  \setcounter{enumi}{1}
  \item \label{itm:k-axiom-prod} $\mathcal{I}$ ist abgeschlossen unter
    endlichen kartesischen Produkten (Produkten in $\Top$).
  \item \label{itm:k-axiom-ev} Sind $X, Y \in \mathcal{I}$, so ist die
    Auswertungsabbildung
    \begin{align*}
      \ev_{X, Y}: \Top_{co}(X, Y) \times X &\to Y, \\
      (f, x) &\mapsto f(x)
    \end{align*}
    stetig. Dabei ist $\Top_{co}(X, Y)$ die Morphismenmenge $\Top(X,
    Y)$ versehen mit der kompakt-offen Topologie.
\end{enumerate}
Dann besitzt $\mathcal{K}$ die Struktur einer kartesisch
abgeschlossenen Kategorie mit Produkten
\[ X \otimes Y \: := \: k(X \times Y) \]
den ``k-ifizierungen'' der Produkte in $\Top$ und internem Hom
\[ X \iHom Y \: := \: k(\Top_{co}(X, Y)) \] 
(\cite{Vogt}, 3).

\begin{defn} \label{def:loc-compact}
  Ein topologischer Raum heißt \emph{lokalkompakt} (im starken Sinne),
  wenn jeder Punkt eine Umgebungsbasis aus kompakten Mengen besitzt.
\end{defn}
\begin{bem} \label{loc-compact-usage}
  Dies ist eine stärkere Bedingung als \emph{lokal kompakt} (im
  schwachen Sinne) wie in \ref{cg-crit} zu sein. Jene stimmt überein
  mit unserer Konvention für ``lokal Eigenschaft'' und wird daher
  getrennt geschrieben. Für Hausdorffräume stimmen beide Begriffe
  überein.
\end{bem}
\begin{prop}[\cite{Vogt}, 5]
  Die folgenden vollen Unterkategorien der Kategorie der topologischen
  Räume erfüllen die Axiome \ref{itm:k-axiom-subspace} -
  \ref{itm:k-axiom-ev}.
  \begin{enumerate}[label=(\roman*)]
  \item die Kategorie der kompakten Hausdorffräume $\mathcal{I}_K$,
  \item die Kategorie der lokalkompakten topologischen Räume
    $\mathcal{I}_L$.
  \end{enumerate}
\end{prop}
Für das Axiom \ref{itm:k-axiom-subspace} weisen wir das nach. Da es
sich um eine lokale Eigenschaft handelt, gilt die Aussage im Fall der
lokalkompakten Räume sofort. Für die kompakten Hausdorffräume bemerkt
man, dass nach dem folgenden Lemma eine offene Teilmenge eines
kompakten Hausdorffraums lokalkompakt ist und lokalkompakte
Hausdorffräume mit den kompakt erzeugten Hausdorffräumen allgemein
(vgl. \ref{cg-crit} \ref{itm:cg-crit-lc}) in der koreflektiven Hülle
der kompakten Hausdorffräume enthalten sind: in der Tat ist für diese
das System der Inklusionen kompakter Teilmengen konfinal im System der
von kompakten Hausdorffräumen ausgehenden stetigen Abbildungen, da das
Bild von Kompakta unter stetigen Abbildungen kompakt ist. Die
Bedingung, kompakt erzeugt zu sein, bedeutet aber gerade, die
Finaltopologie bezüglich dieser Inklusionen zu tragen.
\begin{lemma}
  Sei $K$ ein kompakter Hausdorffraum und $U \open K$ eine offene
  Teilmenge. Dann ist $U$ mit der induzierten Topologie lokalkompakt.
\end{lemma}
\begin{proof}
  Sei $V \open U$ eine offene Umgebung eines Punktes $x \in U$. Der
  Rand $\del V$ ist als abgeschlossene Teilmenge eines kompakten
  Hausdorffraums kompakt und kann somit durch endlich viele offene
  Mengen überdeckt werden, die disjunkt zu einer offenen Umgebung
  $W_0$ von $x$ sind. Bezeichne die Vereinigung dieser Mengen mit
  $W$. Wegen $W \supset \del V$ ist $V \setminus W = \overline{V}
  \setminus W$ abgeschlossen und somit eine kompakte Umgebung von $x$,
  die die offene Umgebung $W_0$ von $x$ enthält.
\end{proof}
Auch Axiom \ref{itm:k-axiom-prod} sieht man direkt: ein Produkt von
Hausdorffräumen ist bekanntermaßen wieder Hausdorffsch und ein Produkt
kompakter Räume wieder kompakt. Mit dieser Aussage finden wir auch bei
einem Produkt lokalkompakter Räume Umgebungsbasen aus Kompakta durch
die Umgebungsbasen aus Produktmengen.

Für das Axiom \ref{itm:k-axiom-ev} verweisen wir auf die Literatur,
siehe etwa \cite{TM}, 1.12.12.

\begin{kor}
  Die koreflektiven Hüllen von $\mathcal{I}_K$ und $\mathcal{I}_L$
  sind kartesisch abgeschlossen und enthalten mit jedem Objekt $X$
  auch alle offenen Unterräume $Y \open X$.
\end{kor}

Damit können wir die für uns entscheidende Eigenschaft zeigen:
\begin{prop} \label{k-etale-closed}
  Ist $X \in \mathcal{K}$ für $\mathcal{K}$ die koreflektive Hülle von
  $\mathcal{I}_K$ bzw. $\mathcal{I}_L$ und $F \to X$ eine étale
  Abbildung, so ist auch $F \in \mathcal{K}$.
\end{prop}
\begin{proof}
  Wir können den étalen Raum $F \to X$ als Kolimes mittels der
  Schnitte $F(U)$ über offene Mengen $U \open X$ darstellen:
  \[ F \iso \bigg( \coprod_{U \open X} F(U) \times U \bigg) \big/ \sim. \]
  Dabei läuft das Koprodukt über alle offenen Teilmengen von $X$ und
  ist die Äquivalenzrelation die Identifikation gleicher Keime, d. h.
  \[ (s, p) \sim (t, q) \Iff p = q \text{ und } s_p = t_p. \]
  Die étale Abbildung $F \to X$ ist dann von der Projektion auf die
  zweiten Faktoren induziert und wohldefiniert. Man erkennt leicht den
  Isomorphismus als die Koeinheit der Adjunktion $(\et, S)$ aus
  \cite{TG}, 2.1.24, eingeschränkt auf die Kategorie der étalen Räume
  über $X$.

  Nach den Stabilitätseigenschaften von $\mathcal{K}$ sind die offenen
  Teilmengen $U \subset X$ Objekte von $\mathcal{K}$ und dann auch der
  Kolimes $F$ bestehend aus Koprodukt und Koegalisator. Man beachte,
  dass es sich bei $F(U) \times U$ mit der diskreten Topologie auf
  $F(U)$ formal um das Koprodukt $\coprod_{F(U)} U$ handelt.
\end{proof}
\begin{bem}
  Der Beweis wiederholt bei genauerer Betrachtung die Aussage, dass
  jede Prägarbe auf $X$ ein Kolimes über darstellbare Prägarben
  $\OffX(\cdot, U)$ ist (\ref{presheaf-colimit-representable}).
\end{bem}

\section{Kartesischer Abschluss der Garben auf topologischen Räumen}
\sectionmark{Kartesischer Abschluss}
\label{sec:enstop-cart-closed}

Wir können uns nun der Frage nach einer kartesisch abgeschlossenen
Struktur auf $\EnsTop$ zuwenden. Ganz allgemein gilt:
\begin{prop} \label{coreflective-cart-closed}
  Ist $C \subset D$ eine koreflektive Unterkategorie einer kartesisch
  abgeschlossenen Kategorie $C$ und stimmen die Produkte in $C$ und
  $D$ überein, dann ist $D$ kartesisch abgeschlossen.
\end{prop}
\begin{proof}
  In diesem Fall ist die Koreflektion $(X \iHom Y)^+$ des internen Hom
  in $D$ ein internes Hom in $C$.
\end{proof}

Damit können wir folgern:
\begin{prop} \label{full-enstop-not-cart-closed}
  Die Kategorie der Garben über topologischen Räumen mit Morphismen
  $\EnsTop$ ist nicht kartesisch abgeschlossen.
\end{prop}
\begin{proof}
  Betrachte die volle Unterkategorie $\Top \subset \EnsTop$ gegeben
  durch die Einbettung $X \mapsto (\varnothing \to X)$. Diese ist
  koreflektiv mit dem Koreflektor $(G \to Y) \mapsto (\varnothing \to
  Y)$. In der Tat gilt:
  \[ \EnsTop((\varnothing \to X), (G \to Y)) \iso \Top(X, Y)
  \iso \EnsTop((\varnothing \to X), (\varnothing \to Y)).
  \]
  Die Produkte in $\EnsTop$ und $\Top \subset \EnsTop$ stimmen nach
  \ref{enstop-prod} überein. Somit müsste laut
  \ref{coreflective-cart-closed} $\Top$ kartesisch abgeschlossen sein,
  ein Widerspruch zu der bekannten Aussage, dass dies für $\Top$ nicht
  möglich ist (\cite{Borceux}, Prop. 7.1.2).
\end{proof}

Da das einzige Problem die fehlende kartesisch abgeschlossene Struktur
in der Basis war, schränken wir auf eine bequemere Kategorie $\K$
ein. In diesem Abschnitt werden wir einen einschränkenden Grund für
die Existenz eines internen Homs in der Kategorie $\Ens_{\K}$
formulieren, für $\K$ im gesamten Abschnitt eine kartesisch
abgeschlossene, koreflektive Kategorie topologischer Räume, die stabil
ist unter dem Bilden offener Unterräume. Diese Bedingungen sind nicht
sehr restriktiv. Die Koreflektivität ist nach \cite{Herrlich},
Thm. 37.3, äquivalent dazu, dass $\K$ unter Kolimites in $\Top$
abgeschlossen ist. Die Stabilität unter offenen Teilräumen ermöglicht
uns, sinnvoll über Garben über $\K$-Räumen zu sprechen
(vgl. \ref{k-etale-closed}).

Zunächst betrachten wir den Fall von Paaren von $\K$-Räumen mit
stetiger, aber nicht notwendigerweise étaler, Abbildung.
\begin{lemma} \label{k-arrow-cart-closed}
  Sei $\K \subset \Top$ eine koreflektive, kartesisch abgeschlossene
  Kategorie topologischer Räume. Dann ist $\K^{[1]}$ kartesisch
  abgeschlossen.
\end{lemma}
\begin{proof}
  Es handelt sich um eine Prägarbenkategorie auf $[1]\op$ mit Werten in
  einer kartesisch abgeschlossenen Kategorie. Die Aussage folgt somit
  aus \ref{presheaf-cart-closed}. Expliziter ist das interne Hom von
  $F \to X$ mit $G \to Y$ das Paar
  \[ \K(F, G) \times_{\K(F, Y)} \K(X, Y) \to \K(X, Y),
  \]
  das die Menge der kommutativen Quadrate mit einer Topologie
  ausstattet, mit der Projektion auf den zweiten Faktor als
  Abbildung. Die Adjunktion
  \[ \K^{[1]}((F \to X) \times (G \to Y), (H \to Z))
  \iso \K^{[1]}((F \to X), (G \to Y) \iHom (H \to Z))
  \]
  für $(F \to X), (G \to Y), (H \to Z) \in \K^{[1]}$ ist dann die
  Bijektion von Faserprodukten
  \begin{align*}
    &\K\big( F \times G, H\big) \times_{\K(F \times G, Z)} \K\big( X \times Y, Z\big) \\
    \iso \quad &\K\big( F, \K(G, H) \times_{\K(G, Z)} \K(Y, Z)\big)
    \times_{\K(F, \K(Y, Z))} \K\big(X, \K(Y, Z)\big).
  \end{align*}
\end{proof}
Im allgemeinen ist das interne Hom in $\K^{[1]}$ für $(F \to X)$ und
$(G \to Y)$ mit étalen Abbildungen nicht wieder étale. Wir können
versuchen, es zu ``étalisieren'':
\begin{lemma}[\cite{TG}, 2.1.40]
  Für $X$ einen topologischen Raum ist die volle Unterkategorie
  $\etTop_X \inj \Top_X$ koreflektiv. Der zur Inklusion
  Rechtsadjungierte heißt \emph{Étalisierung}.
\end{lemma}
\begin{proof} (\cite{TG}, 2.1.40)
  Wir erhalten die Étalisierung als die Verknüpfung $\et \circ S$ für
  $S$ den Funktor der Schnittgarbe und $\et$ den Funktor des étalen
  Raums einer Garbe. Es handelt sich um die Verknüpfung von
  Adjunktionen
  \[(\et, S) \circ (S, \et) = (\et \circ S, \et \circ S):
  \Top_X \rightleftarrows \EnsX \rightleftarrows \etTop_X,
  \]
  wobei letztere Adjunktion die bekannte Äquivalenz von Kategorien
  ist.
\end{proof}
Ist $\K \subset \Top$ nun eine Kategorie topologischer Räume, die mit
jedem Raum $X$ auch jeden étalen Raum über $X$ enthält (etwa wie in
\ref{k-etale-closed}), so können wir die Kategorie der étalen Räume
über $\K$ als volle Unterkategorie von $\K^{[1]}$ auffassen:
\[ \et \K^{[1]} := \Ens_{/\K} \subset \K^{[1]}. \]
Gäbe es nun einen Koreflektor $^+: \K^{[1]} \to \et \K^{[1]}$, so
hätten wir für Objekte $F, G, H \in \et \K^{[1]}$ mit
\ref{coreflective-cart-closed}
\[ \et \K^{[1]}(F \times G, H)
  \iso \et \K^{[1]}(F, (G \iHom H)^+),
\]
mit dem internen Hom $(G \iHom H)$ aus \ref{k-arrow-cart-closed}. Die
Produkte in $\K^{[1]}$ und $\et \K^{[1]}$ stimmen dabei nach dem
folgenden Lemma überein. Umgekehrt ist, falls ein internes Hom $(G
\iHom H)^+$ in $\et \K^{[1]}$ existiert, die Zuordnung
\[ (G \iHom H) \mapsto (G \iHom H)^+ \]
ein partiell definierter Koreflektor. Unsere Aufgabe ist es nun, zu
zeigen, dass eine solche relative Form der Étalisierung nicht möglich
ist.
\begin{lemma}
  Ist $k: \Top \to \K$ ein Koreflektor und $p: F \to X$ étale in
  $\Top$, so ist $kp: kF \to kX$ étale.
\end{lemma}
\begin{proof}
  Dass $p$ étale ist, bedeutet, dass es für jedes $x \in F$ ein
  kommutatives Quadrat
  \[ 
  \begin{tikzcd}
    U \dar{\sim} \rar[hook] & F \dar{p} \\
    p(U) \rar[hook] & X
  \end{tikzcd}
  \]
  gibt mit $U \open F$ und $p(U) \open X$. Anwendung von $k$ ergibt
  das entsprechende Diagramm, in welchem $kU \subset kF$ und $k(p(U))
  \subset kX$ offene Teilmengen sind, da $\mathcal{K}$ mit jedem Raum
  auch seine offenen Unterräume enthält. Da $k$ ein Funktor ist,
  kommutiert das Quadrat und $kU \iso k(p(U)) = (kp)(kU)$ ist ein
  Homöomorphismus.
\end{proof}
Das Übereinstimmen der Produkte in $\K^{[1]}$ und $\et \K^{[1]}$ ergibt sich
nun daraus, dass für $(F \to X), (G \to Y) \in \et \K^{[1]}$ zunächst das
Produkt
\[ F \times G \to X \times Y \]
in der Kategorie der topologischen Räume étale ist, und dann nach dem
Lemma auch das Produkt
\[ k(F \times G) \to k(X \times Y) \]
in $\K^{[1]}$ étale und somit in $\et \K^{[1]}$ ist.

\begin{satz} \label{enstop-missing-coreflector}
  Sei $\K \subset \Top$ eine volle Unterkategorie. Dann ist der
  partiell definierte Koreflektor $^+: \K^{[1]} \to \et \K^{[1]}$ nur
  auf $\et \K^{[1]}$ definiert.
\end{satz}
\begin{proof}
  Dies liegt daran, dass die Testobjekte, mit denen wir Objekte in
  $\K^{[1]}$ eindeutig festlegen können, bereits étale sind. Betrachte
  dazu die Einbettungen $\iota, \tau: \K \to \et \K^{[1]}$ durch über
  dem betreffenden Raum initiale bzw. terminale Garben:
  \begin{align*}
    \iota X := (\varnothing \to X), \\
    \tau X := (X \xrightarrow{\id} X).
  \end{align*}
  Gelte also
  \begin{equation} \label{eq:coreflection-adjunction}
    \K^{[1]}((F \to X), (G \to Y)) \iso \et \K^{[1]}((F \to X), (G \to Y)^+)
  \end{equation}
  für alle étalen $F \to X$. Ein Objekt $(H \to Z) \in \K^{[1]}$ ist
  nach dem Yoneda-Lemma (bis auf eindeutigen Isomorphismus) eindeutig
  festgelegt durch die Funktoren
  \[ \K^{[1]}(\iota \cdot, (H \to Z)) = \K(\cdot, Z) \qquad \text{und} \\
  \K^{[1]}(\tau \cdot, (H \to Z)) = \K(\cdot, H)
  \]
  und ihre aus $\iota \cdot \Trafo \tau \cdot$ entstehende
  Transformation, die dem Morphismus $H \to Z$ entspricht. Nun sind
  $\iota X, \tau X \in \et \K^{[1]}$ für alle $X \in \K$ und es folgt,
  dass im Fall von Gl. \ref{eq:coreflection-adjunction} $(G \to Y)$
  und $(G \to Y)^+$ isomorph sind und die Koreflektion genau dann
  definiert ist, wenn $(G \to Y)$ sowieso schon étale ist.
\end{proof}
\begin{kor}
  Es gibt ein internes Hom für $F \to X$ und $G \to Y$ in $\et
  \K^{[1]}$ genau dann, wenn die Projektion auf den zweiten Faktor
  \[ \K(F, G) \times_{\K(F, Y)} \K(X, Y) \to \K(X, Y) \]
  étale ist.
\end{kor}
\begin{proof}
  Dies folgt aus \ref{k-arrow-cart-closed} und
  \ref{enstop-missing-coreflector}.
\end{proof}
\begin{bem}
  Wenn $\K(F, G)$ die kompakt-offen Topologie trägt, erhielten wir
  folgendes Gegenbeispiel zum kartesischen Abschluss von $\et
  \K^{[1]}$: Für $X$ und $Y$ den einpunkzigen Raum sind étale Räume $F
  \to X$ und $G \to Y$ diskret. Das interne Hom von $F \iHom G$ lautet
  \[ \K(F, G) \to \K(X, Y) = \point \]
  und muss folglich selbst diskret sein. Dies ist im Fall, dass $F$
  unendlich ist, ein Widerspruch zum folgenden Lemma. Im Fall der
  koreflektiven Hüllen von $\K = \mathcal{I}_K$ oder $\K =
  \mathcal{I}_L$ aus dem vorangegangenen Abschnitt gilt allerdings
  \[ \K(F, G) = k(\Top_{co}(F, G)) . \]
  Die Frage ist, ob der Koreflektor die Topologien so sehr verfeinern
  kann, dass $\K(F, G)$ auch bei unendlichem $F$ diskret wird.
\end{bem}
\begin{lemma} \label{compact-open-discrete}
  Ist $G$ diskret, so ist die kompakt-offen Topologie auf $\Top(F,
  G)$, $F \in \Top$ genau dann diskret, wenn $X$ endlich viele
  Zusammenhangskomponenten hat.
\end{lemma}
\begin{proof}
  Die kompakt-offen Topologie ist genau dann diskret, wenn jede
  stetige Abbildung $f: F \to G$ durch endlich viele Aussagen der Form
  $f(K) \subset U$ für $K \subset F$ kompakt und $U \open G$ eindeutig
  festgelegt ist. Ein Kompaktum $K \subset F$ trifft nur endlich viele
  Zusammenhangskomponenten von $F$. Daher kann $f$ so nicht eindeutig
  festgelegt werden, wenn $F$ unendlich viele Zusammenhangskomponenten
  hat. Umgekehrt ist $f$ als stetige Funktion in einen diskreten Raum
  konstant auf Zusammenhangskomponenten und kann durch die Angabe des
  Funktionswerts (Punkte in $G$ sind offen) eines Punktes in jeder
  Zusammenhangskomponente eindeutig festgelegt werden.
\end{proof}

\section[Vollständigkeit der Garben auf topologischen Räumen]
        {Vollständigkeit der Garben auf topologischen Räumen
        \sectionmark{Vollständigkeit}}
        \sectionmark{Vollständigkeit}
        \label{sec:enstop-completeness}

Mit denselben Techniken können wir auch zeigen, dass die Kategorie
$\EnsTop$ der Garben auf topologischen Räumen über die endlichen
Limites aus \ref{enstop-fin-complete} und die trivialen Koprodukte
hinaus weder vollständig noch kovollständig ist.

\begin{prop} \label{enstop-not-complete}
  Die Kategorie $\EnsTop$ der Garben über topologischen Räumen besitzt
  nicht alle unendlichen Produkte.
\end{prop}
\begin{proof}
  Dies sieht man schon am Beispiel eines unendlichen Produkts von
  Inklusionen $\iota: U \inj X$ einer echten offenen Teilmenge $U
  \subsetneq X$. Die Abbildung
  \[ \prod_{\N} \iota: \prod_{\N} U \to \prod_{\N} X \]
  ist nicht étale, denn es gibt keine Schnitte von $\prod_{\N} \iota$
  über offene Teilmengen $V \open \prod_{\N} X$, weil eine offene
  Menge in der Basis der Produkttopologie von $\prod_{\N} X$ in fast
  allen Faktoren $X$ die Projektion $X$ hat. Die Abbildung ist also
  nicht étale, was aber wegen des Fehlens eines Koreflektors
  (\ref{enstop-missing-coreflector}) für die Existenz von unendlichen
  Produkten nötig wäre.
\end{proof}
\begin{kor} \label{enstop-not-reflective}
  Die volle Unterkategorie $\EnsTop \subset \Top^{[1]}$ ist nicht
  reflektiv.
\end{kor}
\begin{proof}
  Gäbe es einen zur Inklusion Linksadjungierten, so wäre die Inklusion
  linksexakt, würde also Limites erhalten.
\end{proof}
Wir können sogar angeben, wann es den Reflektor gibt. Wir erinnern
daran, dass wir einen Punkt eines topologischen Raums über $X$
\emph{étale} nennen, wenn er eine Umgebung für die lokale
Homöomorphismus-Eigenschaft einer Garbe über $X$ besitzt.
\begin{prop}
  Der partiell definierte Reflektor $\Top^{[1]} \to \EnsTop$ ist auf
  $F \in \Top_X$ genau dann definiert, wenn jeder nichtétale Punkt
  $\sigma \in F$ eine kleinste offene Umgebung $U(\sigma)$ besitzt.
\end{prop}
\begin{proof}
  Besitzt jeder nichtétale Punkt in $F$ eine kleinste offene Umgebung,
  so ergänzt der Reflektor in $F$ einen nichtétalen Punkt $\sigma$
  lokal zu einer Kopie von $U(\sigma)$. Ein Morphismus von $F$ in eine
  Garbe $G \in \Ens_{/Y}$ schickt $\sigma$ auf einen Keim $f(\sigma)$
  eines Schnitts in $G(V)$ für eine offene Umgebung $V$ des
  Basispunkts von $f(\sigma)$, deren Urbild $f^{-1}(V) \supset
  U(\sigma)$ die kleinste offene Umgebung von $\sigma$ enthält. Da
  Übereinstimmungsmengen von Schnitten offen sind, enthält deren
  Urbild mit $\sigma$ auch $U(\sigma)$ und es lässt sich somit ein
  Morphismus $F \to G$ eindeutig über den Reflektor $F \to F^+$
  faktorisieren.
  
  Sei umgekehrt der Reflektor $^+$ definiert auf einem Objekt $F \to
  X$. Durch Testen mit Zielobjekten $\tau Y$, $Y \in \Top$, ist nach
  dem Yoneda-Lemma die Basis von $(F \to X)^+$ wieder (bis auf
  eindeutigen Homöomorphismus) $X$ und die Einheit der Adjunktion
  $\kappa: F \to F^+$ liegt über der Identität auf $X$. Wegen der
  eindeutigen Faktorisierung über $\kappa$ müssen sich nicht nur die
  gesamten Mengen von kommutativen Quadraten von $F \to X$ bzw. $(F
  \to X)^+$ in eine Garbe entsprechen, sondern sogar die Morphismen
  über jeder Abbildung $f$ in der Basis. Sei $\sigma \in F$ nun ein
  nichtétaler Punkt, $U \open X$ eine étale Umgebung von
  $\kappa(\sigma)$ und $V \subsetneq U$ eine echt kleinere offene
  Umgebung von $\kappa(\sigma)$. Wir wählen als Testraum die Garbe $G$
  über dem Sierpinski-Raum mit zweielementigem Halm über dem offenen
  Punkt und genau einem globalen Schnitt und als Abbildung in der
  Basis die Abbildung, für die das Urbild des offenen Punktes $V$
  ist. Nun gibt es Morphismen $F \to G$, die $\sigma$ auf den
  nichtfortsetzbaren Schnitt über dem offenen Punkt schicken, aber
  keine Morphismen $F^+ \to G$, die $\kappa(\sigma)$ auf den
  nichtfortsetzbaren Schnitt über dem offenen Punkt schicken, ein
  Widerspruch zur eindeutigen Faktorisierung über $\kappa$.
\end{proof}
\begin{kor} \label{enstop-coequalizers}
  Ein Kolimes in $\EnsTop$ existiert genau dann, wenn alle nicht\-étalen
  Punkte des Kolimes in $\Top^{[1]}$ eine kleinste offene Umgebung
  haben.
\end{kor}
\begin{bsp}
  Betrachte den Basisraum $X = [0, 1] \sqcup \point$ und die über $[0,
    1]$ konstant einelementige und über $\point$ zweielementige Garbe
  $F \in \EnsX$. Durch Identifikation (d. h. einen Koegalisator mit
  der konstant einelementigen Garbe über dem Punkt) des Punkts in
  $F_0$ mit einem der Punkte aus $F_{\point}$ und der zugehörigen
  Identifaktion $0 \sim \point$ entsteht ein topologischer Raum über
  dem Quotienten $[0, 1]$ der Basis mit einem globalen Schnitt und
  einem weiteren, nichtétalen Punkt über $0$.
\end{bsp}

