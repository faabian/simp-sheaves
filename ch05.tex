% Emacs mode: -*-latex-*-
% include latex header (\usepackage, \newcommand etc.) 
\documentclass[a4paper]{article}
% \usepackage[left=3cm,right=3cm,top=3cm,bottom=2cm]{geometry} % page settings
\usepackage{amsmath}
\usepackage{amssymb}
\usepackage{amsthm}
\usepackage{etoolbox}
\usepackage[ngerman]{babel}
\usepackage[utf8]{inputenc}
\usepackage{mathtools}
\usepackage{tikz-cd}
\usepackage{enumitem}
\usepackage{hyperref}

\setlength{\parskip}{\medskipamount}
\setlength{\parindent}{0pt}

\theoremstyle{plain}
\newtheorem{theorem}{Theorem}
\newtheorem{lemma}[theorem]{Lemma}
\newtheorem{prop}[theorem]{Proposition}
\newtheorem{kor}[theorem]{Korollar}
\newtheorem{satz}[theorem]{Satz}
%% \providecommand*{\lemmaautorefname}{Lemma}
%% \providecommand*{\propautorefname}{Prop.}
%% \providecommand*{\korautorefname}{Korollar}
%% \providecommand*{\satzautorefname}{Satz}

\theoremstyle{definition}
\newtheorem{defn}[theorem]{Definition}

\theoremstyle{remark}
\newtheorem{bem}[theorem]{Bemerkung}

\DeclareMathOperator{\Cat}{Cat}
\DeclareMathOperator{\poset}{poset}
\DeclareMathOperator{\EnsX}{Ens_{/X}}
\DeclareMathOperator{\pEnsX}{pEns_{/X}}
\DeclareMathOperator{\AbX}{Ab_{/X}}
\DeclareMathOperator{\pAbX}{pAb_{/X}}
\DeclareMathOperator{\OffX}{Off_X}
\DeclareMathOperator{\Ens}{Ens}
\DeclareMathOperator{\Ob}{Ob}
\DeclareMathOperator{\Der}{Der}
\DeclareMathOperator{\Ab}{Ab}
\DeclareMathOperator{\sKons}{s-Kons}
\DeclareMathOperator{\Ket}{Ket}
\DeclareMathOperator{\EnsB}{Ens_{/\B}}
\DeclareMathOperator{\im}{im}
\DeclareMathOperator{\Id}{Id}
\DeclareMathOperator{\id}{id}
\DeclareMathOperator{\colf}{colf}
\DeclareMathOperator{\limf}{limf}
\DeclareMathOperator{\Top}{Top}

\newcommand{\etalespace}[1]{\overline{#1}}
\newcommand{\B}{\mathcal{B}}
\newcommand{\op}{^\mathrm{op}}
\newcommand{\iso}{\xrightarrow{\sim}}
\newcommand{\qiso}{\xrightarrow{\approx}}
\newcommand{\fromqiso}{\xleftarrow{\approx}}
\newcommand{\open}{\subset\kern-0.58em\circ}  % only possible in math mode
\newcommand{\K}{\mathcal{K}}
\newcommand{\Z}{\mathbb{Z}}
\newcommand{\R}{\mathbb{R}}
\newcommand{\DerAbK}{\Der(\Ab_{/|\K|})}
\newcommand{\DerskK}{\Der_{\mathrm{sk}}(|\K|)}
\newcommand{\DerpskK}{\Der^+_{\mathrm{sk}}(|\K|)}
\newcommand{\AbKr}{\Ab_{/|\K|}}
\newcommand{\sKonsK}{\sKons(\K)}
\newcommand{\inj}{\hookrightarrow}
\newcommand{\surj}{\twoheadrightarrow}
\newcommand{\Iff}{\Leftrightarrow}
\newcommand{\Implies}{\Rightarrow}
\newcommand{\cc}{^{\bullet}}  % chain complex
\newcommand{\from}{\leftarrow}


\begin{document}

\title{Simpliziale Garben}
\author{Fabian Glöckle}
\date{\today}
% \maketitle

\section{Realisierung simplizialer Garben}

Die Beschreibung der geometrischen Realsierung als Tensorprodukt von
Funktoren eröffnet uns eine Reihe weiterer geometrischer
Realisierungen, die diejenige simplizialer Mengen verallgemeinern.

Zunächst stellen wir fest, dass wir in unserer Konstruktion
simpliziale Mengen immer als diskrete simpliziale topologische Räume
betrachtet haben, und die Diskretheit genauso gut auch fallen lassen
können. Wir erhalten die geometrische Realisierung $|X| = X \otimes R$
eines simplizialen topologischen Raums $X: \Delta\op \to \Top$.
\begin{bsp}
  Wir betrachten den simplizialen topologischen Raum $X: \Delta\op \to
  \Top$, den wir aus dem (kombinatorischen) Standard-1-Simplex
  $\Delta^1$ erhalten, indem wir disjunkte Vereinigungen von Punkten
  durch disjunkte Vereinigungen von Intervallen $I = [0, 1]$ mit von
  den Identitäten induzierten Abbildungen ersetzen. Offenbar ist die
  geometrische Realiserung das Produnkt $I \times
  |\Delta^1|$. Ersetzen wir $X_0$ wieder durch zwei Punkte ${0, 1}$
  mit beliebigen Degenerationen, so erhalten wir eine zu einer
  Kreisscheibe verdickte Linie zwischen den beiden Punkten als
  Realisierung. Ersetzen wir die höheren $X_n, n \geq 2$ ebenfalls
  wieder durch Punkte mit beliebigen Randabbildungen, so sorgen deren
  Identifikationen dafür, dass die geometrische Realisierung wieder
  $|\Delta^1|$ wird.
  % TODO Interessantere Beispiele? RP^n oder so?
\end{bsp}

Weiter verallgemeinert die Konstruktion auch auf
Diagrammkategorien. Ist $I$ eine kleine Kategorie und $X: \Delta\op
\to \Top^I$ ein simpliziales $I$-System topologischer Räume, so
erhalten wir eine geometrische Realisierung $|X| = X \otimes R$ von
$X$, wenn wir $R: \Delta \to \Top \to \Top^I$ mittels des Funktors der
konstanten Darstellung auf die Diagrammkategorie
fortsetzen. Insbesondere erhalten wir eine geometrische Realisierung
für die Kategorie der Paare topologischer Räume mit stetiger
Abbildung, d. h. die Diagrammkategorie $\Top^I$ für $I$ die von $\{
\bullet \to \bullet \}$ erzeugte Kategorie. Eine Realisierung für
Garben über topologischen Räumen erhalten wir so aber nicht: Sind in
einem simplizialen $\Top^I$ alle Abbildungen étale, bedeutet das noch
nicht, dass auch die induzierte Abbildung in der geometrischen
Realisierung étale ist. Sind etwa alle Basisräume $X_n$ einpunktig, so
ist die Realisierung $|X|$ ebenfalls einpunktig. Étale Räume $F_n \to
X_n$ sind diskret, d. h. die geometrische Realisierung $|F|$ ist die
einer simplizialen Menge, also im Allgemeinen ein höherdimensionaler
CW-Komplex. Ein solcher ist nicht diskret, also nicht étale über
$|X|$. Wir erhalten die korrekte Realisierung für $\Ens_{/\Top}$,
indem wir den Kolimes in der Garbenkategorie statt in den
topologischen Räumen bilden.
\begin{prop}
  Sei $F \in [\Delta\op, \Ens_{/\Top}]$ eine simpliziale Garbe über
  topologischen Räumen mit Morphismen und $R: \Delta \to \Ens_{/\Top}$
  eine kosimpliziale Garbe über topologischen Räumen. Dann ist die
  Realisierung $F \otimes R \in \Ens_{/\Top}$ eine Garbe über der
  geometrischen Realisierung der Basisräume.
\end{prop}
\begin{proof} \label{real-enstop}
  Die Existenz der Realisierung folgt direkt aus der Beschreibung von
  Koenden als Kolimites \ref{coend-col} und der Kovollständigkeit von
  $\Ens_{/\Top}$ nach \ref{enstop-complete}. Dort wird auch der
  Basisraum des Kolimites zum Kolimes der Basisräume bestimmt.
\end{proof}
\begin{bem}
  Eine kanonische Wahl für $R$ ist etwa die konstant einelementige
  Garbe $|\Delta^n| \to |\Delta^n|$ oder ihre plumpe Variante
  $\blacktriangle^n \to \blacktriangle^n$.
\end{bem}
\begin{bem} \label{real-ensx}
  Diese geometrische Realisierung simplizialer Garben auf
  topologischen Räumen spezialisiert zu einer geometrischen
  Realisierung simplizialer Garben auf $X$: Ist
  $F: \Delta\op \to \EnsX$ eine simpliziale Garbe auf $X$, so ist ihre
  geometrische Realisierung aus \ref{real-enstop} eine Garbe
  $|F| \in \EnsX$, da die geometrische Realisierung des konstanten
  simplizialen topologischen Raums $X: [n] \mapsto X$ selbst $X$ ist.

  Für Garben $E_n$ auf diskreten Räumen $D_n$ handelt es sich um die
  geometrische Realisierung eines Pfeils simplizialer Mengen. Die
  relative Version über $X$ hiervon ist die folgende: für Garben $E_n$
  auf $X \times D_n$ für diskrete $D_n$ und zu für $f: [m] \to [n]$
  monoton von $D_n \to D_m$ induzierten Basen $D_n \times X \to
  D_m \times X$ der Garbenmorphismen ist die geometrische Realisierung
  eine Garbe über $X \times |D|$, für $|D|$ die Realisierung der
  simplizialen Menge $[n] \mapsto D_n$.
\end{bem}

Ziel unserer Überlegungen wird es sein, die Aussagen zu simplizial
konstanten Garben auf der geometrischen Realisierung eines
Simplizialkomplexes $\K$ als Garben auf dem topologischen Raum $\K$
auf die Situation simplizialer Mengen zu übertragen. Die
angesprochenen Realisierungen in \ref{real-enstop} und \ref{real-ensx}
sind dafür nicht geeignet. Das liegt daran, dass wir, um aus Garben
auf der Realisierung wieder ein Diagramm von Garben zu erhalten,
generisierende Randabbildungen benötigen. Im Fall einer simplizialen
Garbe $\Delta\op \to \Ens_{/\Top}$ sind die Randabbildungen im
Garbensystem dagegen gegenläufig zu den generisierenden Einbettungen
$|d_i|: |\Delta^{n-1}| \inj |\Delta^n|$ der Basisräume.

Wir erklären eine neue Realisierung, die diesem Anspruch gerecht wird.
Sei dazu $R: \Delta \to \Ens_{/\Top}$ eine kosimplizialer
topologischer Raum und $F: \Delta\op \to \Ens_{\sslash\Top}, [n]
\mapsto F_n \in Ens_{/X_n}$ eine simpliziale Garbe über topologischen
Räumen mit Komorphismen. Für $f: [n] \to [m]$ monoton gibt es also
eine stetige Abbildung $Ff: X_m \to X_n$ und einen Morphismus von
Garben über $X_m$: $Ff^* F_n \to F_m$. Wir erhalten einen Funktor $K$
von der Unterteilungskategorie $\Sub(\Delta)$ von $\Delta$ in die
Garben über topologischen Räumen
\begin{equation} \label{dg:cov-real}
  \begin{tikzcd}
    \Sub(\Delta) \arrow{dd}
    & {[n]}^\S \dar[mapsto, shorten <= 3em, yshift=1.5em]
    & \lar f^\S \dar[mapsto, shorten <= 3em, yshift=1.5em] \rar
    & {[m]}^\S \dar[mapsto, shorten <= 3em, yshift=1.5em] \\[25]
    & F_n \times R[n]
    & \lar Ff^* F_n \times R[n] \rar
    & F_m \times R[m] \\[-2em]
    \Ens_{/\Top}
    & \dar[shorten <= -1em]
    & \dar[shorten <= -1em]
    & \dar[shorten <= -1em] \\[-2em]
    & X_n \times R[n]
    & \lar X_m \times R[n] \rar
    & X_m \times R[m],
  \end{tikzcd}
\end{equation}
der für $f: [n] \to [m]$ in $\Delta$ auf Morphismen $f^\S \to [n]^\S$
vom universellen Morphismus $Ff^* F_n \to F_n$ über $Ff$ induziert ist
und auf Morphismen $f^\S \to [m]^\S$ durch die Morphismen $Ff^* F_n
\to F_m$ in $\Ens_{/X_m}$ sowie $Rf$. Letzterer ist tatsächlich ein
Morphismus in $\Ens_{/\Top}$, denn es kommutiert
\[ \begin{tikzcd}
  Ff^* F_n \times R[n] \dar \rar
  & F_m \times R[n] \dar \rar
  & F_m \times R[m] \dar \\
  X_m \times R[n] \rar
  & X_m \times R[n] \rar
  & X_m \times R[m].
\end{tikzcd} \]
Wir erhalten die folgende \emph{kovariante Realisierung}:
\begin{prop} \label{real-enstop-cov}
  Sei $F \in [\Delta\op, \Ens_{\sslash \Top}]$ eine simpliziale Garbe
  über topologischen Räumen mit Komorphismen und $R: \Delta \to \Top$
  ein kosimplizialer topologischer Raum. Dann ist der Kolimes $|F|$
  über den oben definierten zugehörigen Funktor $K: \Sub(\Delta) \to
  \Ens_{/\Top}$ eine Garbe über der geometrischen Realisierung $X
  \otimes R$ der Basisräume.
\end{prop}
\begin{bem} \label{real-ensx-cov}
  Diese geometrische Realisierung simplizialer Garben auf
  topologischen Räumen mit Komorphismen spezialisiert zu einer
  geometrischen Realisierung simplizialer Garben auf $X$: Ist $F:
  \Delta\op \to \Ens_{\sslash X}$ eine simpliziale Garbe auf
  topologischen Räumen mit Komorphismen und konstantem Basisraum $X$
  alias eine kosimpliziale Garbe $F\op: \Delta \to \EnsX$, so
  vereinfacht das Diagramm \ref{dg:cov-real} zu
  \[
  \begin{tikzcd}
    \Sub(\Delta) \arrow{dd}
    & {[n]}^\S \dar[mapsto, shorten <= 3em, yshift=1.5em]
    & \lar f^\S \dar[mapsto, shorten <= 3em, yshift=1.5em] \rar
    & {[m]}^\S \dar[mapsto, shorten <= 3em, yshift=1.5em] \\[25]
    & F_n \times R[n]
    & \lar[equal] Ff^* F_n \times R[n] \rar
    & F_m \times R[m] \\[-2em]
    \Ens_{/\Top}
    & \dar[shorten <= -1em]
    & \dar[shorten <= -1em]
    & \dar[shorten <= -1em] \\[-2em]
    & X \times R[n]
    & \lar[equal] X \times R[n] \rar
    & X \times R[m],
  \end{tikzcd}
  \]
  und ihre geometrische Realisierung aus \ref{real-enstop-cov} ist
  eine Garbe $|F| \in \EnsX$, der Kolimes über den Funktor $\Delta \to
  \Ens_{/\Top}$, der $f: [n] \to [m]$ monoton auf den Morphismus
  \[ \begin{tikzcd}
    F_n \times R[n] \dar \rar{F\op f \times |f|}
    & F_m \times R[m] \dar \\
    X \times R[n] \rar
    & X \times R[m]
  \end{tikzcd} \]
  schickt.

  Sind alle $X_n$ diskret, so bestimmt für $\sigma \in X_n$ der Halm
  $(F_m)_\sigma$ die konstante Garbe $(F_m)_\sigma \times R[n] \to
  {\sigma} \times R[n]$ und wir erhalten für monotones $f: [n] \to
  [m]$ Abbildungen $(F_n)_{f(\sigma)} \to (F_m)_\sigma$, die diese
  konstanten Garben verkleben. Insbesondere sind für Randabbildungen
  $d_i$ die Verklebungen Generisierungen, die angeben, wie ein Element
  des Halms am Rand eines Simplex einen Schnitt über eine Umgebung
  dieses Punkts (auch im Inneren des Simplex) definiert. Wir werden
  diese Beobachtungen in \ref{real-enstop-cov-equiv} präzisieren und
  die zunächst seltsam anmutende Konstruktion als etwas natürlicher
  wahrnehmen.

  Für die relative Version betrachten wir Basisräume $X_n = X \times
  D_n$ mit diskreten $D_n$ und von $D_m \to D_n$ induzierten
  Abbildungen. Zu $\sigma \in D_n$ gehört dann eine Garbe $F_\sigma :=
  F_n|_{\sigma \times X} \in \EnsX$ und wir erhalten für monotones $f:
  [n] \to [m]$ Garbenmorphismen $F_{f(\sigma)} \to F_\sigma$, die
  diese Garben verkleben. Wieder sind diese generisierend, erlauben
  also die Ausweitung eines $U$-Schnitts von einem Randpunkt auf einen
  $U$-Schnitt im Inneren.
\end{bem}

\subsection{Die Dualität von Nerv und Realisierung}

Wir suchen Rechtsadjungierte für unsere geometrischen
Realisierungen. Für die Realisierung simplizialer Mengen gelingt uns
das einfach.
\begin{satz}
  Der Funktor der singulären Ketten $S: \Top \to \s\Ens$, $SY
  = \Top(R \, \cdot, Y): [n] \mapsto \Top(|\Delta^n|, Y)$ ist
  rechtsadjungiert zur geometrischen Realisierung
  $|\cdot|: \s\Ens \to \Top$.
\end{satz}
\begin{proof}
  Die Rand- und Degenerationsabbildungen von $SY$ sind für $f: [n] \to
  [m]$ gegeben durch Vorschalten von $|f|: |\Delta^n| \to
  |\Delta^m|$. Wir berechnen
  \begin{align*}
    \Top(|X|, Y)
    & = \Top(\col_{\slicecat{\Delta}{r}{X}} |\Delta^n|, Y) \\
    & \iso \col_{\slicecat{\Delta}{r}{X}} \Top(|\Delta^n|, Y) \\
    & \iso \col_{\slicecat{\Delta}{r}{X}} \s\Ens(\Delta^n, \Top(R \, \cdot, Y)) \\
    & \iso \s\Ens(\col_{\slicecat{\Delta}{r}{X}} \Delta^n, \Top(R \, \cdot, Y)) \\
    & \iso \s\Ens(X, SY)
  \end{align*}  
  mit der Definition der geometrischen Realisierung im ersten Schritt
  (Gl. \ref{eq:real-colim}), der Verträglichkeit von $\Hom:
  C\op \times C \to \Ens$ mit Limites im zweiten und vierten Schritt,
  unserer Bestimmung der $n$-Simplizes als Morphismenmenge
  (Gl. \ref{eq:simp-as-hom}) im dritten Schritt und unserer
  Beschreibung einer simplizialen Menge als Kolimes über ihre
  Simplexkategorie (\ref{sset-colim}) im letzten Schritt.
\end{proof}
Während dieses Argument wieder ein sehr anschauliches ist, möchten wie
wie in \ref{coend-correct-real} erklärt, unser Argument mit den
Begriffen und Techniken von Koenden führen, um es automatisch
verallgemeinern zu können. Wir geben hier noch einmal die direkte
Übersetzung obigen Beweises in die Sprache der Koenden an, und dann
sofort die Verallgemeinerung.
\begin{proof} (\cite{Lore}, 3.2)
  Wir berechnen mit den Regeln des Koenden-Kalküls:
  \begin{align*}
     \Top(|X|, Y)
     &= \Top \left( \int^{[n]} X[n] \times R[n], Y \right) \\
     & \iso[\ref{coend-cocont}]
       \int_{[n]} \Top \big( X[n] \times R[n], Y \big) \\
     & \iso[\ref{copower}]
       \int_{[n]} \Ens \big( X[n], \Top(R[n], Y) \big) \\
     & \iso[\ref{trans-end}]
       [\Delta\op, \Ens] \big( X, \Top(R \, \cdot, Y) \big) \\
     &= \s\Ens(X, SY).
  \end{align*}
\end{proof}

\begin{theorem} [Allgemeine Nerv-Realisierungs-Dualität, \cite{Lore}, 3.2]
  \label{nerve}
  Seien $C$ eine $V$-Kategorie mit Koexponentialen und ein Funktor $R:
  S \to C$ gegeben. Dann gibt es eine Adjunktion $(|\cdot|, N)$  
  \[ C \xtofrom[N]{|\cdot|} [S\op, V] \]
  mit
  \begin{alignat*}{3}
    &|\cdot|: && X &&\mapsto \int^s X(s) \odot R(s) \qquad \text{und} \\
    &N: && Y &&\mapsto C(R \, \cdot, Y).
  \end{alignat*}
\end{theorem}
\begin{proof}
  In wörtlicher Verallgemeinerung des Vorangegangenen:
  \begin{align*}
     C(|X|, Y)
     &= C \left( \int^s X(s) \odot R(s), Y \right) \\
     & \iso[\ref{coend-cocont}]
       \int_s C \big( X(s) \odot R(s), Y \big) \\
     & \iso[\ref{copower}]
       \int_s V \big( X(s), C(R(s), Y) \big) \\
     & \iso[\ref{trans-end}]
       [S\op, V] \big( X, C(R \, \cdot, Y) \big) \\
     &= [S\op, V] (X, NY).
  \end{align*}
\end{proof}

% TODO:
% * erster Ansatz: halmweise Realisierung funktioniert nicht
% * erklären, wieso das für Ens/Top nicht funktioniert
%   bzw. Ens/Top anreichern / kart. abschließen

\subsection{Die kartesisch abgeschlossene Struktur der Garben auf $X$}

Für unsere allgemeine Dualität von Nerv und Realisierung \ref{nerve}
benötigen wir also eine bessere $V$-angereichterte Struktur auf
$C$. Wenn wir uns auf $\EnsX$ beschränken, erhalten wir sogar die
Struktur einer kartesisch abgeschlossenen Kategorie
(engl. \emph{cartesian closed category}), d. h. einer Kategorie mit
endlichen Produkten, für deren kartesische monoidale Struktur es ein
internes Hom gibt.

\begin{prop} \label{ensx-cart-closed}
  Die Kategorie $\EnsX$ ist kartesisch abgeschlossen mit Produkt
  \[ (F \times G)(U) = F(U) \times G(U) \]
  und internem Hom
  \[ (F \Implies G)(U) = \Ens_{/U}(F|_U, G|_U) \]
  jeweils mit den von den Restriktionen von $F$ und $G$ induzierten
  Restriktionen. Der étale Raum des Produkts ist gegeben durch das
  Faserprodukt über $X$:
  \[ \etalespace{F \times G}
     \iso \etalespace{F} \times_X \etalespace{G}. \]
\end{prop}
\begin{proof}
  Das Produkt erfüllt offenbar die universelle Eigenschaft in $\pEnsX$
  und ist eine Garbe, da Produkte mit dem Limes der Garbeneigenschaft
  vertauschen (Spezialfall von \ref{ensx-complete}). Das interne Hom
  besteht aus stetigen Abbildungen über $U$ und erfüllt somit die
  Garbenbedingung, die ja sogar nach der Verklebbarkeit stetiger
  Abbildungen modelliert war. Für die Adjunktion müssen wir zeigen
  \[ \EnsX(F \times G, H) \iso \EnsX(F, G \Implies H). \]
  Links stehen restriktionsverträgliche Systeme $F(U) \times G(U) \to
  H(U)$ alias $F(U) \to \Ens(G(U), H(U))$, rechts
  restriktionsverträgliche Systeme $F(U) \to \Ens_{/U}(G|_U,
  H|_U)$. Wir erhalten eine Abbildung von rechts nach links durch den
  globalen Teil $G(U) \to H(U)$ des Garbenmorphismus $G|_U \to H|_U$
  und das Exponentialgesetz in $\Ens$ und von links nach rechts durch
  Ergänzen des globalen Teils $G(U) \to H(U)$ durch verträgliche $G(V)
  \to H(V)$ als die Bilder unter $F(U) \to F(V) \to \Ens(G(V),
  H(V))$. Diese Abbildungen sind zueinander invers.

  Für den étalen Raum des Produkts erhalten wir nach der universellen
  Eigenschaft des Faserprodukts eine stetige Abbildung über $X$
  \[ \etalespace{F \times G}
  \to \etalespace{F} \times_X \etalespace{G}.
  \]
  Diese induziert auf den Halmen die Bijektionen
  \[ (F \times G)_x \iso F_x \times G_x \]
  aus dem Vertauschen endlicher Limites mit filtrierenden Kolimites.
\end{proof}
% TODO: allg. Aussage (kl. Prägarbenkat. über kart. abg. Kat. ist
% kart. abg. aus Ninja-Yoneda-Lemma herleiten)

Diese Struktur einer kartesisch abgeschlossenen Kategorie macht
$\EnsX$ insbesondere zu einer über sich selbst tensorierten Kategorie
im Sinne von \ref{copower}. Wir erhalten einen Nerv-Funktor für die
geometrische Realisierung simplizialer Garben auf $X$ aus \ref{nerve}.

\subsection{Kategorien von Garben über topologischen Räumen}

Wir betrachten die Kategorienfaserungen $\Ens_{/\Top} \to \Top$ mit
Morphismen den stetigen Abbildungen zwischen den étalen Räumen über
der stetigen Abbildung in der Basis sowie $\Ens{\sslash \Top} \to
\Top$ mit Opkomorphismen als Morphismen, d. h. für $F \in \EnsX$ und
$G \in \Ens_{/Y}$:
\[ \Ens_{\sslash \Top}(F, G) = \coprod_{f: X \to Y} \EnsX(f^* G, F). \]

Wir möchten einen Nerv-Funktor nicht nur für die Realisierung
simplizialer Garben über $X$ finden, sondern auch für simpliziale
Garben über variablen topologischen Räumen, also für simpliziale
Objekte in $\Ens_{/\Top}$ und $\Ens_{\sslash \Top}$. Dafür benötigen
wir wieder eine monoidal abgeschlossene Struktur auf diesen
Kategorien.

Die Kategorie $\Ens_{/\Top}$ besitzt endliche Produkte, die
algebraisch gegeben sind durch Rückzug und Produkt und topologisch
durch Bilden der Produkträume. Konkret:
\begin{prop}
  Seien $F_{1,2} \in \Ens_{/X_{1,2}}$ Garben über topologischen Räumen
  $X_1$ und $X_2$. Dann erfüllt die Garbe
  \[ F_1 \times F_2 := \pr_1^* F_1 \times \pr_2^* F_2 \in \Ens_{/X_1 \times X_2} \]
  mit $\pr_{1,2}: X_1 \times X_2 \to X_{1,2}$ den Projektionen die
  universelle Eigenschaft des Produkts von $F_1$ und $F_2$ in
  $\Ens_{/\Top}$. Für ihren étalen Raum gilt:
  \[ \etalespace{F_1 \times F_2} = \etalespace{F_1} \times \etalespace{F_2} \]
  und $\etalespace{F_1 \times F_2} \to X_1 \times X_2$ ist durch das
  Produkt der $\etalespace{F_{1,2}} \to X_{1,2}$ gegeben.
\end{prop}
\begin{proof}
  % TODO: formal Faserprodukt-Bijektion angeben
  Die Beschreibung von $\Ens_{/\Top}$ als Paare topologischer Räume
  mit étaler Abbildung zeigt die Aussage über den étalen Raum des
  Produkts. Die induzierte Abbildung $\etalespace{F_1}
  \times \etalespace{F_2} \to X_1 \times X_2$ ist ein Homöomorphismus
  auf der Produktmenge der Umgebungen, auf denen $\etalespace{F_{1,2}}
  \to X_{1,2}$ Homöomorphismen sind.
  
  Für die algebraische Beschreibung erhalten wir mit der Offenheit der
  Projektionen $\pr_{1,2}$ und \ref{ensx-cart-closed} für die Schnitte
  über Basismengen $U_1 \times U_2$:
  \begin{align*}
    (F_1 \times F_2)(U_1 \times U_2)
    &\iso (\pr_1^* F_1)(U_1 \times U_2) \times (\pr_2^* F_2)(U_1 \times U_2) \\
    &\iso F_1(U_1) \times F_2(U_2).
  \end{align*}
  Wir erhalten also einen Garbenmorphismus über $X_1 \times X_2$ von
  der algeraischen zur topologischen Beschreibung, indem einem Paar
  $(s, t) \in F_1(U_1) \times F_2 \times U_2$ der Schnitt $s \times t:
  U_1 \times U_2 \to \etalespace{F_1} \times \etalespace{F_2}$
  zugeordnet wird. Dieser Morphismus induziert auf den Halmen die
  Bijektion $(F_1 \times F_2)_{x, y} \iso (F_1)_x \times (F_2)_y$ aus
  dem Vertauschen von endlichen Produkten mit filtrierenden Kolimites.
\end{proof}
\begin{bem}
  Auf ähnliche Weise kann man auch für $\Ens_{\sslash \Top}$ endliche
  Produkte konstruieren: es handelt sich (wegen der opponierten
  Fasern) um das \emph{Ko}produkt der mit den Projektionen auf den
  Produktraum zurückgezogenen Garben.
\end{bem}
Auch dieses Verfahren können wir für beliebige Limites und Kolimites
durchführen und so \ref{ensx-complete} übertragen:
\begin{satz} \label{enstop-complete}
  Die Kategorie der Garben auf topologischen Räumen mit Morphismen
  $\Ens_{/\Top}$ ist vollständig und kovollständig.
\end{satz}
\begin{proof}
  Die Kategorie der topologischen Räume $\Top$ ist vollständig und
  kovollständig mit der Initial- bzw. Finaltopologie für die Limes-
  bzw. Kolimesmengen. Sei $F_i \in \Ens_{/X_i}$ für eine kleine
  Kategorie $I$ ein System von Garben auf topologischen Räumen. Wir
  erhalten für $\pr_i: \lim_i X_i \to X_i$ die Projektionen den Limes
  durch
  \[ \lim_i F_i := \lim_i \pr_i^* F_i \in \Ens_{\lim_i X_i}. \]
  Für eine Garbe $G \in \Ens_{/Y}$ und einen Kegel $G \to F_i$
  faktorisieren zunächst die Morphismen in der Basis $f_i: Y \to X_i$
  eindeutig über $\pr_i$: $f_i = \pr_i \circ g_i$. Wir erhalten mit
  der universellen Eigenschaft des Rückzugs und der Adjunktion $(f^*,
  f_*)$:
  \[ \Ens_{/\Top}(G, F_i) \iso \Ens_{/\lim_i X_i}(g_{i*} G, \pr_i^* F_i) \]
  einen Kegel in $\Ens_{/\lim_i X_i}$, für den der angegebene Limes
  nach \ref{ensx-complete} das universelle Problem löst.

  Für den Kolimes finden wir ebenfalls eine eindeutige Faktorisierung
  über den Kolimes $\ins_i: X_i \to \col_i X_i$ in der Basis und
  erhalten den Kolimes in $\Ens_{/\Top}$ als
  \[ \col_i F_i := \col_i \ins_{i*} F_i \in \Ens_{/\col_i X_i}. \]
  Man beachte, dass hierbei der Prägarbenkolimes garbifiziert wird.
\end{proof}

Wir könnten erwarten, dass wie das Produkt auch das interne Hom von
$EnsX$ in unsere relative Situation übertragen werden kann. Dies
gelingt tatsächlich aber im Allgemeinen nicht, denn in diesem Fall
erhielten wir durch Nachschalten des Faserfunktors $\Ens_{/\Top} \to
\Top$ bzw. $\Ens_{\sslash \Top} \to \Top$ ein zum kartesischen Produkt
adjungiertes internes Hom in der Kategorie der topologischen Räume,
was bekanntermaßen in dieser Allgemeinheit nicht möglich ist
(\cite{???}). Wir müssen uns also wieder auf eine bequeme Kategorie
topologischer Räume mit internem Hom einschränken.

Die häufige Wahl $\CGHaus$ ist für uns ungeeignet, denn der étale Raum
einer Garbe über einem kompakt erzeugten Hausdorffraum ist im
Allgemeinen kein Hausdorffraum mehr (betrachte etwa die Garbe der
stetigen Funktionen nach $\R$). Abhilfe schafft uns eine Konstruktion
aus \cite{Vogt}, die die den kompakt erzeugten Räumen
zugrundeliegenden Gedanken verallgemeinert. Wir geben hier nur die
Ergebnisse an.

Äquivalent zu unserer (der \emph{point-set-}Topologie entspringenden)
Definition kompakt erzeugter Räume ist die folgende Charakterisierung:
\begin{lemma}[\ref{def:cg}, Variante]
  Ein topologischer Raum $X$ ist kompakt erzeugt genau dann, wenn
  gilt: Eine Teilmenge $U \subset X$ ist offen genau dann, wenn ihr
  Urbild unter allen stetigen Abbildungen $K \to X$, $K$ kompakt,
  offen ist.
\end{lemma}
\begin{proof}
  Unsere Bedingung besagt, dass $X$ die Finaltopologie bezüglich des
  Systems der $K \to X$, $K$ kompakt tragen soll. Die Bedingung aus
  der ursprünglichen Definition ist dieselbe für das System der
  Inklusionen kompakter Mengen $K \subset X$. Da jede stetige Abbilung
  $K \to X$, $K$ kompakt, über die Inklusion ihres kompakten Bilds
  faktorisiert, ist letzteres System in ersterem konfinal und die
  Finaltopologien stimmen überein.
\end{proof}
Der in \ref{real-prodcuts} angesprochene zur Inklusion
Rechtsadjungierte $k: \Top \to \CG$ lässt sich nun auch beschreiben
als das Versehen der $X$ zugrundeliegenden Menge mit der genannten
Finaltopologie. Der Raum $kX$ ist dann sogar ein Kolimes über das
System der $K \to X$, $K$ kompakt, mitsamt den Morphismen über $X$
(\cite{Vogt}, 1.1).

Nun verallgemeinern wir (\cite{Vogt}, 1): Sei $\mathcal{I}$ eine
nichtleere volle Unterkategorie von $\Top$ (für $\CG$ die kompakten
Räume). Betrachte die Kategorie $\slicecat{\mathcal{I}}{}{X}$ und $kX
:= \col_{\slicecat{mathcal{I}}{}{X}} X$. Bezeichne die volle
Unterkategorie der topologischen Räume $X$ mit $kX \cong X$ mit
$\mathcal{K}$. Dann ist $k: \Top \to \mathcal{K}$ ein Funktor und
rechtsadjungiert zur Inklusion $\mathcal{K} \to \Top$. Es gilt
$\mathcal{I} \subset \mathcal{K}$.
\begin{bem}
  Dual zu \ref{refl-sub} heißt eine volle Unterkategorie mit zur
  Inklusion Rechtsadjungiertem \emph{koreflektiv}, der
  Rechtsadjungierte heißt \emph{Koreflektor}.
\end{bem}

Wir nehmen nun an, dass $\mathcal{I}$ die folgenden Axiome erfüllt:
\begin{enumerate}
  \item Ist $A \subset X$ ein abgeschlossener Unterraum eines Objekts
    $X \in \mathcal{I}$, dann gilt $A \in \mathcal{K}$.
  \item $\mathcal{I}$ ist abgeschlossen unter endlichen kartesischen
    Produkten (Produkten in $\Top$).
  \item Sind $X, Y \in \mathcal{I}$, so ist die Auswertungsabbildung
    \begin{align*}
      \ev_{X, Y}: \Top_{co}(X, Y) \times X &\to Y,
      (f, x) &\mapsto f(x)
    \end{align*}
    stetig. Dabei ist $\Top_{co}(X, Y)$ die Morphismenmenge $\Top(X,
    Y)$ versehen mit der kompakt-offen Topologie.
\end{enumerate}
Dann besitzt $\mathcal{K}$ die Struktur einer kartesisch
abgeschlossenen Kategorie mit Produkten
\[ X \otimes Y \: := \: k(X \times Y) \]
den ``k-ifizierungen'' der Produkte in $\Top$ und internem Hom
\[ X \Implies Y \: := \: k(\Top_{co}(X, Y)) \] 
(\cite{Vogt}, 3).

\begin{defn}
  Ein topologischer Raum heißt \emph{lokalkompakt} (im starken Sinne),
  wenn jeder Punkt eine Umgebungsbasis aus kompakten Mengen besitzt.
\end{defn}
\begin{bem}
  Dies ist eine stärkere Bedingung als \emph{lokal kompakt} (im
  schwachen Sinne) wie in \ref{cg-crit} zu sein. Jene stimmt überein
  mit unserer Konvention für ``lokal Eigenschaft'' und wird daher
  getrennt geschrieben. Für Hausdorffräume stimmen beide Begriffe
  überein.
\end{bem}
Die volle Unterkategorie der lokalkompakten topologischen Räume
$\mathcal{L} \subset \Top$ erfüllt die Axiome und hat deshalb eine
zugehörige kartesisch abgeschlossene Unterkategorie $\mathcal{K}$, die
die kompakt erzeugten Hausdorffräume und lokalkompakten Räume umfasst
(\cite{Vogt}, 5.3).

Der Vorteil dieser Konstruktion im Vergleich zu den kompakt erzeugten
Hausdorffräumen ist folgende Beobachtung:
\begin{prop}
  \begin{enumerate}[label=(\roman*)]
    \item Ein étaler Raum über einem lokalkompakten topologischen Raum
      ist lokalkompakt.
      % TODO golt das sogar für die koreflektive Hülle??
    \item Die Kategorie der lokalkompakten topologischen Räume
      $\mathcal{L}$ ist
      %% TODO ???
      %% tensoriert über K oder so
      %% kartesisch abgeschlossen. <-- stimmt (vmtl.) nicht!
  \end{enumerate}
\end{prop}
\begin{proof}
  \begin{enumerate}[label=(\roman*)]
    \item Étale Abbildungen sind nach Definition lokale
      Homöomorphismen, Lokalkompaktheit ist eine lokale Eigenschaft.
    \item Das Produkt $X \times Y$ aus $\Top$ ist lokalkompakt, falls
      beide Faktoren es sind und stimmt dann auch mit den Produkten in
      $\mathcal{L}$ und $\mathcal{K}$ überein.
      % TODO ???
  \end{enumerate}
\end{proof}

Unsere Hoffnung, eine kartesisch abgeschlossene Struktur auf
$\Ens_{/\Top}$ zu finden, wird sich nicht erfüllen. Wir können aber
erklären, wieso.

\begin{satz}
  Es gibt keine kartesisch abgeschlossene Struktur auf
  $\Ens_{/C} \subset \Ens_{/\Top}$ für $C \subset \Top$ eine
  nichtleere volle Unterkategorie der topologischen Räume.
\end{satz}
\begin{proof}
  Wir erinnern daran, dass ein Morphismus in $\Ens_{/C}$ ein
  kommutatives Quadrat
  \[ \begin{tikzcd}
    F \arrow[r] \dar{p} & G \dar{q} \\
    X \arrow[r] & Y
  \end{tikzcd} \]
  ist. In diesem Beweis notieren wir Objekte in $\Ens_{/C}$ mitsamt
  ihrer Basis als $F \to X$.
  
  Zunächst zeigen wir, dass Produkt und internes Hom von $\Ens_{/C}$
  ein Produkt und internes Hom in $C$ liefern. Betrachte dazu die
  volltreuen Einbettungen $C \to \Ens_{/C}$ durch faserweise initiale
  Garben $\iota: X \mapsto (\varnothing \to X)$ (konstant leer) und
  durch faserweise terminale Garben $\tau: X \mapsto (X
  \xrightarrow{\id} X)$ (konstant einelementig). Wir haben natürliche
  Bijketionen
  \begin{align*}
    \Ens_{/C}(\iota X, U \to L) &\iso C(X, L) \qquad \text{und} \\
    \Ens_{/C}(\tau X, U \to L) &\iso C(X, U).
  \end{align*}
  durch Betrachten des unteren bzw. oberen Pfeils des kommutativen
  Quadrats.

  Sei $(U \to L) = \tau Y \times \tau Z$ das Produkt in
  $\Ens_{/C}$. Dieses erfüllt die universelle Eigenschaft des Produkts
  insbesondere für Testobjekte der Form $\iota X$ und wir erhalten aus
  der universellen Eigenschaft durch Einschränken auf den unteren
  Pfeil
  \[ \begin{tikzcd}
    \Ens_{/C}(\iota X, U \to L) \rar{\sim}
    \arrow{d}{\sim}
    & \Ens_{/C}(\iota X, \tau Y) \times \Ens_{/C}(\iota X, \tau Z)
    \dar{\sim} \\
    C(X, L) \rar{\sim}
    & C(X, Y) \times C(X, Z).
  \end{tikzcd} \]
  Es erfüllt also $L$ die universelle Eigenschaft des Produkts $Y
  \times Z$ in $C$.  Betrachten wir Testobjekte der Form $\tau X$, so
  erhalten wir entsprechend durch Einschränken auf den oberen Pfeil
  auch, dass $U \iso Y \times Z$ ist und die Abbildung dazwischen nach
  dem Yoneda-Lemma der eindeutige Isomorphismus zwischen den
  universellen Objekten. Es ist also $\tau X \times \tau Y \iso \tau
  (X \times Y)$ in $\Ens_{/C}$. Analog finden wir
  \begin{equation} \label{eq:iota-prod}
    \iota X \times (G \to Y) \iso \iota (X \times Y).
  \end{equation}
  
  Mit derselben Technik erhalten wir auch für $(U \to L) = (\tau Y
  \Implies \tau Z)$ das interne Hom, dass wegen
  \begin{align*}
    C(X, U)
    &\iso \Ens_{/C}(\tau X, U \to L) \\
    &\iso \Ens_{/C}(\tau X \times \tau Y, \tau Z) \\
    &\iso C(X \times Y, Z)
  \end{align*}
  $U$ ein internes Hom in $C$ für $Y$ und $Z$ ist und wegen
  \begin{align*}
    C(X, L)
    &\iso \Ens_{/C}(\iota X, U \to L) \\
    &\iso \Ens_{/C}(\iota X \times \tau Y, \tau Z) \\
    &\iso \Ens_{/C}(\iota (X \times Y), \tau Z) \\
    &\iso C(X \times Y, Z)
  \end{align*}
  auch $L$. Wieder ist nach dem Yoneda-Lemma $U \to L$ der eindeutige
  Isomorphismus zweier einen Funktor darstellenden
  Objekte \footnote{Solche bis auf eindeutigen Isomorphismus
    eindeutigen Objekte werden wir wie üblich kanonisch wählen und die
    eindeutigen Isomorphismen als Identitäten schreiben.}.

  Es muss also $C$ kartesisch abgeschlossen sein.

  Ein Morphismus
  \[ \begin{tikzcd}
    F \arrow[r] \dar{p} & G \dar{q} \\
    X \arrow[r] & Y,
  \end{tikzcd} \]
  ist nach der Kommutativität des Quadrats ein Element des
  Faserprodukts von Mengen
  \[ \Ens_{/C}(F \to X, G \to Y) = \Top(F, G) \times_{\Top(F, Y)} C(X, Y), \]
  wobei die Abbildungen nach $\Top(F, Y)$ durch Nachschalten von $q$
  bzw. Vorschalten von $p$ gegeben sind.

  Wir bestimmen nun mit einem bekannten Trick die den internen
  Hom-Objekten zugrundeliegenden Mengen. Bezeichne $v: \Top \to \Ens$
  den Vergissfunktor. Wir können den Funktor
  \begin{alignat*}{2}
    \Ens_{/C} \: &\to \: \Top \: &&\to \: \Ens, \\
    (U \to L) \: &\mapsto \: U \: &&\mapsto \: vU
  \end{alignat*}
  darstellen durch $\Ens_{/C}(\tau \point, \cdot)$
  und den Funktor
  \begin{alignat*}{2}
    \Ens_{/C} \: &\to \: C \: &&\to \: \Ens, \\
    (U \to L) \: &\mapsto \: L \: &&\mapsto \: vL
  \end{alignat*}
  durch $\Ens_{/C}(\iota \point, \cdot)$. Wir erhalten
  \begin{align*}
    \Ens_{/C}(\tau \point, (F \to X) \Implies (G \to Y))
    &\iso \Ens_{/C}(\tau \point \times (F \to X), G \to Y) \quad & \\
    &\iso \Ens_{/C}(F \to X, G \to Y) & \text{und} \\
    \Ens_{/C}(\iota \point, (F \to X) \Implies (G \to Y))
    &\iso \Ens_{/C}(\iota \point \times (F \to X), G \to Y) \\
    &\iso \Ens_{/C}(\iota X, G \to Y) \\
    &\iso C(X, Y),
  \end{align*}
  da $\tau \point$ das terminale Objekt von $\Ens_{/C}$ ist sowie nach
  Gleichung \ref{eq:iota-prod}.

  Die Abbildung im internen Hom-Objekt bestimmen wir mit den
  Funktorialitäten: Das Vorschalten von $\iota X \to (F \to X)$
  induziert einen Morphismus $\big( (F \to X) \Implies (G \to Y) \big)
  \to \big( \iota X \Implies (G \to Y) \big)$ bzw.
  \[ \begin{tikzcd}
    \Top(F, G) \times_{\Top(F, Y)} C(X, Y) \dar \rar
    & \Top(\varnothing, G) \times_{\Top(\varnothing, Y)} C(X, Y) \dar \rar{\sim}
    & C(X, Y) \dar{\id} \\
    C(X, Y) \rar{\id}
    & C(X, Y) \rar{\id}
    & C(X, Y).
  \end{tikzcd} \]
  Dabei kommt die Identität in der Basis aus dem Yoneda-Lemma für
  \[ \Ens_{/C}(\iota \point, (F \to X) \Implies (G \to Y))
  \Isotrafo \Ens_{/C}(\iota \point, \iota X \Implies (G \to Y)). \]

  Die Abbildung im Hom-Objekt ist also (wie nicht anders zu erwarten)
  die Projektion auf den zweiten Faktor. Soll diese étale sein, so
  muss %???

  % Bijektion der Adjunktion ist sogar Homöo über off. Homöo in Basis!
\end{proof}
  

%% TODO:
%% * Rainer M. Vogt: internes Hom für lokalkompakte gibt einen Nerv für
%%   Ens/Top; Nope, erklären wieso nicht
%% * erwähnen, dass keine Einheit/Koeinheits-Isos zu erwarten, aber auf
%%   Homotopie: Quillen-Äq. von Modellkategorien



% TODO:
% * konkret machen (im Fall von Y_n diskret vllt)
% * charakterisiere das Bild -> kleiner machen mit kleinerem Intervall?
%   (Moer)
% * deriviert???

% V-enriched Struktur durch stetige-Funktionen-Garbe?

\end{document}
