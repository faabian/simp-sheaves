% include latex header (\usepackage, \newcommand etc.) 
\documentclass[a4paper]{article}
% \usepackage[left=3cm,right=3cm,top=3cm,bottom=2cm]{geometry} % page settings
\usepackage{amsmath}
\usepackage{amssymb}
\usepackage{amsthm}
\usepackage{etoolbox}
\usepackage[ngerman]{babel}
\usepackage[utf8]{inputenc}
\usepackage{mathtools}

\setlength{\parskip}{\medskipamount}
\setlength{\parindent}{0pt}

\newtheorem{theorem}{Theorem}
\newtheorem{lemma}[theorem]{Lemma}
\newtheorem{prop}[theorem]{Proposition}
\newtheorem{kor}[theorem]{Korollar}
\newtheorem{satz}[theorem]{Satz}
\newtheorem{defn}[theorem]{Definition}

\DeclareMathOperator{\Cat}{Cat}
\DeclareMathOperator{\poset}{poset}
\DeclareMathOperator{\EnsX}{Ens_{/X}}
\DeclareMathOperator{\pEnsX}{pEns_{/X}}
\DeclareMathOperator{\AbX}{Ab_{/X}}
\DeclareMathOperator{\pAbX}{pAb_{/X}}
\DeclareMathOperator{\OffX}{Off_X}
\DeclareMathOperator{\Ens}{Ens}
\DeclareMathOperator{\Ob}{Ob}
\DeclareMathOperator{\Der}{Der}
\DeclareMathOperator{\Ab}{Ab}
\DeclareMathOperator{\sKons}{s-Kons}
\DeclareMathOperator{\Ket}{Ket}
\DeclareMathOperator{\EnsB}{Ens_{/\B}}

\newcommand{\B}{\mathcal{B}}
\newcommand{\op}{^\mathrm{op}}
\newcommand{\iso}{\xrightarrow{\sim}}
\newcommand{\qiso}{\xrightarrow{\approx}}
\newcommand{\open}{\subset\kern-0.58em\circ}  % only possible in math mode
\newcommand{\K}{\mathcal{K}}
\newcommand{\Kreal}{|\K|}
\newcommand{\Z}{\mathbb{Z}}
\newcommand{\DerAbK}{\Der(\Ab_{/\Kreal})}
\newcommand{\DerskK}{\Der_{\mathrm{sk}}(\Kreal)}
\newcommand{\AbKr}{\Ab_{/\Kreal}}
\newcommand{\sKonsK}{\sKons(\K)}


\begin{document}

\title{Simpliziale Garben}
\author{Fabian Glöckle}
\date{\today}
% \maketitle

\chapter{Simpliziale Garben}

In diesem Kapitel beschäftigen wir uns mit simplizialen Objekten in
der Kategorie der Garben auf einem topologischen Raum $X$. Während
Simplizialkomplexe ungerichtete Graphen verallgemeinern,
verallgemeinern simpliziale Mengen gerichtete Graphen, und werden uns
so bei der geometrischen Realisierung von Objekten in
Diagrammkategorien von Garben zur Verfügung stehen.

Wir betrachten die Menge $\Delta$ der nichtleeren endlichen
Ordinalzahlen. Ihre Elemente sind von der Form $\{0, 1, \dots, n\}$
für ein $n \in \N$, welche wir kurz mit $[n]$ bezeichnen werden. Wir
verstehen diese Mengen $[n]$ als angeordnete Mengen.

\begin{defn}
  Die Simplex-Kategorie $\Delta$ ist die Kategorie der endlichen
  nichtleeren Ordinalzahlen versehen mit monotonen Abbildungen als
  Morphismen.

  Ist $C$ eine Kategorie, so bezeichnen wir eine Prägarbe auf $\Delta$
  mit Werten in $C$ (d. h. einen Funktor $\Delta\op \to C$) als
  simpliziales Objekt in $C$.
\end{defn}

Unseren gewohnten Sprechweisen folgend nennen wir simpliziale Objekte
in $C$ auch kurz ``simpliziale $C$'' und sprechen etwa von
simplizialen Mengen und simplizialen Garben. Wir notieren die
Funktorkategorien simplizialer Objekte in $C$ auch kurz mit $\s C =
[\Delta\op, C]$.

Die monotonen Abbildungen $[m] \to [n]$ werden von zwei besonders
einfachen Klassen monotoner Abbildungen erzeugt: Zum einen von den
Randabbildungen, den eindeutigen Injektionen $d_i^n: [n - 1] \to [n]$,
die $i \in \{0, \dots, n\}$ nicht treffen, zum anderen von den
Degenerationsabbildungen, den eindeutigen Surjektionen $s_i^n: [n +
1] \to [n]$, für die $i \in \{0, \dots n\}$ zweielementiges Urbild
hat.

\begin{lemma}[\cite{??}]
  \begin{enumerate}[label=(\roman*)]
  \item Die Rand- und Degenerationsabbildungen erfüllen die Relationen
    \begin{align*}
    d_j^{n+1} d_i^n = d_i^{n+1} d_{j-1}^n \quad \text{für} \quad i < j
    s_j^n s_i^{n+1} = s_i^n s_{j+1}^{n+1} \quad \text{für} \quad i \leq j
    s_j^{n-1} d_i^n = d_i^{n-1} s_{j-1}^{n-2} \quad \text{für} \quad i < j
    s_j^{n-1} d_i^n = \id_{[n-1]} \quad \text{für} \quad i = j \text{ oder } i = j+1
    s_j^{n-1} d_i^n = d_{i-1}^{n-1} s_j^{n-2} \quad \text{für} \quad i > j + 1
    \end{align*}
 
  \item Sei $f: [m] \to [n]$ monoton. Dann hat $f$ eine eindeutige
    Darstellung
    \[ f = d_{i_1}^n \dots d_{i_s}^{n-s+1} s_{j_t}^{m-t} \dots s_{j_1}^{m-1} \]
    mit $n \geq i_1 > \dots > i_s \geq 0$, $m > j_1 > \dots > j_t \geq
    0$ und $n = m-t+s$.
    
  \item Die vom Köcher der endlichen nichtleeren Ordinalzahlen mit den
  Rand- und Korandabbildungen und den angegebenen Relationen erzeugte
  Kategorie ist isomorph zu $\Delta$.
  \end{enumerate}
\end{lemma}
\begin{proof}
% TODO
\end{proof}


Die einfachsten Beispiele nichttrivialer simplizialer Mengen sind die
Standard-$n$-Simplizes, die sich als die darstellbaren Funktoren
$\Delta^n = \Delta(\cdot, [n])$ beschreiben lassen. Wir erhalten einen
Funktor
\begin{align*}
  \Delta &\to \s\Ens, \\
  [n] &\mapsto \Delta^n = \Delta(\cdot, [n]).
\end{align*}

Ist $X \in \s\Ens$ eine simpliziale Menge, so bezeichnen wir $X_n :=
X([n])$ als die Menge der $n$-Simplizes von $X$. Nun sind die
$n$-Simplizes von $X$ gerade die ``Bilder des $n$-ten
Standardsimplizes in $X$'' sind, präziser
\[ X_n \iso \s\Ens(\Delta^n, X). \]
In der Tat besagt das Yoneda-Lemma, dass die Transformationen des
freien Funktors $\Delta^n = \Delta(\cdot, [n])$ zum Funktor $X$ in
natürlicher Bijektion stehen zu $X([n])$.

Das folgende Lemma zeigt, dass sich eine simpliziale Menge $X$
vollständig durch ihre Simplizes $\Delta^n \to X$ verstehen lässt.
Wir betrachten dazu die Simplexkategorie von $X$, d. h. die Kategorie
$\Delta \downarrow X$, wobei wir in der Notation unterschlagen, dass
wir mittels $[n] \mapsto \Delta^n$ abstrakte Simplizes $[n]$ als
simpliziale Mengen auffassen. Konkret ist also ein Objekt unserer
Simplexkategorie von $X$ ein Morphismus $\Delta^n \to X$ und ein
Morphismus in der Simplexkategorie ein von $[n] \to [m]$ induzierter
Morphismus simplizialer Mengen $\Delta^n \to \Delta^m$ über $X$.

\begin{lemma}
  Sei $X \in \s\Ens$ eine simpliziale Menge. Dann gilt
  \[ X \iso \col_{\Delta^n \to X} \Delta^n, \]
  mit dem Kolimes über die Simplexkategorie von $X$.
\end{lemma}
\begin{proof}
  Bezeichne $C$ obigen Kolimes. Das System, über das der Kolimes
  gebildet wird, sichert uns nach der universellen Eigenschaft des
  Kolimes einen eindeutigen Morphismus $C \to X$. Wir müssen also
  zeigen, dass dieser Bijektionen auf den $n$-Simplizes
  induziert. Tatsächlich definiert ein $n$-Simplex $\Delta^n \to C$
  durch Nachschalten von $C \to X$ einen $n$-Simplex in $X$. Diese
  Zuordnung ist bijektiv, denn wir erhalten eine Umkehrabbildung, wenn
  wir $\Delta^n \to X$ den zugehörigen Morphismus in den Kolimes
  $\Delta^n \to C$ zuordnen.
\end{proof}

Wir erklären nun die geometrische Realisierung simplizialer
Mengen. Der Unterschied zur geometrsichen Realsierung von
Simplizialkomplexen ist im Wesentlichen die Möglichkeit, Simplizes
``wiederzuverwenden'', was zu Identifikationen in der geomtrischen
Realisierung führt. Beispielsweise soll die $S^1$ als geometrische
Realisierung eines 1-Simplizes realisiert werden, dessen Endpunkte
übereinstimmen.

Formal definieren wir zunächst die geometrische Realsierung des
Standard-$n$-Simplex als den abgeschlossenen geometrischen
Standard-$n$-Simplex
\[ |\Delta^n| = \{ x \in \R^{n+1} | 0 \leq x_i \leq 1, \sum_{i=0}^n x_i = 1 \}. \]

% TODO geometrische Standardsimplizes und Randabb., Degenerationen

und fordern, dass sich die Realisierung mit Kolimites vertrage:
\[ |\col_i X_i| \iso \col_i |X_i|. \]
Das erreichen wir, indem wir für eine simpliziale Menge $X$ setzen
\[ |X| := \col_{\Delta^n \to X} |\Delta^n|, \]
wieder mit dem Kolimes über die Simplexkategorie von $X$.

Ein Morphismus simplizialer Mengen $X \to Y$ induziert nun durch
Nachschalten von $X \to Y$ einen Funktor auf den Simplexkategorien
$\Delta \downarrow X \to \Delta \downarrow Y$ und damit auch auf den
Kolimites eine stetige Abbildung $|X| \to |Y|$. Wir erhalten also den
Funktor der geometrischen Realisierung $| \cdot |: \s \Ens \to \Top$.
\end{document}

