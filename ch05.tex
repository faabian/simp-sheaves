% include latex header (\usepackage, \newcommand etc.) 
\documentclass[a4paper]{article}
% \usepackage[left=3cm,right=3cm,top=3cm,bottom=2cm]{geometry} % page settings
\usepackage{amsmath}
\usepackage{amssymb}
\usepackage{amsthm}
\usepackage{etoolbox}
\usepackage[ngerman]{babel}
\usepackage[utf8]{inputenc}
\usepackage{mathtools}
\usepackage{tikz-cd}
\usepackage{enumitem}
\usepackage{hyperref}

\setlength{\parskip}{\medskipamount}
\setlength{\parindent}{0pt}

\theoremstyle{plain}
\newtheorem{theorem}{Theorem}
\newtheorem{lemma}[theorem]{Lemma}
\newtheorem{prop}[theorem]{Proposition}
\newtheorem{kor}[theorem]{Korollar}
\newtheorem{satz}[theorem]{Satz}
%% \providecommand*{\lemmaautorefname}{Lemma}
%% \providecommand*{\propautorefname}{Prop.}
%% \providecommand*{\korautorefname}{Korollar}
%% \providecommand*{\satzautorefname}{Satz}

\theoremstyle{definition}
\newtheorem{defn}[theorem]{Definition}

\theoremstyle{remark}
\newtheorem{bem}[theorem]{Bemerkung}

\DeclareMathOperator{\Cat}{Cat}
\DeclareMathOperator{\poset}{poset}
\DeclareMathOperator{\EnsX}{Ens_{/X}}
\DeclareMathOperator{\pEnsX}{pEns_{/X}}
\DeclareMathOperator{\AbX}{Ab_{/X}}
\DeclareMathOperator{\pAbX}{pAb_{/X}}
\DeclareMathOperator{\OffX}{Off_X}
\DeclareMathOperator{\Ens}{Ens}
\DeclareMathOperator{\Ob}{Ob}
\DeclareMathOperator{\Der}{Der}
\DeclareMathOperator{\Ab}{Ab}
\DeclareMathOperator{\sKons}{s-Kons}
\DeclareMathOperator{\Ket}{Ket}
\DeclareMathOperator{\EnsB}{Ens_{/\B}}
\DeclareMathOperator{\im}{im}
\DeclareMathOperator{\Id}{Id}
\DeclareMathOperator{\id}{id}
\DeclareMathOperator{\colf}{colf}
\DeclareMathOperator{\limf}{limf}
\DeclareMathOperator{\Top}{Top}

\newcommand{\etalespace}[1]{\overline{#1}}
\newcommand{\B}{\mathcal{B}}
\newcommand{\op}{^\mathrm{op}}
\newcommand{\iso}{\xrightarrow{\sim}}
\newcommand{\qiso}{\xrightarrow{\approx}}
\newcommand{\fromqiso}{\xleftarrow{\approx}}
\newcommand{\open}{\subset\kern-0.58em\circ}  % only possible in math mode
\newcommand{\K}{\mathcal{K}}
\newcommand{\Z}{\mathbb{Z}}
\newcommand{\R}{\mathbb{R}}
\newcommand{\DerAbK}{\Der(\Ab_{/|\K|})}
\newcommand{\DerskK}{\Der_{\mathrm{sk}}(|\K|)}
\newcommand{\DerpskK}{\Der^+_{\mathrm{sk}}(|\K|)}
\newcommand{\AbKr}{\Ab_{/|\K|}}
\newcommand{\sKonsK}{\sKons(\K)}
\newcommand{\inj}{\hookrightarrow}
\newcommand{\surj}{\twoheadrightarrow}
\newcommand{\Iff}{\Leftrightarrow}
\newcommand{\Implies}{\Rightarrow}
\newcommand{\cc}{^{\bullet}}  % chain complex
\newcommand{\from}{\leftarrow}


\begin{document}

\title{Simpliziale Garben}
\author{Fabian Glöckle}
\date{\today}
% \maketitle

\chapter{Simpliziale Garben}

In diesem Kapitel beschäftigen wir uns mit simplizialen Objekten in
der Kategorie der Garben auf einem topologischen Raum $X$. Während
Simplizialkomplexe ungerichtete Graphen verallgemeinern,
verallgemeinern simpliziale Mengen gerichtete Graphen, und werden uns
so bei der geometrischen Realisierung von Objekten in
Diagrammkategorien von Garben zur Verfügung stehen.

\section{Simpliziale Mengen}

Wir betrachten die Menge $\Delta$ der nichtleeren endlichen
Ordinalzahlen. Ihre Elemente sind von der Form $\{0, 1, \dots, n\}$
für ein $n \in \N$, welche wir kurz mit $[n]$ bezeichnen werden. Wir
verstehen diese Mengen $[n]$ als angeordnete Mengen.

\begin{defn}
  Die Simplex-Kategorie $\Delta$ ist die Kategorie der endlichen
  nichtleeren Ordinalzahlen versehen mit monotonen Abbildungen als
  Morphismen.

  Ist $C$ eine Kategorie, so bezeichnen wir eine Prägarbe auf $\Delta$
  mit Werten in $C$ (d. h. einen Funktor $\Delta\op \to C$) als
  simpliziales Objekt in $C$.
\end{defn}

Unseren gewohnten Sprechweisen folgend nennen wir simpliziale Objekte
in $C$ auch kurz ``simpliziale $C$'' und sprechen etwa von
simplizialen Mengen und simplizialen Garben. Wir notieren die
Funktorkategorien simplizialer Objekte in $C$ auch kurz mit $\s C =
[\Delta\op, C]$. Opponiert bezeichnen wir einen Funktor $\Delta \to C$
als kosimpliziales Objekt in $C$.

Für $X \in \s C$ ein simpliziales Objekt notieren wir kurz $X_n =
X([n])$ und für eine monotone Abbildung $f: [m] \to [n]$ auch $f^* =
X(f)$.

Die monotonen Abbildungen $[m] \to [n]$ werden von zwei besonders
einfachen Klassen monotoner Abbildungen erzeugt: von den
\emph{Randabbildungen}, den eindeutigen Injektionen $d_i^n: [n - 1] \to [n]$,
die $i \in \{0, \dots, n\}$ nicht treffen, und den
\emph{Degenerationsabbildungen}, den eindeutigen Surjektionen $s_i^n: [n +
1] \to [n]$, für die $i \in \{0, \dots n\}$ zweielementiges Urbild
hat.

\begin{lemma}[\cite{GM}, I.2, ex. 1] \label{face-gen}

  \begin{enumerate}[label=(\roman*)] \item \label{itm:face-gen-rel}
  Die Rand- und Degenerationsabbildungen er\-fül\-len die Relationen
  \begin{align*}
    d_j^{n+1} d_i^n &= d_i^{n+1} d_{j-1}^n \quad \text{für} \quad i < j, \\
    s_j^n s_i^{n+1} &= s_i^n s_{j+1}^{n+1} \quad \text{für} \quad i \leq j, \\
    s_j^{n-1} d_i^n &=
    \begin{cases}
    d_i^{n-1} s_{j-1}^{n-2} \quad \text{für} \quad i < j, \\
    \id_{[n-1]} \quad \text{für} \quad i = j \text{ oder } i = j+1, \\
    d_{i-1}^{n-1} s_j^{n-2} \quad \text{für} \quad i > j + 1.
    \end{cases}
  \end{align*}
 
  \item \label{itm:face-gen-form} Sei $f: [m] \to [n]$ monoton. Dann
  hat $f$ eine eindeutige Darstellung
  \[ f = d_{i_1}^n \dots d_{i_s}^{n-s+1} s_{j_t}^{m-t} \dots s_{j_1}^{m-1} \]
  mit $n \geq i_1 > \dots > i_s \geq 0$, $m > j_1 > \dots > j_t \geq
  0$ und $n = m-t+s$.
    
  \item \label{itm:face-gen-cat} Die vom Köcher der endlichen
  nichtleeren Ordinalzahlen mit den Rand- und Korandabbildungen und
  den angegebenen Relationen auf den Morphismenmengen erzeugte
  Pfadkategorie ist isomorph zu $\Delta$.
\end{enumerate}
\end{lemma}
\begin{proof}
  \begin{enumerate}[label=(\roman*)]
  \item Durch Bildchen Zeichnen oder explizites Nachrechnen: Wir
  werten beide Abbildungen simultan auf allen Elementen aus und
  erhalten zum Beispiel für die erste Aussage:
  \begin{align*}
  &(0, 1, \dots, i-1, i, i+1, \cdots, j-2, j-1, j, \cdots, n-1) \\
  \xmapsto{d_i^n} \quad
  &(0, 1, \cdots, i-1, i+1, i+2, \cdots, j-1, j, j+1, \cdots, n) \\
  \xmapsto{d_j^{n+1}} \quad
  &(0, 1, \cdots, i-1, i+1, i+2, \cdots, j-1, j+1, j+2, \cdots, n+1)
  \end{align*}
  sowie
  \begin{align*}
  &(0, 1, \cdots, i-1, i, i+1, \cdots, j-2, j-1, j, \cdots, n-1) \\
  \xmapsto{d_{j-1}^n} \quad
  &(0, 1, \cdots, i-1, i, i+1, \cdots, j-2, j, j+1, \cdots, n) \\
  \xmapsto{d_i^{n+1}} \quad
  &(0, 1, \cdots, i-1, i+1, i+2, \cdots, j-1, j+1, j+2, \cdots, n+1).
  \end{align*}

  \item In einer solchen Darstellung gibt $m - t$ die Anzahl der
  Funktionswerte der Abbildung an, die $j_k$ beschreiben die Partition
  von $(0, \cdots, m)$ in zusammenhängende Abschnitte mit demselben
  Funktionswert und die $i_k$ bestimmen die Funktionswerte selbst.

  \item Jeden Morphismus in der Pfadkategorie, d. h. jedes Tupel
  komponierbarer Rand- und Degenerationsabbildungen kann mit den
  Relationen aus dem ersten Teil auf eine Form wie
  in \ref{itm:face-gen-form} gebracht werden. Umgekehrt ist
  nach \ref{itm:face-gen-form} jede monotone Abbildung aber auch als
  ein solcher Pfad darstellbar. Den Isomorphismus definiert also der
  Funktor, der auf Objekten durch die Identität und auf Morphismen
  durch die offensichtliche Zuordnung der Erzeuger der Pfadkategorie
  auf die jeweiligen Rand- und Degenerationsabbildungen gegeben ist.
\end{enumerate}
\end{proof}

Die einfachsten Beispiele nichttrivialer simplizialer Mengen sind die
Standard-$n$-Simplizes, die sich als die darstellbaren Funktoren
$\Delta^n = \Delta(\cdot, [n])$ beschreiben lassen. Wir erhalten den
Funktor
\begin{align*}
  R: \Delta\op &\to \s\Ens, \\
  [n] &\mapsto \Delta^n = \Delta(\cdot, [n]),
\end{align*}
der auf Morphismen durch Vorschalten gegeben ist, welches wir mit
$f \mapsto f^*$ notieren.

Ist $X \in \s\Ens$ eine simpliziale Menge, so bezeichnen wir $X_n :=
X([n])$ als die Menge der $n$-Simplizes von $X$. Nun sind die
$n$-Simplizes von $X$ gerade die ``Bilder des $n$-ten
Standardsimplizes in $X$'' sind, präziser
\[ X_n \iso \s\Ens(\Delta^n, X). \]
In der Tat besagt das Yoneda-Lemma, dass die Transformationen des
freien Funktors $\Delta^n = \Delta(\cdot, [n])$ zum Funktor $X$ in
natürlicher Bijektion stehen zu $X([n])$.

Das anschließende Lemma zeigt, dass sich eine simpliziale Menge $X$
vollständig durch ihre Simplizes $\Delta^n \to X$ verstehen
lässt. Dazu erklären wir zunächst Slice-Kategorien, Kategorien von
Objekten über einem gegebenen Objekt bzw. Spezialfälle von
Komma-Kategorien.
\begin{defn}
  Sei $r: C \to D$ ein Funktor und $X \in D$ ein Objekt. Dann
  bezeichnet $\slicecat{C}{r}{X}$ die Slice-Kategorie der Objekte von
  $C$ über $X$ mittels $r$, deren Objekte Objekte $Y \in C$ samt einem
  Morphismus $\pi_Y: rY \to X$ sind und deren Morphismen Morphismen
  $f: Y \to Z$ in $C$ sind, für die $rf$ ein Morphismus über $X$ ist,
  d. h. $\pi_Y = \pi_Z \circ rf$ gilt.
\end{defn}
Wir bezeichnen die Slice-Kategorie $\slicecat{\Delta}{r}{X}$ als die
Simplexkategorie von $X$. Konkret ist darin ein Objekt ein Morphismus
$\Delta^n \to X$ und ein Morphismus ein von $[n] \to [m]$ induzierter
Morphismus simplizialer Mengen $\Delta^n \to \Delta^m$ über $X$.

\begin{lemma}
  Sei $X \in \s\Ens$ eine simpliziale Menge. Dann gilt
  \[ X \iso \col_{\Delta^n \to X} \Delta^n, \]
  mit dem Kolimes über die Simplexkategorie von $X$.
\end{lemma}
\begin{proof}
  Bezeichne $K$ obigen Kolimes. Das System, über das der Kolimes
  gebildet wird, sichert uns nach der universellen Eigenschaft des
  Kolimes einen Morphismus $K \to X$. Wir müssen also zeigen, dass
  dieser Bijektionen auf den $n$-Simplizes induziert. Tatsächlich
  definiert ein $n$-Simplex $\Delta^n \to K$ durch Nachschalten von
  $K \to X$ einen $n$-Simplex in $X$. Diese Zuordnung ist bijektiv,
  denn wir erhalten eine Umkehrabbildung, wenn wir $\Delta^n \to X$
  den zugehörigen Morphismus in den Kolimes $\Delta^n \to K$ zuordnen.
\end{proof}
Der Beweis benötigte keine konkreten Eigenschaften der simplizialen
Mengen. Wir halten mit wörtlich übertragenen Beweis allgemein fest:
\begin{prop}
  Sei $F: C \to \Ens$ ein Funktor. Dann ist $F$ ein Kolimes über
  darstellbare Funktoren $C(X, \cdot)$.
\end{prop}
\begin{proof}
  Wir betrachten mittels des Funktors $r: C \to \Ens^C, X \mapsto
  C(X, \cdot)$ die Slice-Kategorie $\slicecat{C}{r}{F}$ der
  darstellbaren Funktoren über $F$ und dann das System in $\Ens^C$,
  das daraus durch Vergessen der Transformationen nach $F$
  hervorgeht. Wir bezeichnen wieder den Kolimes darüber mit $K$ und
  erhalten mit der universellen Eigenschaft eine Tranformation
  $K \Trafo F$. Diese ist eine Isotransformation, wenn sie auf allen
  Objekten $X \in C$ Bijektionen $KX \iso FX$ induziert. Nach dem
  Yoneda-Lemma entsprechen diese den Transformationen
  $C(X, \cdot) \Trafo K$ bzw. $C(X, \cdot) \Trafo F$. Diese stehen
  aber nach der universellen Eigenschaft des Kolimes und der
  Definition der Kategorie $\slicecat{C}{r}{F}$ in Bijektion durch
  Nachschalten von $K \Trafo F$.
\end{proof}

\section{Der kosimpliziale Raum der Standardsimplizes}

Die geometrsiche Realsierung einer simplizialen Menge soll sich mit
geeigneten Identifikationen aus den geometrischen Realisierungen von
Standard-$n$-Simplizes zusammensetzen. Diese definieren wir als die
abgeschlossenen geometrischen Standard-$n$-Simplizes
\[ |\Delta^n| = \{ x \in \R^{n+1} | 0 \leq x_i \leq 1, \sum_{i=0}^n x_i = 1 \}. \]

Ein Morphismus $f: [m] \to [n]$ induziert eine stetige Abbildung $|f|:
|\Delta^m| \to |\Delta^n|$ auf den zugehörigen Simplizes. Nach
\ref{face-gen} reicht es, diese für Rand- und Degenerationsabbildungen
anzugeben. Für die Randabbildung $|d_i|$ handelt es sich dabei um die
Inklusion der $i$-ten Kante, d. h. in Koordinaten das Einfügen einer
Null an der $i$-ten Stelle, für die Degenerationen $|s_i|$ um den
Kollaps der $i$-ten Kante, d. h. in Koordinaten die Ersetzung der
$i$-ten und ihrer darauffolgenden Koordinate durch ihre Summe. In
Formeln:
\begin{align*}
  |d_i^n|(x_0, \cdots, x_{n-1})
  &= (x_0, \cdots, x_{i-1}, 0, x_i, \cdots, x_{n-1}), \\
  |s_i^n|(x_0, \cdots, x_{n+1})
  &= (x_0, \cdots, x_{i-1}, x_i + x_{i+1}, x_{i+2}, \cdots, x_{n+1}).
\end{align*}
Wir erhalten einen Funktor $R: \Delta \to \Top, [n] \mapsto \Delta^n$,
genannt der \emph{kosimpliziale Raum der Standardsimplizes}.

\section{Geometrische Realisierung simplizialer Mengen}

Wir erklären nun die geometrische Realisierung simplizialer
Mengen. Der Unterschied zur geometrischen Realsierung von
Simplizialkomplexen ist im Wesentlichen die Möglichkeit, Simplizes
wiederzuverwenden und zu degenerieren, was zu Identifikationen in der
geometrischen Realisierung führt. Ein Fall von ``Wiederverwendung''
ist etwa die Realisierung der $S^1$ als 1-Simplex, dessen Endpunkte
übereinstimmen. Degeneration bedeutet, dass niedererdimensionale
Simplizes auch die Rolle höherdimensionaler Simplizes übernehmen
können. Wir können etwa unser Beispiel modifizieren und die $S^n$ als
$n$-Simplex realisieren, bei dem alle niederdimensionalen Kanten in
einem Punkt zusammenfallen.

Die geometrische Realisierung von Standard-$n$-Simplizes haben wir
gerade gesehen. Nun fordern wir, dass sich die Realisierung mit
Kolimites vertrage:
\[ |\col_i X_i| \iso \col_i |X_i|. \]
Wenn dies der Fall sein soll, so müssen wir mit der Darstellung
\[ X := \col_{\Delta^n \to X} \Delta^n \]
aus \ref{sset-colim} auf jeden Fall setzen
\[ |X| := \col_{\Delta^n \to X} |\Delta^n|. \]
Debei wird der Kolimes wieder über die Simplexkategorie von $X$
gebildet, nun allerdings mit den induzierten stetigen Abbildungen aus
\ref{} als Systemmorphismen.

Ein Morphismus simplizialer Mengen $X \to Y$ induziert nun durch
Nachschalten von $X \to Y$ einen Funktor auf den Simplexkategorien
$\Delta \downarrow X \to \Delta \downarrow Y$ und damit auch auf den
Kolimites eine stetige Abbildung $|X| \to |Y|$. Wir erhalten also den
Funktor der geometrischen Realisierung $| \cdot |: \s \Ens \to \Top$.

Unsere Konstruktion erfüllt tatsächlich:
\begin{prop}
  Der Funktor der geometrischen Realisierung simplizialer Mengen $|
  \cdot|$ vertauscht mit beliebigen Kolimites über kleine
  Indexkategorien.
\end{prop}
\begin{proof}
  Zu zeigen ist, dass die Bildung der Simplexkategorie mit Kolimites
  vertauscht, d. h. für einen Funktor $X: I \to \s\Ens$ gilt:
  \[ \slicecat{\Delta}{r}{\col_i X_i} \iso \col_i \slicecat{\Delta}{r}{X_i}. \]
  Beide Kategorien bestehen aber aus Objekten der Form $\Delta^n \to
  X_i$ für ein $i \in I$, die identifiziert werden, falls sie durch
  Nachschalten von $X_i \to X_j$ auseinander hervorgehen.
\end{proof}

Wie ganz allgemein können wir den Kolimes topologischer Räume auch
explizit mittels Koprodukt und Koegalisator ausschreiben. Wir
verstehen die Mengen $X_n$ als diskrete topologische Räume und
erhalten:
\[ |X| \iso (\coprod_n X_n \times |\Delta^n|) / \sim \]
mit der Quotiententopologie, die durch die von
\[ (x, |f|(p)) \sim (f^*(x), p) \]
für alle monotonen $f: [m] \to [n]$ erzeugte Äquivalenzrelation
gegeben ist.

\begin{bem}
Diese Konstruktion lässt sich interpretieren als das Tensorprodukt der
Funktoren
\begin{align*}
  X: \Delta\op &\to \Top \quad \text{und} \\
  R: \Delta &\to \Top, [n] \mapsto |\Delta^n|,
\end{align*}
wobei die Mengen $X_n$ als diskrete topologische Räume aufgefasst
werden (\cite{Moer}, III.1):
\[ |X| = X \otimes R. \]
\end{bem}

\section{Sparsame Realisierung durch nichtdegenerierte Simplizes}

Während die Definition simplizialer Mengen und ihrer Realisierung mit
Degenerationsabbildungen wie gesehen von einem formalen Standpunkt aus
sehr elegant ist, ist für die konkrete Arbeit häufig eine explizitere
Form der Realisierung praktischer, die ``unnötige Simplizes von
vornherein weglässt''.
\begin{defn}[\cite{GM}, I.2.9]
  Sei $X \in \s\Ens$ eine simpliziale Menge. Ein Simplex $x \in X_n$
  heißt degeneriert, falls es einen Simplex $y \in X_m$ und eine
  surjektive monotone Abbildung $s: [n] \to [m], n > m$ gibt mit $x =
  s^* y$.
\end{defn}
Andernfalls heißt ein Simplex nichtdegeneriert. Für $X \in \s\Ens$
bezeichne $NX_n$ die Menge der nichtdegenerierten $n$-Simplizes von
$X$.
\begin{lemma}[\cite{MIT}, Prop. 9] \label{degen-repr}
  Sei $X \in \s\Ens$ eine simpliziale Menge und $x \in X_n$ ein
  $n$-Simplex. Dann gibt es eine eindeutige Darstellung $x = s^* y$
  für $y \in NX_m$ einen nichtdegenerierten Simplex von $X$ und $s:
  [n] \to [m]$ eine surjektive monotone Abbildung.
\end{lemma}
\begin{proof}
  Ist $x$ nichtdegeneriert, wähle $y = x$ und $s =
  \id_{[n]}$. Andernfalls gibt es nach Definition ein $y$ eine
  Surjektion mit der gewünschten Eigenschaft. Diese sind eindeutig
  nach den Relationen aus \ref{face-gen}, vergleiche \cite{MIT},
  Prop. 9.
\end{proof}

Wir definieren nun die \emph{sparsame geometrsiche Realisierung} einer
simplizialen Menge $X$ wie folgt:
\[ \| X \| := \coprod_n NX_n \times |\Delta^n| / \sim_N ,\]
wobei wie gehabt $NX_n$ die diskrete Topologie trägt und die
Äquivalenzrelation erzeugt ist von
\[ (x, d_i p) \sim_N (y, s p), \]
falls $s^* y = d_i^* x$ die eindeutige Darstellung von $d_i^* x$ aus
\ref{degen-repr} ist. Diese Äquivalenzrelation lässt sich
interpretieren als das Umgehen der mittleren Schritte in der Rechnung
\[ (x, d_i p) \sim (d_i^* x, p) \sim (s^* y, p) \sim (y, s p). \]

\begin{satz}
  Die von der Inklusion
  \[ \coprod_n NX_n \times |\Delta^n| \inj \coprod_n X_n \times |\Delta^n| \]
  induzierte Abbildung $\|X\| \iso |X|$ ist ein Homöomorphismus.
\end{satz}
\begin{proof}
  Die Abbildung existiert und ist stetig nach der universellen
  Eigenschaft topologischer Quotienten, denn die Äquivalenzrelation
  $\sim$ umfasst $\sim_N$. Sie ist bijektiv, denn für die
  dazukommenden Punkte in degenerierten Simplizes $s^* x$ gilt ohnehin
  $(s^* x, p) \sim (x, s p)$. Weiter ist sie offen: Ist $U \open
  \|X\|$ eine offene Teilmenge, so berechnen wir ihr Bild in $|X|$
  durch das Bild ihres Urbilds $V$ in $\coprod_n NX_n \times
  |\Delta^n|$ unter
  \[ \coprod_n NX_n \times |\Delta^n|
  \inj \coprod_n X_n \times |\Delta^n| \xrightarrow{q} |X|. \]

  Bezeichne $\overline{V}$ den Abschluss von $V \open \coprod_n X_n
  \times |\Delta^n|$ unter $\sim$-Äquivalenz. Es gilt $\overline{V} =
  q^{-1}(q(V))$ und wir müssen nach Definition der Quotiententopologie
  zeigen, dass $\overline{V}$ offen ist. Bezeichne für $x$ einen
  nichtdegenerierten Simplex von $X$ den Schnitt von $V$ mit dem zu
  $x$ gehörigen geometrischen Simplex $|\Delta^n|$ mit $V_x$. Bei
  Übergang von $V$ zu $\overline{V}$ kommen dann alle Punkte $(s^* x,
  p)$ mit $(x, s p) \in U_x$ hinzu. Das Urbild der offenen Menge $U_x$
  unter dem stetigen $s$-Kollaps $|s|$ ist dann natürlich wieder
  offen.
\end{proof}

\section{Iterative Konstruktion der geometrischen Realisierung}

Wir geben eine weitere Interpretation der geometrischen Realisierung
als iteratives Ankleben geometrischer Simplizes an ihre Ränder an und
werden so insbesondere sehen, dass die geometrische Realisierung einen
Hausdorffraum liefert (\cite{Moer}, III.1).

Bezeichne dazu $\|X^{\leq k}\| = \coprod_{n = 0}^k NX_n \times
|\Delta^n| / \sim_N $ die geometrische Realisierung durch
nichtdegenerierte Simplizes der Dimension $\leq k$. Wir erhalten
Einbettungen $\|X^{\leq k}\| \inj \|X\|$ sowie $\|X\| = \bigcup_k
\|X^{\leq k}\|$, denn die Äquivalenzrelation von ganz $\|X\|$ fügt
keine neuen Identifikationen auf den Teilmengen $\|X^{\leq k}\|
\subset \|X\|$ hinzu.

Wir können die $\|X^{\leq k}\|$ iterativ konstruieren. Betrachte dazu
die stetigen Abbildungen $\pi_k$, die uns den ``$k$-dimensionalen
Teil'' von $\|X\|$ liefern:
\[ pi_k: NX_k \times |\Delta^k|
  \inj \coprod_{n=0}^k NX_n \times |\Delta^n|
  \surj \|X^{\leq k}\|. \]

\begin{prop}
  Sei $X \in \s\Ens$ eine simpliziale Menge. Dann ist
  \shorthandoff{"}
  \[ \begin{tikzcd}
    NX_k \times \del |\Delta^k| \arrow[r, hook] \arrow[d]
    \arrow[dr, phantom, "\ulcorner"]
    & NX_k \times |\Delta^n| \arrow[d, "\pi_k"] \\
    \|X^{\leq k-1}\| \arrow[r, hook]
    & \|X^{\leq k}\|
  \end{tikzcd} \]
  ein Pushout topologischer Räume ist. Dabei ist die Abbildung links
  auf der $i$-ten Kante $d_i: |\Delta^{k-1}| \inj \del |\Delta^k|$
  gegeben durch die Verknüpfung
  \[ NX_k \times |\Delta^{k-1}|
  \xrightarrow{d_i^* \times \id} NX_{k-1} \times |\Delta^{k-1}|
  \xrightarrow{\pi_{k-1}} \|X^{\leq k-1}\|. \]
\end{prop}
\begin{proof}
  Sowohl der Pushout des Diagramms als auch $\|X^{\leq k}\|$ sind
  Quotienten von $\coprod_{n=0}^k NX_n \times |\Delta^n|$ nach
  gewissen Äquivalenzrelationen. Bezeichne die Einschränkung von
  $\sim_N$ auf $\coprod_{n=0}^k NX_n \times |\Delta^n|$ mit
  $\sim_N^k$. Dann ist $(x, d_i p) \sim (d_i^* x, p)$ eine erzeugende
  Relation für $\sim_N^k$ genau dann, wenn es eine erzeugende Relation
  für $\sim_N^{k-1}$ ist oder es gilt $x \in NX_k$ und wegen

  % Problem: die Relation hat ja eigentlich degenerierte ausgespart
  % und lautet deshalb anders..

  %% Dann sind Paare $(x, p), (y, q) \in \coprod_{n=0}^k NX_n \times
  %% |\Delta^n|$ äquivalent bezüglich $\sim_N^k$ genau dann, wenn sie
  %% äquivalent bezüglich $\sim_N^{k-1}$ sind oder es gilt: Es ist
  %% o. E. $x \in NX_k$.
\end{proof}

\section{Kompakt erzeugte Hausdorffräume}

Die geometrischen Realisierungen simplizialer Mengen liegen
tatsächlich sogar in einer in ihren kategoriellen Eigenschaften
bequemen (\emph{convenient}) Kategorie topologischer Räume, den
kompakt erzeugten Hausdorffräumen.

\begin{defn}[\cite{Steenrod}]
  Ein topologischer Raum $X$ heißt kompakt erzeugt, falls gilt: eine
  Teilmenge $A \subset X$ ist abgeschlossen, falls $A \cap K$
  abgeschlossen ist für jedes Kompaktum $K \subset X$.
\end{defn}
\begin{bem}
  Äquivalent dazu ist: eine Teilmenge $U \subset X$ ist offen, falls
  $U \cap K$ offen ist für jedes Kompaktum $K \subset X$.
\end{bem}
Wir notieren volle Unterkategorie der kompakt erzeugten Hausdorffräume
in den topologischen Räumen mit $\CGHaus$. Die kompakt erzeugten
Hausdorffräume umfassen eine sehr große Klasse an relevanten
topologischen Räumen, etwa CW-Komplexe oder erstabzählbare Räume.
% TODO citation needed
Für uns relevant sind die folgenden Kriterien:
\begin{lemma}
  \begin{enumerate}[label=(\roman*)]
    \item Ist $X$ ein lokal kompakter topologischer Raum, so ist $X$
      kompakt erzeugt.
  \item Ist $p: X \surj Y$ eine Quotientenabbildung (d. h. $Y$ trägt
    die Finaltopologie bezüglich $p$) und $X$ kompakt erzeugt, so ist
    auch $Y$ kompakt erzeugt.
  \end{enumerate}
\end{lemma}
\begin{proof}
  \begin{enumerate}[label=(\roman*)]
    \item Sei $U \subset X$ mit $U \cap K$ offen für alle $K \subset
      X$ kompakt. Mit der Lokalkompaktheit von $X$ wählen wir zu jedem
      $x \in X$ eine kompakte Umgebung $K_x$ und erhalten, dass $U =
      \bigcup_{x \in X} U \cap K_x$ offen ist als Vereinigung offener
      Mengen.
    \item Sei $A \subset Y$ mit $A \cap K$ abgeschlossen für alle $K
      \subset Y$ kompakt. Mit der Finaltopologie auf $Y$ gilt, dass $A
      \subset Y$ genau dann abgeschlossen ist, wenn $p^{-1}(A) \subset
      X$ abgeschlossen ist. Sei nun $\tilde{K} \subset X$ kompakt. Es
      ist $p^{-1}(A) \cap \tilde{K}$ abgeschlossen, denn $p(p^{-1}(A)
      \cap \tilde{K}) = p(\tilde{K}) \cap A$ ist abgeschlossen nach
      Voraussetzung ($p(\tilde{K}) \subset Y$ ist kompakt als stetiges
      Bild eines Kompaktums). Da $X$ kompakt erzeugt ist, folgt nun,
      dass $p^{-1}(A)$ abgeschlossen ist und somit auch $A \subset Y$.
  \end{enumerate}
\end{proof}



\end{document}
